\documentclass[prb,12pt]{revtex4-2}
%fonts
% Palatino for main text and math
\usepackage[osf,sc]{mathpazo}

% Helvetica for sans serif
% (scaled to match size of Palatino)
\usepackage[scaled=0.90]{helvet}

% Bera Mono for monospaced
% (scaled to match size of Palatino)
\usepackage[scaled=0.85]{beramono}
\usepackage{amsmath, amssymb,physics,amsfonts,amsthm}
\usepackage[most]{tcolorbox}
\usepackage{enumitem}
\usepackage{cancel}
\usepackage{booktabs}
\usepackage{tikz}
\usepackage{hyperref}
\usepackage{enumitem}
\usepackage{transparent}
\usepackage{float}
\usepackage{multirow}
\newtheorem{Theorem}{Theorem}
\newtheorem{Proposition}{Theorem}
\newtheorem{Lemma}[Theorem]{Lemma}
\newtheorem{Corollary}[Theorem]{Corollary}
\newtheorem{Example}[Theorem]{Example}
\newtheorem{Remark}[Theorem]{Remark}
\theoremstyle{definition}
\newtheorem{Problem}{Problem}
\theoremstyle{definition}
\newtheorem{Definition}[Theorem]{Definition}
\newenvironment{parts}{\begin{enumerate}[label=(\alph*)]}{\end{enumerate}}
%tikz
\usetikzlibrary{patterns}
% definitions of number sets
\newcommand{\N}{\mathbb{N}}
\newcommand{\R}{\mathbb{R}}
\newcommand{\Z}{\mathbb{Z}}
\newcommand{\Q}{\mathbb{Q}}
\newcommand{\C}{\mathbb{C}}
\begin{document}
\title{Mechanics}
	\author{Jun Wei Tan}
	\email{jun-wei.tan@stud-mail.uni-wuerzburg.de}
	\affiliation{Julius-Maximilians-Universit\"{a}t W\"{u}rzburg}
	\date{\today}
	\maketitle
	\section{The Problem}
\begin{Problem}
	Consider an $n$-dimensional harmonic oscillator with Lagrangian
	\[
		L=\frac{1}{2}m\dot{x}_i\dot{x}_i-\frac{1}{2}m\omega^2x_ix_i
	.\] 
	(We sum over repeated indices here, so there is an implicit sum due to the repeated $i$). Show that
	\[
		x_i\to x_i+\varphi_{ij}\dot{x}_j
	\]
	is a symmetry transformation of the system if $\varphi$ is a symmetric $n\times n$ matrix, and find the corresponding conserved quantities.
\end{Problem}
\begin{proof}
	We consider the variation in L
	\begin{align*}
		\delta L=&\pdv{L}{x_i}\delta x_i+\pdv{L}{\dot{x}_i}\delta\dot{x}_i\\
		=&m\dot{x}_i\varphi_{ij}\ddot{x}_j-m\omega^2x_i\varphi_{ij}\dot{x}_j\\
		=&\frac{1}{2}\dv{t}\left[ \dot{x}_i\varphi_{ij}\dot{x}_j-\omega^2 x_i\varphi_{ij}x_j \right]\equiv \dv{G}{t}
	\end{align*}
	where we have used the symmetry of $\varphi$ in the last step (one can verify this by direct differentiation). It then follows from Noether's Theorem that
	\[
		M=\pdv{L}{x_i}\delta x_i-G
	\] 
	(the Noether Charge) is a conserved quantity. We compute this explicitly and get
	\begin{align*}
		M&=m\dot{x}_i\varphi_{ij}\dot{x}_j-G\\
		 &=\frac{1}{2}m\dot{x}_i\varphi_{ij}\dot{x}_j+\frac{1}{2}m\omega^2 x_i\varphi_{ij}x_j\\
		 &=\frac{1}{2}m\left( \dot{x}_i\varphi_{ij}\dot{x}_j+x_i\varphi_{ij}x_j \right) 
	\end{align*}
	Now, since $\varphi$ is an arbitrary symmetric matrix, we can actually impose additional conditions on it, such as $\varphi_{ij}\neq 0$ for all except a certain pair $(i,j)$. It then follows that for the above quantity to be conserved for \emph{all} $(i,j)$, we must have
	\[
		\Lambda_{ij}=	\dot{x}_i\dot{x}_j+x_ix_j
	\]
	be conserved for \emph{all} $(i,j)$. \qedhere 
\end{proof}
\section{Interpretation}
It is clear that the trace is simply the Hamiltonian (energy function) of the system. The other components are not so easy to interpret. 

\section{The Issue}
A serious issue in the interpretation arises when we let $n=6$ (physically, this is achieved by considering 2 3-dimensional simple harmonic oscillators). The two oscillators should then be independent. However, the cross terms relate the velocities of the two oscillators\ldots

We can draw even more nonsensical conclusions by translating the 2nd harmonic oscillator a distance $d$ away (this is achieved by the transformation $x_4\to x_4+d$). Then the composition of the inverse transformation and the transformation we considered above is a symmetry transformation, with almost identical conserved quantities. 

These conserved quantities are then equations relating the first and second harmonic oscillators (which could be arbitrarily far away). Is there a physical way to understand this? Is there some coincidental cancellation?
\end{document}
