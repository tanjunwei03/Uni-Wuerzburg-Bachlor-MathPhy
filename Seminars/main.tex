\documentclass[prb,12pt]{revtex4-2}
%fonts
% Palatino for main text and math
\usepackage[osf,sc]{mathpazo}

% Helvetica for sans serif
% (scaled to match size of Palatino)
\usepackage[scaled=0.90]{helvet}

% Bera Mono for monospaced
% (scaled to match size of Palatino)
\usepackage[scaled=0.85]{beramono}
\usepackage{amsmath, amssymb,physics,amsfonts,amsthm}
\usepackage{enumitem}
\usepackage{cancel}
\usepackage[most]{tcolorbox}
\usepackage{booktabs}
\usepackage{tikz}
\usepackage{hyperref}
\usepackage{enumitem}
\usepackage{transparent}
\usepackage{float}
\usepackage{multirow}
\newtheorem{Theorem}{Theorem}
\newtheorem{Proposition}{Proposition}
\newtheorem{Lemma}[Theorem]{Lemma}
\newtheorem{Corollary}[Theorem]{Corollary}
\newtheorem{Example}[Theorem]{Example}
\newtheorem{Remark}[Theorem]{Remark}
\theoremstyle{definition}
\newtheorem{Problem}{Problem}
\theoremstyle{definition}
\newtheorem{Definition}[Theorem]{Definition}
\newenvironment{parts}{\begin{enumerate}[label=(\alph*)]}{\end{enumerate}}
%tikz
\usetikzlibrary{patterns}
\usepackage{tikz-cd}
% definitions of number sets
\newcommand{\N}{\mathbb{N}}
\newcommand{\R}{\mathbb{R}}
\newcommand{\Z}{\mathbb{Z}}
\newcommand{\Q}{\mathbb{Q}}
\newcommand{\C}{\mathbb{C}}
\allowdisplaybreaks
\begin{document}
\title{11.04.24 Want to Publish in Nature Seminar}
	\author{Jun Wei Tan}
	\email{jun-wei.tan@stud-mail.uni-wuerzburg.de}
	\affiliation{Julius-Maximilians-Universit\"{a}t W\"{u}rzburg}
	\date{\today}
	\maketitle

	\section{Writing of a News Article}
	Use the standard structure of essays

	Intro, rising action, climax, falling action, resolution
\begin{enumerate}
\item Rising action is why this work needs to be done now.
\item Climax is usually one sentence in the abstract 
\end{enumerate}

In secondary science, it is possible to start with the results as the introduction. Then, the rising action is the previous technical difficulties.

\begin{tcolorbox}
	Use short sentences. $\le 20$ words.
\end{tcolorbox}
One word sentences are okay. Can be bold and have a bit of personality when doing secondary science.

\section{Primary Manuscripts}

Intent when reporting results is to explain, not hype. Don't hype with words like remarkable, novel, paradigm shift, unexpected. 

Hyperbolic phrases cloud what the work is meant to accomplish

Avoid using the word \underline{novel} and any related words.

``see why this is important'' not only ``see what we've done''

Avoid descriptive or chronological writing style. No ``and i did this, and i did that''.

\section{Cover Letter}
\begin{enumerate}
	\item Summarise the advance (in bullets!)
	\item Note related work
	\item Suggest and/or exclude referees
	\item Anything else you want the editor to know \emph{in confidentiality}
	\item Do not paraphrase the abstract
	\item Do not have $\ge 1$ page of strictly text
	\item Do not overuse claims of primacy
	\item Do not forget diversity of gender and geographic location when recommending referees
	\item Do not suggest friends or former mentors as referees.
\end{enumerate}

Cover letter can really help the paper.

Nature editors can find out that editor and peer reviewer had an affair

nytimes science jargon caves

Results can note down relations to other works.

In an abstract
\begin{enumerate}
	\item Make the question being addressed clear
	\item Mention critical methodology
	\item DO NOT references figures
\end{enumerate}

Nature
\begin{enumerate}
	\item 200 words or less
	\item Fully references paragraph
	\item Contains 1-2 essential keywords for indexing and search engines
	\item ``here we show'' statement
	\item NO figures, or mention of figures
\end{enumerate}
\end{document}
