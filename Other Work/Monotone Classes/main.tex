\documentclass[prb,12pt]{revtex4-2}

\usepackage{amsmath, amssymb,physics,amsfonts,amsthm}
\usepackage{enumitem}
\usepackage[most]{tcolorbox}
\usepackage{cancel}
\usepackage{booktabs}
\usepackage{tikz}
\usepackage{hyperref}
\usepackage{enumitem}
\usepackage{transparent}
\usepackage{float}
\usepackage{multirow}
\newtheorem{Theorem}{Theorem}
\newtheorem{Proposition}{Theorem}
\newtheorem{Lemma}[Theorem]{Lemma}
\newtheorem{Corollary}[Theorem]{Corollary}
\newtheorem{Example}[Theorem]{Example}
\newtheorem{Remark}[Theorem]{Remark}
\theoremstyle{definition}
\newtheorem{Problem}{Problem}
\theoremstyle{definition}
\newtheorem{Definition}[Theorem]{Definition}
\newenvironment{parts}{\begin{enumerate}[label=(\alph*)]}{\end{enumerate}}
%tikz
\usetikzlibrary{patterns}
\usepackage{pgfplots}
\pgfplotsset{compat=1.18}
% definitions of number sets
\newcommand{\N}{\mathbb{N}}
\newcommand{\R}{\mathbb{R}}
\newcommand{\Z}{\mathbb{Z}}
\newcommand{\Q}{\mathbb{Q}}
\newcommand{\C}{\mathbb{C}}
\allowdisplaybreaks
\begin{document}
\title{Monotone Class Theorem \& Product Measures}
	\author{Jun Wei Tan}
	\email{jun-wei.tan@stud-mail.uni-wuerzburg.de}
	\affiliation{Julius-Maximilians-Universit\"{a}t W\"{u}rzburg}
	\date{\today}
	\maketitle
	The monotone class theorem is often presented rather confusingly. I write this document in an attempt to clarify the concept (mostly to myself, but to anyone else who wishes to read this too).
	\section{Definitions}
	\begin{Definition}
		Here we denote the \emph{power set} of $X$ by $\mathcal{P}(X)$, the set of all subsets of $X$.
	\end{Definition}
	\begin{Definition}
		An algebra of sets $\mathcal{D}\subseteq  \mathcal{P}(X)$ is a collection of subsets of $X$ with the following properties:
		\begin{enumerate}
			\item $A\in \mathcal{D}\iff A^c\in \mathcal{D}$ 
			\item $\varnothing$ (or $X$) $\in \mathcal{D}$ 
			\item Closure under finite unions.
		\end{enumerate}
	\end{Definition}
\begin{Definition}
	A monotone class is a family of sets $\mathcal{M}\subseteq \mathcal{P}(X)$, so that for all monotone sequences of sets $A_1\subseteq A_2\dots, A_i\in \mathcal{M}$, we have $\bigcup_{i=1}^\infty A_i\in \mathcal{M}$, as well as if $B_1\supseteq B_2\dots$, then $\bigcap_{i=1} ^\infty B_i\in \mathcal{M}$.
\end{Definition}
\begin{Definition}
	A $\sigma$-algebra $\mathcal{A}$ is a family of subsets $M\subseteq \mathcal{P}(X)$ with the following properties
	\begin{enumerate}
		\item $X,\varnothing\in \mathcal{A}$
		\item $A\in \mathcal{A}\iff A^c\in \mathcal{A}$ 
		\item Countable unions of elements in $\mathcal{A}$ are also elements of $\mathcal{A}$.
	\end{enumerate}
	\section{Additional Properties}
\end{Definition}
We state the following well known theorem without proof (in any case, it is not difficult to prove).
\begin{Theorem}
The final condition, together with the second condition, is equivalent to countable intersections of elements in $\mathcal{A}$ being also in $\mathcal{A}$.	
\end{Theorem}
\begin{Corollary}
	It is now obvious that all $\sigma$-algebras are also monotone classes.
\end{Corollary}
\begin{Theorem}
	Algebras are closed under set differences.
\end{Theorem}
\begin{proof}
	Let $\mathcal{D}\subseteq \mathcal{P}(X)$ be an algebra of sets, and let $A,B$ be sets in $\mathcal{D}$. It follows that
	\[
	A\backslash B = A\cap B^c
	.\]
	which is in $\mathcal{D}$.
\end{proof}
\section{The Theorem}
The goal of this will be to prove the \emph{Monotone Class Theorem}:
\begin{Theorem}
	Let $\mathcal{D}\subseteq \mathcal{P}(X)$ be an algebra of subsets of $X$. Then the smallest monotone class containing $\mathcal{D}$ ist also the sigma-algebra generated by $\mathcal{D}$, or the smallest $\sigma$-algebra containing $\mathcal{D}$. 
\end{Theorem}
\begin{Remark}
	What is exactly the problem? The problem is that not all monotone classes are $\sigma$-algebras. Consider, for example, the following monotone class on $X=\{1,2,3\} $ :
	\[
		\mathcal{M}=\{\varnothing,\{1\} ,\{2\} ,\{3\} ,X \} 
	.\] 
	Clearly, it is trivially a monotone class, but not a sigma algebra.

	Here, the main problem is that finite unions are not elements of the monotone class. We will see eventually that this is the \emph{only} problem. Near the end, we will consider countable unions $\cup_{i=1}^\infty A_i$ by setting $B_i=\cup_{j=1}^i A_i$, which is a monotone sequence and using the monotone union property.
\end{Remark}
We now embark on the
\begin{proof}
	Let $\mathcal{C}\subseteq \mathcal{P}(X)$ be a collection of subsets of $X$. We define $\mathcal{A}$ to be the smallest $\sigma$-algebra containing $\mathcal{C}$ and $\mathcal{M}$ to be the smallest monotone class containing $\mathcal{C}$. 

	Since all $\sigma$-algebras are monotone classes, then $\mathcal{M}\subseteq \mathcal{A}$. It remains to prove the opposite inclusion.

	Let $P\subseteq X$ be a subset of $X$. We define a set $\Omega(P)$ to be the class of sets $Q\subseteq X$ such that
	\[
	\Omega(P)=\{Q|Q\subseteq X,P-Q\in \mathcal{M},Q-P\in \mathcal{M},P\cup Q\in \mathcal{M}\}
	.\] 
Clearly, since $\mathcal{M}$ is a monotone class, $\Omega(P)$ is also a monotone class for all $P\subseteq X$. The symmetry also shows that
\[
Q\in \Omega(P) \iff P\in \Omega(Q)
.\] 
Now we fix $P\in \mathcal{D}$. Recall that $\mathcal{D}\subseteq \mathcal{M}$. Thus, by the closure of the algebra $\mathcal{D}$ under set differences, we have $\mathcal{D}\subseteq \Omega(P)$ for all $P\in \mathcal{D}$. Since $\Omega(P)$ is a monotone class, it follows that
\[
\mathcal{M}\subseteq \Omega(P)~\forall P\in \mathcal{D}
.\] 
Thus all elements in $\mathcal{M}$ also have the defining properties of $\Omega(P)$. In particular, we conclude that $\mathcal{M}$ is closed under finite differences and unions. Since $\mathcal{D}$ is an algebra, we have $X\in \mathcal{M}$. The closure under complement property follows then from closure under differences.

The final step is as promised: We consider a countable union $\bigcup_{i=1}^\infty A_i, A_i\in \mathcal{M}$. We set $B_i=\cup_{j=1}^i A_i$. Every set $B_i$ is in $\mathcal{M}$ because of the closure of finite unions, and each $B_i$ is included in the next $B_{i+1}$. Applying the monotone property, we see that $\mathcal{M}$ is closed under countable unions, or $\mathcal{M}$ is a sigma algebra. However, since $\mathcal{A}$ is the smallest $\sigma$-algebra containing $\mathcal{D}$, we have $\mathcal{A}\subseteq \mathcal{M}$. The equality follows.
\end{proof}
\section{Applications}
\begin{Definition}
	We consider two measure spaces $(X,\mathcal{A},\mu)$ and $(Y,\mathcal{B},\nu)$. We wish to construct a measure and $\sigma$-algebra on the product $X\times Y$. We define a \emph{measurable rectangle} to be a set of the form $A\times B$, where $A\in \mathcal{A}$ and $B\in \mathcal{B}$. The set of \emph{elementary sets} $\mathcal{E}$ is the set of all finite unions of disjoint measurable rectangles.
\end{Definition}
\begin{Theorem}
$\mathcal{E}$ is an algebra	
\end{Theorem}
\begin{proof}
	Closure under intersections and finite differences follows from
	\begin{align*}
		A_1\times B_1\cap A_2\times B_2=&(A_1\cap A_2)\times (B_1\cap B_2)\\
		A_1\times B_1\backslash A_2\times B_2=&[(A_1\backslash A_2)\times B_1]\cap [(A_1\cap A_2)\times (B_1\backslash B_2)]
	\end{align*}
	Closure under finite unions follow from
	\[
	P\cup Q = (P-Q)\cup Q
	.\qedhere\] 
\end{proof}

\end{document}
