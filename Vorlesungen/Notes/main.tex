\documentclass[twoside,symmetric, openany, 12pt]{./tuftebook}

\usepackage{geometry}
\geometry{
	left=10mm, % left margin
	textwidth=170mm, % main text block
	marginparsep=0mm, % gutter between main text block and margin notes
	marginparwidth=10mm % width of margin notes
}


\usepackage{amsmath, amssymb,physics,amsfonts,amsthm}
\usepackage{enumitem}
\usepackage[most]{tcolorbox}
\usepackage{cancel}
\usepackage{booktabs}
\usepackage{tikz}
\usepackage{xurl}
\usepackage{hyperref}
\usepackage{enumitem}
\usepackage[normalem]{ulem}
\usepackage{transparent}
\usepackage{float}
\usepackage{multirow}
\usepackage{subcaption}
\usepackage{mathtools}
\usepackage{thmtools}
\usepackage{thm-restate}
\usepackage[framemethod=TikZ]{mdframed}
\mdfsetup{skipabove=1em,skipbelow=0em, innertopmargin=12pt, innerbottommargin=8pt}
\declaretheoremstyle[headfont=\bfseries\sffamily, bodyfont=\normalfont, mdframed={  } ]{thmbox}
\declaretheoremstyle[headfont=\bfseries\sffamily, bodyfont=\normalfont]{thmnobox}
%put nobreak in mdframed
\declaretheorem[numberwithin=chapter,style=thmnobox, name=Theorem]{Theorem}
\declaretheorem[sibling=Theorem, style=thmnobox, name=Definition]{Definition}
\declaretheorem[sibling=Theorem, style=thmnobox, name=Corollary]{Corollary}
\declaretheorem[sibling=Theorem, style=thmnobox, name=Postulate]{Postulate}
\declaretheorem[sibling=Theorem, style=thmnobox, name=Example]{Example}
\declaretheorem[sibling=Theorem, style=thmnobox, name=Proposition]{Proposition}
\declaretheorem[sibling=Theorem, style=thmnobox, name=Lemma]{Lemma}
\usepackage{subcaption}
\captionsetup[subfigure]{% changed <<<<<<<<<<<<<<<<
	singlelinecheck = false,
	justification=raggedright, 
	margin = {-3ex, 0mm}, % make margin font size dependent
}
%\newtheorem{Theorem}{Theorem}
%\numberwithin{Theorem}{chapter}
%\newtheorem{Proposition}{Proposition}
%\newtheorem{Lemma}[Theorem]{Lemma}
%\newtheorem{Corollary}[Theorem]{Corollary}
%\newtheorem{Example}[Theorem]{Example}
%\theoremstyle{definition}
\theoremstyle{definition}
\newtheorem{Remark}[Theorem]{Remark}
\theoremstyle{definition}
\newtheorem{Problem}{Problem}
\theoremstyle{definition}
\newenvironment{parts}{\begin{enumerate}[label=(\alph*)]}{\end{enumerate}}
%tikz	
\tcbset{breakable=true,toprule at break = 0mm,bottomrule at break = 0mm}
\usetikzlibrary{patterns}
\usetikzlibrary{matrix}
\usepackage{pgfplots}
\pgfplotsset{compat=1.18}
% definitions of number sets
\newcommand{\N}{\mathbb{N}}
\newcommand{\R}{\mathbb{R}}
\newcommand{\Z}{\mathbb{Z}}
\newcommand{\Q}{\mathbb{Q}}
\newcommand{\C}{\mathbb{C}}
\allowdisplaybreaks
%polar set
\newcommand{\polar}{\textrm{\tiny \fontencoding{U}\fontfamily{ding}\selectfont\symbol{'136}}}


\begin{document}
	\newgeometry{margin=1in}
	\begin{titlepage}
		{\begingroup% AW, Design of Books
			%\FSfont{5pl} % FontSite URW Palladio (Palatino)
			%\drop = 0.14\textheight
			\centering
			%\vspace*{\drop}
			{\Large Notes for the course in}\\[\baselineskip]
			{\Huge\bfseries QUANTUM MECHANICS}\\[\baselineskip]
			{\Large Held in SS25\\ At the JMU Würzburg\\ By Prof. Dr. }\\[\baselineskip]
			{\LARGE By}\\[\baselineskip]
			{\LARGE Jun Wei Tan}\par
			\vfill
			{Julius-Maximilians-Universit\"{a}t W\"{u}rzburg}
			\vfill
			{\small\sffamily \href{mailto:jun-wei.tan@stud-mail.uni-wuerzburg.de}{jun-wei.tan@stud-mail.uni-wuerzburg.de}}\par
			\endgroup}
	\end{titlepage}
	\restoregeometry
	\tableofcontents
\chapter{Quantum Postulates \& Density Operators}
\section{Postulates Revision}
\begin{Postulate}[Schr\"{o}dinger Equation]
	The time evolution of a state is given by
	\[
		i\hbar \pdv{t}\ket{\psi} = H\ket{\psi}
	.\] 
\end{Postulate}
\begin{Theorem}[Time Evolution]
	The time evolution of a state can be given by the exponential of the Hamiltonian operator, if it is not explicitly time dependent
	\[
		\ket{\psi(t)} = e^{-iH t / \hbar}\ket{\psi(0)}
	.\] 
\end{Theorem}
\begin{Theorem}[Time-Dependent Hamiltonians]
	Where the Hamiltonian is time-dependent, the we must use the \textbf{time-ordering operator}
	\[
		\ket{\psi(t)} = T_+ e^{i H t / \hbar} \ket{\psi(0)}
	.\] 
\end{Theorem}
The time ordering operator is given by
\begin{Definition}
	The time-ordering operator is a metaoperator acting on operators
	\[
		T_+\{ H(t_1)H(t_2)\} = \begin{cases}
			H(t_1)H(t_2) & t_1 < t_2\\
			H(t_2)H(t_1) & \text{otherwise}.
		\end{cases}
	\] 
\end{Definition}
\begin{Postulate}[General Measurement]
	A general measurement of a quantum state is defined by a set of measurement operators $\{M_k\}_{k=1}^n$ satisfying the normalisation condition
	\[
	\sum_{k=1}^{n} M_k^\dagger M_k =1
	.\] 
	After the measurement, the quantum state is given by
	\[
		\ket{\psi}\to \frac{M_k \ket{\psi}}{\sqrt{p_k} }
	\] 
	with probability
\[
	p_k = \bra{\psi}M_k^\dagger M_k\ket{\psi}
.\] 
\end{Postulate}
\begin{Example}[Projection Valued Measurement]
	In a projection valued measurement, we have a set of orthogonal projectors for the measurement operators. 
\end{Example}

\begin{Remark}[Positive Operator Valued Measurement]
	Notice that the operator $E_k = M_k^\dagger M_k$ is a positive operator. Hence, the set of measurement operators is associated with a set of positive operators $E_k$. We call these operators the POVM elements.
\end{Remark}
\section{Density Operator}
The density operator describes ``mixed states'', or states where we are not entirely certain what quantum state the quantum system is in. For example, in quantum computing, we do not know the output of a measurement, only that it will be a few states with a certain probability.
\begin{Definition}[Density Operator]
	A density operator is a positive trace 1 operator.
\end{Definition}
\begin{Remark}[Physical Construction]
	Where a system can be in any of a set of states $\{\ket{\psi_k}\}$ with probability $p_k$, the density operator is given by
	\begin{equation}\label{eq:densitymatrix}
		\rho = \sum_{k=1}^n p_k \ket{\psi_k}\bra{\psi_k}
	.\end{equation}
	We can show that it has trace 1, either by picking the states $\ket{\psi_k}$ as a basis and using the normalisation of probability, or by computing this in any other basis.
\end{Remark}
\begin{Definition}[Pure State]
	A state is pure if there is only one state in the expansion Eq. \eqref{eq:densitymatrix}. Equivalently, the density matrix is an orthogonal projector.
\end{Definition}
\begin{Remark}
	This means that we know what state the density operator is in. Note: This state could be entangled.
\end{Remark}
\begin{Corollary}
	A state is pure if and only if $\Tr \rho^2 = 1$.
\end{Corollary}
\begin{Theorem}[Time Evolution]
	The time evolution of a density matrix is given by
	\[
	\rho(t) = U\rho U^\dagger
\]
where $U$ is the propagator.
\end{Theorem}
\begin{proof}
	This is done simply by considering the expansion Eq. \eqref{eq:densitymatrix}.
\end{proof}
\begin{Theorem}[Liouville Equation]
	The time evolution of a density matrix is given by
	\[
		\dv{\rho(t)}{t} = -i\hbar [H, \rho]
	.\] 
\end{Theorem}
\begin{proof}
	We write the Schr\"{o}dinger equation and its transpose
	\begin{align*}
		i\hbar \pdv{t}\ket{\psi(t)}&=H \ket{\psi(t)}  \\
		-i\hbar \bra{\psi(t)} &= \bra{\psi(t)} H.
	\end{align*}
	Then, we differentiate Eq. \eqref{eq:densitymatrix} to get
	\begin{align*}
		\dv{t} \sum_{k=1}^{n} p_k \ket{\psi_k(t)}\bra{\psi_k(t)} &= \sum_{k=1}^{n} p_k \left[ \ket*{\dot{\psi}(t)} \bra{\psi(t)} + \ket{\psi(t)}\bra*{\dot{\psi}(t)} \right]  \\
									 &= \sum_{k=1}^{n} p_k \left[ -i \hbar H \ket{\psi_k(t)}\bra{\psi_k(t)} + i \hbar \ket{\psi_k(t)} \bra{\psi_k(t)} H\right]  \\
									 &= -i \hbar[H, \rho].\qedhere
	\end{align*}
\end{proof}
\begin{Theorem}
	A measurement of a density matrix $\rho$ has outcomes with probability
	\[
		\mathbb{P}(j) = \Tr (M_j^\dagger M_j \rho)
\]
where we denote the probability as $\mathbb{P}(j)$ to distinguish this from the probabilities in Eq. \eqref{eq:densitymatrix} and the state after the measurement is
\[
\rho \to \frac{M_k \rho M_K^\dagger}{p_k}
.\] 
\end{Theorem}
\begin{proof}
	Let us consider the expansion Eq. \eqref{eq:densitymatrix} of $\rho$ in states $\{\ket{\psi_k}\}_{k=1}^n$ and a measurement defined by measurement operators $\{M_j\}_{j=1}^m$. The probability of a measurement outcome $j$ is given by the sum of the probabilities over all states $k$:
	\begin{align*}
		\mathbb{P}(j) &= \sum_{k=1}^n p_k \bra{\psi_k}M_j^\dagger M_j \ket{\psi_k} \\ 
			      &= \sum_{k=1}^n p_k\Tr(M_j^\dagger M_j \ket{\psi_k}\bra{\psi_k}) \\
			      &= \Tr (M_k^\dagger M_k \rho),
	\end{align*}
	The density matrix after the measurement is given by
	\begin{align*}
		\rho &= \sum_{k=1}^{n} p_k \ket{\psi_k}\bra{\psi_k}\\
		     &\to \sum_{k=1}^{n} p_k \frac{M_j \ket{\psi_k}}{\sqrt{\mathbb{P}(j)} } \frac{\bra{\psi_k} M_j^\dagger}{\sqrt{\mathbb{P}(j)} }.\qedhere
	\end{align*}
\end{proof}
We have used the following lemma in the above proof;
\begin{Lemma}
	\[
		\Tr(A \ket{\psi}\bra{\psi}) = \matrixelement{\psi}{A}{\psi}
	.\] 
\end{Lemma}
\begin{proof}
	We have
	\begin{align*}
	\Tr(A \ket{\psi}\bra{\psi}) &= \sum_{k=1}^n \matrixelement{\varphi_k}{A}{\psi}\braket{\psi}{\varphi_k}\\
					    &= \braket{\psi}{\varphi_k} \matrixelement{\varphi_k}{A}{\psi}\\
					    &= \matrixelement{\psi}{A}{\psi}.\qedhere
	\end{align*}
\end{proof}
\begin{Proposition}[Composite Systems]
	For composite systems, the tensor product basis is given by the tensor products of the basis elements $\ket{i} \otimes \ket{j}$. Hence, the density matrix can be expanded in its matrix elements
	\[
		\rho = \sum_{ij\mu\nu}\lambda_{ij\mu\nu} \ket{i}\bra{j}\otimes \ket{\mu}\bra{\nu}
	.\] 
\end{Proposition}
\begin{Remark}[Separable Systems]
	If there are no cross terms, i.e. $\lambda_{ij\mu\nu}=\lambda_{ij}^A\lambda_{\mu\nu}^B$, we can write the density matrix as
	\[
	\rho = \rho_A \otimes \rho_B
\]
where
\[
	\rho_A = \sum_{ij} \lambda_{ij}^A \ket{i}\bra{j},\qquad \rho_B = \sum_{\mu\nu} \lambda_{\mu\nu}^B \ket{\mu}\bra{\nu}
.\] 
\end{Remark}
However, systems are not in general separable. The next question is how to deal with observables of one system
\section{The Interaction Picture}
The interaction picture describes evolution under a composite Hamiltonian
\[
H=H_0+H_1
.\]
While this can in principle be done all the time, in general we will use this where $H_0$ is an easy Hamiltonian to solve while $H_1$ is difficult. We want to apply concepts like in the Heisenberg picture to eliminate the rotation of the state from $H_0$. Thus, we define the state in the interaction picture to be
\[
	\ket{\psi_I(t)} = U_0^\dagger(t)\ket{\psi_S(t)}
\]
where $\ket{\psi_I}$ and $\ket{\psi_S}$ are the state vectors in the interaction and schr\"{o}dinger picture respectively and $U_0(t)$ is the propagator.

\section{The Lindblad Equation}

\end{document}
