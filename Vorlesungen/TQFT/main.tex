\documentclass[twoside,symmetric, openany, 12pt]{./tuftebook}
%\documentclass[12pt]{article}

\usepackage{geometry}
\geometry{
	left=10mm, % left margin
	textwidth=170mm, % main text block
	marginparsep=0mm, % gutter between main text block and margin notes
	marginparwidth=10mm % width of margin notes
}


\usepackage{amsmath, amssymb,physics-patch,amsfonts,amsthm}
\usepackage{enumitem}
\usepackage[most]{tcolorbox}
\usepackage{cancel}
\usepackage{booktabs}
\usepackage{tikz}
\usepackage{xurl}
\usepackage{hyperref}
\usepackage{enumitem}
\usepackage[normalem]{ulem}
\usepackage{transparent}
\usepackage{float}
\usepackage{multirow}
\usepackage{subcaption}
\usepackage{mathtools}
\usepackage{thmtools}
\usepackage{thm-restate}
\usepackage{tensor}
\usepackage[framemethod=TikZ]{mdframed}
\mdfsetup{skipabove=1em,skipbelow=0em, innertopmargin=12pt, innerbottommargin=8pt}
\declaretheoremstyle[headfont=\bfseries\sffamily, bodyfont=\normalfont, mdframed={  } ]{thmbox}
\declaretheoremstyle[headfont=\bfseries\sffamily, bodyfont=\normalfont]{thmnobox}
%put nobreak in mdframed
\declaretheorem[numberwithin=chapter,style=thmnobox, name=Theorem]{Theorem}
\declaretheorem[sibling=Theorem, style=thmnobox, name=Definition]{Definition}
\declaretheorem[sibling=Theorem, style=thmnobox, name=Corollary]{Corollary}
\declaretheorem[sibling=Theorem, style=thmnobox, name=Postulate]{Postulate}
\declaretheorem[sibling=Theorem, style=thmnobox, name=Example]{Example}
\declaretheorem[sibling=Theorem, style=thmnobox, name=Proposition]{Proposition}
\declaretheorem[sibling=Theorem, style=thmnobox, name=Lemma]{Lemma}
\usepackage{subcaption}
\captionsetup[subfigure]{% changed <<<<<<<<<<<<<<<<
	singlelinecheck = false,
	justification=raggedright, 
	margin = {-3ex, 0mm}, % make margin font size dependent
}
%\newtheorem{Theorem}{Theorem}
%\numberwithin{Theorem}{chapter}
%\newtheorem{Proposition}{Proposition}
%\newtheorem{Lemma}[Theorem]{Lemma}
%\newtheorem{Corollary}[Theorem]{Corollary}
%\newtheorem{Example}[Theorem]{Example}
%\theoremstyle{definition}
\theoremstyle{definition}
\newtheorem{Remark}[Theorem]{Remark}
\theoremstyle{definition}
\newtheorem{Problem}{Problem}
\theoremstyle{definition}
\newenvironment{parts}{\begin{enumerate}[label=(\alph*)]}{\end{enumerate}}
%tikz	
\tcbset{breakable=true,toprule at break = 0mm,bottomrule at break = 0mm}
\usetikzlibrary{patterns}
\usetikzlibrary{matrix}
\usepackage{pgfplots}
\pgfplotsset{compat=1.18}
% definitions of number sets
\newcommand{\N}{\mathbb{N}}
\newcommand{\R}{\mathbb{R}}
\newcommand{\Z}{\mathbb{Z}}
\newcommand{\Q}{\mathbb{Q}}
\newcommand{\C}{\mathbb{C}}
\allowdisplaybreaks
%polar set
\newcommand{\polar}{\textrm{\tiny \fontencoding{U}\fontfamily{ding}\selectfont\symbol{'136}}}
\allowdisplaybreaks
\tcbset{colback=white}
\renewcommand\comment[1]{ {\color{red} Comment: #1}}
\begin{document}
\iffalse
	\newgeometry{margin=1in}
	\begin{titlepage}
		{\begingroup% AW, Design of Books
			%\FSfont{5pl} % FontSite URW Palladio (Palatino)
			%\drop = 0.14\textheight
			\centering
			%\vspace*{\drop}
			{\Large Notes in}\\[\baselineskip]
			{\Huge\bfseries Field Theory}\\[\baselineskip]
			{\LARGE By}\\[\baselineskip]
			{\LARGE Jun Wei Tan}\par
			\vfill
			{Julius-Maximilians-Universit\"{a}t W\"{u}rzburg}
			\vfill
			{\small\sffamily \href{mailto:jun-wei.tan@stud-mail.uni-wuerzburg.de}{jun-wei.tan@stud-mail.uni-wuerzburg.de}}\par
			\endgroup}
	\end{titlepage}
	\restoregeometry
	\tableofcontents
\fi
\chapter{Conventions}
\section{Spacetime Coordinates}
\begin{align*}
	x^0 &= x_0 = t = -i\tau \\
	x^1 &= -x_1 = x\\
	\mathcal{L}_E &= - \mathcal{L}_M(t \to -i\tau, \partial_t \to -i\partial_\tau)\\
	x^{\pm} &= \frac{1}{\sqrt{2} }(x^0 \pm x^1) \\
	\partial_{\pm} &= \frac{1}{\sqrt{2} }(\partial_0 \pm \partial_1)
\end{align*}
In 1-dim,
\[
A_\mu B^\mu = A^0B_0 - A^1 B^1
.\] 
Complex coordinates for euclidean space:
\[
z, \overline{z}=\tau \pm ix = i\sqrt{2} x^\pm
,\]
\[
	\partial, \overline{\partial}=\frac{1}{2}\left( \pdv{\tau}\pm i\pdv{x} \right)=-\frac{i}{\sqrt{2} }(\partial_+, \partial_-) 
.\] 
and
\[
\partial \overline{\partial}=\frac{1}{4}(\partial_\tau^2+\partial_x^2)=-\frac{1}{4}(\partial_t^2 - \partial_x^2)=\frac{1}{2}\partial_+ \partial_-
.\] 
then
\[
	\dd{\tau}\dd{x}=\dd{\tau}\wedge \dd{x} = \frac{i}{2}\dd{z}\wedge \dd{\overline{z}}
.\] 
\subsection{Einstein Summation Convention}
Replace indices $\mu,\nu\in \{0,1\} $ with indices $\alpha, \beta \in \{z, \overline{z}\} $ s.t.
\begin{align*}
	x^z &=  z = \tau + ix \\
	x^{\overline{z}} &= \overline{z} = \tau - ix
\end{align*}
with derivatives
\begin{align*}
	\partial_z &= \pdv{x^z} = \pdv{z} = \frac{1}{2}(\partial_\tau - i\partial_x)\\
	\partial_{\overline{z}} &= \pdv{x^{\overline{z}}} = \frac{1}{2}(\partial_\tau + i\partial_x)
\end{align*}
We define the covariant operators
\begin{align*}
	x_z &= \frac{1}{2}(\tau - ix)\\
	x_{\overline{z}} &= \frac{1}{2}(\tau + ix)\\
	\partial^z &= \pdv{x_z} = \partial_\tau + i\partial_x\\
	\partial^{\overline{z}} &= \partial_\tau - i\partial_x
\end{align*}
This leads us to the metric
\begin{align*}
	g_{\alpha\beta} = \begin{pmatrix} 0 & \frac{1}{2} \\ \frac{1}{2} & 0 \end{pmatrix} , g^{\alpha\beta} = \begin{pmatrix}  0 & 2 \\ 2 & 0 \end{pmatrix} , \tensor{g}{_\alpha^\beta} = \begin{pmatrix} 1 & 0 \\ 0  & 1 \end{pmatrix} 
\end{align*}
\chapter{Bosonisation in Field Theory}
\section{Majorana Fermions}
Euclidean space action
\[
	S = \int \dd{\tau}\dd{x} \mathcal{L}_{MF},
\] 
where
\begin{align*}
	\mathcal{L}_{MF} &= \frac{1}{2\pi}(\overline{\psi}, \psi)\begin{pmatrix} \partial & 0 \\ 0 & \overline{\partial} \end{pmatrix} \begin{pmatrix} \overline{\psi} \\ \psi \end{pmatrix} \\
			 &= \frac{1}{2\pi}(\overline{\psi}\partial \overline{\psi}- \psi \overline{\partial}\psi)
.\end{align*} 
The term $(\overline{\psi}, \psi)^T$ is a spinor; $\psi$ and $\overline{\psi}$ are independent \emph{real} fields. The fact that these are real suggests that the fermions are Majorana fermions.
\begin{tcolorbox}[title=Euler-Lagrange Equations]
	In $d+1$-dimensional Minkowski space, the action is given by
	\[
		S = \int\dd{t}\dd[d]{x} \mathcal{L}(\phi, \partial_\mu \phi)
	.\] 
	We seek stable points in the action $S$ and do so by performing a variation in the field
	\begin{align*}
		0 &= \delta S = \int \dd{t} \dd[d]{x} \left\{ \fdv{\mathcal{L}}{(\partial_\mu \phi)} \underbrace{\var{(\partial_\mu\phi)}}_{\partial_\mu(\delta\phi)} - \fdv{\mathcal{L}}{\phi}\delta \phi \right\} \\
		  &= \int \dd{t} \dd[d]{x} \left\{ -\left( \partial_\mu \fdv{\mathcal{L}}{(\partial_\mu\phi)} -\fdv{\mathcal{L}}{\phi} \right) \delta\phi + \text{bound. terms} \right\} 
.\end{align*} 
\end{tcolorbox}
For the MF, we use the $z,\overline{z}$convention to get the EL equation
\[
	\underbrace{\partial \fdv{\mathcal{L}}{\partial \overline{\psi}}}_{-\frac{1}{2\pi}\overline{\psi}} + \underbrace{\overline{\partial}\fdv{\mathcal{L}}{(\overline{\partial}\overline{\psi})}}_0 - \underbrace{\fdv{\mathcal{L}}{\overline{\psi}}}_{\frac{1}{2\pi}\partial \overline{\psi}} = 0
.\] 
This leads to
\[
\partial \psi = 0, \qquad \overline{\psi}= \overline{\psi}(\overline{z})
.\] 
This tells us that $\overline{\psi}$ is the right mover $R$, because it depends on $\tau - ix = -i(t-x)$. By performing something similar with the other field, we get that
\[
	\overline{\partial}\psi = 0,\qquad \psi = \psi(z)
\]
which tells us that the field $\psi$ is the left mover $L$. Here, it is important to note that this does not mean that the two fields do not depend on the two different complex variables from the start; this is the \emph{equation of motion}. We started out with a Lagrangian that depended on two fields, and the equations of motion tell us that the fields are constant on the other variable.
\paragraph{Quantum Propagator}
For a feld theory defined by the action
\[
	S = \frac{1}{2}\int \dd{\tau}\dd[d]{x} \phi M \phi
\]

where $M$ is some kind of matrix, for e.g. $-\partial_\mu\partial^\mu$, the propagator is given by the time ordered correlation function
\[
\langle \mathcal{T}\phi(x) \phi(y)\rangle
.\] 
In this course, we will omit the time ordering as it is understood in all correlation functions. By direct evaluation of the functional integral, we arrive at the expression 
\[
	\langle \mathcal{T}\phi(x)\phi(y)\rangle = (M^{-1})_{xy}
.\] 
The above expression holds for both bosonic and fermionic variables.
\begin{Example}
	Compute the inverse of $M=\frac{1}{\pi}\overline{\partial}$. 
\end{Example}
\begin{proof}
	We use the identity $\overline{\partial}\partial \ln (z \overline{z}) = \pi \delta^{(2)}(z)=\pi \delta(\tau)\delta(x)$. Thus 
	\[
	\langle \psi(z) \psi(w)\rangle = \frac{1}{z-w}
\]
and
\[
\langle \overline{\psi}(\overline{z})\overline{\psi}(\overline{w})\rangle = \frac{1}{\overline{z}-\overline{w}}
.\] 
\end{proof}
\section{Dirac Fermions}
We assemble two real majorana fermions $(\overline{\psi}_1, \psi_1), (\overline{\psi}_2, \psi_2)$ into 1 complex dirac fermion
\begin{align*}
	\begin{pmatrix} \overline{\psi}_D \\ \psi_D \end{pmatrix} &= \frac{1}{\sqrt{2} } \begin{pmatrix}  \overline{\psi}_1 - i \overline{\psi}_2 \\ \psi_1 + i\psi_2 \end{pmatrix} \\
	\mathcal{L}_D &=  \frac{1}{\pi} \begin{pmatrix} \overline{\psi}_D^* & \psi_D^* \end{pmatrix} \begin{pmatrix} \partial & 0 \\ 0 & \overline{\partial} \end{pmatrix} \begin{pmatrix} \overline{\psi}_D \\ \psi_D\end{pmatrix} =\mathcal{L}_{MF_1} + \mathcal{L}_{MF_2} 
\end{align*}
which leads us to the propagator
\begin{align*}
	\langle \psi_D^*(z) \psi_D(w) \rangle &= \frac{1}{z-w}\\
	\langle \overline{\psi}_D^*(\overline{z}) \overline{\psi}(\overline{z})\rangle &= \frac{1}{\overline{z}-\overline{w}}
\end{align*}
\section{Noether's Theorem}
Suppose we have a continuous symmetry of a field $\phi(x) \to \phi(x, \lambda)$ such that
\[
	\mathcal{D}\mathcal{L}= \dv{\mathcal{L}(x, \lambda)}{\lambda}|_{\lambda=0}=\partial_\mu F^\mu(x)
.\] 
Then
\[
	J^\mu = \fdv{\mathcal{L}}{(\partial_\mu \phi)} \mathcal{D}\phi - F^\mu(x)
\]
is conserved.
\section{Bosons \& Bosonisation}
We have
\begin{align*}
	J(z) &= i\partial \Phi\\
	\overline{J}(\overline{z})&=-i \overline{\partial}\Phi
\end{align*}
The conservation law
\[
\overline{\partial}J(z)=\partial \overline{J}(\overline{z})=0
\]
is equivalent to the equation of motion. These currents can be derived from a ``symmetry'' of the bosonic theory: If we translate the fields, we get
\[
\Phi(\lambda)=\Phi+\lambda
,\]
the Lagrangian transforms as
\[
\mathcal{L}_B = \frac{1}{2\pi}\partial \Phi \overline{\partial}\Phi
\]
and thus $D\mathcal{L}_B=0$. Thus, we get the conserved currents
\begin{align*}
	J^z &= \fdv{\mathcal{L}}{\partial_z \Phi} 1 = \frac{i}{2\pi}\overline{J}(\overline{z}) \\
	J^{\overline{z}} &= \fdv{\mathcal{L}}{(\partial_{\overline{z}}\Phi)} = -\frac{i}{2\pi}J(z)
\end{align*}
The current conservation tells us that
\[
	\partial_z J^z + \partial_{\overline{z}} J^{\overline{z}} = 0
\]
or
\[
\frac{1}{\pi}\partial \overline{\partial}\Phi=0
.\] 
This is not a problem since $\Phi\to \Phi+\lambda$ since $\Phi$ is not a physical field! 

If $\Phi \to \Phi+\lambda$ were a physical symmetry, it could be spontaneously broken, contradicting the Mermin-Wagner Theorem (no goldstone modes in 1 + 1 dim!).
\section{Why non-abelian bosonization?}
Consider $N$ Dirac fermions $\begin{pmatrix} \overline{\psi}_r \\ \psi_r \end{pmatrix} , r= 1,2,\ldots, N$. The Lagrangian is
\[
	\mathcal{L}_F = \frac{1}{\pi}\sum_{r=1}^N (\overline{\psi}_r^* \partial \overline{\psi}_r + \psi_r^* \overline{\partial}\psi_r
.\] 
Abelian bosonisation in Sec 1.3 suggests that this is physically equivalent to
\[
	\mathcal{L}_B = \frac{1}{2\pi}\sum_{r=1}^N \partial \Phi_r \overline{\partial}\Phi_r
.\] 
Note: While $\mathcal{L}_F$ has a $U(N) \times U(N)$ symmetry with $\psi(z) = \begin{pmatrix}  \psi_1 \\ \vdots \\ \psi_N \end{pmatrix} $, $\mathcal{L}_F$ is also invariant under
\[
\psi(z) \to \Lambda \psi(z), \qquad \psi^\dagger(z) \to \psi^\dagger(z) \Lambda^\dagger
\]
where $\Lambda\in U(N)_L$ is a unitary $N\times N$ matrix such that $\Lambda^\dagger \Lambda = 1$. The conserved currents generate
\[
	J_{rr'}(z) = \frac{1}{\pi}:\psi_r^*\psi_{r'}:(z)
\]
and
\[
	\overline{J}_{rr'}(\overline{z}) = \frac{1}{\pi}:\overline{\psi}_r^* \overline{\psi}_{r'}:(\overline{z}) 
.\] 
We note that $J_{rr'}^\dagger = J_{r'r}$ so that the off diagonal maps map into each other under hermitian conjugation. Only $J_{rr'}^\dagger + J_{rr'}$ is hermitian. Thus, there are a total of $2 \cdot \frac{1}{2}N(N+1)$, where the first $2$ comes from the left and right movers.
\begin{align*}
	\overline{\partial}J_{rr'}(z) &= 0\\
	\partial \overline{J}_{rr'}(z) &= 0
\end{align*}
for all $r,r\in 1,\ldots, N$. However, $\mathcal{L}_B$ only has $2N$ conserved currents $\partial \Phi_r$, $\overline{\partial}\Phi_r$. Note that this problem shows up already when particles have $SU(2)$ spin. The solution is non-abelian bosonization.

For example, the WZW action
\[
S = S_0 + K \Gamma
,\]
where
\[
	S_0 = \frac{1}{16\pi}\int_{S^2}\dd{x}\dd{\tau} \Tr(\partial_\mu g \partial^\mu g^{-1}), ~\mu = 0,1
\]  
where $g(z,\overline{z})\in U(N)$ are matrix-valued fields. $\Gamma$ is defined by
\[
	\Gamma = \frac{1}{24\pi}\int_{B^3} \tilde{g}^* W
,\]
where $\tilde{g}^*$ is the pullback of 3-form $W$ onto $B^3$,
\[
	W = \Tr(\tilde{g}^{-1}\dd{\tilde{g}} \wedge \tilde{g}^{-1} \dd{\tilde{g}} \wedge \tilde{g}^{-1}\dd{\tilde{g}})
.\] 
The conserved currents in this case would be given by
\begin{align*}
	J(z) &=  \frac{1}{\pi}g^{-1}\partial g, \qquad \overline{\partial}J(z) = 0 \\
	\overline{J}(\overline{z}) &= -\frac{1}{\pi}(\overline{\partial}g)g^{-1},\qquad \partial \overline{J}(\overline{z}) = 0
\end{align*}
The WZW model is far more general than the problem of non-Abelian bosonization! Instead of $U(N)\sim SU(N)\text{ ``spin''}\times U(1)\text{ ``charge''}$. we can take any lie group  $G$.
\chapter{$\theta$ Terms}
\section{Example: Particle on a Ring}
Consider a particle on a ring, parametrised by coordinate $\phi$ with a flux $\Phi$ through the ring. The Hamiltonian is given by
\[
	H = \frac{1}{2}(-i\partial_\phi - A)^2
\] 
where $A = \frac{\Phi}{\Phi_0}$ and $\Phi_0=2\pi$. The wavefunction is naturally periodic $\psi(2\pi)=\psi(0 )$. The solutions are 
\[
	\psi_n = \frac{1}{\sqrt{2\pi} }e^{in\phi},\qquad \epsilon_n = \frac{1}{2}\left( n - \frac{\Phi}{\Phi_0} \right)^2 
.\] 
We reformulate the problem as a path integral. First, we express the Hamiltonian as a function of $p$ and $x$:
\[
H=\frac{1}{2}(p_\varphi-A)^2
.\] 
The Hamiltonian equations of motion are
\begin{align*}
	\dv{\varphi}{t} &= \pdv{H}{p_\varphi}=p_\varphi - A \\
	\dv{p_\varphi}{t} &= -\pdv{H}{\varphi} = 0
\end{align*}
The Lagrangian is
\[
	L = p_\varphi \dot{\phi} - H = (\dot{\varphi} + A)\dot{\varphi} -\frac{1}{2}\dot{\varphi}^2 = \frac{1}{2}\dot{\varphi}^2 + A\dot{\varphi}
.\] 
In euclidean space, this is
\[
	L_E = -L(t\to -i\tau, \partial_t \to i\partial_\tau) =\frac{1}{2}(\partial_\tau\varphi)^2 -iA\partial_\tau\varphi
.\] 
The equation of motion is
\[
	\partial_\tau \fdv{L_E}{(\partial_\tau\varphi)} - \fdv{L_E}{\varphi} = 0
\] 
which is
\[
\partial_\tau^2 \varphi = 0
.\] 
We generate the path integral using the standard formula
\[
	Z = \int_{\varphi(\beta)-\varphi(0)\in 2\pi \Z} \mathcal{D}\varphi e^{-S_E},\qquad S_E = \int\dd{\tau} L_E
.\] 
This gives a family of classical solutions
\[
\varphi_W(\tau) = 2\pi W \frac{\tau}{\beta}
.\]
We note that the solutions are mappings
\[
\varphi:S^1\to S^1,\qquad \tau\mapsto \varphi(\tau)
.\] 
The former must be periodic because of the periodic boundary conditions in the partition function, while $\varphi(\tau)$ lives on a circle. There is an integer winding number $W$ associated with each path that cannot be changed by a continuous deformation. We can perform the path integral separately over each of these sectors, and there will be a classical solution in each of them:
\[
	Z = \sum_W \int_{\varphi(\beta) - \varphi(0) = 2\pi W} \mathcal{D}\varphi e^{-\dd{\tau}L_E(\varphi, \partial_\tau \varphi)}
.\] 
The integral over the second term of the euclidean space action can be performed directly
\[
	\int_{\varphi(\beta) - \varphi(0)=2\pi W}(-iA\partial_\tau\varphi) = -iA \varphi(\tau)|_0^\beta = -2\pi i A W
\] 
leading to the euclidean space action
\[
	Z = \sum_W e^{2\pi i A W}\int_{\varphi(\beta) - \varphi(0)=2\pi W} \mathcal{D}\varphi e^{-\int \dd{\tau}(\partial_\tau\varphi)^2}
.\] 
The first term is an example of a topological term: It depends on the boundary conditions of the path, in the same sense that a gauge field can only be measured when one moves along a complete path. The kinetic information remains in the quadratic term.

Some notes
\begin{enumerate}
	\item $S_\text{top}=-2\pi i W A$ is the simplest example of a topological term and belongs to the class of $\theta$ terms.
	\item $S_\text{top}$ is sensitive only to the topological sector, which is a global property, and does not depend on the local properties of the path. Thus, it cannot affect the equation of motion.
	\item $S_\text{top}$ is invariant to changes of the metric of the base manifold (e.g. scaling $\tau \to \alpha_\tau$)
	\item $e^{-S_\text{top}}$ (or $e^{i S_\text{top}^M}$ in Minkowski space) is always a pure phase. 
\end{enumerate}
Note that $e^{-S_\text{top}}$ is invariant under $A\to A+\Z$.
\section{Homotopy}
Fields are mappings on a manifold $\phi: M \to T$. $M$ is known as the base manifold, and $T$ is known as the target space. Usually, the base manifold is $\R^d$. For low energy physics, it is sensible to consider field configurations which approach a constant value at the boundary of $M$. Thus, we can consider $M\cong S^d$. 

We say that two field configurations $\phi_1,~\phi_2$ are topologically equivalent if they can be deformed continuously into each other, meaning there exists a homotopy
\[
	\hat{\phi}: S^d \times [0,1]\to T,~(x,\lambda) \to \hat{\phi}(x,\lambda)
\]
suuch that $\phi$ is continuous as a function from $S^d\times [0,1]$ in the usual product topology. Clearly, this is an equivalence relation; we call this the homotopy class. In general, we have $\pi_n(S^d)=\{0\} $ if $n<d$ and $\pi_n(S^n)=\Z$.

Some examples:
\begin{enumerate}
	\item $\pi_n(S^d)=\{0\} $ if $n<d$, $\pi_n(S^n) = \Z$.
	\item $\pi_1(T^d) = \pi_1(S^1\times S^1 \dots \times S^1) = \Z \times \Z \dots \times Z$
\end{enumerate}
\subsection{$\theta$-terms in general}
Recall that each $\phi: M\to T$ belongs to a certain homotopy class $[\phi]$. Thus, we can organise $Z$.
\[
	Z = \sum_{w\in G} \int \Dd{\phi_W} e^{-S[\phi]}
.\] 
where $W$ labels the homotopy class, $G$ is the homotopy group and we integrate over all $\phi$s in the homotopy class. This leads to an action
\[
	S[\phi] = S_0[\phi]+\underbrace{S_\text{top}[\phi]}_{=F(W)}
\]
and
\[
	Z = \sum_{w\in G}e^{-F(w)} \int \Dd{\phi_W} e^{-S_0[\phi]}
.\] 
Now, we can ask how $F(W)$ depends on $W$. In general, we will see that this is linear - $F(w_1+w_2) = F(w_1)+F(w_2)$.
\end{document}
