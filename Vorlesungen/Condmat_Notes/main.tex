\documentclass[twoside,symmetric, openany, 12pt]{./tuftebook}
%\documentclass[12pt]{article}

\usepackage{geometry}
\geometry{
	left=10mm, % left margin
	textwidth=170mm, % main text block
	marginparsep=0mm, % gutter between main text block and margin notes
	marginparwidth=10mm % width of margin notes
}


\usepackage{amsmath, amssymb,physics,amsfonts,amsthm}
\usepackage{enumitem}
\usepackage[most]{tcolorbox}
\usepackage{cancel}
\usepackage{booktabs}
\usepackage{tikz}
\usepackage{xurl}
\usepackage{hyperref}
\usepackage{enumitem}
\usepackage[normalem]{ulem}
\usepackage{transparent}
\usepackage{float}
\usepackage{multirow}
\usepackage{subcaption}
\usepackage{mathtools}
\usepackage{thmtools}
\usepackage{thm-restate}
\usepackage[framemethod=TikZ]{mdframed}
\mdfsetup{skipabove=1em,skipbelow=0em, innertopmargin=12pt, innerbottommargin=8pt}
\declaretheoremstyle[headfont=\bfseries\sffamily, bodyfont=\normalfont, mdframed={  } ]{thmbox}
\declaretheoremstyle[headfont=\bfseries\sffamily, bodyfont=\normalfont]{thmnobox}
%put nobreak in mdframed
\declaretheorem[numberwithin=chapter,style=thmnobox, name=Theorem]{Theorem}
\declaretheorem[sibling=Theorem, style=thmnobox, name=Definition]{Definition}
\declaretheorem[sibling=Theorem, style=thmnobox, name=Corollary]{Corollary}
\declaretheorem[sibling=Theorem, style=thmnobox, name=Postulate]{Postulate}
\declaretheorem[sibling=Theorem, style=thmnobox, name=Example]{Example}
\declaretheorem[sibling=Theorem, style=thmnobox, name=Proposition]{Proposition}
\declaretheorem[sibling=Theorem, style=thmnobox, name=Lemma]{Lemma}
\usepackage{subcaption}
\captionsetup[subfigure]{% changed <<<<<<<<<<<<<<<<
	singlelinecheck = false,
	justification=raggedright, 
	margin = {-3ex, 0mm}, % make margin font size dependent
}
%\newtheorem{Theorem}{Theorem}
%\numberwithin{Theorem}{chapter}
%\newtheorem{Proposition}{Proposition}
%\newtheorem{Lemma}[Theorem]{Lemma}
%\newtheorem{Corollary}[Theorem]{Corollary}
%\newtheorem{Example}[Theorem]{Example}
%\theoremstyle{definition}
\theoremstyle{definition}
\newtheorem{Remark}[Theorem]{Remark}
\theoremstyle{definition}
\newtheorem{Problem}{Problem}
\theoremstyle{definition}
\newenvironment{parts}{\begin{enumerate}[label=(\alph*)]}{\end{enumerate}}
%tikz	
\tcbset{breakable=true,toprule at break = 0mm,bottomrule at break = 0mm}
\usetikzlibrary{patterns}
\usetikzlibrary{matrix}
\usepackage{pgfplots}
\pgfplotsset{compat=1.18}
% definitions of number sets
\newcommand{\N}{\mathbb{N}}
\newcommand{\R}{\mathbb{R}}
\newcommand{\Z}{\mathbb{Z}}
\newcommand{\Q}{\mathbb{Q}}
\newcommand{\C}{\mathbb{C}}
\allowdisplaybreaks
%polar set
\newcommand{\polar}{\textrm{\tiny \fontencoding{U}\fontfamily{ding}\selectfont\symbol{'136}}}

\tcbset{colback=white}
\begin{document}
	\newgeometry{margin=1in}
	\begin{titlepage}
		{\begingroup% AW, Design of Books
			%\FSfont{5pl} % FontSite URW Palladio (Palatino)
			%\drop = 0.14\textheight
			\centering
			%\vspace*{\drop}
			{\Large Notes in}\\[\baselineskip]
			{\Huge\bfseries Field Theory}\\[\baselineskip]
			{\LARGE By}\\[\baselineskip]
			{\LARGE Jun Wei Tan}\par
			\vfill
			{Julius-Maximilians-Universit\"{a}t W\"{u}rzburg}
			\vfill
			{\small\sffamily \href{mailto:jun-wei.tan@stud-mail.uni-wuerzburg.de}{jun-wei.tan@stud-mail.uni-wuerzburg.de}}\par
			\endgroup}
	\end{titlepage}
	\restoregeometry
	\tableofcontents

	\chapter{Introduction}
	\section{The Path Integral}

	\section{The Grassman Algebra}
	\subsection{Introduction}
	Because every operator can be written in the formalism of second quantisation as a product of creation and annihilation operators, coherent states turn these operators into scalars, which are then easier to deal with. We define a fermionic coherent state by the usual equation
	\[
		a_k \ket{\phi} = \phi_k \ket{\phi}
	.\] 
	Because annihilation operators for different $k$ anticommute rather than commute, we must have
	\[
	\phi_i \phi_j = -\phi_j\phi_i
	.\] 
	Thus, the $\phi_i$s cannot be part of a field, because they must anticommute rather than commute! We define the Grassman algebra to be generated by $n$ generators $\xi_i$, with the basis coming from all products $\xi_i\xi_j$ etc. We will assume that there is an even number of generators, and to each generator $\xi_i$ we assign an inversion $(\xi_i)^* = \xi_j$ such that the inversion satisfies $(\xi^*)^* = \xi$ and $(\xi_i\xi_j)^* = \xi_j^*\xi_i^*$. 

	Because of the anticommutativity, we have $\xi^2=-\xi^2=0$ for all Grassman numbers $\xi$. Explicitly, we can construct the Grassman algebra as the exterior algebra on some differential forms. Thus, all analytic functions can be expressed in terms of their Taylor series
	\[
	f(\xi) = f_0+ f_1\xi
	.\] 
	All operators are then bilinear:
	\[
	A(\xi^*, \xi) = a_0 + a_1\xi + a_2 \xi^* + a_{12} \xi^* \xi
	.\] 
	We define the derivatives to be equal to the integral
	\[
		\pdv{\xi}f(\xi) = f_1 = \int \dd{\xi} f(\xi)
	.\] 
	Notably, we work in spirit analogously to the Wirtinger derivatives, and let $\xi$ and $\xi^*$ be independent. For reasons of anticommutativity, we require that the derivative be next to $\xi$ in order to act on it, for example
	\[
		\pdv{\xi}(\xi^*\xi) = \pdv{\xi}(-\xi\xi^*) = -\xi^*
	.\] 
	Next, we seek to deal with Gaussian integrals. We will see how they pop up later; for now, it suffices to say that the partition function is the integral of an exponential. After substituting in the fermionic coherent states, we will get something that looks like a Gaussian integral. The desired result is
	\begin{tcolorbox}[title=Gaussian Integrals]
		\[
			\int \pi_\alpha \dd{\xi_\alpha^*}\dd{\xi_\alpha} \exp \left[ -\sum_{\alpha,\beta} \xi_\alpha^* M_{\alpha,\beta} \xi_\alpha + \sum_\alpha \left( J_\alpha^* \xi_\alpha + \xi_\alpha^* J_\alpha \right)  \right] = \det(M) \exp\left( \sum_{\alpha,\beta}J_\alpha^* (M^{-1})_{\alpha, \beta} J_\beta \right) 
		\]
		where the $J$s are Grassman variables and $M$ is Hermitian.
	\end{tcolorbox}
	We show this by diagonalising $\lambda=(\lambda_i)_{ii} = UMU^\dagger$. Then, 
	\begin{align*}
		-\xi^\dagger M \xi + J^\dagger\xi +\xi^\dagger J &= -\xi^\dagger U^\dagger \lambda U \xi + J^\dagger U^\dagger U \xi + \xi^\dagger U^\dagger U J \\
								 &= \sum_\alpha (-\lambda_\alpha \eta_\alpha^* \eta_\alpha + \tilde{J}_\alpha^\dagger + \eta_\alpha + \eta_\alpha^\dagger \tilde{J}_\alpha
	\end{align*}
	and hence the integral simplifies to
	\begin{align*}
	& \int \prod_\alpha \dd{\eta_\alpha^\dagger}\dd{\eta_\alpha} \exp\left[ \sum_\alpha -\lambda_\alpha \eta_\alpha^\dagger \eta_\alpha + \tilde{J}_\alpha^* \eta_\alpha + \eta_\alpha^* \tilde{J}_\alpha \right] \\
		=& \prod_\alpha \int \dd{\eta_\alpha^\dagger}\dd{\eta_\alpha}\exp\left[ -\lambda \eta_\alpha^\dagger \eta_\alpha \right] \exp\left[ J_\alpha^* \eta_\alpha + \eta_\alpha^* J_\alpha \right] \\
		=& \det(M) \exp(J^\dagger M^{-1}J)
	\end{align*}
	\subsection{Wick's Theorem}
	Now, we are in a good position to prove Wick's theorem, the statement of which is
\begin{tcolorbox}[title=Wick's Theorem]
\[
	\frac{\int \prod_\alpha \dd{\psi_\alpha^*}\dd{\psi_\alpha} e^{-\sum_{i,j} \psi_i^* M_{ij} \psi_j + \sum_i (J_i^* \psi_i + \psi_i^* J_i)}}{\int \prod_\alpha \dd{\psi_\alpha^*}\dd{\psi_\alpha} e^{-\sum_{ij} \psi_i^* M_{ij} \psi_j}}=\sum_P \zeta^P M^{-1}_{i_{P_n}, j_n} \dots M^{-1}_{i_{P_1}, j_1}
.\] 	
\end{tcolorbox}
	To do so, we consider the generating function
	\[
		G(J^*, J) = \frac{\int \prod_\alpha \dd{\psi_\alpha^*}\dd{\psi_\alpha} e^{-\sum_{i,j} \psi_i^* M_{ij} \psi_j + \sum_i (J_i^* \psi_i + \psi_i^* J_i)}}{\int \prod_\alpha \dd{\psi_\alpha^*}\dd{\psi_\alpha} e^{-\sum_{ij} \psi_i^* M_{ij} \psi_j}}=e^{\sum_{ij} J_i^* (M^{-1})_{ij}J_j}
	\]
	(note that the action of dividing is to take away the $\det M$). We differentiate the first line \comment{TODO}
	\subsection{Fermionic Coherent States}
	Now, we move on to construct fermionic coherent states. We define that the Grassman numbers anticommute with the annihilation operators, and define
	\[
		\ket{\xi} = e^{-\sum_\alpha \xi_\alpha a_\alpha^\dagger} \ket{0} = \prod_\alpha (1-\xi_\alpha a_\alpha^\dagger)\ket{0}
	.\]
	To show that this is a coherent state, we simply act on this with $a_\beta$:
	\begin{align*}
		a_\beta\ket{\xi} &= a_\beta \prod_\alpha (1-\xi_\alpha a_\alpha^\dagger)\ket{0}\\
				 &= \prod_{\alpha \neq \beta} (1-\xi_\alpha a_\alpha^\dagger)a_\beta (1-\xi_\beta a_\beta^\dagger) \ket{0} \\
				 &= \prod_{\alpha \neq \beta} (1-\xi_\alpha a_\alpha^\dagger)\xi_\beta (1-\xi_\beta a_\beta^\dagger) \ket{0} \\
				 &=\xi_\beta \prod_\alpha (1-\xi_\alpha a_\alpha^\dagger)\ket{0}  \\
	\end{align*}
	Note that here we used the commutativity where $\alpha\neq \beta$ as well as the relation
	\begin{align*}
		\alpha_\beta (1-\xi_\beta a_\beta^\dagger)\ket{0} &= -\alpha_\beta \xi_\beta a_\beta^\dagger\ket{0} \\
								  &= \xi_\beta a_\beta a_\beta^\dagger\ket{0} \\
								  &= \xi_\beta\ket{0} \\
								  &= \xi_\beta(1 - \xi_\beta a_\beta^\dagger)\ket{0}
	\end{align*}
	Then, we can easily show the expressions for the scalar product
	\begin{align*}
		\braket{\xi}{\xi'} &= \mel{0}{\prod_{\alpha,\beta} (1+\xi_\alpha a_\alpha)(1-\xi'_\beta a^\dagger_\beta)}{0}\\
			&= \prod_\alpha ( 1+\xi_\alpha^* \xi'_\alpha ) \\
			&= e^{\sum_\alpha \xi^*_\alpha \xi_\alpha'}
	\end{align*}
	Similarly, we can show that we can produce a partition of unity using
	\[
		1 = \int \prod_\alpha \dd{\xi_\alpha^*}\dd{\xi_\alpha} e^{-\sum_\alpha \xi_\alpha^* \xi_\alpha} \ket{\xi}\bra{\xi}
	.\] 
	\chapter{The Functional Renormalisation Group}
\end{document}
