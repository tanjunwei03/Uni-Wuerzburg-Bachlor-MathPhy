\documentclass[prb,12pt]{revtex4-2}

\usepackage{amsmath, amssymb,physics,amsfonts,amsthm}
\usepackage{enumitem}
\usepackage[most]{tcolorbox}
\usepackage{cancel}
\usepackage{booktabs}
\usepackage{tikz}
\usepackage{xurl}
\usepackage{hyperref}
\usepackage{enumitem}
\usepackage[normalem]{ulem}
\usepackage{transparent}
\usepackage{float}
\usepackage{multirow}
\usepackage{subcaption}
\usepackage{mathtools}
\usepackage{thmtools}
\usepackage{thm-restate}
\usepackage[framemethod=TikZ]{mdframed}
\mdfsetup{skipabove=1em,skipbelow=0em, innertopmargin=12pt, innerbottommargin=8pt}
\declaretheoremstyle[headfont=\bfseries\sffamily, bodyfont=\normalfont, mdframed={  } ]{thmbox}
%put nobreak in mdframed
\declaretheorem[style=thmbox, name=Theorem]{Theorem}
\declaretheorem[sibling=Theorem, style=thmbox, name=Definition]{Definition}
\declaretheorem[sibling=Theorem, style=thmbox, name=Corollary]{Corollary}
\declaretheorem[sibling=Theorem, style=thmbox, name=Example]{Example}
\declaretheorem[sibling=Theorem, style=thmbox, name=Proposition]{Proposition}
\declaretheorem[sibling=Theorem, style=thmbox, name=Lemma]{Lemma}
%\newtheorem{Theorem}{Theorem}
%\numberwithin{Theorem}{chapter}
%\newtheorem{Proposition}{Proposition}
%\newtheorem{Lemma}[Theorem]{Lemma}
%\newtheorem{Corollary}[Theorem]{Corollary}
%\newtheorem{Example}[Theorem]{Example}
%\theoremstyle{definition}
\theoremstyle{definition}
\newtheorem{Remark}[Theorem]{Remark}
\theoremstyle{definition}
\newtheorem{Problem}{Problem}
\theoremstyle{definition}
\newenvironment{parts}{\begin{enumerate}[label=(\alph*)]}{\end{enumerate}}
%tikz	
\tcbset{breakable=true,toprule at break = 0mm,bottomrule at break = 0mm}
\usetikzlibrary{patterns}
\usetikzlibrary{matrix}
\usepackage{pgfplots}
\pgfplotsset{compat=1.18}
% definitions of number sets
\newcommand{\N}{\mathbb{N}}
\newcommand{\R}{\mathbb{R}}
\newcommand{\Z}{\mathbb{Z}}
\newcommand{\Q}{\mathbb{Q}}
\newcommand{\C}{\mathbb{C}}
\allowdisplaybreaks
\begin{document}
	\title{Funktionalanalysis Notizen}
	\author{Jun Wei Tan}
	\email{jun-wei.tan@stud-mail.uni-wuerzburg.de}
	\affiliation{Julius-Maximilians-Universit\"{a}t W\"{u}rzburg}
	\date{\today}
	\maketitle
	\section{Mathematics}
	Gamma Function
	\begin{align*}
		\Gamma(z)&=\int_0^\infty t^{z-1}e^{-t}\dd{t}\\
		\Gamma(z+1)&=z\Gamma(z)\\
		\Gamma\left(\frac 12\right)&=\sqrt{\pi}\\
		\Gamma(n)&=(n-1)!,~n\in \N
	\end{align*}
	Gaussian Integral
	\[\int_0^\infty x^n e^{-k x^2}\dd{x} = \frac 12 k^{-(n+1)/2}\Gamma\left(\frac{1+n}{2}\right)\]
	Fermi-Dirac Integral
	\[f_\alpha(z) =\frac 1{\Gamma(\alpha)}\int_0^\infty \frac{\xi^{\alpha-1}}{e^{\xi}/z + 1}\dd{\xi}\]
	General Series Expansion
	\[f_\alpha(z) \approx f_\alpha(z_0) + \frac{f_{\alpha - 1}(z_0)}{z_0}(z-z_0)+\frac{f_{\alpha -2}(z_0) - f_{\alpha - 1}(z_0)}{2z_0^2}(z-z_0)^2\]
	Expansion around 0
	\[f_\alpha(z) = \sum_{n=1}^\infty \frac{(-1)^{n-1}z^n}{n^\alpha}\]
	Derivative
	\[f_{n-1}(z) =z\pdv{z}f_n(z)\]
	Bose-Einstein Integral
	\[g_\alpha(z) = \frac{1}{\Gamma(\alpha)}\int_0^\infty \frac{\xi^{n-1}}{e^\xi/z-1}\dd{\xi}\]
	General Series Expansion
	\[g_\alpha(z) \approx g_\alpha(z_0) + \frac{g_{\alpha - 1}(z_0)}{z_0}(z-z_0)+\frac{g_{\alpha -2}(z_0) - g_{\alpha - 1}(z_0)}{2z_0^2}(z-z_0)^2\]
	Around 0
	\[g_\alpha(z)=\sum_{n=1}^\infty\frac{z^n}{n^\alpha}\]
	Derivatives
	\[g_{\alpha - 1}(z) =z\pdv{z}g_\alpha(z)\]
\section{Thermodynamic Potentials}
\textbf{Energy}
\begin{align*}
	U&= TS - PV + \mu N\\
	\dd{U} &= T\dd{S}-P\dd{V}+\mu\dd{N}
\end{align*}
\textbf{Helmholtz Free Energy}
Helmholtz free energy is the maximal work in an isothermal process.
\begin{align*}
	F&= U - TS\\
	F &= -PV + \mu N\\
	\dd{F}&= -S\dd{T} -P\dd{V} + \mu \dd{N}
\end{align*}
\textbf{Enthalpy}
\begin{align*}
	H &= U + PV\\
	H &= TS + \mu N\\
	\dd{H} = T\dd{S} + V\dd{P} + \mu \dd{N}
\end{align*}
\textbf{Gibbs Free Energy}
\begin{align*}
	G&= U + PV - TS\\
	G&=\mu N\\
	\dd{G}&= - T\dd{S} + V\dd{P} + \mu \dd{N}
\end{align*}
\textbf{Grand Canonical Potential}
\begin{align*}
	J &=F - \mu N\\
	J &= - PV\\
	\dd{J} &= -S\dd{T} - P \dd{V} - N \dd{\mu}
\end{align*}
\section{Ensembles}
\textbf{General Definitions}
Gibbs Entropy
\[S = -k_B \sum p_n \ln p_n=-k_B \tr(\hat{\rho}\ln \hat{\rho})\]
\textbf{Microcanonical Ensemble}
Energy fixed, 
\[\mathbb{P} \propto \Omega\]
Partition function
\[\Omega = e^{S(E)}\]
\textbf{Canonical Ensemble}

Partition function
\[Z = \tr e^{-\beta \hat{H}}\]
Thermodynamic Potential
\[F = -k_B T \ln Z\]

\textbf{Grand Canonical Ensemble}

Partition Function
\begin{align*}
Z_G
\end{align*}
\end{document}

