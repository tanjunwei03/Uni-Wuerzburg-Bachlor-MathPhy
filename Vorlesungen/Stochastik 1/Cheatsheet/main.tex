\documentclass[prb,12pt, twocolumn]{revtex4-2}

\usepackage{amsmath, amssymb,physics,amsfonts,amsthm}
\usepackage{enumitem}
\usepackage[most]{tcolorbox}
\usepackage{cancel}
\usepackage{booktabs}
\usepackage{tikz}
\usepackage{pgf}
\usepackage{xurl}
\usepackage{hyperref}
\usepackage{enumitem}
\usepackage[normalem]{ulem}
\usepackage{transparent}
\usepackage{float}
\usepackage{multirow}
\usepackage{subcaption}
\usepackage{mathtools}
\usepackage{thmtools}
\usepackage{thm-restate}
\usepackage[framemethod=TikZ]{mdframed}


\renewcommand{\familydefault}{\sfdefault}
\renewcommand{\thesection}{\arabic{section}}
\renewcommand{\thesubsection}{\thesection.\arabic{subsection}}
\renewcommand{\thesubsubsection}{\thesubsection.\arabic{subsubsection}}


\mdfsetup{skipabove=1em,skipbelow=0em, innertopmargin=12pt, innerbottommargin=8pt}
\declaretheoremstyle[headfont=\bfseries\sffamily, bodyfont=\normalfont]{thmbox}
%put nobreak in mdframed
\declaretheorem[style=thmbox, name=Theorem, numberwithin=section]{Theorem}
\declaretheorem[sibling=Theorem, style=thmbox, name=Definition]{Definition}
\declaretheorem[sibling=Theorem, style=thmbox, name=Corollary]{Corollary}
\declaretheorem[sibling=Theorem, style=thmbox, name=Example]{Example}
\declaretheorem[sibling=Theorem, style=thmbox, name=Proposition]{Proposition}
\declaretheorem[sibling=Theorem, style=thmbox, name=Lemma]{Lemma}
\declaretheorem[sibling=Theorem, style=thmbox, name= ]{Note}
%\newtheorem{Theorem}{Theorem}
%\numberwithin{Theorem}{chapter}
%\newtheorem{Proposition}{Proposition}
%\newtheorem{Lemma}[Theorem]{Lemma}
%\newtheorem{Corollary}[Theorem]{Corollary}
%\newtheorem{Example}[Theorem]{Example}
%\theoremstyle{definition}
\theoremstyle{definition}
\newtheorem{Remark}[Theorem]{Remark}
\theoremstyle{definition}
\newtheorem{Problem}{Problem}
\theoremstyle{definition}
\newenvironment{parts}{\begin{enumerate}[label=(\alph*)]}{\end{enumerate}}
%tikz	
\tcbset{breakable=true,toprule at break = 0mm,bottomrule at break = 0mm}
\usetikzlibrary{patterns}
\usetikzlibrary{matrix}
\usepackage{pgfplots}
\pgfplotsset{compat=1.18}
% definitions of number sets
\newcommand{\N}{\mathbb{N}}
\newcommand{\R}{\mathbb{R}}
\newcommand{\Z}{\mathbb{Z}}
\newcommand{\Q}{\mathbb{Q}}
\newcommand{\C}{\mathbb{C}}
\allowdisplaybreaks
\begin{document}
	\title{Stochastik 1}
	\author{Jun Wei Tan}
	\email{jun-wei.tan@stud-mail.uni-wuerzburg.de}
	\affiliation{Julius-Maximilians-Universit\"{a}t W\"{u}rzburg}
	\date{\today}
	\maketitle
	\section{Laplace-Räume}
	\begin{Note}
		mit Zurücklegen mit Beachtung der Reihenfolge
		\[\Omega_I^{k,n}=\{1,\dots, n\}^k,~\text{card }\Omega_I^{k,n} = n^k\]
	\end{Note}
	\begin{Note}
		ohne Zurücklegen mit Beachtung der Reihenfolge
		\begin{align*}
		\Omega_{II}^{k,n}=\{\omega =& (\omega_1, \dots, \omega_k)\in \{1,\dots, n\}^k\\
		&|\omega_i \neq \omega_j, i\neq j\}\\
		\text{card }\Omega_{II}^{k,n} =&\frac{n!}{(n-k)!}=(n)_k
		\end{align*}
	\end{Note}
	\begin{Note}
		ohne Zurücklegen ohne Beachtung der Reihenfolge
		\begin{align*}
		\Omega_{III}^{k,n}&=\{A\subseteq \{1,\dots, n\}|\text{card }A=k\}\\
		\text{card }\Omega_{III}^{k,n}&=\frac{n!}{(n-k)!k!}=\binom{n}{k}
	\end{align*}
	\end{Note}
	\begin{Note}
		mit Zurücklegen ohne Beachtung der Reihenfolge
		\begin{align*}
			\Omega_{IV}^{k, n}=&\{\omega=(\omega_1, \dots, \omega_k)\in \{1, \dots, n\}^k\\
			&|\omega_1\le \dots \le \omega_k\}\\
			\text{card }\Omega_{IV}^{k,n}=&\binom{k+n-1}{k}=\binom{k+n-1}{n-1}
		\end{align*}
	\end{Note}
\section{Wahrscheinlichkeitsräume}
\begin{Note}[Einschluss-Ausschluss]
\begin{widetext}
\[\mathbb{P}(A_1\cup \dots \cup A_n)=\sum_{k=1}^n (-1)^{k-1}\sum_{1\le i_1< i_2<\dots < i_k\le n}\mathbb{P}(A_{i_1}\cap \dots \cap A_{i_k})\]
\end{widetext}
\end{Note}
\begin{Note}[Bedingte Wahrscheinlichkeit]
\[\mathbb{P}(A|B)=\frac{\mathbb{P}(A\cap B)}{\mathbb{P}(B)}\]
\end{Note}
\begin{Note}[Bayes-Formel]
\[\mathbb{P}(B|A)=\frac{\mathbb{P}(A|B)\mathbb{P}(B)}{\mathbb{P}(A)}\]
\end{Note}
\begin{Note}[Unabhängige Mengensysteme] $\mathcal{A}=\{A_1, \dots, A_n\}$ unabhängig, falls
\[\mathbb{P}\left(\bigcap_{i\in I} A_i\right)=\prod_{i\in I}\mathbb{P}(A_i)\]
für alle $I\subseteq \{1,\dots, n\}$.	
\end{Note}
\begin{Note}[Diskret]
	Ein abzählbarer Maßraum mit einem Wahrscheinlichkeitsmaß definiert auf der Potenzmenge heißt diskreter Wahrscheinlichkeitsraum.
\end{Note}
\begin{Note}[Wahrscheinlichkeitsfunktion]
	Die Wahrscheinlichkeits funktion $p(x)$ ist $p(x) = \mathbb{P}(\{x\})$. 
\end{Note}
\section{Zufallsvariablen}
\begin{Note}
	\[\mathbb{P}(X = x) = \mathbb{P}(X^{-1}(x))\]
\end{Note}
\begin{Note}[Verteilung]
	Die Verteilung ist definiert durch das Maß
	\[\mathbb{P}\circ X^{-1}\]
	auf $\R$.
\end{Note}
\begin{Note}[Bernoulli-Verteilung]
	Eine Verteilung $\mathbb{P}_X$ auf $\{0, 1\}$ mit $\mathbb{P}_X (1) = p_X(1) = p \in [0, 1]$ heißt bernoulli-Verteilung Ber($p$).
\end{Note}
\end{document}

