\documentclass[prb,12pt]{revtex4-2}

\usepackage{amsmath, amssymb,physics,amsfonts,amsthm}
\usepackage{enumitem}
\usepackage{cancel}
\usepackage{booktabs}
\usepackage{tikz}
\usepackage{hyperref}
\usepackage{enumitem}
\usepackage{transparent}
\usepackage{float}
\usepackage{multirow}
\newtheorem{Theorem}{Theorem}
\newtheorem{Proposition}{Theorem}
\newtheorem{Lemma}[Theorem]{Lemma}
\newtheorem{Corollary}[Theorem]{Corollary}
\newtheorem{Example}[Theorem]{Example}
\newtheorem{Remark}[Theorem]{Remark}
\theoremstyle{definition}
\newtheorem{Problem}{Problem}
\theoremstyle{definition}
\newtheorem{Definition}[Theorem]{Definition}
\newenvironment{parts}{\begin{enumerate}[label=(\alph*)]}{\end{enumerate}}
%tikz
\usetikzlibrary{patterns}
% definitions of number sets
\newcommand{\N}{\mathbb{N}}
\newcommand{\R}{\mathbb{R}}
\newcommand{\Z}{\mathbb{Z}}
\newcommand{\Q}{\mathbb{Q}}
\newcommand{\C}{\mathbb{C}}
\begin{document}
\title{Theoretische Mechanik - Vorlesungen}
\author{Jun Wei Tan}
\email{jun-wei.tan@stud-mail.uni-wuerzburg.de}
\affiliation{Julius-Maximilians-Universit\"{a}t W\"{u}rzburg}
\date{\today}
\maketitle

\begin{Definition}
	Sei
	\[
	\va F : \R^3 \to \R^3
	,\] 
	und nicht
	\[
	\va F : \R^3 \times \R^3 \times \R \to \R^3
	.\]
	d.h die Kraft ist nur vom Ort abhängig.

	Dann definieren wir
	\[
	W: \Gamma \to \R,\]
	\[
		\gamma \to W(\gamma)=-\int_\gamma \va F\cdot \dd{\va x}
	.\] 
\end{Definition}
\end{document}
