\documentclass[twoside,symmetric, openany, 12pt]{./tuftebook}

\usepackage{geometry}
\geometry{
	left=10mm, % left margin
	textwidth=170mm, % main text block
	marginparsep=0mm, % gutter between main text block and margin notes
	marginparwidth=10mm % width of margin notes
}


\usepackage{amsmath, amssymb,physics,amsfonts,amsthm}
\usepackage{enumitem}
\usepackage[most]{tcolorbox}
\usepackage{cancel}
\usepackage{booktabs}
\usepackage{tikz}
\usepackage{xurl}
\usepackage{hyperref}
\usepackage{enumitem}
\usepackage[normalem]{ulem}
\usepackage{transparent}
\usepackage{float}
\usepackage{multirow}
\usepackage{subcaption}
\usepackage{mathtools}
\usepackage{thmtools}
\usepackage{thm-restate}
\usepackage[framemethod=TikZ]{mdframed}
\mdfsetup{skipabove=1em,skipbelow=0em, innertopmargin=12pt, innerbottommargin=8pt}
\declaretheoremstyle[headfont=\bfseries\sffamily, bodyfont=\normalfont, mdframed={  } ]{thmbox}
%put nobreak in mdframed
\declaretheorem[numberwithin=chapter,style=thmbox, name=Theorem]{Theorem}
\declaretheorem[sibling=Theorem, style=thmbox, name=Definition]{Definition}
\declaretheorem[sibling=Theorem, style=thmbox, name=Corollary]{Corollary}
\declaretheorem[sibling=Theorem, style=thmbox, name=Example]{Example}
\declaretheorem[sibling=Theorem, style=thmbox, name=Proposition]{Proposition}
\declaretheorem[sibling=Theorem, style=thmbox, name=Lemma]{Lemma}
\usepackage{subcaption}
\captionsetup[subfigure]{% changed <<<<<<<<<<<<<<<<
	singlelinecheck = false,
	justification=raggedright, 
	margin = {-3ex, 0mm}, % make margin font size dependent
}
%\newtheorem{Theorem}{Theorem}
%\numberwithin{Theorem}{chapter}
%\newtheorem{Proposition}{Proposition}
%\newtheorem{Lemma}[Theorem]{Lemma}
%\newtheorem{Corollary}[Theorem]{Corollary}
%\newtheorem{Example}[Theorem]{Example}
%\theoremstyle{definition}
\theoremstyle{definition}
\newtheorem{Remark}[Theorem]{Remark}
\theoremstyle{definition}
\newtheorem{Problem}{Problem}
\theoremstyle{definition}
\newenvironment{parts}{\begin{enumerate}[label=(\alph*)]}{\end{enumerate}}
%tikz	
\tcbset{breakable=true,toprule at break = 0mm,bottomrule at break = 0mm}
\usetikzlibrary{patterns}
\usetikzlibrary{matrix}
\usepackage{pgfplots}
\pgfplotsset{compat=1.18}
% definitions of number sets
\newcommand{\N}{\mathbb{N}}
\newcommand{\R}{\mathbb{R}}
\newcommand{\Z}{\mathbb{Z}}
\newcommand{\Q}{\mathbb{Q}}
\newcommand{\C}{\mathbb{C}}
\allowdisplaybreaks
%polar set
\newcommand{\polar}{\textrm{\tiny \fontencoding{U}\fontfamily{ding}\selectfont\symbol{'136}}}


\begin{document}
\newgeometry{margin=1in}
\begin{titlepage}
	{\begingroup% AW, Design of Books
		%\FSfont{5pl} % FontSite URW Palladio (Palatino)
		%\drop = 0.14\textheight
		\centering
		%\vspace*{\drop}
		{\Large Notes for the course in}\\[\baselineskip]
		{\Huge\bfseries FUNCTIONAL ANALYSIS}\\[\baselineskip]
		{\Large Held in WS24/25\\ At the JMU Würzburg\\ By Prof. Dr. Stefan Waldmann}\\[\baselineskip]
		{\LARGE By}\\[\baselineskip]
		{\LARGE Jun Wei Tan}\par
		\vfill
		{Julius-Maximilians-Universit\"{a}t W\"{u}rzburg}
		\vfill
		{\small\sffamily \href{mailto:jun-wei.tan@stud-mail.uni-wuerzburg.de}{jun-wei.tan@stud-mail.uni-wuerzburg.de}}\par
		\endgroup}
\end{titlepage}
\restoregeometry
	\tableofcontents
\chapter{$C^*$ Algebras}
\begin{Theorem}
	If we have an algebra $\mathcal{A}$, there is an algebra $\bar{\mathcal{A}}$ containing $\mathcal{A}$ as a sub algebra that has an identity. If $\mathcal{A}$ is $C^*$, then so is $\overline{\mathcal{A}}$.
\end{Theorem}
\begin{Definition}
	The set of $\lambda$ such that $\lambda 1 - A$ is invertible is called the resolvent set of $A$. Its complement in the complex plane is called the spectrum.
\end{Definition}
\begin{Definition}[Von-Neumann Series]
	We can expand the resolvent using the geometric series
	\[
	\frac{1}{\lambda - A}= \frac{1}{\lambda}\frac{1}{1 - A / \lambda} = \frac{1}{\lambda}\sum_{n=0}^{\infty} \left( \frac{A}{\lambda} \right)^n
	.\] 
\end{Definition}
\begin{Remark}
	This shows that for $\lambda>\|A\|$, the series is absolutely convergent. This implies that the spectrum is bounded.
\end{Remark}
\begin{Theorem}
	The spectral radius is given, for general $A$, by
	\[\rho(A) = \inf_n \|A^n\|^{1/n} = \lim_{n\to \infty} \|A^n\|^{1/n} \le \|A\|\]
\end{Theorem}
\begin{proof}
	The former follows from the root test. 
\end{proof}
\begin{Theorem}
	Let $\mathcal{A}$ be a unital $C^*$ algebra. 
	\begin{enumerate}
		\item If $A$ is normal then
		\[\rho(A) = \|A\|\]
		\item If $A$ is an isometry, then
		\[\rho(A) = 1\]
		\item If $A$ is unitary, then
		\[\sigma_A\subseteq S^1\]
		\item If $A$ is self adjoint, its spectrum is real
		\item For all polynomials $P$,
		\[\sigma(P(A))= P(\sigma(A))\]
	\end{enumerate}
\end{Theorem}
\begin{proof}
	\begin{enumerate}
		\item We use the $C^*$ property to move the power out of the norm:
			\begin{align*}
				\|A^{2^n}\|^2&= \|(A^* A)^{2^n}\|\\
					     &= \|(A^* A)^{2^{n - 1}}\|^2\\
					     &= \|A^* A\|^{2^{n}}\\
					     &= \|A\|^{2^{n+1}}
			\end{align*}
		\item We have
			\begin{align*}
				\|A^{2n}\|^2 &= \|(A^*)^{2n}A^{2n}\|\\
					     &= \|1\|=1
			\end{align*}
		\item We have
			\[
				\sigma(A) = \overline{\sigma(A^*)}=\overline{\sigma(A^{-1})}=\left( \overline{\sigma(A)} \right)^{-1}
			.\qedhere\] 
	\end{enumerate}
\end{proof}
\begin{Theorem}[Uniqueness]
	The norm of a $C^*$ algebra is unique.
\end{Theorem}
\begin{proof}
	The norm is given by, for normal elements,
	\[
	\|A\|=\rho(A)
	.\] 
	For not normal elements, we have
	\[
	\|A\|= \sqrt{\|A^* A\|}=\rho(A^*A)
	.\qedhere\] 
\end{proof}
\chapter{Homomorphisms}
\begin{Theorem}
	Homomorphisms preserve positivity
\end{Theorem}
\begin{Theorem}
	The kernel of a homomorphism is a 2 sided ideal.
\end{Theorem}
\begin{Definition}[Approximative Identity]
	If $\mathcal{I}$ is a right sided ideal, an approximate identity is defined such that
	\[
		\|E_\alpha I - I\|\xrightarrow{\alpha \to\infty}0
	.\] 
\end{Definition}
\begin{Theorem}
	Every right ideal possesses an approximate identity.
\end{Theorem}
\begin{proof}
	The idea is: We partially order the set of all finite families $\alpha = \{A_1, \dots, A_{|\alpha|}\} $, then we construct
	\[
	F_\alpha = \sum_{i=1}^{|\alpha|} A_\alpha A_\alpha^*
	.\] 
	Then, by adjoining a unit to $\mathcal{A}$ if necessary, we can define
	\[
	E_\alpha = |\alpha| F_\alpha \frac{1}{1+|\alpha|F_\alpha}=1 - \frac{1}{1+|\alpha|F_\alpha}
	.\] 
	The proof idea is showing that $\alpha\to\infty$, hence this looks like a 1.
\end{proof}
\begin{Theorem}
	Every closed two sided ideal is self adjoint and the factor algebra is a $C^*$ algebra.
\end{Theorem}
\chapter{Quantum Mechanics}
\begin{Definition}[Symplectic Vector Space]
	A symplectic vector space is a real vector space with a nondegenerate antisymmetric bilinear map. Nondegenerate means that if
	\[
	\theta(u,v)=0
\]
for all $u\in V$, then $v=0$.
\end{Definition}
\begin{Definition}[Weyl System]
	A Weyl system of a symplectic vector space is a map $V\to \mathcal{A}$ such that
	\begin{align*}
		W(0) &= 1 \\
		(W(\phi))^* &= W(-\phi) \\
		W(\phi)W(\psi) &= e^{-\frac{i}{2}\theta(\phi,\psi)}W(\phi+\psi) 
	\end{align*}
\end{Definition}
\begin{Definition}[Weyl Algebra]
	We define a particular vector space by letting $V=\R^2$, and defining the symplectic product as
	\[
	\theta((\xi_1,\eta_1), (\xi_2, \eta_2) )= \eta_1\xi_2-\xi_1\eta_2
	.\] 
	The algebra generated by $W(\xi, 0),~W(0,\eta)$ is called the Weyl algebra.
\end{Definition}
\begin{Theorem}
	Let $(\mathcal{A},W)$ be a Weyl system of a symplectic vector space $(V,\theta)$. Then
	\begin{enumerate}
		\item $W(\phi)$ is unitary for all $\phi\in V$ 
		\item $\|W(\phi)-W(\psi)\|=2$ for all $\phi\neq\psi \in V$.
		\item $\mathcal{A}$ is not separable, unless $V=\{0\} $.
		\item The family $\{W(\phi)\}_{\phi\in V}$ is linearly independent.
	\end{enumerate}
\end{Theorem}
\begin{proof}
	\begin{enumerate}
		\item We have
			\begin{align*}
				[W(\phi)]^*W(\phi) &= W(-\phi)W(\phi) \\
						   &= e^{-\frac{i}{2}\theta(-\phi,\phi)}W(0) 
			\end{align*}
			Because $\theta$ is antisymmetric, it follows that $\theta(-\phi,\phi) =-\theta(\phi,\phi) = 0$.
		\item 
	\end{enumerate}
\end{proof}
\renewcommand{\listtheoremname}{List of Definitions}
\listoftheorems[ignoreall, onlynamed={Definition}]
\renewcommand{\listtheoremname}{List of Theorems}
\listoftheorems[ignoreall, onlynamed={Theorem}]
\renewcommand{\listtheoremname}{List of Examples}
\listoftheorems[ignoreall, onlynamed={Example}]
\end{document}
