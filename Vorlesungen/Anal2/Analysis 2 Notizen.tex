\documentclass[prb,12pt]{revtex4-2}

\usepackage{amsmath, amssymb,physics,amsfonts,amsthm}
\usepackage{enumitem}
\usepackage{cancel}
\usepackage{booktabs}
\usepackage{tikz}
\usepackage{hyperref}
\usepackage{enumitem}
\usepackage{transparent}
\usepackage{float}
\usepackage{multirow}
\newtheorem{Theorem}{Theorem}
\newtheorem{Proposition}{Theorem}
\newtheorem{Lemma}[Theorem]{Lemma}
\newtheorem{Corollary}[Theorem]{Corollary}
\newtheorem{Example}[Theorem]{Example}
\newtheorem{Remark}[Theorem]{Remark}
\theoremstyle{definition}
\newtheorem{Problem}{Problem}
\theoremstyle{definition}
\newtheorem{Definition}[Theorem]{Definition}
\newenvironment{parts}{\begin{enumerate}[label=(\alph*)]}{\end{enumerate}}
%tikz
\usetikzlibrary{patterns}
% definitions of number sets
\newcommand{\N}{\mathbb{N}}
\newcommand{\R}{\mathbb{R}}
\newcommand{\Z}{\mathbb{Z}}
\newcommand{\Q}{\mathbb{Q}}
\newcommand{\C}{\mathbb{C}}
\begin{document}
	\title{Analysis 2 (Vorlesungen)}
	\author{Jun Wei Tan}
	\email{jun-wei.tan@stud-mail.uni-wuerzburg.de}
	\affiliation{Julius-Maximilians-Universit\"{a}t W\"{u}rzburg}
	\date{\today}
	\maketitle
\begin{Definition}
	$f$ is differentiable at $x_0\in x$ if and only if
	\[
	\lim_{x \to 0} \frac{f(x)-f(x_0)}{x-x_0}=f'(x_0)
	.\] 
	(This definition means that the limit exists and is finite.) We define the limit as the derivative.
\end{Definition}

\begin{Definition}
	Let $X$ be a set and $f:x\to \R$ a function. A point $x_0\in X$ is called a global maximum if and only if
	 \[
	f(x)\le f(x_0)
	\] 
	holds for all $x\in X$
\end{Definition}

\begin{Definition}
	If it is also true that $f(x)<f(x_0)$ for all $x\in X$, then we call $x_0$ a strict global maximum ``striktes globales Maximum".
\end{Definition}

\begin{Definition}
	$x\in X$ heißt lokales (striktes) Maximum, wenn es eine Umgebung $U \subseteq X$ gibt, sodass $x_0$ eine Maximum von $f|_U:U\to \R$ ist. 
\end{Definition}

\begin{Theorem}
	(Mittelwertsatz) Sei $I=[a,b]\subseteq \R$ mit $a<b$, und $f,g:[a,b]\to \R$ stetig und differenzierbar. 

	Dann gibt es $x_0\in (a,b)$ mit
	\[
		\left( f(b)-f(a) \right) g'(x_0)=\left( g(b)-g(a) \right) f'(x_0)
	.\] 
\end{Theorem}

\begin{proof}
	Sei
	\[
	\varphi(x)=\left( f(b)-f(a) \right) g(x)-\left( g(b)-g(a) \right) f(x)
	.\] 
	$\varphi(x)$ ist stetig und differenzierbar auf $[a,b]$ bzw. $(a,b)$. Wir haben
	\[
	\varphi(a)=\dots=\varphi(b)
	.\] 
	Dann k\"{o}nnen wir den Satz Rolles verwenden: $\exists x_0\in (a,b)$ mit $\varphi'(x_0)=0$, d.h.
	\[
	\varphi'(x_0)=\left( f(b)-f(a) \right) g'(x_0)-\left( g(b)-g(a) \right) f'(x_0)
	.\] 
\end{proof}

\begin{Corollary}
	Sei $f:[a,b]\to \R$ stetig und differenzierbar in $(a,b)$ mit $f'(x)=0$ f\"{u}r alle $x\in (a,b)$. Dann ist $f$ konstant.
\end{Corollary}

\begin{Corollary}
	Sei $f:[a,b]\to \R$ stetig und in $(a,b)$ differenzierbar. Dann

	\begin{enumerate}[label=(\roman*)]
		\item Gilt $f'\left(x \right) >0$ f\"{u}r alle $x\in (a,b)$, so ist $f$ strikt monoton wachsend.
		\item Gilt $f'<0$, so ist $f$ monoton fallend.
	\end{enumerate}
\end{Corollary}

\begin{Corollary}
	Sei $f:[a,b]\to \R$ stetig und auf $(a,b)$ differenzierbar mit beschr\"{a}nkter Ableitung, dann sie die Differenzquotienten auch beschr\"{a}nkt. Wenn
	\[
	m\le f'(x) \le M
	,\] 
	dann ist
	\[
	m\le \frac{f(b)-f(a)}{b-a}\le M
	.\] 
\end{Corollary}

\begin{Corollary}
	\[
	\left| \frac{f(x_2)-f(x_1)}{x_2-x_1} \right| <\|f'\|
	.\] 
	Wobei $\|f'\|=\sup_{x\in [a,b]}f'(x)$
\end{Corollary}

\begin{Theorem}
	Sei $X\subseteq \C$ offene Teilmenge und $f:X\to \C$ differenzierbar mit lokal beschr\"{a}nkter Ableitung $f':X\to \C$. Dann sei f\"{u}r alle kompakten Teilmengen $K\subseteq X$ und alle $z_1,z_0\in K$ 
	\[
	|f(z_1)-f(z_0)|<\|f'\|_K|z_1-z_0|
	.\] 
\end{Theorem}

\begin{proof}
	Wir bezeichnet
	\[
	z(t)=z_1t+z_0(1-t)
	,\] 
	und wahlen ein komplexe Zahl $c$, womit $c(z_1-x_0)=|z_1-z_0|$. Dann ist
	\[g(t)=Re\left[ cf(z(t)) \right] 
	\] 
	differenzierbar und reelle. Dann ist
	\begin{align*}
		g'(t)=&Re\left[ cf'(z(t))(z_1-z_0) \right]
	\end{align*}
	Daher gilt auch
	\begin{align*}
		|g'(t)|<& |cf'(z(t))(z_1-z_0)\\
		=& |c| |f'(z(t))| |z_1-z_0|\\
		=& |f'(z(t))| |z_1-z_0|\\
		<& \|f'\| |z_1-z_0|
	\end{align*}
\end{proof}

\begin{Theorem}
	(Zwischenwertsatz f\"{u}r Ableitung) Sei $f:[a,b]\to \R$ diffbar mit
	\[
	f'(a)\neq f'(b)
	.\] 
	Dann nimmt $f'$ jeder Wert zwischen $f'(a)$ und $f'(b)$ in $(a,b)$ an.
\end{Theorem}

\begin{proof}
	Nimm an, dass $f'(a)<f'(b)$, und sei $y_0\in (f'(a), f'(b))$. Dann behandelt
	\[
		\varphi(x)=f(x)-y_0x, x\in [a,b]
	.\] 
	$\varphi$ ist diffbar mit $\varphi'(x)=f'(x)-y_0$. Dann ist

	\begin{gather*}
		\varphi'(a)=f'(a)-y_0<0\\
		\varphi'(b)=f'(b)-y_0>0
	\end{gather*}

	Dann existiert $\epsilon_1, \epsilon_2>0$ mit
	\begin{gather*}
		\varphi(x)<\varphi(a), \qquad 
	\end{gather*}
\end{proof}
\section{17/10/23}
Wir befassen uns mit Grenze wie
\[
\lim_{x \to x_0} \frac{f(x)}{g(x)}
.\] 
Es wäre gut, wenn wir das als
\[
\frac{\lim_{x \to x_0} f(x)}{\lim_{x \to x_0} g(x)}
\] 
schreiben könnten. Das ist nur richtig, wenn $\lim_{x \to x_0} g(x)\neq 0$. Was passiert, wenn
\[
\lim_{x \to x_0} g(x)=0=\lim_{x \to x_0} f(x)?\] 
\begin{Lemma}
	Sei $g(x_0)=0$ und $g'(x_0)\neq 0$. Dann existiert eine Umgebung $U$, daf\"{u}r gilt
	\[
	g(x)\neq 0 \qquad x\in U \\ \left\{ 0 \right\} 
	.\] 
\end{Lemma}
\begin{proof}
	Angenommen, dass es falsch ist. Dann existiert in jeder offene Ball $B_{1 / n}(x_0)$ ein punkt, der wie als $x_n$ bezeichnen und daf\"{u}r gilt, dass $g(x_n)=0$. 
\end{proof}
\begin{Proposition}
	Seien $f,g: X\to \mathbb{K}$ bei $x_0\in X$ differenzierbar und
	\[
	f(x_0)=0=g(x_0) \qquad g'(x_0)\neq 0.\]
	Dann gilt
	\[
	\lim_{x \to x_0} \frac{f(x)}{g(x)}=\frac{f'(x_0)}{g'(x_0)}
	.\] 
\end{Proposition}

\begin{Theorem}
	(L'Hopital) Seien $I=(a,b)\subseteq \R$ offen und $f,g:I\to \R$ differenzierbar auf $I$ mit $g'(x)\neq 0$ f\"{u}r alle $x\in I$. Weitere gilt auch entweder
	\begin{enumerate}[label=(\roman*)]
		\item $\lim_{x \to a^{-}} f(x)=0= \lim_{x \to a^{-}} f(x)$
		\item $\lim_{x \to a^{-}} g'(x)=\infty\text{ oder }-\infty$
	\end{enumerate}
	In diesem Fall gilt
	\[
		\lim_{x \to a^{-}} \frac{f(x)}{g(x)}=\lim_{x \to a^{-}} \frac{f'(x)}{g'(x)}
	,\]
	sofern der Grenzwert der Ableitung in $\R\cup \left\{ \pm\infty \right\} $ existiert. Eine entsprechende Aussage gilt f\"{u}r $b$.
\end{Theorem}

\begin{Definition}
	Sei $X$ eine offene Teilmenge $\subseteq \R\text{ oder }\C$. Dann
	\begin{enumerate}
		\item Eine Funktion $f:X\to \mathbb{K}$ heißt k-mal stetig differenzierbar, wenn $f':X\to \mathbb{K}$ (k-1)-mal stetig differenzierbar ist.
		\item Wenn das f\"{u}r alle $k\in \mathbb{N}$ passt, heißt $f$ glatt.
		\item Die Menge alle k-mal stetig differenzierbar Funktionen heißt $\mathcal{C}^k$
		\item Wenn f\"{u}r a Funktion es f\"{u}r alle $k$ passt, kann die Funktion als glatt gennant werden.
		\item Die Menge alle glatte Funktionen heißt $\mathcal{C}^\infty$
	\end{enumerate}
\end{Definition}
\begin{proof}
	$f,g$ sind unbedingt stetig bei $x_0$, also
	\[
	\lim_{x \to x_0} f(x)=0=\lim_{x \to x_0} g(x)
	.\] 
	Da $g'(x_0)\neq 0$, gibt es eine Umgebung $U\subseteq X$ von $x_0$ mit $g(x)\neq 0$ f\"{u}r $x\in U\\ \left\{ x_0 \right\} $. Dann gilt daf\"{u}r
	\[
		\frac{f(x)}{g(x)}=\frac{f(x)-f(x_0)}{g(x)-g(x_0)}=\frac{f(x)-f(x_0)}{x-x_0}\frac{x-x_0}{g(x)-g(x_0)}
	.\] 
Weil die beiten Grenzwerte existieren und $g'(x_0)\neq 0$ gilt, folgt also
\[
\lim_{x \to x_0} \frac{f(x)}{g(x)}=\frac{f'(x_0)}{g'(x_0)}
.\] 
\end{proof}
\begin{Example}
	\begin{enumerate}
		\item Polynome sind glatt, die Ableitung eines Polynoms ist immer einen anderen Polynom.
		\item Rationale Abbildungen sind glatt, die Ableitung eine rationale Abbildung ist rational.
		\item Die Ableitung die exponentiale Abbildung ist wieder die exponentiale Abbildung.
	\end{enumerate}
\end{Example}

\begin{Definition}
	Eine Algebra $\mathcal{A}$ von Funktionen ist eine Menge, wobei f\"{u}r alle $f,g\in \mathcal{A}$ gilt.
 \[
	 af+bg\in \mathcal{A},\]
	 \[
		 fg\in \mathcal{A}
	 .\] 
\end{Definition}

\begin{Proposition}
	Sei $X$ offene Teilmenge der rellen oder komplexen Zahlen und $k\in\mathbb{N}\cup \left\{ \infty \right\} $. Dann
	\begin{enumerate}
		\item $\mathcal{C}^k$ bietet eine Unteralgebra alle Funktionen
		\item Ist $f\neq 0$ auf ganze $X$ eine $\mathcal{C}^k$ Funktion, so ist $\frac{1}{f}\in \mathcal{C}^k (X, \mathbb{K})$.
		\item Ist Y ein weitere Teilmenge und $g\in \mathcal{C}(Y< \mathbb{K})$ mit $f(X)\subseteq Y$, dann ist $g\circ f \in C^k(X, \mathbb{K})$
	\end{enumerate}
\end{Proposition}
\end{document}
