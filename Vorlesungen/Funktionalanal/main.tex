\documentclass[prb,12pt]{revtex4-2}

\usepackage{amsmath, amssymb,physics,amsfonts,amsthm}
\usepackage{enumitem}
\usepackage[most]{tcolorbox}
\usepackage{cancel}
\usepackage{booktabs}
\usepackage{tikz}
\usepackage{xurl}
\usepackage{hyperref}
\usepackage{enumitem}
\usepackage[normalem]{ulem}
\usepackage{transparent}
\usepackage{float}
\usepackage{multirow}
\usepackage{subcaption}
\usepackage{mathtools}
\usepackage{thmtools}
\usepackage{thm-restate}
\usepackage[framemethod=TikZ]{mdframed}
\mdfsetup{skipabove=1em,skipbelow=0em, innertopmargin=12pt, innerbottommargin=8pt}
\declaretheoremstyle[headfont=\bfseries\sffamily, bodyfont=\normalfont, mdframed={ nobreak } ]{thmbox}
\declaretheorem[style=thmbox, name=Theorem]{Theorem}
\declaretheorem[sibling=Theorem, style=thmbox, name=Definition]{Definition}
\declaretheorem[sibling=Theorem, style=thmbox, name=Corollary]{Corollary}
\declaretheorem[sibling=Theorem, style=thmbox, name=Example]{Example}
\declaretheorem[sibling=Theorem, style=thmbox, name=Proposition]{Proposition}
\declaretheorem[sibling=Theorem, style=thmbox, name=Lemma]{Lemma}
%\newtheorem{Theorem}{Theorem}
%\numberwithin{Theorem}{chapter}
%\newtheorem{Proposition}{Proposition}
%\newtheorem{Lemma}[Theorem]{Lemma}
%\newtheorem{Corollary}[Theorem]{Corollary}
%\newtheorem{Example}[Theorem]{Example}
%\theoremstyle{definition}
\theoremstyle{definition}
\newtheorem{Remark}[Theorem]{Remark}
\theoremstyle{definition}
\newtheorem{Problem}{Problem}
\theoremstyle{definition}
\newenvironment{parts}{\begin{enumerate}[label=(\alph*)]}{\end{enumerate}}
%tikz	
\tcbset{breakable=true,toprule at break = 0mm,bottomrule at break = 0mm}
\usetikzlibrary{patterns}
\usetikzlibrary{matrix}
\usepackage{pgfplots}
\pgfplotsset{compat=1.18}
% definitions of number sets
\newcommand{\N}{\mathbb{N}}
\newcommand{\R}{\mathbb{R}}
\newcommand{\Z}{\mathbb{Z}}
\newcommand{\Q}{\mathbb{Q}}
\newcommand{\C}{\mathbb{C}}
\allowdisplaybreaks
\begin{document}
	\title{Funktionalanalysis Notizen}
	\author{Jun Wei Tan}
	\email{jun-wei.tan@stud-mail.uni-wuerzburg.de}
	\affiliation{Julius-Maximilians-Universit\"{a}t W\"{u}rzburg}
	\date{\today}
	\maketitle
	\section{Baire Spaces}
	\begin{Definition}[Baire Spaces]
		...
	\end{Definition}
	\begin{Theorem}[Baire I]
		...
	\end{Theorem}
	\begin{Theorem}[Baire II]
		...
	\end{Theorem}
	\section{Topological Vector Spaces}
	\begin{Definition}
		A topological vector space is a vector space with a topology such that addition and scalar multiplication are continuous.
	\end{Definition}
	\begin{Theorem}
		Translation $T_v:x\mapsto x + v$ and multiplication $\lambda: x \mapsto \lambda x$ with $\lambda\neq0$ are continuous.
	\end{Theorem}
	\begin{proof}
		They are invertible with inverse $T_{-v}$ and $\frac 1\lambda$ respectively
	\end{proof}
	\begin{Definition}[Uniform Continuity]
	\end{Definition}
	\begin{Theorem}[Equivalence of Completeness Conditions]
	\end{Theorem}
	\begin{Definition}
		content...
	\end{Definition}
	\section{Banach Spaces}
\end{document}

