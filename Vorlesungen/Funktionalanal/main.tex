\documentclass[prb,12pt]{revtex4-2}

\usepackage{amsmath, amssymb,physics,amsfonts,amsthm}
\usepackage{enumitem}
\usepackage[most]{tcolorbox}
\usepackage{cancel}
\usepackage{booktabs}
\usepackage{tikz}
\usepackage{xurl}
\usepackage{hyperref}
\usepackage{enumitem}
\usepackage[normalem]{ulem}
\usepackage{transparent}
\usepackage{float}
\usepackage{multirow}
\usepackage{subcaption}
\usepackage{mathtools}
\usepackage{thmtools}
\usepackage{thm-restate}
\usepackage[framemethod=TikZ]{mdframed}
\mdfsetup{skipabove=1em,skipbelow=0em, innertopmargin=12pt, innerbottommargin=8pt}
\declaretheoremstyle[headfont=\bfseries\sffamily, bodyfont=\normalfont, mdframed={ nobreak } ]{thmbox}
\declaretheorem[style=thmbox, name=Theorem]{Theorem}
\declaretheorem[sibling=Theorem, style=thmbox, name=Definition]{Definition}
\declaretheorem[sibling=Theorem, style=thmbox, name=Corollary]{Corollary}
\declaretheorem[sibling=Theorem, style=thmbox, name=Example]{Example}
\declaretheorem[sibling=Theorem, style=thmbox, name=Proposition]{Proposition}
\declaretheorem[sibling=Theorem, style=thmbox, name=Lemma]{Lemma}
%\newtheorem{Theorem}{Theorem}
%\numberwithin{Theorem}{chapter}
%\newtheorem{Proposition}{Proposition}
%\newtheorem{Lemma}[Theorem]{Lemma}
%\newtheorem{Corollary}[Theorem]{Corollary}
%\newtheorem{Example}[Theorem]{Example}
%\theoremstyle{definition}
\theoremstyle{definition}
\newtheorem{Remark}[Theorem]{Remark}
\theoremstyle{definition}
\newtheorem{Problem}{Problem}
\theoremstyle{definition}
\newenvironment{parts}{\begin{enumerate}[label=(\alph*)]}{\end{enumerate}}
%tikz	
\tcbset{breakable=true,toprule at break = 0mm,bottomrule at break = 0mm}
\usetikzlibrary{patterns}
\usetikzlibrary{matrix}
\usepackage{pgfplots}
\pgfplotsset{compat=1.18}
% definitions of number sets
\newcommand{\N}{\mathbb{N}}
\newcommand{\R}{\mathbb{R}}
\newcommand{\Z}{\mathbb{Z}}
\newcommand{\Q}{\mathbb{Q}}
\newcommand{\C}{\mathbb{C}}
\allowdisplaybreaks
\begin{document}
	\title{Funktionalanalysis Notizen}
	\author{Jun Wei Tan}
	\email{jun-wei.tan@stud-mail.uni-wuerzburg.de}
	\affiliation{Julius-Maximilians-Universit\"{a}t W\"{u}rzburg}
	\date{\today}
	\maketitle
	\section{Baire Spaces}
	\begin{Theorem}[List of Useful Identities]
		\noindent
		\begin{parts}
			\item
			\[(A\cup B)^\text{cl}=A^\text{cl}\cup B^\text{cl}.\]
			\item 
			\[
			(A\cup B)^{\circ}\supseteq A^\circ \cup B^\circ
			.\] 
			\item \[
			(A\cap B)^\text{cl}\subseteq A^\text{cl}\cap B^\text{cl}
			.\]
			\item 	\[
			(A\cap B)^\circ = A^\circ \cap B^\circ
			.\] 
			\item \[
			(M\setminus A)^\text{cl}=M\setminus A^{\circ}
			.\]
			\item \[
			(M\setminus A)^\circ = M \setminus A^\text{cl}
			.\] 
		\end{parts}
	\end{Theorem}
\begin{proof}
	\begin{parts}
		\item If all neighbourhoods of $x$ intersects $A$, then they must certainly intersect $A\cup B$. The same thing happens if all neighbourhoods intersect $B$. Conversely, we suppose that $x$ is not in $A^\text{cl}$ or $B^\text{cl}$. Then we have an open neighbourhood not intersecting $A$, and an open neighbourhood not intersecting $B$. Considering the intersection of these two neighbourhoods shows that $x$ is not in the closure of $A\cup B$ either.
		\item $A^\circ\cup B^\circ$ is an open set contained in $A\cup B$.
		\item Suppose every neighbourhood of $x$ intersects $A\cap B$. Then every neighbourhood intersects $A$, and also intersects $B$.
		\item Clearly, $A^\circ \cap B^\circ$ is an open set contained in $A\cap B$. Conversely, it is also true that $(A\cap B)^\circ$ is an open set contained in $A$, and thus its interior, and it is also contained in $B$.
		\item Clearly, $M \setminus A^\circ$ is a closed set containing $M\setminus A$. This shows one inclusion.
		
		Now suppose $x\in M \setminus A^\circ$. Since it is not in the interior, no open neighbourhood of $x$ is completely contained in $A$; in particular, every neighbourhood must intersect $M\setminus A$. This shows the reverse inclusion.
		\item Clearly, $M\setminus A^\text{cl}$ is an open subset of $M \setminus A$.
		
		Conversely, suppose $x$ is an element of $(M \setminus A)^\circ$. Then there is an open neighbourhood of $x$ contained in $M \setminus A$, in particular not intersecting A. This shows that $x$ is in $M \setminus A^\text{cl}$.\qedhere
	\end{parts}
\end{proof}
	\begin{Definition}[Nowhere Dense]
		A subset $A$ of a topological space is called nowhere dense if the interior of its closure is open, $(A^\text{cl})^\circ=\varnothing$. 
	\end{Definition}
\begin{Theorem}
	A subset $A$ of a topological space $(M, \mathcal{M})$ is nowhere dense iff its complement contains a dense open set.
\end{Theorem}
\begin{proof}
		We perform the following computation:
		\begin{align*}
			A\text{ nowhere dense}&\iff (A^\text{cl})^\circ = \varnothing\\
			&\iff M \setminus (A^\text{cl})^\circ = M\\
			&\iff (M \setminus A^\text{cl})^\text{cl}=M\\
			&\iff ((M \setminus A)^\circ)^\text{cl}=M\\
			&\iff (M \setminus A)^\circ\text{ is dense}.\qedhere
		\end{align*}
\end{proof}
\begin{Definition}[Meager]
	A subset is called meager if it is a countable union of nowhere dense sets.
\end{Definition}
\begin{Theorem}
	Let $(M, \mathcal{M})$ be a topological space. Then the following are equivalent:
	\begin{parts}
		\item Any countable union of closed subsets of $M$ without inner points has no inner points.
		\item Any countable intersection of open dense subsets of $M$ is dense.
		\item Every non-empty open subset of $M$ is not meager
		\item The complement of every meager subset of $M$ is dense.
	\end{parts}
\end{Theorem}
\begin{proof}
	The proof follows (a) $\implies$ (b) $\implies$ (c) $\implies$ (d) $\implies$ (a).
	\begin{enumerate}
		\item Let $(U_i)_{i\in \N}$ be a collection of dense open subsets of $M$. We consider their complements, which all have empty interior. Since
		\[M \setminus \left(\bigcap_{i=1}^\infty U_i\right)=\bigcup_{i=1}^\infty \left(M \setminus U_i\right)\]
		and the sets in the union on the right are all closed subsets without interior points, their union has no interior points, hence the countable intersection is dense.
		\item Suppose we had a meager nonempty open subset $U$ of $M$, that is, we have $U=\bigcup_{i=1}^\infty A_i$ with $A_i$ nowhere dense sets. Then $M \setminus A_i$ is a dense subset for all $i$, by (b), their intersection is still dense. Then
		\begin{align*}
			\varnothing &= M \setminus \left(\bigcap_{i=1}^\infty (M \setminus A_i^\text{cl})\right)^\text{cl}\\
			&= M \setminus \left(M \setminus \bigcup_{i=1}^\infty A_i^\text{cl}\right)^\text{cl}\\
			&= M \setminus \left(M \setminus\left(\bigcup_{i=1}^\infty A_i^\text{cl}\right)^\circ\right)\\
			&=\left(\bigcup_{i=1}^\infty A_i^\text{cl}\right)^\circ,
		\end{align*}
	which is a contradiction, since we had $U$ as a nonempty open subset of the final union. 
	\item Suppose we have a meager subset
	\end{enumerate}
\end{proof}
	\begin{Definition}[Baire Spaces]
		...
	\end{Definition}
	\begin{Theorem}[Baire I]
		...
	\end{Theorem}
	\begin{Theorem}[Baire II]
		...
	\end{Theorem}
	\section{Topological Vector Spaces}
	\begin{Definition}
		A topological vector space is a vector space with a topology such that addition and scalar multiplication are continuous.
	\end{Definition}
	\begin{Theorem}
		Translation $T_v:x\mapsto x + v$ and multiplication $\lambda: x \mapsto \lambda x$ with $\lambda\neq0$ are continuous.
	\end{Theorem}
	\begin{proof}
		They are invertible with continuous inverse $T_{-v}$ and $\frac 1\lambda$ respectively
	\end{proof}
	\begin{Definition}[Uniform Continuity]
	\end{Definition}
	\begin{Theorem}[Equivalence of Completeness Conditions]
	\end{Theorem}
	\begin{Definition}
		content...
	\end{Definition}
	\section{Banach Spaces}
\end{document}

