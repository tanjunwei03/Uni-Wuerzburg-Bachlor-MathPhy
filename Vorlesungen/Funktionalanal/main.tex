\documentclass[prb,12pt]{revtex4-2}

\usepackage{amsmath, amssymb,physics,amsfonts,amsthm}
\usepackage{enumitem}
\usepackage[most]{tcolorbox}
\usepackage{cancel}
\usepackage{booktabs}
\usepackage{tikz}
\usepackage{xurl}
\usepackage{hyperref}
\usepackage{enumitem}
\usepackage[normalem]{ulem}
\usepackage{transparent}
\usepackage{float}
\usepackage{multirow}
\usepackage{subcaption}
\usepackage{mathtools}
\usepackage{thmtools}
\usepackage{thm-restate}
\usepackage[framemethod=TikZ]{mdframed}
\mdfsetup{skipabove=1em,skipbelow=0em, innertopmargin=12pt, innerbottommargin=8pt}
\declaretheoremstyle[headfont=\bfseries\sffamily, bodyfont=\normalfont, mdframed={  } ]{thmbox}
%put nobreak in mdframed
\declaretheorem[style=thmbox, name=Theorem]{Theorem}
\declaretheorem[sibling=Theorem, style=thmbox, name=Definition]{Definition}
\declaretheorem[sibling=Theorem, style=thmbox, name=Corollary]{Corollary}
\declaretheorem[sibling=Theorem, style=thmbox, name=Example]{Example}
\declaretheorem[sibling=Theorem, style=thmbox, name=Proposition]{Proposition}
\declaretheorem[sibling=Theorem, style=thmbox, name=Lemma]{Lemma}
%\newtheorem{Theorem}{Theorem}
%\numberwithin{Theorem}{chapter}
%\newtheorem{Proposition}{Proposition}
%\newtheorem{Lemma}[Theorem]{Lemma}
%\newtheorem{Corollary}[Theorem]{Corollary}
%\newtheorem{Example}[Theorem]{Example}
%\theoremstyle{definition}
\theoremstyle{definition}
\newtheorem{Remark}[Theorem]{Remark}
\theoremstyle{definition}
\newtheorem{Problem}{Problem}
\theoremstyle{definition}
\newenvironment{parts}{\begin{enumerate}[label=(\alph*)]}{\end{enumerate}}
%tikz	
\tcbset{breakable=true,toprule at break = 0mm,bottomrule at break = 0mm}
\usetikzlibrary{patterns}
\usetikzlibrary{matrix}
\usepackage{pgfplots}
\pgfplotsset{compat=1.18}
% definitions of number sets
\newcommand{\N}{\mathbb{N}}
\newcommand{\R}{\mathbb{R}}
\newcommand{\Z}{\mathbb{Z}}
\newcommand{\Q}{\mathbb{Q}}
\newcommand{\C}{\mathbb{C}}
\allowdisplaybreaks
\begin{document}
	\title{Funktionalanalysis Notizen}
	\author{Jun Wei Tan}
	\email{jun-wei.tan@stud-mail.uni-wuerzburg.de}
	\affiliation{Julius-Maximilians-Universit\"{a}t W\"{u}rzburg}
	\date{\today}
	\maketitle
	\tableofcontents
	\section{Topology}
	\subsection{Basic Concepts}
	\begin{Theorem}[List of Useful Identities]
		\noindent
		\begin{parts}
			\item
			\[(A\cup B)^\text{cl}=A^\text{cl}\cup B^\text{cl}.\]
			\item 
			\[
			(A\cup B)^{\circ}\supseteq A^\circ \cup B^\circ
			.\] 
			\item \[
			(A\cap B)^\text{cl}\subseteq A^\text{cl}\cap B^\text{cl}
			.\]
			\item 	\[
			(A\cap B)^\circ = A^\circ \cap B^\circ
			.\] 
			\item \[
			(M\setminus A)^\text{cl}=M\setminus A^{\circ}
			.\]
			\item \[
			(M\setminus A)^\circ = M \setminus A^\text{cl}
			.\] 
		\end{parts}
	\end{Theorem}
\begin{proof}
	\begin{parts}
		\item If all neighbourhoods of $x$ intersects $A$, then they must certainly intersect $A\cup B$. The same thing happens if all neighbourhoods intersect $B$. Conversely, we suppose that $x$ is not in $A^\text{cl}$ or $B^\text{cl}$. Then we have an open neighbourhood not intersecting $A$, and an open neighbourhood not intersecting $B$. Considering the intersection of these two neighbourhoods shows that $x$ is not in the closure of $A\cup B$ either.
		\item $A^\circ\cup B^\circ$ is an open set contained in $A\cup B$.
		\item Suppose every neighbourhood of $x$ intersects $A\cap B$. Then every neighbourhood intersects $A$, and also intersects $B$.
		\item Clearly, $A^\circ \cap B^\circ$ is an open set contained in $A\cap B$. Conversely, it is also true that $(A\cap B)^\circ$ is an open set contained in $A$, and thus its interior, and it is also contained in $B$.
		\item Clearly, $M \setminus A^\circ$ is a closed set containing $M\setminus A$. This shows one inclusion.
		
		Now suppose $x\in M \setminus A^\circ$. Since it is not in the interior, no open neighbourhood of $x$ is completely contained in $A$; in particular, every neighbourhood must intersect $M\setminus A$. This shows the reverse inclusion.
		\item Clearly, $M\setminus A^\text{cl}$ is an open subset of $M \setminus A$.
		
		Conversely, suppose $x$ is an element of $(M \setminus A)^\circ$. Then there is an open neighbourhood of $x$ contained in $M \setminus A$, in particular not intersecting A. This shows that $x$ is in $M \setminus A^\text{cl}$.\qedhere
	\end{parts}
\end{proof}
\begin{Definition}[Preorder]
	A relation $\preceq$ on a set $A$ is a preorder if the following conditions hold:
	\begin{parts}
		\item $\alpha \preceq \alpha$ for all $\alpha\in A$
		\item If $\alpha\preceq \beta$ and $\beta\preceq\gamma$, then $\alpha\preceq\gamma$.
	\end{parts}
\end{Definition}
\begin{Definition}
	A \emph{directed set} $A$ is a set with a preorder and the following additional condition: For all $\alpha, \beta\in A$, there is some $\gamma$ with $\alpha \preceq \gamma$ and $\beta\preceq \gamma$. 
\end{Definition}
\begin{Remark}
The additional condition $\alpha\preceq \beta$ and $\beta\preceq \alpha\implies \beta=\alpha$ is known as a \emph{partial order}. This is sometimes too restrictive, and is usually not important. The important property is the defining property of a directed set.
\end{Remark}
\begin{Definition}
	A map from a directed set $I$ into a set $X$ is called a \emph{net} (indexed by $I$). The elements of this net are denoted by $(x_i)_{i\in I}$.
\end{Definition}
\begin{Definition}[Convergence]
	Let $X$ now be a topological space. We say that a net $(x_i)_{i\in I}$ converges to $x$ if for every neighbourhood $U$ of $x$ there is $\alpha\in I$ such that $\beta\succeq \alpha\implies x_\beta\in U$.
\end{Definition}
\begin{Theorem}
	The limit of a net is unique if the space $X$ is Hausdorff.
\end{Theorem}
\begin{proof}
The proof is analogous to that of a sequence.
\end{proof}

It is obvious that all sequences are nets. The difference is as follows: A sequence only has countably many elements. Thus, it is possible that there will be too many open sets. This leads us to our next theorem:

\begin{Theorem}
	Let $X$ be a topological space and $(x_i)_{i\in I}$ be a net converging to $x$. If $X$ is first countable, then there is a sequence converging to $x$.
\end{Theorem}

In general, we have the following theorem

\begin{Theorem}
	Let $A$ be a subset of the topological space $X$. $x\in A^\text{cl}$ if and only if there is a net in $A$ converging to $x$.
\end{Theorem}
\subsection{Baire Spaces}
	\begin{Definition}[Nowhere Dense]
		A subset $A$ of a topological space is called nowhere dense if the interior of its closure is open, $(A^\text{cl})^\circ=\varnothing$. 
	\end{Definition}
\begin{Theorem}
	A subset $A$ of a topological space $(M, \mathcal{M})$ is nowhere dense iff its complement contains a dense open set.
\end{Theorem}
\begin{proof}
		We perform the following computation:
		\begin{align*}
			A\text{ nowhere dense}&\iff (A^\text{cl})^\circ = \varnothing\\
			&\iff M \setminus (A^\text{cl})^\circ = M\\
			&\iff (M \setminus A^\text{cl})^\text{cl}=M\\
			&\iff ((M \setminus A)^\circ)^\text{cl}=M\\
			&\iff (M \setminus A)^\circ\text{ is dense}.\qedhere
		\end{align*}
\end{proof}
\begin{Definition}[Meager]
	A subset is called meager if it is a countable union of nowhere dense sets.
\end{Definition}
\begin{Theorem}\label{thm:bairecond}
	Let $(M, \mathcal{M})$ be a topological space. Then the following are equivalent:
	\begin{parts}
		\item Any countable union of closed subsets of $M$ without inner points has no inner points.
		\item Any countable intersection of open dense subsets of $M$ is dense.
		\item Every non-empty open subset of $M$ is not meager
		\item The complement of every meager subset of $M$ is dense.
	\end{parts}
\end{Theorem}
\begin{proof}
	The proof follows (a) $\implies$ (b) $\implies$ (c) $\implies$ (d) $\implies$ (a).
	\begin{enumerate}
		\item Let $(U_i)_{i\in \N}$ be a collection of dense open subsets of $M$. We consider their complements, which all have empty interior. Since
		\[M \setminus \left(\bigcap_{i=1}^\infty U_i\right)=\bigcup_{i=1}^\infty \left(M \setminus U_i\right)\]
		and the sets in the union on the right are all closed subsets without interior points, their union has no interior points, hence the countable intersection is dense.
		\item Suppose we had a meager nonempty open subset $U$ of $M$, that is, we have $U=\bigcup_{i=1}^\infty A_i$ with $A_i$ nowhere dense sets. Then $M \setminus A_i$ is a dense subset for all $i$, by (b), their intersection is still dense. Then
		\begin{align*}
			\varnothing &= M \setminus \left(\bigcap_{i=1}^\infty (M \setminus A_i^\text{cl})\right)^\text{cl}\\
			&= M \setminus \left(M \setminus \bigcup_{i=1}^\infty A_i^\text{cl}\right)^\text{cl}\\
			&= M \setminus \left(M \setminus\left(\bigcup_{i=1}^\infty A_i^\text{cl}\right)^\circ\right)\\
			&=\left(\bigcup_{i=1}^\infty A_i^\text{cl}\right)^\circ,
		\end{align*}
	which is a contradiction, since we had $U$ as a nonempty open subset of the final union. 
	\item Suppose we have a meager subset $A$ such that $M\setminus A$ is not dense. Then $(M \setminus A)^\text{cl} = M \setminus A^\circ \neq M$. Then $A$ has nonempty interior, contradicting (c).
	\item Finally, let us consider a sequence of closed sets $(A_n)_{n\in \N}$ without interior points, and we suppose that their union has a nonempty interior. Then
	\[\left(M \setminus \bigcup_{n=1}^\infty A_n\right)^\text{cl}=M \setminus \left(\bigcup_{n=1}^\infty A_n\right)^\circ\neq M,\]
	contradicting (d), since $\bigcup_{n=1}^\infty A_n$ is meager and hence has a dense complement.\qedhere
	\end{enumerate}
\end{proof}
	\begin{Definition}[Baire Spaces]
		A topological space $X$ is a Baire space if one (and hence all) of the conditions of Theorem~\ref{thm:bairecond} hold.
	\end{Definition}
Theorem \ref{thm:bairecond} is clearly unwieldy, and one must first show that a space is a Baire space before even beginning to apply this theorem. Thus, we seek easier conditions with which we can verify that a space is a Baire space. These theorems are known as Baire's Theorems.
	\begin{Theorem}[Baire I]
		A complete metric space $(M,d)$ is a Baire space.
	\end{Theorem}
\begin{proof}
	We seek to show the condition in Theorem~\ref{thm:bairecond}(b). Let us consider a collection of dense open sets $O_n$ and a point $x\in M$, as well as an open neighbourhood $U$ of $x$. Since $U$ intersects $O_1$, we can find an open ball $B_{\epsilon_1}(p_1)$ whose closure is contained in $U \cap O_1$. Inductively, we construct open balls $B_{\epsilon_n}(p_n)$ such that $B_{\epsilon_n}(p_n)^\text{cl}\subseteq B_{\epsilon_{n-1}}(p_{n-1})$. Then, 
	\[\bigcap_{i=1}^\infty B_{\epsilon_n}(p_n)^\text{cl}\subseteq U \cap \bigcap_{i=1}^\infty O_n\]
	But since the metric space is complete, the intersection on the left side is nonempty. Thus, the intersection on the right is nonempty. Since $U$ was an arbitrary open set, the intersection is dense. 
\end{proof}
	\begin{Theorem}[Baire II]
		A locally compact Hausdorff space $(M, \mathcal{M})$ is a Baire space.
	\end{Theorem}
	\section{Topological Vector Spaces}
	\begin{Definition}
		A topological vector space is a vector space with a topology such that addition and scalar multiplication are continuous.
	\end{Definition}
	\begin{Theorem}
		Translation $T_v:x\mapsto x + v$ and multiplication $\lambda: x \mapsto \lambda x$ with $\lambda\neq0$ are homeomorphisms.
	\end{Theorem}
	\begin{proof}
		They are invertible with continuous inverse $T_{-v}$ and $\frac 1\lambda$ respectively
	\end{proof}
	It is defined in some books that a topological vector space must be $T_1$ (or $T_0$). The usefulness of this definition comes from the fact that $T_0$ topological vector spaces are automatically $T_2$ and $T_3$. This is, in fact, not a statement about topological vector spaces, but about topological groups. Before we prove the result, we will need the following lemmas
	\begin{Lemma}
		Let $G$ be a group that is also a $T_1$ topological space. $G$ is a topological group iff the map $G \times G \to G, (x,y)\mapsto xy^{-1}$ is continuous.
	\end{Lemma}
\begin{proof}
	Clearly, if $G$ is a topological group, then the map is continuous. 
	
	Conversely, we consider $y \mapsto (e,y) \mapsto y^{-1}$, which is continuous as a composition of continuous functions. Then $(x,y) \mapsto (x, y^{-1}) \mapsto xy$ is continuous, again as a composition of continuous functions.
\end{proof}
\begin{Lemma}
	A topological space $X$ is Hausdorff iff the diagonal $X\times X$ is closed.
\end{Lemma}
	\begin{Theorem}
		Let $G$ be a $T_0$ topological group. Then it is $T_1, T_2$ and $T_3$.
	\end{Theorem} 
\begin{proof}
	We will prove $T_0\implies T_1\implies T_2$. 
	
	The goal is: We show that $\{e\}$ is closed. Pick $g\neq e$. Because $G$ is $T_0$, either $G \setminus \{e\}$ is a neighbourhood of $g$, or $G \setminus \{g\}$ is a neighbourhood of $e$. Suppose the latter. Then applying the translation $T_{g^{-1}}$ shows that $G \setminus \{e\}$ is a neighbourhood of $g^{-1}$, and applying the inversion map shows that $G \setminus \{e\}$ is a neighbourhood of $g$. Thus $\{e\}$ is closed.
	
	Then the translations show that every singleton set is closed.\textcolor{red}{TODO}
	
	Note: If the group is abelian, then the proof is significantly easier: We suppose $U$ is a neighbourhood of $x$ not containing $y$. Then $x+y-U$ is a neighbourhood of $y$ not containing $x$. 
	
	Now let $G$ be $T_1$
\end{proof}
	\subsection{Continuity}
	\begin{Definition}[Uniform Continuity]
		Let $V,W$ be topological vector spaces and $\phi:V\to W$ be a map. Then $\phi$ is called uniformly continuous if for all neighbourhoods $U\subseteq W$ of $0$ there exists a neighbourhood $Z\subseteq V$ such that $v-v'\in Z\implies \phi(v)-\phi(v')\in U$.
	\end{Definition}
	\begin{Theorem}[Equivalence of Continuity Conditions]
		Let $V$ and $W$ be topological vector spaces and let $\phi:V\to W$ be a linear map. Then the following are equivalent
		\begin{parts}
			\item The map $\phi$ is continuous at $0$.
			\item The map $\phi$ is continuous at some $v\in V$.
			\item The map $\phi$ is continuous.
			\item The map $\phi$ is uniformly continuous.
		\end{parts}
	\end{Theorem}
	\begin{proof}
		\begin{enumerate}
			\item Assume $\phi$ is continuous at 0. 
			
			Recall: This means that for all neighbourhoods $U$ of $0_W$, there is a neighbourhood $Z$ of $0_V$ such that $\phi(Z)\subseteq V$. Now choose $v\in V$ and a neighbourhood $U\ni \phi(v)$. Then $T_{-v}(U)$ is a neighbourhood of $0_W$, and we choose a neighbourhood $Z$ of $0$ such that $\phi(Z)\subseteq T_{-v}(U)$ by continuity at $0$. Then $T_v(Z)\subseteq U$ and contains $v$, as desired.
			\item Repeating the same proof, we can get continuity at every point in $V$, which is equivalent to continuity.
			\item We actually only need continuity at 0. Given an open neighbourhood $Z$ of $0_W$, we choose an open neighbourhood $U$ of $0_V$ that maps into $Z$. Then we have, for $v-u\in Z$, that $\phi(v)-\phi(u)=\phi(v-u)\in W$.
			\item Uniform continuity directly implies continuity at 0 if we take $v'$ to be $0$ in the definition. \qedhere
		\end{enumerate}
	\end{proof}
	\begin{Definition}[Algebraic and Topological Dual]
		The algebraic dual of a vector space $V$ is the well known space of linear maps from $V$ to the field over which it is defined. We denote this by $V'=\text{Hom}(V,\mathbb{K})$. The topological dual $V^*=L(V, \mathbb{K})$ is the space of all \emph{continuous} linear maps. Clearly, $L(V, \mathbb{K})\subseteq \text{Hom}(V,\mathbb{K})$.
	\end{Definition}
	\subsection{Subspaces}
	\begin{Theorem}
		Let $V$ be a topological vector space. If $W\subseteq V$ is a subspace, then its closure $W^\text{cl}$ is also a subspace of $V$.
	\end{Theorem}
	\begin{proof}
Suppose $v,w\in W^\text{cl}$. Then there are nets $(v_i)_{i\in I}$ converging to $v$ and $(w_i)_{i\in I}$ converging to $w$. Then $(v_i+w_i)_{i\in I}$ is a net converging to $v+w$, hence $v+w$ is still in $W^\text{cl}$. 
	\end{proof}
	\section{Banach Spaces}
	\subsection{Norms \& Seminorms}
	\begin{Definition}[Norm]
		A norm $\|\cdot\|$ on a vector space $V$ is a function $V\to \R$ that satisfies
		\begin{parts}
			\item (Homogenity) $\|sv\|=|s|\cdot \|v\|$ for all vectors $v$ and scalars $s$
			\item (Triangle Inequality) $\|u+v\|\le \|u\| + \|v\|$
			\item (Positivity) $\|v\|>0$ for all $v\neq 0$.  
		\end{parts}
	\end{Definition}
	\begin{Remark}
		A norm induces a topology through the metric defined by $d(v, w)=\|v-w\|$. If a vector space has this topology, it is known as a \emph{normed space}.
	\end{Remark}
	\begin{Definition}[Banach Space]
		A Banach space is a complete normed space.
\end{Definition}
	\begin{Definition}[Seminorm]
		A seminorm $p$ is a norm without the positivity condition. Instead, we have positive semidefiniteness, i.e.
		\[p(v)\ge 0~\forall v\in V.\]
	\end{Definition}
	Note that a seminorm does not form a metric space. We define the \emph{kernel} of the seminorm as the set $\{v\in V|p(v)=0\}$. Note that this is a subspace. To get an actual norm, we must ``divide'' by the kernel.
	
	We can define a norm on the quotient vector space $V / \text{ker } p$ by letting the seminorm $p$ act on any representative of this space. 
	\begin{Theorem}[Quotient of Norms]
		The norm $\|\cdot \|: V / \text{ker } p\to \R$,  $[v] \mapsto p(v)$ is a norm.
	\end{Theorem}
\begin{proof}
	First, we show that it is well defined. Consider $v\in V$ and let $u\in \text{ker }p$ be arbitrary. Then we have
	\[p(v+u)\le p(v)+p(u)=p(v)\]
	by the triangle inequality, and conversely
	\[p(v)=p(v-u+u)\le p(v+u)+p(u)=p(v+u)\]
	which shows that the norm is independent of the choice of representative.
	
	The triangle inequality and homogenity follow from the same properties of the seminorm. It is also true by definition that the norm is positive.
\end{proof}
Thus, given a seminorm on a vector space, we can construct a new vector space that has a norm, and thus an induced topology. 
\begin{Theorem}[Operator Norm]
	Let $V,W$ be normed spaces and let $A:V\to W$ be a linear map. Then the following statements are equivalent:
	\begin{parts}
		\item $A$ is continuous.
		\item There exists a constant $c\ge 0$ such that
		\[\|A(v)\| \le c\|v\|\]
		for all $v\in V$.
	\end{parts}
\end{Theorem}
\begin{proof}
	We use the fact that continuity is equivalent to continuity at 0. Then, we simply unravel the definitions. $A$ is continuous at 0 if for every neighbourhood $Z\subseteq W$ of $0_W$ we have a neighbourhood $U\subseteq V$ of $0_V$ such that $A(U)\subseteq V$.
	
	Equivalently, we can consider open balls in place of $Z$ and $U$. This is then equivalent to the second condition.
\end{proof}
	\begin{Definition}[Operator Norm]
		The operator norm $\|A\|$ is defined as
		\[\|A\|=\inf \{c| \|Av\| \le c\|v\|~\forall v\in V\}\]
	\end{Definition}
	\begin{Corollary}[Equivalent Statements]\noindent
		\begin{parts}
			\item \[\|A\|=\sup_{V\ni v \neq 0}\frac{\|Av\|}{\|v\|}\]
			\item \[\|A\| = \sup_{\|v\|=1}\|A v\|\]
		\end{parts}
	\end{Corollary}
	\begin{Theorem}[Operator Composition]
		Let $V,W, Z$ be normed spaces, $A\in L(V,W)$, $B\in L(W,Z)$. Then
		\[\|B\circ A\| \le \|A\| \|B\|.\]
	\end{Theorem}
\begin{proof}
	For $v\in V$, we have
	\[\|BAv\|\le \|B\| \|Av\| \le \|B\| \|A\| \|v\|.\qedhere\]
\end{proof}
Note that the above definition shows once again the continuity of composition.
	\subsection{Bases}
	\begin{Definition}[Hamel Basis]
		A Hamel basis is a set $B\subseteq V$ such that for all $v\in V$, we have
		\[v=\sum_{k=1}^n a_k e_k\]
		with $a_i\in \mathbb{K}$ and $e_i\in B$ for all $i$, and
		\[\sum_{k=1}^n a_k e_k=0\implies a_i=0\forall i.\]
	\end{Definition}
	The existence of Hamel bases is equivalent to the axiom of choice, and can be proven by Zorn's Lemma. The proof is as follows: We construct minimal spanning sets and maximally linearly independent sets by Zorn's Lemma (partial order by inclusion), and show that they are the same. These sets are Hamel bases.
	
	A Hamel basis has all the beloved properties of a basis from finite dimensional linear algebra. For example, a linear map is uniquely defined through its action on the Hamel basis:
	\[Av = \sum_{k=1}^n a_k Ae_k.\]
	However, a Hamel bases are usually difficult to come by, as we see with the following theorem:
	\begin{Theorem}
		A Banach space with a countable Hamel basis is finite. 
	\end{Theorem}
\begin{proof}
	Suppose we have a countable Hamel basis $(e_n)_{n\in \N}$ of a Banach space $V$. Denote $M_n=\text{span}(\{e_1, \dots, e_n\})$. This is a closed proper subspace of $V$. Thus it has empty interior. However, because by definition the $e_n$s form a Hamel basis, we have $V=\bigcup_{k=1}^\infty M_k$, contradicting Baire's category theorem.
\end{proof}
	\begin{Definition}[Schauder Basis]
		A Schauder basis is a countable set $\{e_n\}\subseteq V$ such that all vectors $v\in V$ can be uniquely expressed as a sum
		\[v=\sum_{k=1}^\infty a_k e_k,\]
		where the convergence is understood to be in the topology of the vector space. 
	\end{Definition}
The importance of a Schauder basis is that it is countable, and we are still able to define \emph{some} linear maps by their action on the basis. In particular, for any continuous linear map $A$, we have
\[Av = \sum_{k=1}^\infty a_k Ae_k.\]
Also important to note is that this basis must be \emph{ordered}, since the sum does not necessarily converge unconditionally. 
	\subsection{Quotients of Banach Spaces}
	\begin{Theorem}[Quotient (Semi-)Norm]
		Let $V$ be a vector space and $U\subseteq V$ a subspace. 
		\begin{parts}
			\item For every seminorm $p$ on $V$,
			\[[p]([v])=\inf\{p(v+u)|u\in U\}\]
			defines a seminorm on $V/U$.
			\item If $p$ is a norm then $[p]$ is a norm on $V/U$ iff $U$ is closed.
		\end{parts}
	\end{Theorem}
	\begin{Definition}[Quotient (Semi-)Norm]
	\end{Definition}
	\begin{Theorem}[Quotient Norm]
	\end{Theorem}
	\subsection{Various Examples}
	\section{The Main Theorems}
	\begin{Definition}
		Let $V$ be a vector space over $\mathbb{K}$. A functional $p$ is called \emph{sublinear}, if
		\begin{parts}
			\item (Homogenity) $p(rx)=rp(x)$ for all real $r\ge 0$ and $v\in V$.
			\item (Subadditivity) $p(u+v)\le p(u)+p(v)$ for all $u,v\in V$
		\end{parts}
	\end{Definition}
The main property that relates sublinear functionals to linear functionals is as follows:
\begin{Theorem}
	If $p$ is a sublinear functional on a real vector space $V$, then the following are equivalent:
	\begin{parts}
	\item $p$ is linear
	\item $p(v)+p(-v)=0$ for all $v\in V$.
	\item $p(v)+p(-v)\le 0$ for all $v\in V$
	\end{parts}
\end{Theorem}
\begin{proof}
	If $p$ is linear, then $p(v)+p(-v)=p(v-v)=p(0)=0$. Clearly, (b) implies (c).

	Now assume (c). First, we prove (b):
	\[
	0=p(v-v)\le p(v)+p(-v) \le 0
	\]
	or $p(v)+p(-v)=0$, proving (b). Using (b) we have, for $r<0$,
	\[
	p(rv)=-p(-rv)=rp(v)
	,\]
	proving the first aspect of linearity. Then we have
	\[
	p(v+w-w)\le p(v+w)+p(w)=p(v+w)+p(-w)
	,\]
	implying
	\[
	p(v)+p(w)\le p(v+w)
	.\] 
	The subadditivity yields the other inequality, completing the proof of linearity.
\end{proof}

We can define a partial order on the sublinear functions by defining $p$ to be less than $q$ if $p$ is less than $q$ pointwise. Zorn's Lemma then yields minimal elements. We will prove eventually that these minimal elements are linear functionals. Before that, however, we need two lemmas:

\begin{Theorem}[Auxiliary Functionals]\label{thm:auxfunc}
	If $p$ is a sublinear functional on a real vector space $V$, then the auxiliary functional $q(v)=\inf \{p(v+tw)-tp(w)|t\ge 0,w\in V\} $ is a sublinear functional such that $q \le p$.
\end{Theorem}
\begin{proof}
	By considering $t=0$ in the infimum, we see that $q\le p$. We only need to show sublinearity. Consider first $r=0$. Then
	\[
	q(rv)=\inf \{p(rv+tw)-tp(w)|t\ge 0,w\in V\} =\inf \{p(tw)-tp(w)|t\ge 0, w\in V\} =0
	.\] 
	For $r>0$, we have
	\begin{align*}
		q(rv)&=\inf \{p(rv+tw)-tp(w)|t \ge 0, w\in V\} \\
		     &=\inf \left\{ rp\left( v+\frac{t}{a}w \right)-tp(w)|t\ge 0, w\in V \right\} \\
		     &=r\inf \{p(v+tw / a) -tp(w) / a|t\ge 0, w\in V\} \\
		     &=rq(v)
	\end{align*}
	For subadditivity, we take the special points $w=\frac{1}{s+t}(sx+ty)$, or $(s+t)w=sx+ty$. Then
	\begin{align*}
		q(x+y)&\le p(x+y+(s+t)w)-(s+t)p(w)\\
		      &\le p(x+sw)-sp(w)+p(y+tw)-tp(w)
	\end{align*}
	which shows that $q(x+y)\le q(x)+q(y)$.
\end{proof}
Now, we move on to the proof of the main result.
\begin{Theorem}
	A sublinear functional $p$ on a real vector space $V$ is linear iff it is minimal.
\end{Theorem}
\begin{proof}
	Suppose we have $q\le p$, with $q$ sublinear and $p$ linear. Since $q$ is sublinear, we have $0=q(v-v)\le q(v)+q(-v)$, which implies $-q(-v)\le q(v)$. Since $q(-v) \le p(-v)=-p(v)$, we have $p(v)\le -q(-v)\le q(v)$, suggesting that $p\le q$. Thus $p=q$.

	Conversely, suppose that $p$ is a minimal sublinear functional. Then we must have $q=p$ where $q$ is the auxiliary sublinear functional defined in Theorem~\ref{thm:auxfunc}. If we let $t=1$ and $w=-v$ in the above definition, we get
	\[
	p(v)\le p(v-v)-p(-v)
,\]
or $p(v)+p(-v)\le 0$, as desired. Finally, we show the boundedness which we require to apply Zorn's Lemma
\end{proof}
\begin{Theorem}
	For every sublinear functional $p$, we have a linear functional $f\le p$. 
\end{Theorem}
\begin{proof}
	We begin, as always, by considering a chain of sublinear functionals, which is also a totally ordered set of sublinear functionals $P$. Suppose that $q(x)$ is unbounded from below for all $q\in P$ and some $x\in V$. Then we have, for all $n$, a $p_n$ such that $p_n(x)\le -n$.

	Then we construct a decreasing sequence of sublinear functionals $q_n$ using $q_n=\min (p_1, \dots, p_n)$. Since $q_n(x)\le -n$, we have
	\[
	0=q_n(x-x)\le q_n(x)+q_n(-x)\le -n +q_n(-x)
	\]
	or $q_n(-x)\ge n$. Then we have, for all $n$, $n\le q_n(x)\le q_1(x)$, a contradiction. Thus we can take the infimum for all $x\in V$. This yields a functional $q^*$. It remains to show that $q^*$ is sublinear. \textcolor{red}{TODO} 
\end{proof}
\subsection{Separation Properties}
\begin{Definition}
	Let $V$ be a normed space. A subset $A$ of $V$ is called
	\begin{parts}
	\item \emph{Absorbing}, if for every $v\in V$ there is $\lambda>0$ with $v\in \lambda A$.
	\item \emph{Balanced}, if for all $|z|\le 1$ we have $zA\subseteq A$.
	\item \emph{Convex}, if for all $v,w\in A$ and $\lambda\in [0,1]$ we have $\lambda v+(1-\lambda)w\in A$.
	\item \emph{Absolutely convex}, if it is balanced and convex.
	\end{parts}
\end{Definition}
\begin{Theorem}
	Let $V$ be a vector space over $\mathbb{K}$.
	\begin{parts}
	\item If $p:V \to \R_0^+$ is a seminorm then
		\[
			B_{p,1}(0)=\{v\in V|p(v)<1\} 
		\]
		and
		\[
			B_{p,1}(0)^\text{cl}=\{v\in V|p(v)\le 1\} 
		\]
		are absorbing and absolutely convex.
	\item If $C\subseteq V$ is convex, balanced and absorbing then
		\[
		p_c(V)=\inf \{\lambda| \lambda<0,v\in \lambda C\} 
	\]
	is a seminorm.
\item For $C$ absolutely convex and absorbing
	\[
		B_{p_c, 1}(0)\subseteq C \subseteq B_{p_c, 1}(0)^\text{cl}
	.\] 
\item For every seminorm $p$ one has
	\[
		P_{B_{p, 1}(0)}=P=P_{B_{p,1}(0)^\text{cl}}
	.\] 
	\end{parts}
\end{Theorem}
\begin{Definition}[Minkowski Functional]
	Let $C\subseteq V$ be an absorbing subset in a vector space $V$ over a $\mathbb{K}$. Then $p_c:V\to \R_0^+$ defined by
	\[
	p_c(v)=\inf \{\lambda>0|v\in \lambda C\} 
\]
is called the Minkowski functional of $C$.
\end{Definition}
\begin{Theorem}[Separation I]
	Let $V$ be a normed space with two nonempty disjoint convex subsets $A,B\subseteq V$.
	\begin{parts}
	\item Suppose $A$ is open. Then there exists $\Phi\in V'$ and $\alpha\in \R$ such that
		\[
		\Re(\Phi(v))<\alpha\le \Re(\Phi(u))
	\]
	for $v\in A< u\in B$
\item Suppose both $A$ and $B$ are open. Then there exists $\Phi\in V'$ and $\alpha\in \R$ with
	\[
	\Re(\Phi(v))<\alpha<\Re(\Phi(v))
\]
for $v\in A, u\in B$.
	\end{parts}
\end{Theorem}
The picture for this theorem is as follows: A linear functional $\Phi$ defines a hyperplane for each fixed value of $\alpha$, then splitting the space into two parts, one where the functional is less than, and one where the functional is greater than $\alpha$. In $\R^2$, the picture looks as follows:
\begin{center}
	\begin{tikzpicture}[scale=1.3]
		\draw[thick] (0,0) circle (1);
		\draw[thick] (3,0) circle (1);
		\draw[thick, orange] (1.3,-1.5) -- (1.9,1.5) node[anchor=south] {$\Phi(v)=\alpha$};
		\draw[thick, orange] (0,1) node[anchor=south] {$\Phi(v)<\alpha$};
		\draw[thick, orange] (3,1) node[anchor=south] {$\Phi(v)>\alpha$};
	\end{tikzpicture}
\end{center}
\begin{proof}
	First, we prove the extension from $\R$ to $\C$. If we assume that the real case is done, then $\Phi(v)=\Psi(v)-i\Psi(iv)$ is $\C$-linear and $\Re(\Phi)=\Psi$ still fulfils the equation.

	Hence, we focus on the real case. Suppose $A$ is open. Then, for $v_0\in A, w_0\in B$, we have $u_0=w_0-v_0\neq 0$. Then $C=A-B+u_0$ is open, and $0\in C$. This implies that $C$ is absorbing.

	Since $A-B$ is still convex, we can define the Minkowski functional on $C$. This is, however, not a seminorm, as $C$ might not be balanced. However, $p$ is a sublinear functional. 
\end{proof}
\end{document}

