\begin{Problem}
	Es seien $V,W$ zwei nicht notwendigerweise endlich-dimensionale Vektorräume über $\mathbb{K}$. Wir definieren eine lineare Abbildung $\Phi:V\to \text{im}(\Phi)\subseteq W$. Zeigen Sie:
	\begin{parts}
		\item F\"{u}r jedes Komplement $U$ von $\text{ker}(\Phi)$, d.h. $V=\text{ker}(\Phi)\oplus U$, ist
		\[
			\Phi|_U:U\to\text{im}(\Phi)
		\]
		ein Isomorphismus.
	\item Sei $g$ eine Funktion der Art
		\[
			g:\text{im}(\Phi)\to V, b\to x
		\]
für ein ausgewähltes $x$, sodass $\Phi(x)=b$ und sei $g$ linear, dann gilt
\[
	V=\text{ker}(\Phi)\oplus g(\text{im}(\Phi))
.\] 
	\end{parts}
\end{Problem}
\begin{proof}
	\begin{parts}
	\item Es ist ein Homomorphismus, weil es linear ist. Wir müssen nur zeigen, dass es bijektiv ist.

		Es ist injektiv. Nehmen wir an, dass es nicht injektiv ist. Es gibt dann zwei unterschiedliche Vektoren $v,u\in U$, sodass $\Phi(v)=\Phi(u)$. Es gilt dann
		\[
			\Phi(\underbrace{v-u}_{\in U})=\Phi(v)-\Phi(u)=0
		,\]
		also $v-u\in\text{ker}(\Phi)$. Wir wissen jedoch, dass $v-u\neq 0$, ein Widerspruch zu der Voraussetzung, dass $U$ ein Komplement von $\text{ker}(\Phi)$ ist.

		Es ist surjektiv. F\"{u}r jedes $0\neq v\in \text{im}(\Phi)$ gibt es ein Vektor $u\in V$, sodass $v=\Phi(u)$ (Definition des Bilds). Wenn $v\neq 0$, kann $u$ nicht in $\text{ker}(\Phi)$ sein. Also wir schreiben $u=u_1+u_2$, wobei $u_1\in \text{ker}(\Phi)$ und $u_2\in U$. Es gilt
		\[
		v=\Phi(u_1)+\Phi(u_2)=0+\Phi(u_2)=\Phi(u_2)
		,\]
		also f\"{u}r jedes $\text{im}(\Phi)\ni v\neq 0$ gibt es ein Vektor $u_2\in U$, sodass $\Phi(u_2)=v$, also $\Phi|_U$ ist surjektiv.

		Schluss: $\Phi|_U$ ist ein Isomorphismus.
	\item Wir brauchen $x \neq 0$. Sei $B$ ein Basis
		\[
			B=\left\{\smash{\underbrace{v_1,\dots, v_k}_{\text{Basis f\"{u}r ker}(\Phi)}, \underbrace{x, v_{k+2}, \dots, v_n}_{\text{Basis f\"{u}r }U}}\vphantom{v_1}\right\} 
		.\] 

	Sei $\Phi:\R^3\to R^2, (x,y,z)^T\to (x,y,0)$. Sei dann $g: \R^2\to \R^3, (x,y)\to (x,0,y)$
	\end{parts}
\end{proof}

\begin{Problem}
	Es sei $A,B,C,D$ Matrizen \"{u}ber $\mathbb{K}$ in den Dimensionen $m\times m, m \times n, n \times m, n \times n$. 
	\begin{parts}
	\item Zeigen Sie: ist $A$ invertierbar, dann gilt
		\[
			\text{det}\begin{pmatrix} A & B \\ C & D \end{pmatrix} =\text{det}(A)\cdot \text{det}(D-CA^{-1}B)
		.\] 
	\item Beweisen Sie die Gleichheit
		\[
			\text{det}(1_m+BC)=\text{det}(1_n+CB)
		.\] 
	\end{parts}
\end{Problem}
\begin{proof}
	\begin{parts}
	\item Wir beweisen zuerst eine ähnliche Ergebniss zu die diagonale Matrizen: Wenn $B=0$ und $C=0$, ist
		 \[
			 \text{det}\begin{pmatrix} A & 0 \\ 0 & D \end{pmatrix} = \text{det}(A)\text{det}(D)
		.\] 
	\end{parts}
\end{proof}
\begin{Problem}
	Gegeben seien die Vektoren
	\[
	v_1=\begin{pmatrix} 1 \\ 2 \\ 1 \end{pmatrix} , \qquad v_2=\begin{pmatrix} 3 \\ 1 \\ 1 \end{pmatrix} , \qquad v_3(t)=\begin{pmatrix} t^3 \\ \frac{3}{2}t \\ t^2 \end{pmatrix} 
	.\] 
	Finden Sie alle $t\in \R$ f\"{u}r welche das Volumen des durch $v_1,v_2,v_3(t)$ aufgespannten Spat verschwindet. F\"{u}r welche $t\in [1,4]$ wird das Volumen maximal.
\end{Problem}
\begin{Problem}
	Es gilt $v_1\times v_2= (1,2,-5)^T$ und
	\[
	v_3(t)\cdot (v_1\times v_2)=t^3+3t-5t^2=t(t^2-5t+3)
	.\] 
	Die Nullstellen des Polynoms $t^2-5t+3$ sind $\frac{1}{2}(5\pm \sqrt{13} )$, also das Volumen verschwindet f\"{u}r $t=0$ oder $t=\frac{1}{2}(5\pm \sqrt{13} )$. 

	Wir berechnen die Ableitung
	\[
		\dv{t}\left( t^3-5t^2+3t \right) =3t^2-10t+3
	.\]
	Die Ableitung ist $0$ genau dann, wenn $t=3^{\pm 1}$. Die zweite Ableitung ist:
	 \[
		 \dv[2]{t}\left( t^3-5t^2+3t \right) =6t-10
	.\] 
	was $<0$ ist, wenn $t=\frac{1}{3}$ und $>0$ ist, wenn $t=3$, also $t=\frac{1}{3}$ ist ein lokales Maximum. Wir betrachten dann zudem die Grenzpunkte
\end{Problem}
\begin{Problem}
	Berechnen Sie die Determinanten der folgenden Matrizen
	\begin{parts}
	\item $A\in \mathbb{K}^{n\times n}$, mit
		\[
			A=\begin{pmatrix}  a & b & b & \dots & b \\ b & a & b & \dots & b \\ b & b & a & \dots & b \\ \vdots & \vdots & & \ddots & \vdots \\ b & b & \dots & b & a \end{pmatrix} 
		\] 
		und $a\neq 0$ 

		{\footnotesize\emph{Wenden Sie m\"{o}glichst geschickt 2(a) an.}}
		\item $A\in \mathbb{K}^{(n+1)\times (n+1)}$, mit
			\[
				A=\begin{pmatrix} t & 0 & \dots & \dots & a_0 \\ -1 & t & \dots & \dots & a_1 \\ 0 & -1 & t & \dots & a_2 \\ \vdots & \ddots & \ddots & \ddots & \vdots \\ 0 & \dots & 0 & -1 & t+a_n \end{pmatrix} 
			.\] 
			{\footnotesize\emph{Sie dürfen die Entwicklungsformel nach Laplace verwenden.}}
		\item $V_{ij}, R_{i\lambda}, S_{ij}\in \mathbb{K}^{n\times n}$ aus den Lemmata 5.56 - 5.58.
	\end{parts}
\end{Problem}
\begin{proof}
	\begin{parts}
	\item Wir bezeichnen die $n\times n$ solche Matriz als $A_n$ und deren Determinante als $\Delta_n(a,b)$. Per Hinweis definieren wir $A=(a), B=(b,b,\dots, b), C=B^T, D=M_{n-1}$. Falls $a\neq 0$ ist $A$ invertierbar mit inverse $A^{-1}=(1 / a)$.

		Es gilt außerdem
		\[
			CA^{-1}B=\begin{pmatrix} b \\ b \\ \vdots \\ b \end{pmatrix} \left( \frac{1}{a} \right) \begin{pmatrix} b & b & \dots & b \end{pmatrix}  =\frac{1}{a}\begin{pmatrix} b^2 & b^2 & \dots & b^2 \\ b^2 & b^2 & \dots & b^2 \\ \vdots & \vdots & \ddots & \vdots \\ b^2 & b^2 & \dots & b^2 \end{pmatrix} 
		.\] 
		Daraus folgt
		\[
		\Delta_n(a,b)=a\Delta_n\left( a-\frac{b^2}{a},b-\frac{b^2}{a} \right), \qquad n > 2
		.\] 
		F\"{u}r $n=2$ ist $\Delta_n(a,b)=a^2-b^2$.
	\end{parts}
\end{proof}
