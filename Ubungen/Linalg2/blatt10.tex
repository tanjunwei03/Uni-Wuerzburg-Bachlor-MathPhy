\begin{Problem}
	Betrachten Sie den Vektorraum $\C^n$, $n\in \N$ mit dem Standardskalarprodukt $\left<v,w \right> = \overline{v}^Tw$ f\"{u}r alle $v,w\in \C^n$. Sei $A\in M_n(\C)$. Zeigen Sie die folgenden Aussagen.
	\begin{enumerate}[label=(\alph*)]
		\item Die adjungierte Abbildung von $A$, wie in Definition 7.66, ist gerade durch die in Definition 7.13 beschriebene adjungierte Matrix $A^*=\overline{A^T}$ gegeben.
		\item Es gilt $\text{tr}(A^*)=\overline{\text{tr}(A)}$.
		\item Es gilt $\text{det}(A^*)=\overline{\text{det}(A)}$.
		\item $\lambda\in \C$ ist genau dann Eigenwert von $A$, wenn $\overline{\lambda}$ Eigenwert von $A^*$ ist.
	\end{enumerate}
Sei $A$ nun ein invertierbarer und adjungierbarer Endomorphismus auf dem unitären Vektorraum $V$. Dann gilt
\begin{enumerate}[label=(\alph*),resume]
	\item $\left( A^{-1} \right)^*=\left( A^* \right)^{-1}$.
\end{enumerate}
\end{Problem}
\begin{proof}
\begin{parts}
\item Es gilt
	\begin{align*}
		\left<A^*v,w \right> =& (A^* v)^Tw\\
		=&v^T \overline{(A^*)^T}w\\
		=&v^T \overline{(\overline{A^T})^T}w\\
		=& v^T \overline{(\overline{A}^T)^T}w\\
		=& v^T\overline{\overline{A}}w\\
		=& v^TAw\\
		=&\left<v,Aw \right>
	\end{align*}
\item 
	\begin{align*}
		\text{tr}(A^*)=&\sum_{i=1}^n (A^*)_{ii}\\
		=&\sum_{i=1}^n (\overline{A^T})_{ii}\\
		=&\sum_{i=1}^n \overline{A^T_{ii}}\\
		=&\sum_{i=1}^n \overline{A_{ii}}\\
		=&\overline{\sum_{i=1}^n A_{ii}}\\
		=& \overline{\text{tr}(A)}.
	\end{align*}
\item Wir brauchen aus den vorherigen Übungsblätter: $\text{det}(A^T)=\text{det}A$. Wir wissen auch aus den Eigenschaften der komplexe Konjugierte
\begin{equation}\label{eq:linalg2blatt10-1}
	\overline{\sum_{i=1}^n\prod_{j=1}^k a_{ij}}=\sum_{i=1}^n\prod_{j=1}^k \overline{a_{ij}}	
\end{equation}
	Es folgt daraus:
	\begin{align*}
		\text{det}(A^*)=&\text{det}(\overline{A^T})\\
		=&\overline{\text{det}(A^T)} & \eqref{eq:linalg2blatt10-1}\\
		=& \overline{\text{det}(A)}
	\end{align*}
\item 
	Sei $\lambda$ ein Eigenwert von $A$. Es gilt dann
	\begin{align*}
		\text{det}(A^*-\overline{\lambda} I)=&\text{det}(\overline{A^T}-\overline{\lambda} I)\\
		=& \text{det}(\overline{A}^T-\overline{\lambda}I^T)\\
		=&\text{det}(\overline{(A-\lambda I)}^T)\\
		=& 0
	\end{align*}
Umgekehrt: Sei $\overline{\lambda}$ ein Eigenwert von $A^*$. Es gilt:
\begin{align*}
	\text{det}(A-\lambda I)=&\text{det}(\overline{(\overline{A}-\overline{\lambda}I}))\\
	=&\text{det}((\overline{(\overline{A}^T-\lambda I^T)})^T)\\
	=& 0
\end{align*}
\item Sei $v,w\in V$. Es gilt
	\begin{align*}
		\left<(A^*)^{-1}w,v \right> =& \left< (A^*)^{-1}w,A A^{-1}v \right>\\
		=&\left< A^*(A^*)^{-1}w,A^{-1}v \right>\\
		=&\left<w,A^{-1}v \right> \\
	\end{align*}
	also $(A^*)^{-1}=(A^{-1})^*$.\qedhere
\end{parts}
\end{proof}
\begin{Problem}
	Betrachten Sie die Vektoren
	\[
	v_1=\begin{pmatrix} i \\ 0 \\ -1 \\ -2 \end{pmatrix},\qquad v_2=\begin{pmatrix} 2 \\ 2 \\ i \\ 0 \end{pmatrix} 
\]
im unitären Vektorraum $\C^4$, versehen mit dem Standardskalarprodukt $\langle x,y\rangle:=\overline{x}^Ty$. Sei $U=\text{span}\{v_1,v_2\}$. Bestimmen Sie eine Orthonomalbasis von $U$ sowie von $U^\perp$.
\end{Problem}
\begin{proof}
	\[
		U^\perp=\text{ker}\begin{pmatrix} -i & 0 & -1 & -2 \\ 2 & 2 & -i &0 \end{pmatrix} 
	.\] 
Es gilt
\begin{gather*}
\left(
\begin{array}{cccc}
 -i & 0 & -1 & -2 \\
 2 & 2 & -i & 0 \\
\end{array}
\right) \xrightarrow{R_2-2iR_1} \left(
\begin{array}{cccc}
 -i & 0 & -1 & -2 \\
 0 & 2 & i & 4 i \\
\end{array}
\right) \xrightarrow{R_1\times i} \left(
\begin{array}{cccc}
 1 & 0 & -i & -2 i \\
 0 & 2 & i & 4 i \\
\end{array}
\right)	
\end{gather*}
Eine Basis f\"{u}r den Kern ist dann
\[
\begin{pmatrix} i \\ -i / 2 \\ 1 \\ 0 \end{pmatrix} , \begin{pmatrix}  2i \\ -2i \\ 0 \\ 1\end{pmatrix} 
.\] 
Der Gram-Schmidt-Algorithismus ergibt dann Orthonomalbasen. F\"{u}r $U$ ist $\langle v_2,v_1\rangle=3i,~\left<v_1,v_1 \right> = 6$. Dann ist eine Basis
\[
	\left\{	v_1,v_2-\frac{\left<v_2,v_1 \right>}{\left<v_1,v_1 \right>}v_1\right\}=\left\{ \begin{pmatrix} i \\ 0 \\ -1 \\ -2 \end{pmatrix} , \begin{pmatrix} 5 / 2 \\ 2 \\ 3i / 2 \\ i \end{pmatrix}  \right\} 
.\] 
Daraus ergibt sich eine Orthonomalbasis f\"{u}r $U$:
\[
\left\{ \frac{1}{\sqrt{6} }\begin{pmatrix} i \\ 0 \\ -1 \\ -2 \end{pmatrix} , \frac{1}{3\sqrt{6} }\begin{pmatrix} 5 \\ 4 \\ 3i \\ 2i \end{pmatrix}  \right\} 
.\] 
F\"{u}r $U^\perp$ ist ähnlich eine Basis
\[
\left\{ 3\begin{pmatrix} 2i \\ -i \\ 2 \\ 0 \end{pmatrix} , \frac{1}{3\sqrt{5} }\begin{pmatrix} 2i \\ -4i \\ -4 \\ 3 \end{pmatrix}  \right\} 
.\qedhere\] 
\end{proof}
\begin{Problem}
	Sei $V_n\subset \C[x]$ der Unterraum der Polynome vom Grad $\leq n$. Sei
	\[
		\left<f,g \right> = \int_0^1 \overline{f(x)}g(x)\dd{x}
	\]
	f\"{u}r $f,g\in V_n$.
	\begin{parts}
	\item Zeigen Sie, dass durch $\langle\cdot,\cdot\rangle$ ein Skalarprodukt auf $V_n$ definiert ist.
	\item Starten Sie mit der Basis der Monome $1,x,x^2,x^3$ und f\"{u}hren Sie den Gram-Schmidt-Algorithmus durch, um eine Orthonomalbasis von $V_3$ bezüglich $\langle\cdot,\cdot\rangle$ zu finden.
	\end{parts}
\end{Problem}
\begin{proof}
	\begin{parts}
	\item Die linearität Eigenschaften folgen alle aus die Linearität des Integrals. Es bleibt positiv Definitheit zu zeigen.

		(Es wurde nicht geschrieben, welche Definition des Integrals hier benutzt wurde. Ich benutze das Lebesgue-Integral, weil es einfacher ist).
		
		Es gilt
		\[
			\left<f,f \right> =\int_0^1 |f(x)|^2\dd{x}
		.\] 
		Weil $|f|\ge 0$, ist das Integral $\ge 0$. Es ist $0$ genau dann, wenn $|f(x)|^2$ $\lambda_1$-fast überall $0$ ist. Das Integral ist $0$ genau dann, wenn $|f(x)|^2$ $\lambda_1$-fast überall $0$ ist. Aber alle Polynome sind stetig. $|f|^2$ ist dann als Verkettung stetige Funktionen auch stetig und eine $\lambda_1$-fast überall 0 stetige Funktion ist überall $0$. Daraus folgt: $f=0$, das Nullpolynom. 
	\item 
		\begin{enumerate}[label=(\roman*)]
			\item Das erste Element ist $1$. Es gilt
				\[
					\left<1,1 \right> = \int_0^1 1\dd{x}=1
				.\] 
			\item $\langle x,1\rangle=\int_0^1 |x|\dd{x}=1 / 2$. Das zweite Element ist $x-1 / 2$. Es gilt
				\[
				\left<x -\frac{1}{2},x-\frac{1}{2} \right> = \frac{1}{12}
				.\] 
			\item 
				\begin{align*}
					\left<x^2,1 \right> =& \int_0^1 x^2 \dd{x}=1 / 3\\
					\left<x^2, x - \frac{1}{2} \right> =& 1 / 12
				\end{align*}
				Das zweite Element ist
				\[
				x^2-\frac{1}{3}-(x-\frac{1}{2})=x^2-x+\frac{1}{6}
				.\] 
				Es gilt
				\[
				\left<x^2-x+\frac{1}{6},x^2-x+\frac{1}{6} \right> = \frac{1}{180}
				.\] 
			\item
				\begin{align*}
					\left<x^3,1 \right> =& 1 / 4\\
					\left<x^3, x - \frac{1}{2} \right> =& \frac{3}{40}\\
					\left<x^3, x^2-x+\frac{1}{6} \right> =&\frac{1}{120}
				\end{align*}
				Dann ist das dritte Element
				\[
					x^3-\frac{1}{4}-\frac{9}{10}\left( x-\frac{1}{2} \right) -\frac{3}{2}\left( x^2-x+\frac{1}{6} \right)=x^3-\frac{3x^2}{2}+\frac{3x}{5}-\frac{1}{20} 
				.\] 
				Insgesamt ist die Basis
				\[
				\left\{ 1,x-\frac{1}{2},x^2-x+\frac{1}{6},x^3-\frac{3x^2}{2}+\frac{3x}{5}-\frac{1}{20} \right\}  
				.\qedhere\] 
		\end{enumerate}
	\end{parts}
\end{proof}
\begin{Problem}
	Sei $V$ ein euklidischer oder unitärer Vektorraum und $M,N\subseteq V$ Teilmengen von $V$.
	\begin{enumerate}[label=(\alph*)]
	\item Sei $N\subseteq M$. Zeigen Sie, dass dann $M^{\perp}\subseteq N^{\perp}$ gilt.
	\item Zeigen Sie $M\subseteq \left( M^\perp \right)^\perp$ und geben Sie ein Beispiel f\"{u}r einen euklidischen oder unitären Vektorraum $V$, sowie eine Teilmenge $M\subseteq V$ an, sodass $M\neq (M^\perp)^\perp$.
	\item Zeigen Sie, dass das Orthogonalkomplement von $M$ durch
		\[
			M^\perp=\bigcap_{v\in M} \text{ker}v^\flat
		\]
		gegeben ist, wobei $v^\flat = \langle v,\cdot\rangle\in V^*$ wie in Bemerkung 7.2iii.
	\item Zeigen Sie, dass $(M^\perp)^{\perp\perp}=(M^{\perp\perp})^\perp$ gilt.
	\item Zeigen Sie, dass $M^\perp+N^\perp\subset (M\cap N)^\perp$ gilt.
	\item Zeigen Sie, dass $(M+N)^\perp=M^\perp\cap N^\perp$ gilt, wenn $M$ und $N$ Unterräume von $V$ sind.
	\end{enumerate}
	Beweisen oder widerlegen Sie:
	\begin{enumerate}[label=(\alph*),resume]
		\item Es gilt $M^\perp\cup N^\perp\subset (M\cup N)^\perp$.
		\item Es gilt $(M\cup N)^\perp\subset M^\perp \cup N^\perp$.
	\end{enumerate}
\end{Problem}
\begin{proof}
\begin{parts}
\item Sei $v\in M^\perp$. Per Definition gilt $\langle v, w\rangle=0$ f\"{u}r alle $w\in M$. Dann gilt es auch f\"{u}r alle $w\in N$, weil $N\subseteq M$. Dann ist $M^\perp\subseteq N^\perp$.
\item Sei $v\in M^\perp,~w\in M$. Per Definition ist $\langle v,w\rangle$ f\"{u}r alle solche $w$. Daraus folgt: $v\in (M^\perp)^\perp$ und $M\subseteq (M^\perp)^\perp$.

	Gegenbeispiel: Sei $V=\C[x]$ mit innerem Produkt 
	 \[
		 \left<a_0+a_1x+\dots+a_n x_n, b_0+b_1x+\dots+b_mx_m \right> = \sum_{i=0}^{\text{max}(n,m)}a_ib_i
	 \]
	 und $M$ die Menge alle Polynome $f$, so dass f\"{u}r die Polynomefunktion $f(1)=1$ gilt. Dann ist $M^\perp=\{0\} $ und $(M^\perp)^\perp=V$. Aber $M\neq V$ offenbar gilt.
\item Die Definitionen sind gleich: Auf der linken Seite haben wir alle Vektoren $w\in V$, so dass $\langle v, w\rangle=0$ f\"{u}r alle $v\in M$.

	Per Definition ist $w\in \text{ker }v^\flat$ genau dann, wenn $\langle v, w\rangle=0$. Wenn $w$ in alle $\text{ker }v^\flat$ liegt, gilt das f\"{u}r alle $v$, was genau der Definition von $M^\perp$ ist.
\item 
	\[
		(M^\perp)^{\perp\perp}=((M^\perp)^\perp)^\perp=(M^{\perp\perp})^\perp
	.\] 
\item Sei
	\[
		v:=\underbrace{v_M}_{\in M^\perp}+\underbrace{V_N}_{\in N^\perp}\in M^\perp+N^\perp
	\] 
	und $w\in M\cap N$. Dann ist
	\begin{align*}
	\left<w,v \right> =& \cancelto{0}{\left<w, v_M \right>} & w\in M\\
			   &+\cancelto{0}{\left<w,v_N \right>} & w\in N\\
	=&0
	\end{align*}
	also $v\in (M\cap N)^\perp$. Daraus folgt: $M^\perp+N^\perp \subseteq (M\cap N)^\perp$.
\item Sei $v\in M^\perp\cap N^\perp$ und
	\[
		w:=\underbrace{w_M}_{\in M}+\underbrace{w_N}_{\in N}\in M+N
	\]
	Dann gilt
	\begin{align*}
		\left<v,w \right> =& \cancelto{0}{\left<v, w_M \right>} & v\in M^\perp\\
				   &+\cancelto{0}{\left<v,w_N \right>} & v\in N^\perp\\
		=&0
	\end{align*}
	also $(M+ N)^\perp \supseteq M^\perp \cap N^\perp$.

	Sei jetzt $v\in (M+ N)^\perp$ und $w\in M$. Da $M+N\supseteq M$ und $M+N\supseteq N$ gilt, ist
	\begin{align*}
		(M+N)^\perp \subseteq& M^\perp\\
		(M+N)^\perp \subseteq& N^\perp
	\end{align*}
	und daher
	\[
		(M+N)^\perp\subseteq M^\perp\cap N^\perp
	.\] 
\item Falsch. Sei $V=\R^3,~M=\text{span}((1,0,0)^T),~N=\text{span}((0,1,0)^T$. Es gilt
	\begin{align*}
		M^\perp=&\text{span}\left( \begin{pmatrix} 0 \\ 1 \\ 0 \end{pmatrix}, \begin{pmatrix} 0 \\ 0 \\ 1 \end{pmatrix}  \right)\\
		N^\perp=&\text{span}\left( \begin{pmatrix} 1 \\ 0 \\ 0 \end{pmatrix}, \begin{pmatrix} 0 \\ 0 \\ 1 \end{pmatrix}  \right) \\
		(M\cup N)^\perp =& \text{span}\left( \begin{pmatrix} 0 \\ 0 \\ 1 \end{pmatrix}  \right) 
	\end{align*}
\item Wahr. Es gilt $M\cup N\supseteq M$ und $M\cup N\supseteq N$. Daraus folgt:
	\begin{align*}
		(M\cup N)^\perp \subseteq& M^\perp\\
		(M\cup N)^\perp\subseteq& N^\perp
	\end{align*}
	Die Behauptung folgt.\qedhere
\end{parts}	
\end{proof}
