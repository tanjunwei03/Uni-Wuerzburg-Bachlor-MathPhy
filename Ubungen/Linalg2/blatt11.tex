\begin{Problem}
	Betrachten Sie die folgenden $3\times 3$-Matrizen
	\[
		A=\begin{pmatrix} 2 & 1 & 1 \\ 1 & 2 & 1 \\ 1 & 1 & 2 \end{pmatrix} ,\qquad B\begin{pmatrix} 1+i & 1 - i & 0 \\ -1 + i & 1 + i & 0 \\ 0 & 0 & 1+i \end{pmatrix} 
	.\] 
	\begin{parts}
		\item Berechnen Sie die adjungierten Abbildungen $A^*$ und $B^*$.
		\item Bestimmen Sie die Eigenwerte von $A$ und $B$.
		\item Überprüfen Sie, dass sich aus den Eigenvektoren der Matrix $A$ eine Orthonomalbasis des $\C^3$ mit Standardskalarprodukt bauen lässt. Führen Sie das selbe auch mit Matrix $B$ durch.
		\item Bestimmen Sie unitäre Matrizen $U,V\in M_3(\C^3)$, sodass $UAU^*$ und $VBV^*$ diagonal sind. Können Sie zudem erreichen, dass $U$ und $V$ orthogonal sind?
	\end{parts}
\end{Problem}
\begin{proof}
	\begin{parts}
	\item 
		\[
			A^*=A,\qquad B^*=\begin{pmatrix} 1 - i & -1-i & 0 \\ 1+i & 1-i & 0 \\ 0 & 0 & 1-i \end{pmatrix} 
		.\] 
	\item $A$:
		\begin{align*}
			\text{det}(A-\lambda I)=&
			\begin{vmatrix}
				2-\lambda & 1 & 1 \\ 1 & 2-\lambda & 1 \\ 1 & 1 & 2-\lambda
			\end{vmatrix}\\
			=&(2-\lambda)
			\begin{vmatrix}
				2-\lambda & 1 \\ 1 & 2-\lambda
			\end{vmatrix}
			-
			\begin{vmatrix}
				1 & 1 \\ 1 & 2-\lambda
			\end{vmatrix}
			+
			\begin{vmatrix}
				1 & 2-\lambda \\ 1 & 1
			\end{vmatrix}\\
			=&(2-\lambda)((2-\lambda)^2-1)-(1-\lambda)+(\lambda-1)\\
			=&(2-\lambda)^3-(2-\lambda)-2+2\lambda\\
			=&-\lambda^3+6\lambda^2-9\lambda+4
		\end{align*}
		Offensichtlich ist $\lambda=1$ ein Nullstelle des Polynoms. Es gilt
		\begin{align*}
			-\lambda^3+6\lambda^2-9\lambda+4=&(\lambda-1)(-\lambda^2+5\lambda-4)\\
			=&-(\lambda-1)^2(\lambda-4)
		\end{align*}
		also die Eigenwerte von $A$ sind $1$ und $4$.

		Ähnlich für $B$:
\begin{align*}
	\text{det}(B-\lambda I)=&
	\begin{vmatrix}
		1-i-\lambda & -1-i & 0 \\ 1+i & 1-i-\lambda & 0 \\ 0 & 0 & 1-i-\lambda
	\end{vmatrix}\\
	=&-\lambda^3+3(1-i)\lambda^2+4i\lambda\\
	=&-\lambda(\lambda-(1-i))(\lambda-(2-2i))
\end{align*}
also die Eigenwerte sind $0$, $1-i$ und $2-2i$.
\item Wir berechnen die Eigenvektoren:

	F\"{u}r $A$:
	\begin{align*}
		\text{EW}=4:&(1,1,1)^T\\
		\text{EW}=1:&(-1,0,1)^T\\
		\text{EW}=1:&(-1,1,0)^T
	\end{align*} 
	Die Eigenräume f\"{u}r die Eigenwerte $4$ und $1$ sind schon orthogonal. Im Eigenraum EW=1 können wir den Gram-Schmidt-Algorithismus durchführen, da lineare Kombinationen von Eigenvektoren mit gleichen Eigenwerte noch Eigenvektoren sind. Eine Orthonormalbasis ist dann
	\begin{align*}
		\text{EW}=4:&\frac{1}{\sqrt{3} }(1,1,1)^T\\
		\text{EW}=1:&\frac{1}{\sqrt{2} }(-1,0,1)^T\text{ und}\\
			    &\frac{1}{\sqrt{6} }(-1,2,-1)^T
	\end{align*}
	Die Eigenvektoren von $B$ sind
	\begin{align*}
		\text{EW}=0:&(i,1,0)^T\\
		\text{EW}=1-i:&(0,0,1)^T\\
		\text{EW}=2-2i:&(-i,1,0)^T
	\end{align*}
	Die Vektoren sind schon orthogonal, also wir normalisieren die Vektoren, um eine Orthonormalbasis zu bekommen.
	\begin{align*}
		\text{EW}=0:&\frac{1}{\sqrt{2} }(i,1,0)^T\\
		\text{EW}=1-i:&(0,0,1)^T\\
		\text{EW}=2-2i:&\frac{1}{\sqrt{2} }(-i,1,0)
	\end{align*}
\item Wir schreiben die Orthonomalbasis von Eigenvektoren von $A$ als die Zeilen der Matrix $U$. 
	\[
		U=\begin{pmatrix} \frac{1}{\sqrt{3} }& \frac{1}{\sqrt{3} } & \frac{1}{\sqrt{3} } \\ -\frac{1}{\sqrt{2} } & 0 & \frac{1}{\sqrt{2} }\\ -\frac{1}{\sqrt{6} } & \sqrt{\frac{2}{3}} & -\frac{1}{\sqrt{6} } \end{pmatrix} 
	.\] 
	Ähnlich f\"{u}r $V$:
	\[
		V=\begin{pmatrix} \frac{i}{\sqrt{2} } & \frac{1}{\sqrt{2} } & 0 \\ 0 & 0 & 1 \\ -\frac{i}{\sqrt{2} } & \frac{1}{\sqrt{2} } & 0 \end{pmatrix} 
	.\] 
	$U$ ist schon orthogonal. $V$ ist unitär und kann nicht orthogonal werden.\qedhere
	\end{parts}
\end{proof}
\begin{Problem}
	Betrachten Sie die reelle Matrix
	\[
		P=\frac{1}{6}\begin{pmatrix} 5 & 2 & -1 \\ 2 & 2 & 2 \\ -1 & 2 & 5 \end{pmatrix} 
	.\] 
	\begin{parts}
	\item Zeigen Sie, dass $P$ ein Orthogonalprojektor bezüglich des Standardskalarprodukts ist.
	\item Bestimmen Sie den Rang von $P$.
	\item Bestimmen Sie eine Orthonomalbasis des Bildes und des Kerns von $P$ sowie den Basiswechsel $O$ von der Standardbasis auf diese Basis.
	\item Verifizieren Sie, dass $O$ orthogonal ist, Können Sie einen solchen orthonomalen Basiswechsel finden, dass $\text{det }O=1$ gilt?
	\end{parts}
\end{Problem}
\begin{proof}
	\begin{parts}
	\item Schritt 1: $P$ ist ein Projektor. Durch direktes Rechnung:
		\[
			P^2=\frac{1}{36}\begin{pmatrix} 5 & 2 & -1 \\ 2 & 2 & 2 \\ -1 & 2 & 5 \end{pmatrix} \begin{pmatrix} 5 & 2 & -1 \\ 2 & 2 & 2 \\ -1 & 2 & 5 \end{pmatrix}=P 
		.\]
	Da $P^*=P$, ist $P$ ein Orthogonalprojektor nach Proposition 7.79.
\item 
	\begin{gather*}
	\left(
\begin{array}{ccc}
 \frac{5}{6} & \frac{1}{3} & -\frac{1}{6} \\
 \frac{1}{3} & \frac{1}{3} & \frac{1}{3} \\
 -\frac{1}{6} & \frac{1}{3} & \frac{5}{6} \\
\end{array}
\right) \xrightarrow{R_2\times 5} \left(
\begin{array}{ccc}
 \frac{5}{6} & \frac{1}{3} & -\frac{1}{6} \\
 \frac{5}{3} & \frac{5}{3} & \frac{5}{3} \\
 -\frac{1}{6} & \frac{1}{3} & \frac{5}{6} \\
\end{array}
\right) \xrightarrow{R_2-2R_1} \left(
\begin{array}{ccc}
 \frac{5}{6} & \frac{1}{3} & -\frac{1}{6} \\
 0 & 1 & 2 \\
 -\frac{1}{6} & \frac{1}{3} & \frac{5}{6} \\
\end{array}
\right) \\\xrightarrow{R_3\times 5} \left(
\begin{array}{ccc}
 \frac{5}{6} & \frac{1}{3} & -\frac{1}{6} \\
 0 & 1 & 2 \\
 -\frac{5}{6} & \frac{5}{3} & \frac{25}{6} \\
\end{array}
\right) \xrightarrow{R_3+R_1} \left(
\begin{array}{ccc}
 \frac{5}{6} & \frac{1}{3} & -\frac{1}{6} \\
 0 & 1 & 2 \\
 0 & 2 & 4 \\
\end{array}
\right) \xrightarrow{R_3-2R_2} \left(
\begin{array}{ccc}
 \frac{5}{6} & \frac{1}{3} & -\frac{1}{6} \\
 0 & 1 & 2 \\
 0 & 0 & 0 \\
\end{array}
\right)	
	\end{gather*}
	Der Rang ist $2$. 
\item Eine Basis des Bildes ist die erste 2 Spalten, also
	\[
	\left\{ \begin{pmatrix} 5 \\ 2 \\ -1 \end{pmatrix}, \begin{pmatrix} 2 \\ 2 \\ 2 \end{pmatrix}  \right\} 
	.\] 
	Wir ergänzen es bis eine Basis f\"{u}r den ganzen Vektorraum: \[
	\left\{ \begin{pmatrix} 5 \\ 2 \\ -1 \end{pmatrix}, \begin{pmatrix} 2 \\ 2 \\ 2 \end{pmatrix}, \begin{pmatrix} 1 \\ 0 \\ 0 \end{pmatrix}  \right\} 
	.\] 
	Daraus ergibt sich eine Orthonomalbasis durch den Gram-Schmidt-Verfahren:
	\[
	\left\{ \frac{1}{\sqrt{30} }\begin{pmatrix} 5 \\ 2 \\ -1 \end{pmatrix}, \frac{1}{\sqrt{5} }\begin{pmatrix}  0 \\ 1 \\ 2 \end{pmatrix}, \frac{1}{\sqrt{6} }\begin{pmatrix} 1 \\ -2 \\ 1 \end{pmatrix}    \right\} 
	.\] 
	Die ersten 2 Vektoren sind eine Orthonomalbasis des Bildes, der dritte eine Orthonomalbasis des Kerns. Der Basiswechselmatrix ist
	\[
	O=\left(
\begin{array}{ccc}
 \sqrt{\frac{5}{6}} & 0 & \frac{1}{\sqrt{6}} \\
 \sqrt{\frac{2}{15}} & \frac{1}{\sqrt{5}} & -\sqrt{\frac{2}{3}} \\
 -\frac{1}{\sqrt{30}} & \frac{2}{\sqrt{5}} & \frac{1}{\sqrt{6}} \\
\end{array}
\right) 
	.\] 
\item 
	\[
	A A^T=\left(
\begin{array}{ccc}
 \sqrt{\frac{5}{6}} & 0 & \frac{1}{\sqrt{6}} \\
 \sqrt{\frac{2}{15}} & \frac{1}{\sqrt{5}} & -\sqrt{\frac{2}{3}} \\
 -\frac{1}{\sqrt{30}} & \frac{2}{\sqrt{5}} & \frac{1}{\sqrt{6}} \\
\end{array}
\right) 
\left(
\begin{array}{ccc}
 \sqrt{\frac{5}{6}} & \sqrt{\frac{2}{15}} & -\frac{1}{\sqrt{30}} \\
 0 & \frac{1}{\sqrt{5}} & \frac{2}{\sqrt{5}} \\
 \frac{1}{\sqrt{6}} & -\sqrt{\frac{2}{3}} & \frac{1}{\sqrt{6}} \\
\end{array}
\right)=\text{diag}(1,1,1)
	.\] 
	Er hat schon Determinante $1$. Sonst könnten wir die erste zwei Spalten vertauschen.\qedhere
	\end{parts}
\end{proof}
\begin{Problem}
	Beweisen oder widerlegen Sie:
	\begin{parts}
	\item Jede Matrix $P\in M_n(\C)$ mit $P^2=P$ ist ein Orthogonalprojektor bezüglich eines \emph{geeigneten} Skalarprodukts auf $V$.
	\item Eine invertierbare Matrix $A\in M_n(\R)$ ist orthogonal f\"{u}r ein geeignet gewähltes Skalarprodukt auf $\R^n$.
	\end{parts}
\end{Problem}
\begin{proof}
	\begin{parts}
	\item Da $P$ ein Projektor ist, ist $\C^n=\text{ker }P\oplus \text{im }P$. Die Frage ist: Können wir uns für ein Skalarprodukt entscheiden, so dass  $\text{ker }P=(\text{im }P)^\perp$? 
		
		Wir entscheiden uns für eine Basis von $\text{ker }P$ bzw. von $\text{im }P$, was wir mit $\{b_1,b_2,\dots, b_n\} $ bezeichnen. Sei die Matrix $A^{-1}$ so definiert: Die Spalten sind die Basisvektoren von $\text{ker }P$ bzw. $\text{im }P$. Dann schickt $A$ die Basisvektoren nach der Standardbasis. Wir definieren dann $B=A^TA$. Wir definieren dann ein Skalarprodukt $\langle v_1, v_2\rangle = (v_1)^TBv_2$, was noch ein Skalarprodukt ist, da $B$ positiv definit ist. 

		Es gilt dann, f\"{u}r $b_i\neq b_j$:
		\begin{align*}
			\langle b_i, b_j\rangle=& (b_i)^TBb_j\\
			=&(b_i)^TA^TAb_j\\
			=&(Ab_i)^T(Ab_j)\\
			=&0
		\end{align*}
		wobei wir benutzt haben, dass die Standardbasis bzgl. des Standardskalarprodukts orthonomal ist. Daraus folgt, dass $\text{ker }P=(\text{im }P)^\perp$ bzgl. dieses Standardskalarprodukts, also nach Proposition 7.79 ist $P$ ein Orthogonalprojektor.
	\end{parts}
\end{proof}
\begin{Problem}
	Seien $V,W$ euklidische Vektorräume und $A$ eine lineare Abbildung $V\to W$. 
	\begin{parts}
	\item Zeigen Sie die Äquivalenz der folgenden drei Aussagen:
		\begin{enumerate}[label=\roman*.]
			\item F\"{u}r alle $v_1,v_2\in V$ gilt: $\left<v_1,v_2 \right>_V=0\implies \left<Av_1, Av_2 \right>_W=0$.
			\item F\"{u}r alle $v_1,v_2\in V$ gilt: $\|v_1\|_V=\|v_2\|_V\implies \|Av_1\|_W=\|Av_2\|_W$.
			\item Es existieren eine Konstante $\alpha>0$ und eine Isometrie $\Phi:V\to W$ mit $A=\alpha\Phi$.
		\end{enumerate}
	\item Zeigen Sie, dass es genau dann Konstanten $C_1,C_2>0$ gibt, sodass
		\[
		C_1\left<v_1,v_2 \right>_V\le \left<Av_1,Av_2 \right>_W\le C_2\left<v_1,v_2 \right>_V
	\]
	f\"{u}r alle $v_1,v_2\in V$ gilt, wenn $A=\alpha\Phi$ f\"{u}r eine Konstante $\alpha>0$ und eine Isometrie $\Phi:V\to W$ erf\"{u}llt ist. Bestimmen Sie die bestmöglichen Parameter $C_1$ und $C_2$.
	\end{parts}
\end{Problem}
\begin{proof}
	\begin{parts}
	\item Der Plan ist: (i)$\iff$(iii)$\iff$(ii).

		Angenommen (i). Sei $v_1,v_2\in V$, so dass $\langle v_1,v_2\rangle=k$. Sei $U=\text{span}v_1$, was ein Orthogonalkomplement hat. Also wir zerlegen $v_2=k_1v_1+v_2'$, wobei $v_2'$ senkrecht auf $v_1$ liegt. Daraus folgt:
		\begin{align*}
			\left<v_1,v_2 \right> =&\left<v_1,k_1 v_1+v_2' \right>\\
			=&k_1\left<v_1,v_1 \right>+\left<v_1,v_2' \right>\\
			=&k_1\left<v_1,v_1 \right>\\
			\left<Av_1,Av_2 \right> =& \left<Av_1,k_1 Av_1+Av_2' \right>\\
			=&k_1 \left<Av_1,Av_1 \right>+ \left<Av_1,Av_2' \right>\\
			=&k_1\left<Av_1,Av_1 \right> & \text{(i) angenommen}
		\end{align*}
	\end{parts}
\end{proof}
