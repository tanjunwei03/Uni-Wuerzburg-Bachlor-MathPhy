\begin{Problem}
	\begin{parts}
		\item  Berechnen Sie alle möglichen Matrixprodukte der folgenden Matrizen. Was muss jeweils für die Dimensionen erfüllt sein?
	 \begin{gather*}
		 A=\begin{pmatrix} 1 & -1 & 2 \\ 0 & 3 & 5 \\ 1 & 8 & 7 \end{pmatrix}, B=\begin{pmatrix} -1 & 0 & 1 & 0 \\ 0 & 1 & 0 & -1 \\ 1 & 0 & -1 & 0 \end{pmatrix} ,C=\begin{pmatrix} 1 \\ 0 \\ 8 \\ -7 \end{pmatrix}\\
		 D=\begin{pmatrix} -1 & 2 & 0 & 8 \end{pmatrix}, E=\begin{pmatrix} 1 & 4 \\ 0 & 5 \\ 6 & 8 \end{pmatrix} , F=\begin{pmatrix} -1 & 2 & 0 \end{pmatrix} .
	 \end{gather*}
 \item Eine Blockmatrix ist eine Matrix von der Form
	 \[
		 A=\begin{pmatrix} A_1 & A_3 \\ A_2 & A_4 \end{pmatrix} \] 
		 mit Matrizen $A_1\in \mathbb{K}^{n \times m}, A_2\in \mathbb{K}^{n'\times m}, A_3\in \mathbb{K}^{n \times m'}, A_4\in\mathbb{K}^{n'\times m'}$. Sei weiterhin
		 \[
			 B=\begin{pmatrix} B_1 & B_3 \\ B_2 & B_4 \end{pmatrix} \] 
			 mit ebenso Einträgen aus $\mathbb{K}$. Wer nun meint, die Multiplikation von A und B sei so simpel wie
			 \[
				 A\cdot B=\begin{pmatrix} A_1B_1+A_3B_2 & A_1B_3+A_3B_4 \\ A_2B_1+A_4B_2 & A_2B_3+A_4B_4 \end{pmatrix} \] 
hat tatsächlich recht. Beweisen Sie diese Formel und geben Sie gleichzeitig die $B_i'$s für die benötigten Matrizenräume an, sodass die Rechnung wohldefiniert ist.
 \end{parts}
\end{Problem}
\begin{proof}
	\begin{parts}
	\item F\"{u}r $A$ eine $n\times m$ Matrize, und $B$ eine $p\times q$ Matrize, ist $AB$ wohldefiniert, nur wenn $m=p$

		Die Matrizprodukte sind
		\begin{gather*}
			AB=\begin{pmatrix} 1 & -1 & -1 & 1\\5 & 3 & -5 & -3 \\ 6 & 8 & -6 & -8 \end{pmatrix}\\
			AE=\begin{pmatrix} 13 & 15 \\ 30 & 55 \\ 43 & 100 \end{pmatrix} \\
			FA=\begin{pmatrix} -1 & 7 & 8 \end{pmatrix}\\ 
			BC=\begin{pmatrix} 7 \\ -7 \\ -7 \end{pmatrix}\\
			CD=\begin{pmatrix} -1 & 2 & 0 & 8 \\ 0 & 0 & 0 & 0\\-8 & 16 & 0 & 64\\-7 & 14 & 0 & 56 \end{pmatrix}\\
			DC=(55)\\
			CF=\begin{pmatrix} -1 & 2 & 0 \\ 0 & 0 & 0 \\ -8 & 16 & 0\\-7 & 14 & 0 \end{pmatrix}\\ 
			FE=\begin{pmatrix} -1 & 6 \end{pmatrix} 
		\end{gather*}
	\item Wir brauchen $B_1\in\mathbb{K}^{m\times p},B_2\in\mathbb{K}^{m'\times p}, B_3\in\mathbb{K}^{m\times q}, B_4\in\mathbb{K}^{m'\times q}$ f\"{u}r $p,q\in \N$. Wir bezeichnen, f\"{u}r $v_1\in\mathbb{K}^p,v_2\in\mathbb{K}^q$, das Vektor $(v_1,v_2)\in\mathbb{K}^{p+q}$.
	\end{parts}
\end{proof}
\begin{Problem}
	Es seien $V$ und $W$ Vektorräume über K, nicht notwendigerweise endlich-dimensional und
	\[
	\Phi:V\to W
	\]
	eine lineare Abbildung. Beweisen Sie:
	\begin{parts}
	\item Die duale Abbildung $\Phi^*$ ist injektiv genau dann, wenn $\Phi$ surjektiv ist.
		{\footnotesize Hinweis: Die Richtung $\implies$ beweisen Sie am einfachsten als eine Kontraposition.}
	\item Die duale Abbildung $\Phi^*$ ist surjektiv genau dann, wenn $\Phi$ injektiv ist.
		{\footnotesize Hinweis: Die Rückrichtung lässt sich am einfachsten direkt beweisen. Nutzen Sie in dem Fall die Injektivität von $\Phi$ aus, um f\"{u}r ein beliebiges $v^*\in V^*$ eine lineare Abbildung im Bild von $\Phi^*$ zu konstruieren, die die gleichen Werde wie Abbildung $v^*$ liefert.}
	\item Im Falle der Invertierbarkeit gilt
		\[
			\left( \Phi^{-1} \right) ^*=\left( \Phi^* \right) ^{-1}
		.\] 
	\end{parts}
\end{Problem}
\begin{proof}
	\begin{parts}
	\item Sei $\Phi$ surjektiv, und $w_1^*,w_2^*\in W^*$. Es gilt $\Phi w_1^*=w_1^*\circ \Phi, \Phi w_2^*=w_2^*\circ \Phi$. Die zwei Abbildungen  $w_1^*\circ \Phi$ und $w_2^*\circ \Phi$ sind unterschiedliche, solange es mindestens ein $v\in V$ gibt, sodass $(w_1^*\circ\Phi)(v)\neq (w_2^*\circ\Phi)(v)$. Wir haben aber ausgenommen, dass $w_1^*\neq w_2^*$. Das bedeutet, dass es $w\in W$ gibt, so dass $w_1^*(w)\neq w_2^*(w)$. Weil $\Phi$ surjektiv ist, ist $w=\Phi(v)$ f\"{u}r eine $v$. Dann ist $(w_1^*\circ\Phi)(v)\neq (w_2^*\circ\Phi)(v)$, also  $\Phi^*$ ist injektiv.

		Jetzt nehmen wir an, dass $\Phi$ nicht surjektiv ist. Wir definieren zwei lineare Funktionale $w_1^*$ und $w_2^*$, sodass $w_1^*\neq w_2^*$. Sei $w_1^*(w)=w_2^*(w)\forall w\in\text{im}(\Phi)$ 
	\item Zuerst beweisen wir: $\Phi$ nicht injektiv $\implies$ $\Phi^*$ nicht surjektiv.
		Sei $v_1,v_2\in V,v_1\neq v_2$ und $\Phi(v_1)=\Phi(v_2)=w$.Es gibt eine lineare Abbildung $v^*\in V^*$, so dass $v^*(v_1)\neq v^*(v_2)$. Sei aber $w^*\in W^*$. Es gilt $\left( \Phi^* w^* \right)(v) =(w^*\circ\Phi)(v)$. Dann ist
		\[
		\Phi^*w^*(v_1)=\Phi^*(w)=\Phi^*w^*(v_2)
		,\] 
		also $\Phi^*(w^*)\neq v^*$ f\"{u}r alle $w^*\in W^*$. Es folgt: $\Phi^*$ ist nicht surjektiv.

		Jetzt beweisen wir $\Phi$ injektiv $\implies$ $\Phi^*$ surjektiv. Sei $v^*\in V^*$. Wir definieren eine Abbildung (momentan nicht unbedingt linear) so: F\"{u}r alle  $w\in \text{im}(\Phi)$, also  $w=\Phi(v)$, ist $w^*(w)=v^*(v)$. F\"{u}r  $w\not\in \text{im}(\Phi)$ ist $w^*(w)=0$. 

		Es ist klar, dass $w^*\cdot \Phi=v^*$. Wir müssen nur zeigen, dass $w^*$ linear ist, also $w^*\in W^*$.
		 \begin{enumerate}[label=(\arabic*)]
			 \item Sei $w\in W$, $a\in \mathbb{K}$. Falls $w \not\in \text{im}(\Phi)$, ist auch $aw\not\in \text{im}(\Phi)$. Es gilt daher
				 \[
				 w^*(aw)=aw^*(w)=0
				 .\] 
				 Falls $w\in \text{im}(\Phi)$, also $w=\Phi v$ f\"{u}r ein $v\in V$, gilt auch $aw=\Phi(av)$, und
				 \[
				 w^*(aw)=v^*(av)=av^*(v)=aw^*(w)
				 .\] 
		\end{enumerate}
		Daraus folgt: $w^*\in W^*$, und $\Phi^*(w^*)=v^*$.
	\item In den letzten Teilaufgaben haben wir bewiesen, dass wenn $\Phi$ bijektiv ist, ist $\Phi^*$ auch bijektiv. Die Rückrichtung stimmt auch. Wir müssen nur Gleichheit zeigen.

		\begin{tcolorbox}
			\textbf{Vereinfachung:} Wir müssen nur zeigen, per Definition eine Inverseabbildung, dass
			\[
				\Phi^*\circ \left( \Phi^{-1} \right) ^*=\text{id}_{V^*}
			.\] 
		\end{tcolorbox}
		Es gilt, f\"{u}r $v^*\in V^*$, $\left( \Phi^{-1} \right)^*(v^*)=v^*\circ\Phi^{-1}$. Daraus folgt
		\begin{align*}
			\left(\Phi^*\circ\left( \Phi^{-1} \right)^*\right)(v^*)=&\Phi^*\left( v^*\circ \Phi^{-1} \right) \\
			=& v^*\circ\Phi^{-1}\circ\Phi\\
			=& v^*\qedhere
		\end{align*}
	\end{parts}
\end{proof}
