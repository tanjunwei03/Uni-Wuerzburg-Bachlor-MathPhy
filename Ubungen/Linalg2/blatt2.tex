\begin{Problem}
	Es seien die Punkte $x_0, x_1, \dots, x_n$ mit $x_i \in \R$ gegeben. Wir definieren den Operator
	\[
		\Phi:\R_{\le n}[x]\to \R^{n+1}, p\to y, \text{ mit }p(x_i)=y_i, i=0,\dots,n
	\] 
	wobei wir mit $\R_{\le n}[x]$ den Raum der Polynome mit reellen Koeffizienten vom Grad höchsten $n$ bezeichnen und $p(x)$ die Auswertung des Polynoms $p$ im Punkt $x$ beschreibt.
	\begin{parts}
		\item  Zeigen Sie: Sind die Punkte $x_i$ paarweise verschieden, so ist die Abbildung $\Phi$ wohldefiniert und isomorph. (Eine Konsequenz hieraus ist die eindeutige Lösbarkeit der Polynominterpolation.)
		\item Was passiert, wenn Sie nicht fordern, dass die $x_i$ paarweise verschieden sind? Kann $\Phi$ im Allgemeinen noch injektiv (surjektiv) sein?
	\end{parts}
\end{Problem}
\begin{proof}
	\begin{parts}
\item Injektiv: Nehme an, dass es zwei unterschiedliche Polynome $p_1$, $p_2$ gibt, mit $p_1(x_i)=p_2(x_i)\forall i=0,\dots,n$. Dann ist $p(x):=p_1(x)-p_2(x)$ auch ein Polynom, mit $p(x_i):=0\forall i\in \{0,\dots,n\}$. Weil  $ \deg(p)\le n$ist, folgt daraus, dass $\forall x,p(x)=0, p_1(x)=p_2(x)$. Das ist ein Widerspruch.

	Surjektive: Sei $(y_0,\dots,y_n)\in \R^{n+1}$. Dann ist
	\[
	p(x)=(x-y_0)(x-y_1)\dots(x-y_n)
	\]
	auch ein Polynom mit $\Phi(p)=(y_0,\dots,y_n)$.

	Linearität: Sei $p_1(x),p_2(x)\in \R_{\le n}[x], a\in \R$. Sei auch $p(x)=p_1(x)+p_2(x)$. Es gilt dann
	\[
	p(x_i)=p_1(x_i)+p_2(x_i),i=0,\dots,n\] und daher
	\[
	\Phi(p)=\Phi(p_1+p_2)=\Phi(p_1)+\Phi(p_2)
	.\] 
	Es gilt auch, f\"{u}r $p(x):=ap_1(x)$, dass
	\[
	p(x_i)=ap_1(x_i), i=0,\dots,n
	,\]
	und daher
	\[
	\Phi(p)=\Phi(ap_1)=a\Phi(p_1)
	.\] 
\item Nein. Sei, zum Beispiel, $n=1$, $x_0=x_1=0$. Dann gilt
	\begin{align*}
		\Phi(x)=&(0,0)^T\\
		\Phi(x^2)=&(0,0)^T
	\end{align*}
	Aber die zwei Polynome sind ungleich.
	\end{parts}
\end{proof}

\begin{Problem}
	\begin{parts}	
	\item Es sei eine Matrix $A \in \mathbb{K}^{n\times n}$ gegeben. Wir bilden die erweiterte Matrix
	\[
		B=(A|1_n)\]
		mit $1_n$ die Einheitsmatrix in $\R^n$. Zeigen Sie: $A$ ist genau dann invertierbar, wenn $A$ durch elementare Zeilenumformung in die Einheitsmatrix überführt werden kann. Verfizieren Sie weiterhin: Werden die dafür benötigten Zeilenumformungen auf ganz $B$ angewendet, so ergibt sich im hinteren Teil, wo zu Beginn die Einheitsmatrix stand, genau $A^{-1}$.
	\item Es sei nun
		\[
			A=\begin{pmatrix} 1 & 0 & 0 & 1\\0 & -1 & 2 & 0 \\ 0 & 0 & 0 & -2\\3 & 0 & 1 & 2 \end{pmatrix} 
		.\] 
		Bestimmen Sie $A^{-1}$.
	\end{parts}
\end{Problem}
\begin{proof}
	\begin{parts}
	\item Definiert $(x,y),x\in \mathbb{K}^n,y\in\mathbb{K}^m$ durch $\mathbb{K}^{n+m}\ni(x,y)=(x_1,\dots,x_n,y_1,\dots,y_n)$. Eine solche erweiterte Matrix bedeutet eine Gleichungssystem durch
		\[
		B(x, -y)=Ax-1_ny=0
		,\]
		wobei $x,y\in \mathbb{K}^n$. F\"{u}r jeder $x\in\mathbb{K}^n$ gibt es $y\in \mathbb{K}^n,$ so dass $B(x,-y)=0$. Nehme an, dass wir durch elementare Zeilenumformung
		\[
		B=(A|1_n)\to (1_n, A'):=B'
		\]
		kann. Die Gleichungssystem ist dann $x=A'y$. Dadurch können wir f\"{u}r jeder  $y\in\mathbb{K}^n$ eine $A'y=x\in\mathbb{K}^n$ rechnen, f\"{u}r die gilt, dass $Ax=y$. Das heißt, dass $A'=A^{-1}$. 
	\item
		{\allowdisplaybreaks
		\begin{gather*}
			\left(
\begin{array}{cccc|cccc}
 1 & 0 & 0 & 1 & 1 & 0 & 0 & 0 \\
 0 & -1 & 2 & 0 & 0 & 1 & 0 & 0 \\
 0 & 0 & 0 & -2 & 0 & 0 & 1 & 0 \\
 3 & 0 & 1 & 2 & 0 & 0 & 0 & 1 \\
\end{array}
\right) \xrightarrow{R_4-3R_1} \left(
\begin{array}{cccc|cccc}
 1 & 0 & 0 & 1 & 1 & 0 & 0 & 0 \\
 0 & -1 & 2 & 0 & 0 & 1 & 0 & 0 \\
 0 & 0 & 0 & -2 & 0 & 0 & 1 & 0 \\
 0 & 0 & 1 & -1 & -3 & 0 & 0 & 1 \\
\end{array}
\right) \xrightarrow{R_2\times -1}\\ \left(
\begin{array}{cccc|cccc}
 1 & 0 & 0 & 1 & 1 & 0 & 0 & 0 \\
 0 & 1 & -2 & 0 & 0 & -1 & 0 & 0 \\
 0 & 0 & 0 & -2 & 0 & 0 & 1 & 0 \\
 0 & 0 & 1 & -1 & -3 & 0 & 0 & 1 \\
\end{array}
\right) \xrightarrow{R_3\leftrightarrow R_4} \left(
\begin{array}{cccc|cccc}
 1 & 0 & 0 & 1 & 1 & 0 & 0 & 0 \\
 0 & 1 & -2 & 0 & 0 & -1 & 0 & 0 \\
 0 & 0 & 1 & -1 & -3 & 0 & 0 & 1 \\
 0 & 0 & 0 & -2 & 0 & 0 & 1 & 0 \\
\end{array}
\right) \xrightarrow{R_2+2R_3}\\ \left(
\begin{array}{cccc|cccc}
 1 & 0 & 0 & 1 & 1 & 0 & 0 & 0 \\
 0 & 1 & 0 & -2 & -6 & -1 & 0 & 2 \\
 0 & 0 & 1 & -1 & -3 & 0 & 0 & 1 \\
 0 & 0 & 0 & -2 & 0 & 0 & 1 & 0 \\
\end{array}
\right) \xrightarrow{R_2-R_4} \left(
\begin{array}{cccc|cccc}
 1 & 0 & 0 & 1 & 1 & 0 & 0 & 0 \\
 0 & 1 & 0 & 0 & -6 & -1 & -1 & 2 \\
 0 & 0 & 1 & -1 & -3 & 0 & 0 & 1 \\
 0 & 0 & 0 & -2 & 0 & 0 & 1 & 0 \\
\end{array}
\right) \xrightarrow{R_4\times -\frac{1}{2}}\\ \left(
\begin{array}{cccc|cccc}
 1 & 0 & 0 & 1 & 1 & 0 & 0 & 0 \\
 0 & 1 & 0 & 0 & -6 & -1 & -1 & 2 \\
 0 & 0 & 1 & -1 & -3 & 0 & 0 & 1 \\
 0 & 0 & 0 & 1 & 0 & 0 & -\frac{1}{2} & 0 \\
\end{array}
\right) \xrightarrow{R_1-R_4} \left(
\begin{array}{cccc|cccc}
 1 & 0 & 0 & 0 & 1 & 0 & \frac{1}{2} & 0 \\
 0 & 1 & 0 & 0 & -6 & -1 & -1 & 2 \\
 0 & 0 & 1 & -1 & -3 & 0 & 0 & 1 \\
 0 & 0 & 0 & 1 & 0 & 0 & -\frac{1}{2} & 0 \\
\end{array}
\right) \xrightarrow{R_3+R_4} \\\left(
\begin{array}{cccc|cccc}
 1 & 0 & 0 & 0 & 1 & 0 & \frac{1}{2} & 0 \\
 0 & 1 & 0 & 0 & -6 & -1 & -1 & 2 \\
 0 & 0 & 1 & 0 & -3 & 0 & -\frac{1}{2} & 1 \\
 0 & 0 & 0 & 1 & 0 & 0 & -\frac{1}{2} & 0 \\
\end{array}
\right)
		\end{gather*}
	}
	\end{parts}
\end{proof}
\begin{Problem}
	Es seien die Vektorräume $V, W$ über $\mathbb{K}$ gegeben mit $\dim(V) = n$ und $\dim(W ) = m$. Wir betrachten eine lineare Abbildung
	\[
	T:V\to W, v\to T(v)\] 
Seien $B_V$ und $B_W$ Basen von $V$, bzw. $W$. Wir nehmen an $T$ ist nicht die konstante Nullabbildung. Beweisen Sie:
\begin{parts}
\item Der Kern von $_{B_W}[T]_{B_V}$ ist entweder trivial (d.h. nur die 0) oder hängt nur von der Wahl von $B_V$ ab, aber nicht von $B_W$.
\item Das Bild von $_{B_W}[T]_{B_V}$ ist entweder der ganze $\mathbb{K}^m$ oder hängt nur von der Wahl von $B_W$ ab, aber nicht von $B_v$. 
\item Der Rang von $_{B_W}[T]_{B_V}$ ist unabh\"{a}ngig von $B_W$ und $B_V$.
\end{parts}
\end{Problem}
\begin{proof}
Nach Korollar 5.43 gilt, f\"{u}r $A,A' \subseteq V$ und $B,B'\subseteq W$ Basen der Vektorräume $V$ und $W$ über $\mathbb{K}$, und $\Phi\in \text{Hom}(V,W)$.
 \[
	 _{B'}\left[ \Phi \right]_{A'}={}_{B'}[\text{id}_W]_B\cdot{}_B[\Phi]_A\cdot{}_A[\text{id}_V]_{A'}
.\] 
\begin{Lemma}
	Jeder Basiswechsel f\"{u}r sowohl $B_V$ als auch $B_W$ kann als zwei Basiswechseln interpretiert werden, wobei eine Basiswechsel nur $B_V$ verändert, und die andere nur $B_W$.
\end{Lemma}
\begin{proof}
	\[
		_{B'}\left[ \Phi \right]_{A'}={}_{B'}[\text{id}_W]_B\cdot{}_B[\Phi]_A\cdot{}_A[\text{id}_V]_{A'}={}_{B'}[\text{id}_W]_B\left( {}_B[\text{id}_W]_B\cdot {}_B[\Phi]_A\cdot {}_A[\text{id}_V]_{A'} \right) {}_A[\text{id}_V]_A
	.\]
	(In den Klammern gibt es zuerst ein Basiswechsel in $V$, dann ein Basiswechsel in $W$). Ein ähnliche Argument zeigt, dass wir zuerst ein Basiswechseln in $W$ betrachten kann.
\end{proof}
\begin{Corollary}
	In die Aufgabe muss man nur das Fall betrachten, in dem entweder $B_V$ oder $B_W$ sich verändert. 
\end{Corollary}
	\begin{parts}
	\item Nehme an, $\ker({}_{B_W}[T]_{B_V})\neq 0$. Die zwei Fälle
		\begin{enumerate}[label=(\roman*)]
			\item Nur $B_W$ sich verändert.

				Sei $v\in\mathbb{K}^n$, $_B[\Phi]_Av=0$. Es gilt
				\[
					_{B'}[\Phi]_A={}_{B'}[\text{id}_W]{}_B[\Phi]_A{}_A[\text{id}_V]_Av={}_{B'}[\text{id}_W]_B{}_B[\Phi]_Av={}_{B'}[\text{id}_W]_B(0)=0
				.\] 

				Sei jetzt $_B[\Phi]_Av\neq 0$. Solange wir zeigen, dass
				\[
					_{B'}[\text{id}_W]_B u\neq 0
				\]
				f\"{u}r $\mathbb{K}^m\ni u\neq 0$, sind wir fertig. Aber $_{B'}[\text{id}_W]_Bu=0$, nur wenn $u=0v_1+0v_2+\dots+0v_n, v_i\in B'=0$ wegen der linear Unabhängigkeit.

			\item Nur $B_V$ sich verändert. Es stimmt leider nicht, dass $\text{ker}({}_{B_W}[T]_{B_V})$ von $B_V$ abhängig sein \emph{muss}. Sei zum Beispiel $B_K$ ein Basis f\"{u}r $\text{ker}\left( {}_{B_W}[T]_{B_V} \right) $, und $B_V$ und $B_V'$ Basen von $V$, f\"{u}r die gilt $B_K\subset B_V, B_K\subset B_V'$. Jetzt ist der Kern einen invarianten Unterraum von $B$ unter $_{B_V'}[T]_{B_V}$, also der Kern verändert sich nicht, wenn der Basis sich verändert. 

				Wenn der Kern kein invarianter Unterraum ist, gilt es natürlich, das der Kern sich durch das Basiswechsel verändert.
		\end{enumerate}
	\item Nehme an, dass $\text{im}\left( {}_{B_W}[T]_{B_V} \right) \neq \mathbb{K}^m$. Wir betrachten noch einmal die zwei Fälle
		\begin{enumerate}[label=(\roman*)]
			\item Nur $B_V$ sich verändert. Weil $_{B_V}[\text{id}]_{B_V'}:V\to V$ bijektiv ist, gilt
	\begin{align*}
		\text{im}\left( {}_{B_W}[T]_{B_V'} \right) =&\left\{ {}_{B_W}[T]_{B_V'}v|v \in\mathbb{K}^m \right\}=\left\{ {}_{B_W}[T]_{B_V}{}_{B_V}[\text{id}]_{B_V'}v|v\in \mathbb{K}^m \right\} \\=&\left\{ {}_{B_W}[T]_{B_V}v|v\in\mathbb{K}^m \right\}=\text{im}\left( {}_{B_W}[T]_{B_V'} \right)  
				\end{align*}
			\item $B_W$ sich verändert. Jetzt gilt
				\[
					_{B_W'}[T]_{B_V}={}_{B_W'}[\text{id}]_{B_W}{}_{B_W}[T]_{B_V}
				.\] 
				Leider ist es noch falsch, dass das Bild von $B_W$ abhängig sein \emph{muss} wegen eines ähnliches Arguments zu das Kern.
		\end{enumerate}
	\item Weil das Bild von $B_V$ unabhängig ist, ist der Rang auch von $B_V$ unabhängig.

		Weil $_{B_W'}[\text{id}]_{B_W}$ bijektiv als Abbildung $\mathbb{K}^m\to \mathbb{K}^m$ ist, ist es auch bijekive f\"{u}r alle Teilmengen $U\subseteq \mathbb{K}^m$. Das Bild vor und nach dem Basiswechsel sind dann isomorph. Deswegen ist der Rang von $B_W$ unabhängig.
	\end{parts}
\end{proof}
\begin{Problem}
	Es wird gerechnet.
	\begin{parts}
	\item Wir definieren die lineare Abbildung T (x) = A · x mit A gegeben wie in 2(b). Wir definieren die Basen
		\[
		B_1:=\left\{ \begin{pmatrix} 1 \\ 0 \\ 0 \\ 0 \end{pmatrix} , \begin{pmatrix} 0 \\ -1 \\ 1 \\ -1 \end{pmatrix} , \begin{pmatrix} 0 \\ 0 \\ 0 \\ 1 \end{pmatrix} , \begin{pmatrix} 0 \\ -2 \\ 1 \\ 0 \end{pmatrix}  \right\},\qquad B_2:=\left\{ \begin{pmatrix} 1 \\ 1 \\ 0 \\ 0 \end{pmatrix} , \begin{pmatrix} 0 \\ 1 \\ 1 \\ 0 \end{pmatrix} , \begin{pmatrix} 0 \\ 0 \\ 1 \\ 1 \end{pmatrix} , \begin{pmatrix} 0 \\ 0 \\ 0 \\1 \end{pmatrix}  \right\} 
		.\] 
		Berechnen Sie
		\[
			_{B_2}[T]_{B_1}
		.\] 
	\item Wir schauen nochmal auf Aufgabe 1. Es seien die paarweise verschiedene Punkte $x_0, x_1 , \dots, x_n$ gegeben und die Abbildung $\Phi$ wie zuvor. Gegeben sei die kanonische Basis
		\[
		B:=\left\{ e_1,e_2,\dots,e_n \right\} 
		\]
		vom $\R^n$ sowie die Basen
		\[
		B_M:=\left\{ 1,x,x^2,\dots, x^n \right\} 
		\]
		und
		\[
			B_l:=\left\{ l_0(x),l_1(x),l_2(x),\dots,l_n(x) \right\} ,\qquad \text{mit}\qquad l_i(x):=\prod_{i\neq j = 0}^{n} \frac{x-x_j}{x_i-x_j} 
		.\] 
		Bestimmen Sie
		\[
			_B[\Phi]_{B_M},\qquad\text{ und }\qquad {}_B[\Phi]_{B_l}
		.\] 
Ausgehend von den entstandenen Matrizen: Stellen Sie eine Vermutung, welche Basis für große $n$ bevorzugt wird.
	\end{parts}
\end{Problem}
\begin{proof}
	Wir berechnen
	\[
		_{B_2}[\text{id}]_{B_1}
	.\] 
	Es gilt
	\begin{align*}
		\begin{pmatrix} 1 \\ 0 \\ 0 \\ 0 \end{pmatrix} =&\begin{pmatrix} 1 \\ 1 \\ 0 \\ 0 \end{pmatrix} -\begin{pmatrix} 0 \\ 1 \\ 1 \\ 0 \end{pmatrix} +\begin{pmatrix} 0 \\ 0 \\ 1 \\ 1 \end{pmatrix} -\begin{pmatrix} 0 \\ 0 \\ 0 \\1  \end{pmatrix} \\
		\begin{pmatrix}  0 \\ -1 \\ 1 \\ -1 \end{pmatrix} =& -\begin{pmatrix} 0 \\ 1 \\ 1 \\ 0 \end{pmatrix} +2\begin{pmatrix}  0 \\ 0 \\ 1 \\ 1 \end{pmatrix} -3\begin{pmatrix} 0 \\ 0 \\ 0 \\ 1 \end{pmatrix}\\ 
		\begin{pmatrix} 0 \\ 0 \\ 0 \\ 1 \end{pmatrix} =& \begin{pmatrix} 0 \\ 0 \\ 0 \\ 1 \end{pmatrix} \\
		\begin{pmatrix} 0 \\ -2 \\ 1 \\ 0 \end{pmatrix} =& -2\begin{pmatrix} 0 \\ 1 \\ 1 \\ 0 \end{pmatrix} + 3\begin{pmatrix} 0 \\ 0 \\ 1 \\ 1 \end{pmatrix} -3\begin{pmatrix} 0 \\ 0 \\ 0 \\ 1 \end{pmatrix} 
	\end{align*}
	Daraus folgt:
\[
	_{B_2}[\text{id}]_{B_1}=\begin{pmatrix} 1 & 0 & 0 & 0\\
		-1 & -1 & 0 & -2\\
		1 & 2 & 0 & 3\\
		-1 & -3 & 1 & -3
	\end{pmatrix} 
.\]
Wir berechnen $\left\{ {}_{B_2}[\text{id}]_{B_1} \right\}^{-1}$ 
{\allowdisplaybreaks
\begin{gather*}
	\left(
\begin{array}{cccc|cccc}
 1 & 0 & 0 & 0 & 1 & 0 & 0 & 0 \\
 -1 & -1 & 0 & -2 & 0 & 1 & 0 & 0 \\
 1 & 2 & 0 & 3 & 0 & 0 & 1 & 0 \\
 -1 & -3 & 1 & -3 & 0 & 0 & 0 & 1 \\
\end{array}
\right) \xrightarrow{R_2+R_1} \left(
\begin{array}{cccc|cccc}
 1 & 0 & 0 & 0 & 1 & 0 & 0 & 0 \\
 0 & -1 & 0 & -2 & 1 & 1 & 0 & 0 \\
 1 & 2 & 0 & 3 & 0 & 0 & 1 & 0 \\
 -1 & -3 & 1 & -3 & 0 & 0 & 0 & 1 \\
\end{array}
\right) \xrightarrow{R_4+R_1} \\\left(
\begin{array}{cccc|cccc}
 1 & 0 & 0 & 0 & 1 & 0 & 0 & 0 \\
 0 & -1 & 0 & -2 & 1 & 1 & 0 & 0 \\
 1 & 2 & 0 & 3 & 0 & 0 & 1 & 0 \\
 0 & -3 & 1 & -3 & 1 & 0 & 0 & 1 \\
\end{array}
\right) \xrightarrow{R_3-R_1} \left(
\begin{array}{cccc|cccc}
 1 & 0 & 0 & 0 & 1 & 0 & 0 & 0 \\
 0 & -1 & 0 & -2 & 1 & 1 & 0 & 0 \\
 0 & 2 & 0 & 3 & -1 & 0 & 1 & 0 \\
 0 & -3 & 1 & -3 & 1 & 0 & 0 & 1 \\
\end{array}
\right) \xrightarrow{R_2\times -1} \\\left(
\begin{array}{cccc|cccc}
 1 & 0 & 0 & 0 & 1 & 0 & 0 & 0 \\
 0 & 1 & 0 & 2 & -1 & -1 & 0 & 0 \\
 0 & 2 & 0 & 3 & -1 & 0 & 1 & 0 \\
 0 & -3 & 1 & -3 & 1 & 0 & 0 & 1 \\
\end{array}
\right) \xrightarrow{R_3-2R_2} \left(
\begin{array}{cccc|cccc}
 1 & 0 & 0 & 0 & 1 & 0 & 0 & 0 \\
 0 & 1 & 0 & 2 & -1 & -1 & 0 & 0 \\
 0 & 0 & 0 & -1 & 1 & 2 & 1 & 0 \\
 0 & -3 & 1 & -3 & 1 & 0 & 0 & 1 \\
\end{array}
\right) \xrightarrow{R_4+3R_2} \\\left(
\begin{array}{cccc|cccc}
 1 & 0 & 0 & 0 & 1 & 0 & 0 & 0 \\
 0 & 1 & 0 & 2 & -1 & -1 & 0 & 0 \\
 0 & 0 & 0 & -1 & 1 & 2 & 1 & 0 \\
 0 & 0 & 1 & 3 & -2 & -3 & 0 & 1 \\
\end{array}
\right) \xrightarrow{R_4\leftrightarrow R_3} \left(
\begin{array}{cccc|cccc}
 1 & 0 & 0 & 0 & 1 & 0 & 0 & 0 \\
 0 & 1 & 0 & 2 & -1 & -1 & 0 & 0 \\
 0 & 0 & 1 & 3 & -2 & -3 & 0 & 1 \\
 0 & 0 & 0 & -1 & 1 & 2 & 1 & 0 \\
\end{array}
\right) \xrightarrow{R_3+3R_4} \\\left(
\begin{array}{cccc|cccc}
 1 & 0 & 0 & 0 & 1 & 0 & 0 & 0 \\
 0 & 1 & 0 & 2 & -1 & -1 & 0 & 0 \\
 0 & 0 & 1 & 0 & 1 & 3 & 3 & 1 \\
 0 & 0 & 0 & -1 & 1 & 2 & 1 & 0 \\
\end{array}
\right) \xrightarrow{R_2+2R_4} \left(
\begin{array}{cccc|cccc}
 1 & 0 & 0 & 0 & 1 & 0 & 0 & 0 \\
 0 & 1 & 0 & 0 & 1 & 3 & 2 & 0 \\
 0 & 0 & 1 & 0 & 1 & 3 & 3 & 1 \\
 0 & 0 & 0 & -1 & 1 & 2 & 1 & 0 \\
\end{array}
\right) \xrightarrow{R_4\times -1} \\\left(
\begin{array}{cccc|cccc}
 1 & 0 & 0 & 0 & 1 & 0 & 0 & 0 \\
 0 & 1 & 0 & 0 & 1 & 3 & 2 & 0 \\
 0 & 0 & 1 & 0 & 1 & 3 & 3 & 1 \\
 0 & 0 & 0 & 1 & -1 & -2 & -1 & 0 \\
\end{array}
\right)
\end{gather*}
}
also
\[
	_{B_1}[\text{id}]_{B_2}=\left\{_{B_2}[\text{id}]_{B_1}\right\}^{-1}=\begin{pmatrix} 1 & 0 & 0 & 0\\ 1 & 3 & 2 & 0\\1 & 3 & 3 & 1\\-1 & -2 & -1 & 0 \end{pmatrix} 
.\] 
Es gilt dann
\begin{align*}
	_{B_1}[\text{id}]_{B_2}A{}_{B_2}[\text{id}]_{B_1}=&\begin{pmatrix} 1 & 0 & 0 & 0\\-1 & -1 & 0 & -2 \\ 1 & 2 & 0 & 3\\-1 & -3 & 1 & 3 \end{pmatrix} \begin{pmatrix} 1 & 0 & 0 & 1 \\ 0 & -1 & 2 & 0 \\ 0 & 0 & 0 & -2\\3 & 0 & 1 & 2 \end{pmatrix}\begin{pmatrix}  1 & 0 & 0 & 0 \\ 1 & 3 & 2 & 0 \\ 1 & 3 & 3 & 1 \\ -1 & -2 & -1 & 0 \end{pmatrix}\\
	=& \left(
\begin{array}{cccc}
 0 & -2 & -1 & 0 \\
 -5 & 1 & -5 & -4 \\
 8 & 1 & 10 & 7 \\
 -7 & 0 & -12 & -9 \\
\end{array}
\right)
\end{align*}
\item Sei $p_k(x)=x^k\in B_M$. Es folgt, dass $\Phi\left( p_k(x) \right) =\left\{ x_1^k,x_2^k,\dots, x_n^k \right\} $. Deswegen gilt
	\[
		_B[\Phi]_{B_M}=\begin{pmatrix} 1 & x_1 & x_1^2 & \dots & x_1^n \\
			1 & x_2 & x_2^2 & \dots  & x_2^n\\
			1 & x_3 & x_3^2 & \dots & x_3^n\\
			\vdots & \vdots &\vdots & \ddots & \vdots\\
			1 & x_n & x_n^2 & \dots & x_n^n
		\end{pmatrix} 
	.\] 
	Betrachten Sie dann $l_k(x)$. Weil $(x-x_i)$ f\"{u}r $i\neq k$ vorkommt, gilt $l_k(x_i)=0\forall i \neq k$. F\"{u}r $i=k$ gilt $l_k(x_k)=\prod_{i\neq j = 0}^{n} \frac{x_k-x_j}{x_k-x_j}=1$. Es gilt daher
	\[
		_B[\Phi]_{B_l}=I_n=\text{diag}_n(1,1,\dots,1)
	.\]
	Ich vermute, dass $B_l$ f\"{u}r große $n$ bevorzugt wird\ldots
\end{proof}
