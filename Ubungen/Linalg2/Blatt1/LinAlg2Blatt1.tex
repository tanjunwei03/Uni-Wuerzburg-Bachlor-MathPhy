\documentclass[prb,12pt]{revtex4-2}

\usepackage{amsmath, amssymb,physics,amsfonts,amsthm}
\usepackage{enumitem}
\usepackage{cancel}
\usepackage{booktabs}
\usepackage{tikz}
\usepackage{hyperref}
\usepackage{enumitem}
\usepackage{polynom}
\usepackage{transparent}
\usepackage{float}
\usepackage{multirow}
\newtheorem{Theorem}{Theorem}
\newtheorem{Proposition}{Theorem}
\newtheorem{Lemma}[Theorem]{Lemma}
\newtheorem{Corollary}[Theorem]{Corollary}
\newtheorem{Example}[Theorem]{Example}
\newtheorem{Remark}[Theorem]{Remark}
\theoremstyle{definition}
\newtheorem{Problem}{Problem}
\theoremstyle{definition}
\newtheorem{Definition}[Theorem]{Definition}
\newenvironment{parts}{\begin{enumerate}[label=(\alph*)]}{\end{enumerate}}
%tikz
\usetikzlibrary{patterns}
% definitions of number sets
\newcommand{\N}{\mathbb{N}}
\newcommand{\R}{\mathbb{R}}
\newcommand{\Z}{\mathbb{Z}}
\newcommand{\Q}{\mathbb{Q}}
\newcommand{\C}{\mathbb{C}}
\begin{document}
	\title{Lineare Algebra 2 Hausaufgabenblatt Nr. 1}
	\author{Jun Wei Tan}
	\email{jun-wei.tan@stud-mail.uni-wuerzburg.de}
	\affiliation{Julius-Maximilians-Universit\"{a}t W\"{u}rzburg}
	\date{\today}
	\maketitle

\begin{Problem}
	\begin{parts}	
		\item Bestimmen Sie alle komplexwertigen Lösungen der Gleichung
	\[
	x^2=u+iv,\] 
	in Abh\"{a}ngigkeit von $u,v\in \R$

\item F\"{u}hren Sie das Nullstellenproblem
	\[
		ax^2+bx+c=0,\]
		mit $a\in\C\backslash 0, b\in \C, c\in \C$ auf den Fall in (a) zur\"{u}ck. Geben Sie weiterhin eine geschlossene Darstelling aller L\"{o}sungen f\"{u}r den Fall $a=1$ an.

		Hat alles geklappt, sollte bei Ihnen speziell für den Fall $a = 1$ und $\Im(b) = \Im(c) = 0$ die entsprechende Mitternachtsformel dastehen.
		\end{parts}
\end{Problem}
\begin{proof}
	\begin{parts}
	\item $\left| x^2 \right| =|x|^2=|u+iv|=\sqrt{u^2+v^2}$ 

		Daraus folgt:
		\[
			|x|=(u^2+v^2)^{1 / 4}
		,\]
		\[
			x=(u^2+v^2)^{1 / 4}e^{i\theta} 
		.\] 
		Setze es in $x^2=u+iv$ ein und löse die Gleichungen für $\theta$. Sei $\varphi=\atan_2(u,v)$ Dann ist:
		 \[
			 \theta=\frac{\varphi}{2}\text{ oder }\theta=\frac{\varphi+2\pi}{2}
		 .\]

	\item \[
			ax^2+bx+c=0\implies x^2+\frac{b}{a}x+\frac{c}{a}=0
	,\]
	d.h.
	\begin{align*}
		x^2+\frac{b}{a}x+\frac{c}{a}=& \left( x+\frac{b}{2a} \right)^2+\frac{c}{a}-\frac{b^2}{4a^2}=0\\
		\left( x+\frac{b}{2a} \right)^2=&\frac{b^2}{4a^2}-\frac{c}{a}\\
		x=&-\frac{b}{2a}\pm p
	\end{align*}
	wobei $p$ die L\"{o}sung zu $p^2=\frac{b^2}{4a^2}-\frac{c}{a}$ ist. Im Fall $a=1$ und $\Im(b)=\Im(c)=0$, daraus folgt: 
	\[
	x=-\frac{b}{2}\pm \frac{1}{2}\sqrt{b^2-4c} 
	.\] 
	\end{parts}
\end{proof}
\begin{Problem}
	Finden Sie f\"{u}r die Polynome $p,d\in \C[x]$ jeweils solche $q,r\in \C[x]$ mit $\deg(r)<\deg(d)$, dass $p=qd+r$ gilt.
	\begin{parts}
	\item $p=x^7+x^5+x^3+1, d=x^2+x+1$
	\item $p=x^5+(3-i)x^3-x^2+(1-3i)x+1+i,d=x^2+i$
	\item  Wie sehen $s$, $r$ aus, wenn man in (a) und (b) jeweils die Rollen von $p$ und $d$ vertauscht? D.h. bestimmen Sie $s,r\in \C[x]$ mit $\deg r < \deg p$, sodass $d=sp+r$ gilt.
	\end{parts}
\end{Problem}
\begin{proof}
	\begin{parts}
	\item 
		\polylongdiv[style=A]{x^7+x^5+x^3+1}{x^2+x+1}

		Daher
		\[
		q=x^5-x^4+x^3, r=1
		.\]
	\item $q=x^3+(3-2i)x-x, r=-(1+6i)x+(1+2i)$
	\item $r=d, s=0$
	\end{parts}
\end{proof}

\begin{Problem}
	Seien
	\[A=\begin{pmatrix} 1 & 2 & 1 & 5 \\ 2 & 1 & -1 & 4 \\ 1 & 0 & -1 & 1 \end{pmatrix}, b=\begin{pmatrix} -38 \\ -46 \\ -18 \end{pmatrix} , c=\begin{pmatrix} 0 \\ 0 \\ 1 \end{pmatrix}  \]
	gegeben.
	\begin{parts}
	\item Bestimmen Sie $\Im(A)$ und $\ker(A)$ 
	\item Bestimmen Sie L\"{o}s$(A, b)$ und L\"{o}s$(A, c)$.
	\end{parts}
\end{Problem}
\begin{proof}
	\begin{parts}
	\item 	\begin{align*}
		\left(
		\begin{array}{cccc}
			1 & 2 & 1 & 5 \\
			2 & 1 & -1 & 4 \\
			1 & 0 & -1 & 1 \\
		\end{array}
		\right) \xrightarrow{R_2-2R_1} \left(
		\begin{array}{cccc}
			1 & 2 & 1 & 5 \\
			0 & -3 & -3 & -6 \\
			1 & 0 & -1 & 1 \\
		\end{array}
		\right) \xrightarrow{R_3-R_1} \left(
		\begin{array}{cccc}
			1 & 2 & 1 & 5 \\
			0 & -3 & -3 & -6 \\
			0 & -2 & -2 & -4 \\
		\end{array}
		\right) \xrightarrow{R_2\times -\frac{1}{3}} \\\left(
		\begin{array}{cccc}
			1 & 2 & 1 & 5 \\
			0 & 1 & 1 & 2 \\
			0 & -2 & -2 & -4 \\
		\end{array}
		\right) \xrightarrow{R_3+2R_2} \left(
		\begin{array}{cccc}
			1 & 2 & 1 & 5 \\
			0 & 1 & 1 & 2 \\
			0 & 0 & 0 & 0 \\
		\end{array}
		\right) \xrightarrow{R_1-2R_2} \left(
		\begin{array}{cccc}
			1 & 0 & -1 & 1 \\
			0 & 1 & 1 & 2 \\
			0 & 0 & 0 & 0 \\
		\end{array}
		\right)
	\end{align*}
	Daraus folgt
	\[
	\text{im}(A)=\text{span}\left\{ \begin{pmatrix} 1 \\ 2 \\ 1 \end{pmatrix} , \begin{pmatrix} 2 \\ 1 \\ 0 \end{pmatrix}  \right\} 
	.\]
	Sei dann $(x_1,x_2,x_3,x_4)^T\in \R^4$. Wenn $(x_1,x_2,x_3,x_4)^T\in \ker(A)$, gilt
	\begin{align*}
		t_3&:= x_3\\
		t_4&:= x_4\\
		x_1&=x_3-x_4=t_3-t_4\\
		x_2&=-x_3-2x_4=-t_3-2t_4
	\end{align*}
	Daraus folgt:
	\begin{align*}
	\ker(A)=\text{span}\left\{ \begin{pmatrix} 1 \\ -1 \\ 1 \\ 0 \end{pmatrix} ,\begin{pmatrix} -1 \\ -2 \\ 0 \\ 1 \end{pmatrix}  \right\} 
	\end{align*}	
\item 
	\begin{gather*}
\left(
\begin{array}{cccc|c}
 1 & 2 & 1 & 5 & -38 \\
 2 & 1 & -1 & 4 & -46 \\
 1 & 0 & -1 & 1 & -18 \\
\end{array}
\right) \xrightarrow{R_2-2R_1} \left(
\begin{array}{cccc|c}
 1 & 2 & 1 & 5 & -38 \\
 0 & -3 & -3 & -6 & 30 \\
 1 & 0 & -1 & 1 & -18 \\
\end{array}
\right) \xrightarrow{R_3-R_1}\\\left(
\begin{array}{cccc|c}
 1 & 2 & 1 & 5 & -38 \\
 0 & -3 & -3 & -6 & 30 \\
 0 & -2 & -2 & -4 & 20 \\
\end{array}
\right) \xrightarrow{R_2\times -\frac{1}{3}} \left(
\begin{array}{cccc|c}
 1 & 2 & 1 & 5 & -38 \\
 0 & 1 & 1 & 2 & -10 \\
 0 & -2 & -2 & -4 & 20 \\
\end{array}
\right) \xrightarrow{R_3+2R_2} \left(
\begin{array}{cccc|c}
 1 & 2 & 1 & 5 & -38 \\
 0 & 1 & 1 & 2 & -10 \\
 0 & 0 & 0 & 0 & 0 \\
\end{array}
\right)\\ \xrightarrow{R_1-2R_2} \left(
\begin{array}{cccc|c}
 1 & 0 & -1 & 1 & -18 \\
 0 & 1 & 1 & 2 & -10 \\
 0 & 0 & 0 & 0 & 0 \\
\end{array}
\right)
	\end{gather*}
	\begin{align*}
		x_1-x_3+x_4=&-18\\
		x_2+x_3+2x_4=&-10
	\end{align*}	
	Deswegen ist $\text{L\"{o}s}(A,b)$ 
	\begin{align*}
		\begin{pmatrix} -18+x_3-x_4\\-10-x_3-2x_4 \\ x_3 \\ x_4 \end{pmatrix} =& \begin{pmatrix} -18 \\ -10 \\ 0 \\ 0 \end{pmatrix} +x_3\begin{pmatrix} 1 \\ -1 \\ 1 \\ 0 \end{pmatrix} +x_4 \begin{pmatrix} -1 \\ -2 \\ 0 \\ 1 \end{pmatrix} 
	\end{align*}
\item 
	\begin{align*}
\left(
\begin{array}{cccc|c}
 1 & 2 & 1 & 5 & 0 \\
 2 & 1 & -1 & 4 & 0 \\
 1 & 0 & -1 & 1 & 1 \\
\end{array}
\right) \xrightarrow{R_2-2R_1} \left(
\begin{array}{cccc|c}
 1 & 2 & 1 & 5 & 0 \\
 0 & -3 & -3 & -6 & 0 \\
 1 & 0 & -1 & 1 & 1 \\
\end{array}
\right) \xrightarrow{R_3-R_1} \left(
\begin{array}{cccc|c}
 1 & 2 & 1 & 5 & 0 \\
 0 & -3 & -3 & -6 & 0 \\
 0 & -2 & -2 & -4 & 1 \\
\end{array}
\right) \\\xrightarrow{R_2\times -\frac{1}{3}} \left(
\begin{array}{cccc|c}
 1 & 2 & 1 & 5 & 0 \\
 0 & 1 & 1 & 2 & 0 \\
 0 & -2 & -2 & -4 & 1 \\
\end{array}
\right) \xrightarrow{R_3+2R_2} \left(
\begin{array}{cccc|c}
 1 & 2 & 1 & 5 & 0 \\
 0 & 1 & 1 & 2 & 0 \\
 0 & 0 & 0 & 0 & 1 \\
\end{array}
\right) \xrightarrow{R_1-2R_2} \left(
\begin{array}{cccc|c}
 1 & 0 & -1 & 1 & 0 \\
 0 & 1 & 1 & 2 & 0 \\
 0 & 0 & 0 & 0 & 1 \\
\end{array}
\right)
	\end{align*}
	Es gibt keine L\"{o}sungen, weil $0\neq 1$, also  $\text{L\"{o}s}(A, c)=\varnothing$
\end{parts}
\end{proof}
\begin{Problem}
	Gegeben seien die $\R$-Vektorr\"{a}ume $V$ mit Basis $B_V=\left\{ v_1,v_2,v_3 \right\} $ und Basis $B_W=\left\{ w_1,w_2,w_3 \right\} $. Wir definieren einen linearen Operator $T: V \to W$ wie folgt:
	\[
	T(v_1)=w_1+w_3\qquad T(v_2)=w_1+w_2, T(v_3)=-w_1-w_2-w_3
	.\] 
	\begin{parts}
	\item $w_1,w_2,w_3\in \text{span}\left\{ T(v_1),T(v_2),T(v_3) \right\} $, weil
			\begin{align*}
				w_1=&T(v_1)+T(v_2)+T(v_3)\\
				w_2=&(-1)\left( T(v_3)+T(v_1) \right) \\
				w_3=&(-1)\left( T(v_2)+T(v_3) \right) 
			\end{align*}
			Daraus folgt:
			\[
				W=\text{span}\left( w_1,w_2,w_3 \right)=\text{span}\left( T(v_1), T(v_2), T(v_3) \right)  
			.\]
			Daraus folgt:
			\[
				\text{im}(T)=\R^3, \qquad \text{ker}(T)=\left\{ 0 \right\} 
			.\] 
		\item 
\[
	\begin{pmatrix} 1 & 1 & -1\\ 0 & 1 & -1\\1 & 0 & -1 \end{pmatrix} 
.\] 
\item \[
B_W^*=\left\{ w_1+w_3,w_1+w_2,-w_1-w_2-w_3 \right\} 
.\] 
\item \[
B_V^*=\left\{ v_1+v_2+v_3, -(v_1+v_3), -(v_2+v_3) \right\} 
.\] 
		\end{parts}
\end{Problem}
	\end{document}
