\begin{Problem}
	Betrachten Sie die komplexen $3\times 3$-Matrizen
	\[
		A_1=\begin{pmatrix} 5 & 2 & -1 \\ 2 & 2 & 2 \\ -1 & 2 & 5 \end{pmatrix},~A_2=\begin{pmatrix} 1 & 3 & -i \\ 0 & 3 & 4 \\ 0 & 0 & 2 \end{pmatrix},~A_3=\begin{pmatrix} 2 & 0 & 0 \\ 0 & 5 & 2 \\ 0 & 2 & 1 \end{pmatrix} 
	.\] 
	Welche der Matrzen sind positiv, welche sogar positiv definit?
\end{Problem}
\begin{proof}
	Wir berechnen das Spektrum von $A_1$. Es gilt f\"{u}r das charakteristiches Polynom
	\[
		P(\lambda)=\text{det}(A_1-\lambda I)=-\lambda^3+12\lambda^2-36\lambda
	.\] 
	Die Nullstellen bzw. Eigenwerte sind $\lambda=0$ und $\lambda=6$. Dann ist $A_1$ positiv. Weil $\lambda=0$ ein Eigenwert ist, ist $\text{det}(A_1)=0$ und $A_1$ ist nicht invertierbar. 

	$A_2$ ist nicht positiv, weil $A_2\neq A_2^*$. 

	Wir berechnen noch einmal das Spektrum von $A_3$. Es gilt f\"{u}r das charakteristische Polynom.
\[
	P_3(\lambda)=\text{det}(A-\lambda I)=-x^3+8x^2-13x+2
.\] 
Die Nullstellen sind $x=2$ und $x=3\pm 2\sqrt{2} $, also $A_3$ ist positiv. Da $0$ kein Nullstelle ist, ist $\text{det}(A_3)\neq 0$ und $A_3$ ist invertierbar, also $A$ ist positiv definit.
\end{proof}
\begin{Problem}
	Betrachten Sie den unitären Vektorraum $\C^n$ mit dem Standardskalarprodukt.
	\begin{parts}
		\item Sei $A\in M_n(\C)$ selbstadjungiert und $A=U^{-1}DU$, wobei $U$ eine invertierbare Matrix und $D$ eine Diagonalmatrix sind. Sei $P_i=U^{-1}M_iU$ mit Diagonalmatrix $M_i$, sodass
		\[
			(M_i)_{kk}=\begin{cases}
				1 & D_{kk}=\lambda_i\\
				0 & \text{sonst.}
			\end{cases}
		\]
		gilt. Zeigen Sie, dass $P_i$ eine Orthogonalprojektion auf den Eigenraum von $\lambda_i$ ist.
		\item Bestimmen Sie den Positivteil $(A_i)_+$, den Negativteil $(A_i)_{-}$ und den Absolutbetrag $|A_i|$ f\"{u}r $i=1,2$ der folgenden Matrizen
			\[
				A_1=\begin{pmatrix} 1 & -1 & -1 \\ -1 & 1 & -1 \\ -1 & -1 & 1 \end{pmatrix},~A_2=\begin{pmatrix} 0 & 3 & 0 & 0 \\ 3 & 0 & 0 & 0 \\ 0 & 0 & 0 & -2i \\ 0 & 0 & 2i & 0 \end{pmatrix} 
			.\] 
	\end{parts}
\end{Problem}
\begin{proof}
	\begin{parts}
	\item $P_i^2=U^{-1}M_i U U^{-1}M_i U=U^{-1} M_i^2 U=U^{-1}M_iU=P_i$. Sei $v$ ein Eigenvektor mit Eigenwert $\lambda$. D.h. $Uv=e_i$ f\"{u}r eine geeignete Basisvektor $e_i$. Dann ist $(M_j)_{kk} Uv = 0$, wenn $j$ einen anderen Eigenwert entspricht. Dann ist $\text{im }P_i$ der Eigenraum mit Eigenwert $\lambda_i$.

		Da $A$ selbstadjugiert und daher normal ist, sind die Eigenräume orthogonal, und $P_i$ ist ein Orthogonalprojektor.
	\item Wir berechnen die Eigenvektoren und Eigenwerte von den Matrizen. F\"{u}r $A_1$ sind die Eigenwerte $2,2$ und $-1$. Die Eigenvektoren sind $(-1,0,1)$ und $(-1,2,-1)$ bzgl. des Eigenwerts $2$ und $(1,1,1)$ bzg. des Eigenwerts $-1$. 

		Dann diagonalisieren wir $A_1$:
		\begin{align*}
			U=&\begin{pmatrix} -1 & -1 & 1 \\ 0 & 2 & 1 \\ 1 & -1 & 1 \end{pmatrix}\\
			A_1=&U\text{diag}(2,2,-1)U^{-1}
		\end{align*}
		Dann ist das Positivteil $U\text{diag}(2,2,0)U^{-1}$, oder
		 \[
			 (A_1)_+=\frac{1}{3}\begin{pmatrix} 4 & -2 & -2 \\ -2 & 4 & -2 \\ -2 & -2 & 4 \end{pmatrix} 
		.\] 
		und
		\[
			(A_1)_{-}=\frac{1}{3}\begin{pmatrix} -1 & -1 & -1 \\ -1 & -1 & -1 \\ -1 & -1 & -1 \end{pmatrix} 
		.\] 
	Der Betrag ist
	\[
		|A_1|=\begin{pmatrix} 5 & -1 & -1 \\ -1 & 5 & -1 \\ -1 & -1 & 5 \end{pmatrix} 
	.\] 
		Ähnlich sind die Eigenwerte von $B$ $\pm 3$ und $\pm 2$. Die Eigenvektoren sind
		\begin{align*}
			\text{EW}=-3:& (-1,1,0,0)^T\\
			\text{EW}=3:&(1,1,0,0)^T\\
			\text{EW}=-2:&(0,0,i,1)\\
			\text{EW}=2:& (0,0,-i,1)
		\end{align*}
		Dann definieren wir
		 \[
			 U_2=\begin{pmatrix} -1 & 1 & 0 & 0 \\ 1 & 1 & 0 & 0 \\ 0 & 0 & i & -i \\ 0 & 0 & 1 & 1 \end{pmatrix} 
		.\] 
		Es gilt $A_2=U_2\text{diag}(-3,3,-2,2)U_2^{-1}$. Das Positivteil ist durch $A_2=U_2\text{diag}(0,3,0,2)$ definiert und
		\[
			(A_2)_+=\begin{pmatrix} 3 / 2 & 3 / 2 & 0 & 0 \\ 3 / 2 & 3 / 2 & 0 & 0 \\ 0 & 0 & 1 & -i \\ 0 & 0 & i & 1 \end{pmatrix} 
		.\] 
		Das Negativteil ist ähnlich
		\[
			(A_2)_{-}=\begin{pmatrix} - 3 / 2 & 3 / 2 & 0 & 0 \\ 3 / 2 & - 3 / 2 & 0 & 0 \\ 0 & 0 & -1 & -i \\ 0 & 0 & i & -1 \end{pmatrix} 
		.\] 
		Der Betrag ist
		\[
			|A_2|=\text{diag}(3,3,2,2)
		.\qedhere\] 
	\end{parts}
\end{proof}
\begin{Problem}
	Beweisen oder widerlegen Sie:
	\begin{parts}
	\item Eine obere Dreiecksmatrix ist nie orthogonal.
	\item Sei $V$ ein unitärer Vektorraum. Ein Endomorphismus $A$ ist genau dann normal, wenn $\|Av\|=\|A^*v\|$ f\"{u}r alle $v\in V$ gilt.  
	\end{parts}
\end{Problem}

\begin{proof}
	\begin{parts}
	\item Falsch. Die Identität $\text{diag}(1,1,\dots, 1)$ ist eine obere Dreiecksmatrix und jedoch orthogonal.
	\end{parts}
\end{proof}

\begin{Problem}
	Sei $V$ ein endlich-dimensionaler euklidischer oder unitärer Vektorraum. Sei weiter $\text{End}_{sa}(V)\subset \text{End}(V)$ die Teilmenge der selbstadjungierten Endomorphismen auf $V$. F\"{u}r $A,B\in \text{End}_{sa}(V)$ definieren wir $A\le B$, falls $B-A$ ein positiver Endomorphismus ist.
	\begin{parts}
	\item Zeigen Sie, dass $\text{End}_{sa}(V)$ ein reeller Unterraum von $\text{End}(V)$ ist.
	\item Zeigen Sie, dass f\"{u}r $\lambda,\mu\ge 0$ und $A,B,C,D\in \text{End}_{sa}(V)$ mit $A\le B$ und $C\le D$ folgt, dass
		\[
		\lambda A+\mu C\le \lambda B+\mu D
	\]
	gilt.
\item Zeigen Sie, dass f\"{u}r alle $A\le B$ 
	\[
	CAC^*\le CBC^*
\]
f\"{u}r alle $C\in \text{End}(V)$ gilt.
\item Zeigen Sie, dass f\"{u}r $A\ge 0$ und $\lambda>0$ der Endomorphismus $A+\lambda$ invertierbar ist.
\item Betrachten Sie $V=\C^2$ mit Standardskalarprodukt und
	\[
		A=\begin{pmatrix} 1 & 0 \\ 0 & 0 \end{pmatrix} ,~B=\begin{pmatrix} 2 & 1 \\ 1 & 1 \end{pmatrix} 
	.\]
	Zeigen Sie, dass $0\le A \le B$ gilt. Zeigen Sie, dass $A^2\le B^2$ \emph{nicht} gilt.
	\end{parts}
\end{Problem}
\begin{proof}
	\begin{parts}
	\item Linearität von die adjungierte Endomorphismus:
		\begin{align*}
			(\lambda_1A+\lambda_2 B)^*=&(\lambda_1 A)^*+(\lambda_2 B)^*\\
			=&\lambda_1^* A^*+\lambda_2^* B^*\\
			=&\lambda_1 A^*+\lambda_2 B^* & \lambda\text{ reell}\\
			=&\lambda_1 A+\lambda_2 B & A,B\text{ selbstadjugiert}
		\end{align*}
	\item Es gilt
		\begin{align*}
			&\lambda B+\mu D-(\lambda A +\mu C)\\
			=&\lambda(B-A)+\mu(D-C)
		\end{align*}
		Da sowohl $B-A$ als auch $D-C$ positiv sind, ist die lineare Kombination auch positiv. Die Behauptung folgt.
	\item Es gilt $CBC^*-CAC^*=C(B-A)C^*$. Sei $v\in V$. Es gibt dann $w\in W$, so dass $\langle v, C(B-A)C^* v\rangle= \langle w, (B-A)w\rangle$. Dies ist definiert genau durch $w=C^*v$. Da $B-A$ positiv ist, ist das innere Produkt auch immer positiv. 
	\item Ein Endomorphismus ist genau dann invertierbar, wenn $0$ kein Eigenwert ist. Da $A\ge 0$, besitzt $A$ nichtnegative Eigenwerte. Wir zeigen: Sei $\lambda_1\ge 0$ ist genau dann Eigenwert von $A$, wenn $\lambda_1+\lambda$ Eigenwert von $A+\lambda$ ist.
		
		Sei zunächst $\lambda_1$ Eigenwert von $A$. Es gilt
		\begin{align*}
			\text{det}(A+\lambda-(\lambda_1+\lambda))=&\text{det}(A-\lambda_1)\\
			=& 0
		\end{align*}
		Dann ist $\lambda+\lambda_1$ Eigenwert von $A+\lambda$. Sei umgekehrt $\lambda_1+\lambda$ Eigenwert von $A+\lambda$. Es gilt dann
		\begin{align*}
			\text{det}(A-\lambda_1)=&\text{det}(A-\lambda_1+\lambda-\lambda)\\
			=&\text{det}(A+\lambda-(\lambda_1+\lambda))\\
			=&0
		\end{align*}
		Dann ist $\lambda_1$ Eigenwert von $A$.

		Weil $\lambda>0$, sind die Eigenwerte alle strikt positiv. Dann ist $0$ kein Eigenwert, und $A +\lambda$ ist invertierbar.
\item 
	\[
		B-A=\begin{pmatrix} 1 & 1 \\ 1 & 1 \end{pmatrix} 
	.\] 
	Die Eigenwerte sind $2$ und $0$, also $B-A$ ist positiv. Es gilt
	\[
		B^2=\begin{pmatrix} 5 & 3 \\ 3 & 2 \end{pmatrix} 
	\]
	und
	\[
		A^2=\begin{pmatrix} 1 & 0 \\ 0 & 0 \end{pmatrix} 
	.\] 
	Dann ist
	\[
		B^2-A^2=\begin{pmatrix} 4 & 3 \\ 3 & 2 \end{pmatrix} 
	.\] 
	Die Eigenwerte von $B^2-A^2$ sind $3\pm \sqrt{10} $. Aber $3-\sqrt{10} <0$, also $B^2-A^2$ ist nicht positiv.
	\end{parts}
\end{proof}
