\begin{Problem}
	Seien $(X,\mathcal{A},\mu),(Y,\mathcal{B},\nu)$ und $(Z,\mathcal{C},\eta)$ $\sigma$-endliche Maßraume. Definiere $A\otimes B\otimes C:=\mathcal{A}_{\sigma}(\mathcal{A}\boxtimes\mathcal{B}\boxtimes\mathcal{C})$. Zeigen Sie:
	\begin{parts}
		\item Es gilt $(\mathcal{A}\otimes \mathcal{B})\otimes \mathcal{C}=\mathcal{A}\otimes \mathcal{B}\otimes \mathcal{C}=\mathcal{C}\otimes (\mathcal{B}\otimes \mathcal{C})$.
			
			{\footnotesize \emph{Hinweis: Betrachten Sie} $\Pi_{*}(\mathcal{A}\otimes \mathcal{B}\otimes \mathcal{C})$ \emph{mit}
				\[
				\Pi:X\times Y\times Z\to X\times Y,~\Pi(x,y,z):=(x,y)
		.\]} 
			\item Es gilt $(\mu\otimes \nu)\otimes \eta=\mu\otimes(\nu\otimes\eta)$ auf $\mathcal{A}\otimes \mathcal{B}\otimes \mathcal{C}$.
	\end{parts}
\end{Problem}
\begin{proof}
	\begin{parts}
	\item Per Hinweis betrachten wir $\Pi_*(\mathcal{A}\otimes \mathcal{B}\otimes \mathcal{C})$. Sei $P\in \mathcal{A}\otimes \mathcal{B}$. $P\in \Pi_*(\mathcal{A}\otimes \mathcal{B}\otimes \mathcal{C})$ genau dann, wenn $P\times Z\in \mathcal{A}\otimes \mathcal{B}\otimes \mathcal{C}$.
	\end{parts}
\end{proof}
\begin{Problem}\label{pr:advanalblatt8-2}
	Sei $x_0\in \R^2,a<b\in \R,r:[a,b]\to [0,\infty)$ eine stetige Funktion und
	\[
		A:=\left\{ (x,y,z)\in \R^3|x^2+y^2<r(z)^2,z\in [a,b] \right\} 
	.\] 
	\begin{parts}
	\item Bestimmen Sie $\lambda_2(B_R(x_0))$ f\"{u}r $R>0$ mittels dem Satz von Fubini.
	\item Zeigen Sie:
		\[
			\lambda_3(A)=\pi\int_a^b r(z)^2\dd{z}
		.\] 
	\end{parts}
\end{Problem}
\begin{proof}
	\begin{parts}
	\item Aus Bewegungsinvarianz des Lebesgue-Maßes genügt es, $B_R(0)$ zu betrachten. Per Definition ist es definiert durch
		\[
		x^2+y^2<R^2
		.\]
		Insbesondere ist, f\"{u}r jedes $x\in [-1,1]$, 
			\[
				\left\{ y|(x,y)\in B_R(0) \right\} =[-\sqrt{R^2-x^2} ,\sqrt{R^2-x^2}] 
			.\] 
			Aus Satz 2.81 folgt
			\begin{align*}
				\lambda_2(B_R(0))=&\int_{-R}^R \lambda_1([-\sqrt{R^2-x^2} ,\sqrt{R^2-x^2} ])\dd{x}\\
				=&\int_{-R}^R 2\sqrt{R^2-x^2} \dd{x}
			\end{align*}
			Die Funktion ist Riemann-integrierbar, also wir dürfen Sätze vom Riemann-Integral verwenden. Hier integrieren wir es per Substitution. Sei $x=R\sin\theta,\dd{x}=R\cos\theta\dd{\theta}$. Wenn $x=-R$, ist $\theta=-\pi / 2$ und wenn $x=R$ ist $\theta=\pi / 2$. 
			\begin{align*}
				\lambda_2(B_R(0))=&\int_{-\pi / 2}^{\pi / 2} 2R\sqrt{1-\sin^2\theta} R\cos\theta\dd{\theta}\\
				=&\int_{-\pi / 2}^{\pi / 2}2R^2\cos^2\theta\dd{\theta}\\
				=&2R^2\int_{-\pi / 2}^{\pi / 2}\frac{\cos 2\theta+1}{2}\dd{\theta}\\
				=&2R^2\left[ \frac{\sin 2\theta}{4}+\frac{\theta}{2} \right]_{-\pi / 2}^{\pi /2}\\
				=&2R^2\left[ \sin \pi - \sin 0 + \frac{\pi}{2} \right] \\
				=&\pi R^2\\
				=&\lambda_2(B_R(x_0))
			\end{align*}
		\item Wir vorher ist $(\R, \mathcal{L}(1), \lambda_1)$ $\sigma$-endlich. Das Innere der Menge $\{ (x,y,z)\in \R^3|x^2+ y^2<r(z)^2,z\in (a,b) \} $ ist offen und daher messbar. Da $A$ das Innere mit eine Nullmenge hinzugefügt ist, ist $A$ auch messbar. Wir können daher das Maß als Integral schreiben:
			\begin{align*}
				\lambda_3(A)=&\int_{[a,b]} A_z\dd{\lambda_1}\\
				=&\int_{[a,b]} \pi r(z)^2\dd{\lambda_1}\\
				=&\pi \int_a^b r(z)^2\dd{z}.\qedhere
			\end{align*}
	\end{parts}
\end{proof}
\begin{Problem}
	Seien $a,b,c\in \R$ mit $c\ge |a|+|b|$. Definiere
	\[
	E:=\left\{ (x,y,z)\in \R^3|x^2+y^2<1,0\le z\le ax+by+c \right\} 
	.\] 
	Bestimmen Sie $\lambda_3(E)$.
\end{Problem}
\begin{proof}
	Ähnlich betrachten wir $\{(x,y,z)\in\R^3|x^2+y^2<1,0 <z<ax+by+c\} $. Da die Menge offen ist, ist sie messbar. Weil $E$ gleich die Menge mit eine Nullmenge hinzugefügt ist, ist $E$ auch messbar. Wir schreiben dann das Maß als Doppel bzw. Tripelintegral. Hierbei nutzen wir, dass $x^2\le 1$, also $x\in (-1,1)$ aus die Gleichung $x^2+y^2<1$. Sei $x$ fest. Dann ist $y^2 < 1-x^2$, also $y\in (-\sqrt{1-x^2},\sqrt{1-x^2})$.
	\begin{align*}
		\lambda_3(E)=&\int_{-1}^1 \lambda_2(E_x)\dd{\lambda_2} &\text{Satz 2.81}\\
		=&\int_{-1}^1\int_{-\sqrt{1-x^2} }^{\sqrt{1-x^2} }\lambda_1([0,ax+by+c])\dd{\lambda_1(y)}\dd{\lambda_1\left(x \right) } & \text{Satz 2.81}\\
		=&\int_{-1}^1\int_{-\sqrt{1-x^2}}^{\sqrt{1-x^2} }ax+by+c\dd{\lambda_1(y)}\dd{\lambda_1(x)}\\
		=&\int_{-1}^1 \left[ axy+by^2 / 2 + cy \right]_{-\sqrt{1-x^2} }^{\sqrt{1-x^2} }\dd{\lambda_1(x)}\\
		=&\int_{-1}^1 2ax\sqrt{1-x^2} +2c\sqrt{1-x^2} \dd{\lambda_1(x)}\\
		=&2\int_{-1}^1 (ax+c)\sqrt{1-x^2} \dd{\lambda_1(x)}\\
		=&2\left[ \int_{-1}^1 (ax)\sqrt{1-x^2} \dd{x}+c\int_{-1}^1 \sqrt{1-x^2} \dd{x} \right] \\
		=&2c\int_{-1}^1 \sqrt{1-x^2} \dd{x} & x\sqrt{1-x^2}\text{ ist ungerade}\\
		=&c\pi. & \qedhere
	\end{align*}
\end{proof}
\begin{Problem}
	Seien $B_1:=B_1(0,0,0)\subseteq \R^3$ und $B_2:=B_1(0,0,1)\subseteq \R^3$ die offenen Kugeln mit Radius 1 um die Punkte $(0,0,0)$ und $(0,0,1)$. Bestimmen Sie $\lambda_3(B_1\cap B_2)$.
\end{Problem}
\begin{proof}
	Per Definition ist
	\begin{align*}
		B_1=&\{(x,y,z)\in \R^3| x^2+y^2+z^2<1\} \\
		B_2=& \{(x,y,z)\in \R^3|x^2+y^2+(z-1)^2<1\} 
	\end{align*}
	Weil $\lambda_1$ bzw. das Lebesgue-Maß $\sigma$-endlich ist, werden wir das Maß als Integral gemäß Satz 2.81 schreiben. Daher betrachten wir $(B_1\cap B_2)_z$ f\"{u}r beliebiges $z\in [0,1]$. Es gilt
	\begin{align*}
		(B_1\cap B_2)_z=&\{(x,y)\in \R^2|x^2+y^2<\text{min}(1-(z-1)^2,1-z^2)\}\\
		=&B_{\text{min}(\sqrt{1-(z-1)^2} ,\sqrt{1-z^2} )}(0) \subseteq \R^2
	\end{align*}
	Offenbar ist 
	\[
		\text{min}(\sqrt{1-(z-1)^2} ,\sqrt{1-z^2} )=\begin{cases}
			\sqrt{1-z^2}  & z \ge 1 / 2,\\
			\sqrt{1-(z-1)^2}  & z \le 1 / 2.
		\end{cases}
	\] 
	Aus \ref{pr:advanalblatt8-2} ist $\lambda_2(B_R(0))=\pi R^2$. Dann ist
	\begin{align*}
		\lambda_3(B_1\cap B_2)=&\int_0^1 \lambda_2(B_{\text{min}(\sqrt{1-(z-1)^2} ,\sqrt{1-z^2} )}(0))\dd{\lambda_{1}(z)}\\
		=&\int_0^1\pi\text{min}(\sqrt{1-(z-1)^2} ,\sqrt{1-z^2} )^2\dd{\lambda_1(z)}\\
		=&2\int_{1 / 2}^{1}\pi (1-z^2) \dd{\lambda_1(z)}\\
		=&2\pi (z-z^3 / 3)|_{1 / 2}^{1}\\
		=&\frac{5\pi}{12}.\qedhere
	\end{align*}
\end{proof}
