\begin{Problem}
	Sei wie im Beweis von Lemma 1.18
	\[
	p_1:\R^n\to \R, \qquad (x_1,\dots,x_n)\to x_1
	\]
	die Projektion auf die erste Koordinate. F\"{u}r $A\subseteq \R^n$ definiere $\mu^*(A):=\lambda_1^*(p_1(A))$.
	\begin{parts}
		\item Zeigen Sie, dass $\mu^*$ ein äußeres Maß ist.
		\item Beweisen oder widerlegen Sie: Es gilt,$\mu^*(O)>0$ f\"{u}r alle offenen michtleeren Menge $O\subseteq \R^n$
		\item Beweisen oder widerlegen Sie: Es gilt  $\mu^*(C)>0$ f\"{u}r alle nichtleeren abgeschlossenen Mengen $C\subseteq \R^n$
	\end{parts}
\end{Problem}
\begin{proof}	
	\begin{parts}
\item 
	\begin{enumerate}[label=(\roman*)]
		\item $\mu^*(\varnothing)=\lambda_1^*(\varnothing)=0$.
		\item Sei  $\R^n\ni A\subseteq B\in \R^n$. Es gilt auch, dass
			 \[
			p_1(A)\subseteq p_1(B)
			.\]
			Es folgt daraus
			\[
			\mu^*(A)\le \mu^*(B)
			.\]
		\item Sei $(A_i),A_i\subseteq \R^n$ eine Folge von Mengen, und $A=\bigcup_{i=1}^\infty A_i$
			\begin{tcolorbox}
				\begin{Lemma}
					Es gilt $p_1(A)=\bigcup_{i=1} ^\infty p_1(A_i)$
				\end{Lemma}
				\begin{proof}
					Sei $x_1\in p_1(A)$, also es gibt $(x_1,\dots, x_n)\in A$. Das bedeutet, dass $(x_1,\dots, x_n)\in A_k$ f\"{u}r eine $k$ ist. Dann ist $x_1\in p_{1}(A_k)$, und $x_1\in \bigcup_{i=1} ^\infty p_1(A_i)$. 
				\end{proof}
			\end{tcolorbox}
			Weil $\lambda_1^*$ äußeres Maß ist, gilt
			\[
			\mu^*(A)=\lambda_1^*(p_1(A))\le\sum_{i=1} ^\infty \lambda_1^*(p_1(A_i))=\sum_{i=1}^{\infty} \mu^*(A_i)
			.\] 
	\end{enumerate}
\item H 
	\begin{tcolorbox}
		\begin{Lemma}
			$p_1$ ist stetig.
		\end{Lemma}
	\end{tcolorbox}
	\end{parts}
\end{proof}
