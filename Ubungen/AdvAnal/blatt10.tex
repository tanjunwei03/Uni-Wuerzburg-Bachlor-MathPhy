\begin{Problem}
	Sei $R>0$ und $a<b$. Definiere $Z:=B_R(0)\times (a,b)\subseteq \R^3$, wobei $B_R(0)\subseteq \R^2$ die (offene) Kreisscheibe um $0$ mit Radius $R$ ist. Definiere außerdem die Abbildung
	\[
	\Phi:U\to \R^3, \qquad \Phi(r,\varphi,z):=\begin{pmatrix} r\cos\varphi \\ r\sin\varphi \\ z \end{pmatrix} 
\]
mit $U:=(0,R)\times (0,2\pi)\times (a,b)$.
\begin{parts}
	\item Zeigen Sie, dass eine Nullmenge $N\subseteq \R^3$ existiert, sodass $\Phi(U)=Z\backslash N$.
	\item Zeigen Sie, dass $\Phi:U\to Z \backslash N$ ein $C^1$-Diffeomorphismus ist mit $\text{det}(\Phi'(r,\varphi,z))=r$.
	\item Sei $f:\R^3\to r$ definiert durch $f(x,y,z):=z\sqrt{x^2+y^2} $. Bestimmen Sie $\int_Z f\dd{\lambda_3}$.
\end{parts}
\end{Problem}
\begin{proof}
	\begin{parts}
	\item Hypothese: $N=[0,R)\times \{0\} \times (a,b)$.

		Wir schreiben
		\[
		\begin{pmatrix} r\cos\varphi \\r\sin\varphi \\ z \end{pmatrix} =\begin{pmatrix} x \\ y \\ z \end{pmatrix}\in Z 
		.\] 
		Wir finden die $(x,y,z)$, f\"{u}r die die Gleichungen keine Lösung haben. Es ist klar, dass die dritte Gleichung trivialerweise immer erfüllt werden kann. 

		Jetzt betrachten wir $(x,y,z)\in N$. Weil $y=0$, ist $\sin\varphi=0$ und $\varphi=\pi$ ($\varphi=0$ ist keine Lösung in $U$). Dann ist $x=r\cos\varphi=-r$. Weil $r>0$, ist dann $x<0$, also $\Phi(U)\subseteq Z\backslash N$.

		Sei jetzt $(x,y,z)^T\not\in N$. Die (eindeutige) Lösungen sind
		\begin{align*}
			r=&\sqrt{x^2+y^2} \\
			\tan\varphi=&y / x
		\end{align*}
		Man verfiziere sofort, dass $r$ und $\varphi$ Lösungen sind und außerdem in $U$ liegen, insofern $(x,y,z)^T\not\in N$. Daraus folgt:
		\[
		\Phi(U)=Z\backslash N
		.\] 
		Jetzt zeigen wir: $N$ ist eine Nullmenge. Da $N\subseteq [0,R)\times (-\epsilon,\epsilon)\times (a,b)$ f\"{u}r alle $\epsilon>0 $, ist $N$ eine Nullmenge.
\item Die Ableitung ist
	\[
		\Phi'=\begin{pmatrix} \cos\varphi & \sin\varphi & 0 \\ -r\sin\varphi & r\cos\varphi & 0 \\ 0 & 0 & 1 \end{pmatrix} 
	.\] 
	Da die Komponente alle stetig sind, ist $\Phi'$ stetig, und $\Phi$ ist stetig differenzierbar. Die Determinante ist
	\[
		\text{det}\Phi'=
		\begin{vmatrix}
			\cos\varphi & \sin\varphi \\ -r\sin\varphi & r\cos\varphi
		\end{vmatrix}=r
	.\] 
\item 
	\begin{align*}
		\int_Z f\dd{\lambda_3}=&\int_{Z\backslash N}f\dd{\lambda_3} & N\text{ Nullmenge}\\
		=&\int_U |\text{det}\Phi'|(f\circ\Phi)\dd{\lambda_3}\\
		=&\int_U rf(r\cos\varphi,r\sin\varphi,z)\dd{\lambda_3}\\
		=&\int_U rz\sqrt{(r\cos\varphi)^2+(r\sin\varphi)^2} \\
		=&\int_U r^2z\dd{\lambda_3}\\
		=&\int_{U_z}\int_a^b r^2z\dd{z}\dd{\lambda_2}\\
		=&\int_{U_z}r^2(b-a)\dd{\lambda_2}\\
		=&\int_{(U_z)_\theta}\int_0^{2\pi}r^2(b-a)\dd{\lambda_2}\\
		=&\int_{(U_z)_\theta}2\pi r^2(b-a)\dd{\lambda_2}\\
		=&\int_0^R 2\pi r^2(b-a)\dd{\lambda_2}\\
		=&\left.2\pi(b-a)\frac{r^3}{3}\right|_0^R\\
		=&\frac{2}{3}\pi(b-a)R^3.\qedhere
	\end{align*}
	\end{parts}
\end{proof}
\begin{Problem}
	Sei $R>0$ und $K:=B_R(0)\subseteq \R^3$. Definiere die Abbildung
	\[
	\Phi:U\to \R^3,\qquad\Phi(r,\theta,\varphi):=\begin{pmatrix} r\sin\theta\cos\varphi \\ r\sin\theta\sin\varphi \\ r\cos\theta \end{pmatrix} 
\]
mit $U:=(0,R)\times (0,\pi)\times(0,2\pi)$.
\begin{parts}
\item Zeigen Sie, dass eine Nullmenge $N\subseteq \R^3$ existiert, sodass $\Phi(U)=K\backslash N$.
\item Zeigen Sie, dass $\Phi:U\to K\backslash N$ ein $C^1$-Diffeomorphismus ist mit $\text{det}(\Phi'(r,\theta,\varphi))=r^2\sin\theta$.
\item Sei $f:\R^3\to \R$ definiert durch $f(x,y,z):=z\sqrt{x^2+y^2} $ und
	\[
	H:=B_R(0)\cap \{(x,y,z)\in \R^3|z\ge 0\} 
	.\] 
	Bestimmen Sie $\int_H f\dd{\lambda_3}$.
\end{parts}
\end{Problem}
\begin{proof}
	\begin{parts}
	\item Ähnlich ist die Nullmenge $\{(0,0,R),(0,0,-R)\} $. Wir lösen die Gleichungen
		\[
			\begin{pmatrix} x \\ y \\ z \end{pmatrix} =\begin{pmatrix} r\sin\theta\cos\varphi \\ r\sin\theta\sin\varphi \\ r\cos\theta \end{pmatrix} 
		\]
		f\"{u}r $(x,y,z)^T\in B_R(0)$. Ähnlich ist $r=\sqrt{x^2+y^2+z^2}, \tan\theta = (\sqrt{x^2+y^2} ) / z$ und $\tan\varphi=y / x$ eine Lösung, solange $z\neq\pm R$ (sonst wäre $\theta=0$ oder $\pi$, welche nicht in $U$ sind. Als endliche Menge ist $N$ eine Nullmenge.
	\item Es gilt
		\[
			\Phi'=\begin{pmatrix} \sin\theta\cos\varphi & \sin\theta\sin\varphi & \cos\theta \\ r\cos\theta\cos\varphi & r\cos\theta\sin\varphi & -r\sin\theta \\ -r\sin\theta\sin\varphi & r\sin\theta\cos\varphi & 0 \end{pmatrix} 
		.\] 
		Weil alle Komponente stetig sind, ist auch $\Phi'$ stetig, und $\Phi$ ist stetig differenzierbar. F\"{u}r die Determinante f\"{u}hren wir eine Laplaceentwicklung mit der dritten Spalte durch:
		\begin{align*}
			\text{det}\Phi'=&\cos\theta(r^2\cos\theta\sin\theta\cos^2\varphi+r^2\cos\theta\sin\theta\sin^2\varphi)\\
					&+r\sin\theta(r\sin^2\theta\cos^2\varphi+r\sin^2\theta\sin^2\varphi)\\
			=&r^2\cos^2\theta\sin\theta+r^2\sin^3\theta\\
			=&r^2\sin\theta
		\end{align*}
	\item Sei $U=(0,R)\times (0,\pi / 2)\times (0,2\pi)$. Es gilt $\Phi(U)=H\backslash N$.
		 \begin{align*}
			 \int_H f\dd{\lambda_3}=&\int_{H\backslash N}f\dd{\lambda_3}\\
			 =&\int_U |\text{det}\Phi'|(f\circ \Phi)\dd{\lambda_3}\\
			 =&\int_U (r^2\sin\theta)r\cos\theta\sqrt{(r\sin\theta\cos\varphi)^2+(r\sin\theta\sin\varphi)^2} \\
			 =&\int_U (r^2\sin\theta)r\cos\theta(r\sin\theta)\\
			 =&\int_U r^4\sin^2\theta\cos\theta\dd{\lambda_3}\\
			 =&\int_{U_\varphi}\int_0^{2\pi}r^4\sin^2\theta\cos\theta\dd{\varphi}\dd{\lambda_2}\\
			 =&\int_{U_\varphi}2\pi r^4\sin^2\theta\cos\theta\dd{\lambda_2}\\
			 =&\int_{(U_\varphi)_\theta}\int_0^{\pi / 2}2\pi r^4\sin^2\theta\cos\theta\dd{\theta}\dd{\lambda_1}\\
			 =&\int_0^R 2\pi r^4(1 / 3)\dd{r}\\
			 =&\frac{2}{15}\pi r^5|_0^R\\
			 =&\frac{2\pi}{15}R^5.\qedhere
		\end{align*}
	\end{parts}
\end{proof}
\begin{Problem}
	Sei $A\in \R^{3\times 3}$ symmetrisch und positiv definit. Zeigen Sie, dass
	\[
		\int_{\R^3}\exp(-x^TAx)\dd{\lambda_3}=\frac{\pi^{3 / 2}}{\sqrt{\text{det}(A)} }
	.\] 
\end{Problem}
