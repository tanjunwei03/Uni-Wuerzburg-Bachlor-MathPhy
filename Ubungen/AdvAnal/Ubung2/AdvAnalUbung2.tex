\documentclass[prb,12pt]{revtex4-2}

%font
\usepackage[T1]{fontenc}
\usepackage[osf,sc]{mathpazo}
%actual preamble
\usepackage{amsmath, amssymb,physics,amsfonts,amsthm}
\usepackage{enumitem}
\usepackage{cancel}
\usepackage{booktabs}
\usepackage{tikz}
\usepackage{hyperref}
\usepackage{enumitem}
\usepackage{transparent}
\usepackage{float}
\usepackage{multirow}
\newtheorem{Theorem}{Theorem}
\newtheorem{Proposition}{Theorem}
\newtheorem{Lemma}[Theorem]{Lemma}
\newtheorem{Corollary}[Theorem]{Corollary}
\newtheorem{Example}[Theorem]{Example}
\newtheorem{Remark}[Theorem]{Remark}
\theoremstyle{definition}
\newtheorem{Problem}{Problem}
\theoremstyle{definition}
\newtheorem{Definition}[Theorem]{Definition}
\newenvironment{parts}{\begin{enumerate}[label=(\alph*)]}{\end{enumerate}}
%tikz
\usetikzlibrary{patterns}
% definitions of number sets
\newcommand{\N}{\mathbb{N}}
\newcommand{\R}{\mathbb{R}}
\newcommand{\Z}{\mathbb{Z}}
\newcommand{\Q}{\mathbb{Q}}
\newcommand{\C}{\mathbb{C}}
\begin{document}
	\title{Vertiefung Analysis Hausaufgabenblatt Nr. 1}
	\author{Jun Wei Tan}
	\email{jun-wei.tan@stud-mail.uni-wuerzburg.de}
	\affiliation{Julius-Maximilians-Universit\"{a}t W\"{u}rzburg}
	\date{\today}
\maketitle
\begin{Problem}
	(Maß über $\mathcal{P}(\N)$) Für $\lambda \in \R$ definiere
	\[
	\mu_\lambda:\mathcal{P}(\N)\to \overline{\R}, \mu_\lambda(A):=\sum_{k\in A} \exp(\lambda) \frac{\lambda^k}{k!}
	.\] 
	Bestimmen SIe jeweils alle $\lambda\in \R$, f\"{u}r die
	\begin{parts}
		\item $(\N, \mathcal{P}(\N), \mu_\lambda)$ ein Maßraum ist.
		\item $\mu_\lambda$ ein endliches Maß ist.
		\item $\mu_\lambda$ ein Wahrscheinlichkeitsmaß ist.
	\end{parts}
\end{Problem}
\begin{proof}
	\begin{parts}
	\item $\mu_\lambda$ ist auf jedem Fall f\"{u}r endliche Teilmengen von $\N$ wohldefiniert. Weil $\sum_{k=0}^\infty \frac{\lambda^k}{k!}$ konvergiert (absolut) f\"{u}r alle $\lambda\in\R$, konvergiert absolut alle Teilfolge. 

		$\mu_\lambda$ ist auch trivialweise additiv.
	\item Das passt f\"{u}r $\lambda \in \R$ 
	\item Wir brauchen 
		\[
		\sum_{k=1}^{\infty} \exp(\lambda) \frac{\lambda^k}{k!}=\exp(\lambda)(\exp(\lambda)-1)=1
		,\] 
		oder 
		\[
		\exp(\lambda)=\frac{1}{2}(1\pm \sqrt{5} )
		.\]
		Weil $\exp(\lambda), \lambda \in \R$ immer positiv ist, gibt es nur eine reelle Lösung:
		\[
		\lambda=\ln\left[ \frac{1}{2}\left( 1+\sqrt{5}  \right)  \right] 
		.\] 
	\end{parts}
\end{proof}
\begin{Problem}
	(vollständiger Maßraum) Sei $X$ eine nichtleere Menge, $(X, \mathcal{A}, \mu)$ ein Maßraum und $B \in \mathcal{A}$. Definiere $\mu_B : \mathcal{A} \to [0, \infty]$, $\mu_B (A) := \mu(A \cap B)$.
	\begin{parts}
	\item Zeigen Sie, dass $\mu_B$ ein Maß über $\mathcal{A}$ ist.
	\item  Beweisen oder widerlegen Sie: Ist $(X, \mathcal{A}, \mu_B)$ ein vollständiger Maßraum, dann auch $(X, \mathcal{A}, \mu)$.
	\item  Beweisen oder widerlegen Sie: Ist $(X, \mathcal{A}, \mu)$ ein vollständiger Maßraum, dann auch $(X, \mathcal{A}, \mu_B)$.
	\end{parts}
\end{Problem}
\begin{proof}
	\begin{parts}
	\item In der Übungsblatt 1. haben wir schon bewiesen, dass es wohldefiniert ist.

		Sei dann $(A_j), A_j\in \mathcal{A}$ eine Folge disjunkte Menge. Es gilt
	\begin{align*}
		&\mu_B\left( \bigcup_{i\in\N} A_i \right) =\mu\left( B\cap \bigcup_{i\in \N} A_i \right)\\
		&=\underbrace{\mu\left( \bigcup_{i\in \N} \left( B \cap A_i \right)  \right)=\sum_{i\in \N}\mu(B\cap A_i)}_{\sigma-\text{additivität von }\mu}=\sum_{i\in \N}\mu_B(A_i)
	\end{align*}
		$\mu_B$ ist dann $\sigma$-Additiv, und daher Maß.
	\item Ja. Sei $\mu(A)=0$. Weil $A\cap B\subseteq A$ ist, gilt auch $\mu_B(A)=0$. Weil $(X,\mathcal{A},\mu_B)$ vollständig ist, ist jede Teilmenge $\mathcal{A}\ni A' \subseteq A$. $(X, \mathcal{A},\mu)$ ist dann vollständig.
	\item Nein. Sei $X=\left\{ a,b,c \right\},B=\left\{ a \right\}, \mathcal{A}=\left\{\varnothing, \left\{ a \right\} , \left\{ b,c \right\}  \right\} $.

		Sei auch $\mu(\left\{ b,c \right\} \neq 0, \mu(X)\neq 0, \mu(\left\{ a \right\} )\neq 0$. Dann ist $(X, \mathcal{A},\mu)$ trivialweise vollständig (es gibt keine Nullmenge), aber $\mu_B(\left\{ b,c \right\} )=\mu(\left\{ a \right\} \cap \left\{ b,c \right\} )=\mu(\varnothing)=0$. Deswegen ist $\left\{ b,c \right\} $ eine Nullmenge in $(X,A,\mu_B)$, aber $\left\{ b \right\} \subseteq \left\{b ,c \right\} \not\in \mathcal{A}$
	\end{parts}
\end{proof}
\begin{Problem}
	\begin{parts}
\item	Seien $K_1 , K_2 \subseteq \mathcal{P}(\R^n)$ mit $\varnothing \in K_i$ für $i = 1, 2$ und $\nu_i : K_i \to [0, \infty]$ mit $\nu_i (\varnothing) = 0$ für $i = 1, 2$. Bezeichne nun mit $\mu^*_i$ die analog zu Satz 1.37 von $\nu_i$ induzierten äußeren Maße. Es existiere ein $\alpha > 0$, so dass
	\[
		\forall I_1\in K_1\exists I_2\in K_2: I_1\subseteq I_2\text{ und }\alpha \nu_2(I_2)\le \nu_1(I_1)
	.\] 
	Zeigen Sie: F\"{u}r alle $A\subseteq \R^n$ gilt $\alpha\mu_2^*(A)\le \mu_1^*(A)$.
\item Vervollständigen Sie den Beweis zu Satz 1.55: Zeigen Sie, dass
	\[
		\lambda_a^*(A)\le \lambda_l^*(A)\le \lambda_n^*(A)\text{ und }\lambda_a^*(A)\le \lambda_r^* (A)\le \lambda_n^*(A)\]
		f\"{u}r alle $A\subseteq \R^n$ gilt.
	\end{parts}
\end{Problem}
\begin{proof}
	\begin{parts}
	\item F\"{u}r alle $\epsilon>0$ gibt es eine Überdeckung $(A_{1,j}),A_{1,j}\in K_1$, f\"{u}r die gilt
		 \begin{align*}
			 \bigcup_{k=1}^\infty A_{1,k}\supseteq& A\\
			 \sum_{k=1}^{\infty} \nu_1(A_{1,k})\le& \mu_1^*(A)+\epsilon 
		\end{align*}
		Es gibt auch per Hypothese eine Folge $(A_{2,k}), A_{2,k}\in K_2, A_{2,k}\supseteq A_{1,k}, \alpha\nu_2(A_{2,k})\le \nu_2(A_{1,k})$. Dann gilt
		\begin{align*}
			\bigcup_{k=1} ^\infty A_{2,k}\supseteq& A\\
			\sum_{k=1}^{\infty} \alpha\nu_2(A_{2,k})\le& \sum_{k=1}^{\infty} \nu_1(A_{2,k})<\mu_1^*(A)+\epsilon
		\end{align*}
		Weil das f\"{u}r alle $\epsilon$ gilt, ist $\alpha\mu_2^*(A)\le \mu_1^*(A)$
	\end{parts}
\end{proof}
\end{document}
