\documentclass[prb,12pt]{revtex4-2}

\usepackage{amsmath, amssymb,physics,amsfonts,amsthm}
\usepackage{enumitem}
\usepackage{cancel}
\usepackage{booktabs}
\usepackage{tikz}
\usepackage{hyperref}
\usepackage{enumitem}
\usepackage{transparent}
\usepackage{float}
\usepackage{multirow}
\newtheorem{Theorem}{Theorem}
\newtheorem{Proposition}{Theorem}
\newtheorem{Lemma}[Theorem]{Lemma}
\newtheorem{Corollary}[Theorem]{Corollary}
\newtheorem{Example}[Theorem]{Example}
\newtheorem{Remark}[Theorem]{Remark}
\theoremstyle{definition}
\newtheorem{Problem}{Problem}
\theoremstyle{definition}
\newtheorem{Definition}[Theorem]{Definition}
\newenvironment{parts}{\begin{enumerate}[label=(\alph*)]}{\end{enumerate}}
%tikz
\usetikzlibrary{patterns}
% definitions of number sets
\newcommand{\N}{\mathbb{N}}
\newcommand{\R}{\mathbb{R}}
\newcommand{\Z}{\mathbb{Z}}
\newcommand{\Q}{\mathbb{Q}}
\newcommand{\C}{\mathbb{C}}
\begin{document}
	\title{Vertiefung Analysis Hausaufgabenblatt Nr. 1}
	\author{Jun Wei Tan}
	\email{jun-wei.tan@stud-mail.uni-wuerzburg.de}
	\affiliation{Julius-Maximilians-Universit\"{a}t W\"{u}rzburg}
	\date{\today}
	\maketitle
	\begin{Problem}
		Es erfüllt alle Bedingungen.
	\end{Problem}
	\begin{Problem}
		Sei $X=\R, \mathcal{A}_1=\mathcal{A}_2=\left\{ (a,b), a<b,a,b\in\R \right\} $. Dann sind $(0,1)\times (0,1)\text{ und }(1,2)\times (1,2) \in \mathcal{A}_1\text{ [lustiges Symbol etwas wie }\otimes\text{ aber anders (wie tippt man das eigentlich?)]} \mathcal{A_2}$, aber deren Vereinigung nicht.
	\end{Problem}

	\begin{Problem}
		\begin{enumerate}
			\item $A\backslash(A\cap B)=(A\cap B^c)$
			\item  $A=X\cap A, \varnothing = A \cap \varnothing$
			\item  $\bigcup_{i\in \N} (A \cap B_i)=A \cap \left(\bigcup_{i\in \N}B_i\right) $
		\end{enumerate}
	\end{Problem}
	\end{document}
