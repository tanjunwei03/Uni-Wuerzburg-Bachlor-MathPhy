\begin{Problem}
	\textbf{(Hyperbelfunktion)} Sei $d\in \{1,2,3\} , R>0, 1\le p<\infty$ und $\alpha>0$. Definiere
	\[
	B_R(d;0):=\{x\in \R^d|\|x\|<R\} , \qquad f:\R^d\to \R, x\to \begin{cases}
		0 & x=0,\\
		\frac{1}{\|x\|^\alpha} & x\neq 0.
	\end{cases}
	.\] 
	\begin{parts}
		\item Sei zunächst $d=1$. F\"{u}r welche $\alpha, p$ ist die Funktion $\chi_{B_R(1;0)}f$ in $L^p(\lambda_1)$? F\"{u}r welche $\alpha,p$ ist die Funktion $\chi_{\R\backslash B_R(1;0)}f$ in $L^p(\lambda_1)$?
		\item Welche Bedingungen müssen in den Fällen $d=2,3$ f\"{u}r $\alpha, p$ gelten, damit $\chi_{B_R(d;0)}f\in L^p(\lambda_d)$ ist?
		\item Sei $1<p<r<q<\infty$. Geben Sie eine Funktion $g:\R\to \R$ an mit $g\in L^r(\lambda_1)$, $g \not\in L^p(\lambda_1)$, $g \not\in L^q(\lambda_1)$.
		\item Geben Sie ein Beispiel f\"{u}r eine Funktion $g:(0,1)\to \R$ an, sodass $g\in L^p(\lambda_1)$ f\"{u}r alle $p\in [1,\infty)$ gilt, aber $g \not\in L^\infty(\lambda_1)$.
	\end{parts}
\end{Problem}
\begin{proof}
	\begin{parts}
	\item \label{part:advanalblatt11-1a}Die Funktion $\chi_{B_R(1;0)}f$ ist in $L^p(\lambda_1)$ genau dann, wenn
	\begin{align*}
		\int \chi_{B_R(1;0)}|f|^p \dd{\lambda_1}=&\int_{B_R(1;0)} |f|^p \dd{\lambda_1}\\
		=&\int_{B_R(1;0)}\frac{1}{\|x\|^{\alpha p}}\dd{\lambda_1}\\
		=&\int_0^R \frac{1}{\|x\|^{\alpha p}}\dd{\lambda_1}+\int_{-R}^0 \frac{1}{\|x\|^{\alpha p}}\dd{\lambda_1}
	\end{align*}
	Weil die Funktionen positiv sind, sind sie Lebesgue-Integrierbar genau dann, wenn sie (uneigentlich) Riemann-Integrierbar sind. Wir wissen aber auch, dass
	\begin{align*}
		\int_0^R \frac{1}{\|x\|^{\alpha p}}\dd{\lambda_1}=&\int_0^R \frac{1}{x^{\alpha p}}\dd{\lambda_1} & x>0\\
		=&\lim_{a \to 0} \int_a^R \frac{1}{x^{\alpha p}}\dd{x}\\
		=&\lim_{a \to 0}\left[ \frac{x^{-\alpha p + 1}}{1-\alpha p} \right] 
	\end{align*}
	was existiert genau dann, wenn $-\alpha p + 1 \ge 0$. Das Ergebnis stimmt nicht f\"{u}r $\alpha p=1$. In diesem Fall ist
	\[
		\int_a^R \frac{1}{x}\dd{x}=[\ln x]_a^R
	\]
	und der Grenzwert existiert nicht. Aus der Symmetrie von $x\to -x$ gilt genau die gleiche f\"{u}r $\int_{-R}^0 \frac{1}{\|x\|^{\alpha p}}\dd{x}$. Insgesamt ist die Funktion genau dann integrierbar, wenn $-\alpha p + 1 >0$. 

	Ähnlich berechnen wir das Riemann-Integral f\"{u}r $\chi_{\R\backslash B_R(1;0)}$. Es gilt
	\begin{align*}
		\int_R^\infty \frac{1}{\|x\|^{\alpha p}}=&\int_R^\infty \frac{1}{x^{\alpha p}}\dd{x}\\
		=&\lim_{a \to \infty} \int_R^a \frac{1}{x^{\alpha p}}\dd{x}\\
		=&\lim_{a \to \infty} \left[ \frac{x^{-\alpha p+1}}{1-\alpha p} \right]
	\end{align*}
	was genau dann existiert, wwenn $-\alpha p + 1\le 0$. Ähnlich stimmt das Ergebnis nicht f\"{u}r $-\alpha p = 1$ nicht. In diesem Fall ist
	\[
		\int_R^\infty \frac{1}{x}\dd{x}=\lim_{a \to \infty} \ln x |_R^a
	,\]
	was nicht existiert. Also es ist genau dann integrierbar, wenn $-\alpha p + 1 < 0$.
\item $d=2$: Wir berechnen das Integral in Polarkoordinaten. Da die Funktion positiv ist, ist das Integral wohldefiniert. Sie ist genau dann integrierbar, wenn die transformierte Funktion integrierbar ist. Wir wissen, dass $\|x\|=r$. Daraus folgt:
	\begin{align*}
		\int_{B_R(2;0)} \frac{1}{\|x\|^{\alpha p}}\dd{\lambda_2}=&\int_0^R\int_0^{2\pi} \frac{1}{r^{\alpha p}}\dd{\theta}r\dd{r}\\
		=& 2\pi\int_0^R \frac{1}{r^{\alpha p-1}}\dd{r}
	\end{align*}
	Aus dem Argument in \ref{part:advanalblatt11-1a} ist das Integral endlich genau dann, wenn
	\[
		-(\alpha p - 1)+1=-\alpha p+2>0
	.\] 
	Ähnlich für $d=3$ berechnen die das Integral in Kugelkoordinaten. Die Funktion ist integrierbar genau dann, wenn die transformierte Funktion integrierbar ist. Es gilt
	\begin{align*}
		\int_{B_R(3;0)}\frac{1}{\|x\|^{\alpha p}}\dd{\lambda_3}=& \int_0^R\int_0^{2\pi}\int_0^\pi \frac{1}{r^{\alpha p}}r^2\sin\theta\dd{\theta}\dd{\varphi}\dd{r}\\
		=&2\int_0^R \int_0^{2\pi}\frac{1}{r^{\alpha p - 2}}\dd{\varphi}\dd{r}\\
		=&4\pi\int_0^R \frac{1}{r^{\alpha p - 2}}\dd{r}
	\end{align*}
	Das Integral ist endlich genau dann, wenn
	\[
		-(\alpha p - 2)+1=-\alpha p + 3>0
	.\] 
\item Sei $p=1,r=2,q=3$. Sei außerdem
	\[
	g(x)=\begin{cases}
		x^{- 1 / 3} & |x|<1\\
		1 / x & |x| \ge 1
	\end{cases}
	.\] 
	Wir zeigen alle drei Eigenschaften.
	\begin{align*}
		\int |g|\dd{\lambda_1}\ge& \int_{\R \backslash B_1(1;0)} |g|\dd{\lambda_1}\\
		=&\int_{\R\backslash B_1(1;0)} \frac{1}{|x|}\dd{\lambda_1}\\
		=&\infty & \text{\ref{part:advanalblatt11-1a}}
	\end{align*}
	also $g\not\in L^1(\lambda_1)$. Jetzt ist
	\begin{align*}
		\int |g|^3 \dd{\lambda}\ge& \int_{B_1(1;0)} |g|^3 \dd{\lambda_1}\\
		=& \int_{-1}^1 \frac{1}{|x|}\dd{\lambda_1}\\
		=&\infty & \text{\ref{part:advanalblatt11-1a}}
	\end{align*}
	also $g\not\in L^3(\lambda_1)$. Zuletzt ist
	\begin{align*}
		\int |g|^2\dd{\lambda_1}=& \int_{B_1(1;0)}|g|^2\dd{\lambda_1}+\int_{\R\backslash B_1(1;0)}|g|^2\dd{\lambda_1}\\
		=&\underbrace{\int_{B_1(1;0)} x^{-2 / 3}\dd{\lambda_1}}_{<\infty}+\underbrace{\int_{\R\backslash B_1(1;0)} \frac{1}{x^2}\dd{\lambda_1}}_{<\infty}\\
		<& \infty
	\end{align*}
	wobei die zwei Integrale weniger als unendlich aus \ref{part:advanalblatt11-1a} sind, also $g\in L^2(\lambda_1)$. 
\item Sei $g:(0,1)\to \R$ definiert durch $g(x)=\ln x$. Es gilt $\lim_{x \to 0} |g(x)|=\infty$, also $g\not\in L^\infty(\lambda_1)$. $g$ ist jedoch ein Element von $L^p(\lambda_1)$ f\"{u}r alle $p\in [1,\infty)$. 

	Wir zeigen es zunächst für all $p\in \N$ per Induktion. Außerdem berechnen wir das Integral über $(0,1]$, was das Ergebnis nicht verändert, weil $\{1\} $ eine Nullmenge ist. Da $\ln x$ entweder $>0$ oder $<0$ f\"{u}r alle $x\in (1,0]$ ist, schreiben wir $|\ln x|=k\ln x$, wobei $k\in \{-1,1\} $. 

	$p=1$ Fall:
	\begin{align*}
		\int_0^1 \ln x\dd{\lambda_1}=&\lim_{a \to 0} \int_0^1 \ln x\dd{\lambda_1}\\
		=&\lim_{a \to 0}\left[ x\ln x|_a^1 - \int_a^1 \dd{\lambda_1} \right] \\
		=&\lim_{a \to 0} \left[ a\ln x-(1-a) \right] \\
		=&-1
	\end{align*}
	also
	\[
		\int_0^1 |\ln x|\dd{\lambda_1}=1
	.\] 
	Wir nehmen jetzt an, dass es f\"{u}r beliebiges $p\in \N$ gilt. 
	\begin{align*}
		\int_0^1 |\ln x|^{p}\dd{\lambda_1}=& k^{p+1} \int_0^1 (\ln x)^{p+1}\dd{\lambda_1}\\
		=&k^{p+1}\left[ x(\ln x)^{p+1}|_0^1-\int_0^1 x(p+1)(\ln x)^p \frac{1}{x} \right] \\
		=&k^{p+1}\left[ -(p+1)\int_0^1(\ln x)^p\dd{\lambda_1}\right] 
	\end{align*}
	Per Voraussetzung ist das Integral von $|\ln x|^p$ endlich, also das Integral von $|\ln x|^{p+1}$ ist auch endlich. Insgesamt ist $\ln x\in L^p(\lambda_1)$ f\"{u}r alle $p\in \N$.

	Sei jetzt $p$ beliebig. Es gilt
	\[
		\int_0^1 |\ln x|^p\dd{\lambda_1}=\int_0^{1 / e}|\ln x|^p\dd{\lambda_1}+\int_{1 / e}^1 |\ln x|^p\dd{\lambda_1}
	.\] 
	Da $|\ln x|^p$ auf $(1 / e, 1)$ stetig f\"{u}r alle $p$ ist, existiert das Integral von $1 / e$ bis $1$ stets. Also wir vergleichen $|\ln x|^p$ f\"{u}r $x\in (0,1 / e)$. 

	Weil $|\ln x|>1$ f\"{u}r alle $x\in (0, 1 / e)$, ist
	\[
		\int_0^{1 / e}|\ln x|^p\dd{\lambda_1}\le \int_0^{1 / e}|\ln x|^{\left\lceil  p \right\rceil}\dd{\lambda_1} 
	.\] 
	Per Definition ist $\left\lceil p \right\rceil \in \N$ und $|\ln x|^{\left\lceil p \right\rceil }$ integrierbar. Dann ist $|\ln x|^p$ auch integrierbar und $\ln x\in L^p(\lambda_1)~\forall p\in [1,\infty)$.\qedhere
	\end{parts}
\end{proof}
\begin{Problem}
	Sei $(X, \mathcal{A},\mu)$ ein Maßraum und $1\le p<q\le\infty$.
	\begin{parts}
	\item Zeigen Sie, dass $L^p(\mu)\cap L^q(\mu)\subseteq L^r(\mu)$ f\"{u}r alle $r\in (p,q)$ gilt und außerdem
		\[
			\|f\|_{L^r(\mu)}\le \|f\|_{L^p(\mu)}^\theta \|f\|_{L^q(\mu)}^{1-\theta}
		\] 
		f\"{u}r alle $f\in L^p(\mu)\cap L^q(\mu)$ und $\theta\in (0,1)$ mit
		\[
		\frac{1}{r}=\frac{\theta}{p}+\frac{1-\theta}{q}
		.\] 
	\item Sei der Maßraum $(X,\mathcal{A},\mu)$ nun endlich. Zeigen Sie, dass dann $L^q(\mu)\subseteq L^p(\mu)$ und
		\[
			\|f\|_{L^p(\mu)}\le \mu(X)^{\frac{1}{p}-\frac{1}{q}}\|f\|_{L^q(\mu)}
		\] 
		f\"{u}r alle $f\in L^q(\mu)$ gilt.

		\emph{Hinweis: Betrachten Sie den Raum $L^r(\mu)$ f\"{u}r $r:=\frac{q}{p}$.}
	\end{parts}
\end{Problem}
\begin{proof}
	\begin{parts}
	\item Sei $f\in L^p(\mu)\cap L^q(\mu)$. Das Ziel ist: $f\in L^r(\mu)$ f\"{u}r alle $r\in (p,q)$. Sei
\begin{align*}
	A=&\{x|x\in X,~|f(x)|<1\} \\
	B=&\{x|x\in X,~|f(x)|\ge 1\}=X\backslash A
\end{align*}
Weil $|f|^p$ bzw. $|f|^q$ auf der ganzen Menge $X$ integrierbar sind, sind $|f|^p$ bzw. $|f|^q$ auf $A$ und $B$ integrierbar. Es gilt, f\"{u}r alle $x\in A$,
\[
|f|^q \le |f|^r \le |f|^p
\]
also $|f|^r$ ist auf $A$ integrierbar. Ähnlich ist f\"{u}r alle $x\in B$ 
\[
|f|^p\le |f|^r\le |f|^q
\]
und das Integral von $|f|^r$ auf $B$ existiert. Da
\[
	\int |f|^r\dd{\mu}=\int_A |f|^r\dd{\mu}+\int_B |f|^r\dd{\mu}
,\]
ist $|f|^r$ integrierbar und $f\in L^r(\mu)$. Aus der Höldersche Ungleichung folgt, f\"{u}r $1 / p + 1 / q = 1, p,q\in [1,\infty]$
\[
	\|f^2\|_{\mathcal{L}^1(\mu)}\le \|f\|_{\mathcal{L}^p(\mu)}\|f\|_{\mathcal{L}^q(\mu)}
.\] 
Es gilt
\begin{align*}
	\|f\|_{L^r(\mu)}=&\left[ \int |f|^{r}\dd{\mu} \right]^{1 / r}\\
	=&\left[ \int |f|^{\theta r}|f|^{(1-\theta)r}\dd{\mu} \right]^{1 / r}\\
	=& 
\end{align*}
Es gilt
\begin{align*}
	\|f\|_{\mathcal{L}^1(\mu)}=&\int |f|\dd{\mu}\\
	=& \int |f|^{\theta}|f|^{1-\theta}\dd{\mu}\\
	\le&\|f^\theta\|_{\mathcal{L}^q(\mu)} \|f^{1-\theta}\|_{\mathcal{L}^{p}(\mu)}
\end{align*}
	\end{parts}
\end{proof}
\begin{Problem}
	Sei $f:\R^3\to \R$ definiert durch $f(x,y,z):=|xyz|$ und
	\[
		A:=\left\{ (x,y,z)\in \R^3|x^2+y^2+z^2\le 1, z\ge \frac{1}{2} \right\} 
	.\] 
	Bestimmen Sie $\int_A f\dd{\lambda_3}$.
\end{Problem}
\begin{proof}
	Weil $f$ stetig auf einer kompakten Menge definiert ist, ist $f$ auf $A$ integrierbar. F\"{u}r den Schnitt $A_z$ gilt
	\[
	A_z=\left\{ (x,y)\in \R^2|x^2+y^2\le 1 - z^2 \right\} 
	.\] 
	Dann verwenden wir den Satz von Fubini, um das Integral als Doppelintegral zu schreiben. $z$ muss in $(1 / 2, 1)$ sein, damit $A_z$ nichtleerist. $z\ge 1 / 2$ per Definition und $z\le 1$, weil sonst $x^2+y^2\le 1 - z^2\le 0$. Da $x^2+y^2$ positiv ist, wäre der Schnitt dann leer.
	\[
		\int_A f\dd{\lambda_3}=\int_{1 / 2}^1\int_{A_z}|xy| |z|\dd{\lambda_2}\dd{z}
	.\] 
	Wir berechnen jetzt das Integral
	\[
		\int_{A_z}|xy|\dd{\lambda_2}
	\]
	in Polarkoordinaten. Das Integral existiert zumindest f\"{u}r fast alle $z\in (1 / 2, 1)$, weil $|xyz|$ integrierbar in $\R^3$ ist, also wir verwenden den Transformationssatz
	\begin{align*}
		\int_{A_z}|x y|\dd{\lambda_2}=&\int_{(0,\sqrt{1-z^2} )}\int_{(0,2\pi)} |r^2\cos\varphi\sin\varphi|r\dd{\varphi}\dd{r}\\
		=&\int_0^{\sqrt{1-z^2} } 2r^3\dd{r}\\
		=&\frac{1}{2}(1-z^2)^2
	\end{align*}
	Nebenrechnung:
	\begin{align*}
		\int_0^{2\pi} |\sin\varphi\cos\varphi|\dd{\varphi}=&4\int_0^{\pi / 2}|\sin\varphi\cos\varphi|\dd{\varphi}\\
		=&4\int_0^{\pi /2}\left| \frac{1}{2}\sin 2\varphi \right| \dd{\varphi}\\
		=&2\int_0^{\pi / 2}\sin 2\varphi\dd{\varphi}\\
		=&2\left[ -\frac{\cos 2\varphi}{2} \right]_0^{\pi / 2}\\
		=&(1 - (-1))=2\\
	\end{align*}
	Daraus folgt:
	\begin{align*}
		\int_A f\dd{\lambda_3}=&\int_{1 / 2}^1 \frac{|z|}{2}(1-z^2)^2\dd{z}\\
		=&\frac{1}{2}\int_{1 / 2}^1 z(1-z^2)^2\dd{z}\\
		=&\frac{1}{2}\int_{1 / 2}^1 z(1-2z^2+z^4)\dd{z}\\
		=&\frac{1}{2}\left[ \frac{z^2}{2}-\frac{z^4}{2}+\frac{z^6}{6} \right]_{1 / 2}^1\\
		=&\frac{9}{256}.\qedhere
	\end{align*}
\end{proof}
