\begin{Problem}
	\textbf{(Hyperbelfunktion)} Sei $d\in \{1,2,3\} , R>0, 1\le p<\infty$ und $\alpha>0$. Definiere
	\[
	B_R(d;0):=\{x\in \R^d|\|x\|<R\} , \qquad f:\R^d\to \R, x\to \begin{cases}
		0 & x=0,\\
		\frac{1}{\|x\|^\alpha} & x\neq 0.
	\end{cases}
	.\] 
	\begin{parts}
		\item Sei zunächst $d=1$. F\"{u}r welche $\alpha, p$ ist die Funktion $\chi_{B_R(1;0)}f$ in $L^p(\lambda_1)$? F\"{u}r welche $\alpha,p$ ist die Funktion $\chi_{\R\backslash B_R(1;0)}f$ in $L^p(\lambda_1)$?
		\item Welche Bedingungen müssen in den Fällen $d=2,3$ f\"{u}r $\alpha, p$ gelten, damit $\chi_{B_R(d;0)}f\in L^p(\lambda_d)$ ist?
		\item Sei $1<p<r<q<\infty$. Geben Sie eine Funktion $g:\R\to \R$ an mit $g\in L^r(\lambda_1)$, $g \not\in L^p(\lambda_1)$, $g \not\in L^q(\lambda_1)$.
		\item Geben Sie ein Beispiel f\"{u}r eine Funktion $g:(0,1)\to \R$ an, sodass $g\in L^p(\lambda_1)$ f\"{u}r alle $p\in [1,\infty)$ gilt, aber $g \not\in L^\infty(\lambda_1)$.
	\end{parts}
\end{Problem}
\begin{proof}
	\begin{parts}
	\item Die Funktion $\chi_{B_R(1;0)}f$ ist in $L^p(\lambda_1)$ genau dann, wenn
	\begin{align*}
		\int \chi_{B_R(1;0)}|f|^p \dd{\lambda_1}=&\int_{B_R(1;0)} |f|^p \dd{\lambda_1}\\
		=&\int_{B_R(1;0)}\frac{1}{\|x\|^{\alpha p}}\dd{\lambda_1}\\
		=&\int_0^R \frac{1}{\|x\|^{\alpha p}}\dd{\lambda_1}+\int_{-R}^0 \frac{1}{\|x\|^{\alpha p}}\dd{\lambda_1}
	\end{align*}
	Weil die Funktionen positiv sind, sind sie Lebesgue-Integrierbar genau dann, wenn sie (uneigentlich) Riemann-Integrierbar sind. Wir wissen aber auch, dass
	\begin{align*}
		\int_0^R \frac{1}{\|x\|^{\alpha p}}\dd{\lambda_1}=&\int_0^R \frac{1}{x^{\alpha p}}\dd{\lambda_1} & x>0\\
		=&\lim_{a \to 0} \int_a^R \frac{1}{x^{\alpha p}}\dd{x}\\
		=&\lim_{a \to 0}\left[ \frac{x^{-\alpha p + 1}}{1-\alpha p} \right] 
	\end{align*}
	was existiert genau dann, wenn $-\alpha p + 1 \ge 0$. Das Ergebnis stimmt nicht f\"{u}r $\alpha p=1$. In diesem Fall ist
	\[
		\int_a^R \frac{1}{x}\dd{x}=[\ln x]_a^R
	\]
	und der Grenzwert existiert nicht. Aus der Symmetrie von $x\to -x$ gilt genau die gleiche f\"{u}r $\int_{-R}^0 \frac{1}{\|x\|^{\alpha p}}\dd{x}$. Insgesamt ist die Funktion genau dann integrierbar, wenn $-\alpha p + 1 >0$. 

	Ähnlich berechnen wir das Riemann-Integral f\"{u}r $\chi_{\R\backslash B_R(1;0)}$. Es gilt
	\begin{align*}
		\int_R^\infty \frac{1}{\|x\|^{\alpha p}}=&\int_R^\infty \frac{1}{x^{\alpha p}}\dd{x}\\
		=&\lim_{a \to \infty} \int_R^a \frac{1}{x^{\alpha p}}\dd{x}\\
		=&\lim_{a \to \infty} \left[ \frac{x^{-\alpha p+1}}{1-\alpha p} \right]
	\end{align*}
	was genau dann existiert, wwenn $-\alpha p + 1\le 0$. Ähnlich stimmt das Ergebnis nicht f\"{u}r $-\alpha p = 1$ nicht. In diesem Fall ist
	\[
		\int_R^\infty \frac{1}{x}\dd{x}=\lim_{a \to \infty} \ln x |_R^a
	,\]
	was nicht existiert. Also es ist genau dann integrierbar, wenn $-\alpha p + 1 < 0$.
\item $d=2$: Wir berechnen das Integral in Polarkoordinaten. Da die Funktion positiv ist, ist das Integral wohldefiniert. Sie ist genau dann integrierbar, wenn die transformierte Funktion integrierbar ist. Wir wissen, dass $\|x\|=r$. Daraus folgt:
	\begin{align*}
		\int_{B_R(2;0)} \frac{1}{\|x\|^{\alpha p}}\dd{\lambda_2}=&\int_0^R\int_0^{2\pi} \frac{1}{r^{\alpha p}}\dd{\theta}r\dd{r}\\
		=& 2\pi\int_0^R \frac{1}{r^{\alpha p-1}}\dd{r}
	\end{align*}
	Aus dem Argument in (a) existiert das Integral genau dann, wen
	\[
		-(\alpha p - 1)+1=\alpha p+2>0
	.\] 
	\end{parts}
\end{proof}
\begin{Problem}
	Sei $(X, \mathcal{A},\mu)$ ein Maßraum und $1\le p<q\le\infty$.
	\begin{parts}
	\item Zeigen Sie, dass $L^p(\mu)\cap L^q(\mu)\subseteq L^r(\mu)$ f\"{u}r alle $r\in (p,q)$ gilt und außerdem
		\[
			\|f\|_{L^r(\mu)}\le \|f\|_{L^p(\mu)}^\theta \|f\|_{L^q(\mu)}^{1-\theta}
		\] 
		f\"{u}r alle $f\in L^p(\mu)\cap L^q(\mu)$ und $\theta\in (0,1)$ mit
		\[
		\frac{1}{r}=\frac{\theta}{p}+\frac{1-\theta}{q}
		.\] 
	\item Sei der Maßraum $(X,\mathcal{A},\mu)$ nun endlich. Zeigen Sie, dass dann $L^q(\mu)\subseteq L^p(\mu)$ und
		\[
			\|f\|_{L^p(\mu)}\le \mu(X)^{\frac{1}{p}-\frac{1}{q}}\|f\|_{L^q(\mu)}
		\] 
		f\"{u}r alle $f\in L^q(\mu)$ gilt.

		\emph{Hinweis: Betrachten Sie den Raum $L^r(\mu)$ f\"{u}r $r:=\frac{q}{p}$.}
	\end{parts}
\end{Problem}
\begin{proof}
	\begin{parts}
	\item Sei $f\in L^p(\mu)\cap L^q(\mu)$. Das Ziel ist: $f\in L^r(\mu)$ f\"{u}r alle $r\in (p,q)$. Sei
\begin{align*}
	A=&\{x|x\in X,~|f(x)|<1\} \\
	B=&\{x|x\in X,~|f(x)|\ge 1\}=X\backslash A
\end{align*}
Weil $|f|^p$ bzw. $|f|^q$ auf der ganzen Menge $X$ integrierbar sind, sind $|f|^p$ bzw. $|f|^q$ auf $A$ und $B$ integrierbar. Es gilt, f\"{u}r alle $x\in A$,
\[
|f|^q \le |f|^r \le |f|^p
\]
also $|f|^r$ ist auf $A$ integrierbar. Ähnlich ist f\"{u}r alle $x\in B$ 
\[
|f|^p\le |f|^r\le |f|^q
\]
und das Integral von $|f|^r$ auf $B$ existiert. Da
\[
	\int |f|^r\dd{\mu}=\int_A |f|^r\dd{\mu}+\int_B |f|^r\dd{\mu}
,\]
ist $|f|^r$ integrierbar und $f\in L^r(\mu)$. Aus der Höldersche Ungleichung folgt, f\"{u}r $1 / p + 1 / q = 1, p,q\in [1,\infty]$
\[
	\|f^2\|_{\mathcal{L}^1(\mu)}\le \|f\|_{\mathcal{L}^p(\mu)}\|f\|_{\mathcal{L}^q(\mu)}
.\] 
	\end{parts}
\end{proof}
\begin{Problem}
	Sei $f:\R^3\to \R$ definiert durch $f(x,y,z):=|xyz|$ und
	\[
		A:=\left\{ (x,y,z)\in \R^3|x^2+y^2+z^2\le 1, z\ge \frac{1}{2} \right\} 
	.\] 
	Bestimmen Sie $\int_A f\dd{\lambda_3}$.
\end{Problem}
