\begin{Problem}
	 Seien $X, Y$ nichtleere Mengen, $f : X \to Y$ eine Abbildung und $\mathcal A, \mathcal S$ $\sigma$-Algebren über $X$ sowie $B$ eine $\sigma$-Algebra über $Y$. Beweisen oder widerlegen Sie:
	 \begin{parts}
		 \item $\mathcal A \cup \mathcal S$ ist eine $\sigma$-Algebra \"{u}ber $X$.
		 \item $\mathcal A \cap \mathcal S$ ist eine $\sigma$-Algebra \"{u}ber $X$. 
		 \item $\mathcal A \backslash \mathcal S$ ist eine $\sigma$-Algebra \"{u}ber $X$.
		 \item $f ^{-1}(\mathcal B)=\left\{ f^{-1}(B)\subseteq X | B\in\mathcal B \right\}$ ist eine $\sigma$-Algebra \"{u}ber $X$.
		 \item $f(\mathcal A)=\left\{ f(A)\subseteq Y|A\in \mathcal A \right\} $ ist eine $\sigma$-Algebra \"{u}ber $Y$.
	 \end{parts}
\end{Problem}
\begin{proof}
	\begin{parts}
	\item Falsch. Sei
		\begin{align*}
			X=&\left\{ a,b,c \right\} \\
			\mathcal A=&\left\{ \varnothing, \left\{ a,b \right\} , \left\{ c \right\} , X \right\} \\
			\mathcal S=&\left\{ \varnothing, \left\{ a \right\} , \left\{ b,c \right\} , X \right\}\\
		\end{align*}

		Dann ist 
		\[
		A\cup S=\left\{ \varnothing, \left\{ a \right\} ,\left\{ a,b \right\} , \left\{ c \right\} , \left\{ b,c \right\}, X  \right\} 
		.\] 
		keine $\sigma$-Algebra, weil
		\[
		\left\{ a,b \right\} \cap \left\{ b,c \right\} =\left\{ b \right\} \not\in \mathcal A \cup \mathcal S
		.\] 

	\item Richtig.
		\begin{enumerate}[label=(\arabic*)]
			\item $X\in \mathcal A, X\in \mathcal S\implies X \in\mathcal A \cap \mathcal S$
			\item Sei $A\in \mathcal A\cap \mathcal S$. Dann  $A\in \mathcal A$ und $A\in\mathcal S$. 

				Daraus folgt: $A^c\in\mathcal A$ und $A^c\in \mathcal S$. Deswegen ist $A^c\in \mathcal A \cap \mathcal S$.

			\item Sei $(A_j), A_j\in\mathcal A\cap \mathcal S$. Dann gilt:
\begin{align*}
	\bigcup_{j=1} ^\infty A_j\in& \mathcal A\\
	\bigcup_{j=1} ^\infty A_j\in& \mathcal S
\end{align*}
Daraus folgt
\[
	\bigcup_{j=1} ^\infty A_j\in \mathcal A\cap \mathcal S
.\] 
		\end{enumerate}
	\item Falsch. $X\in \mathcal A$, $X\in \mathcal S\implies X\not\in \mathcal A \backslash\mathcal S$
	\item Richtig.
		 \begin{enumerate}[label=(\arabic*)]
			 \item $f^{-1}(Y)=X\in f^{-1}\mathcal{B}$ 
			 \item Sei $A=f^{-1}(B)$
				 \[
					 X-A=f^{-1}(\underbrace{Y-B}_{\in \mathcal B})\in f^{-1}(\mathcal B)
			 .\] 

		 \item Es folgt aus
			 \[
				 \bigcup_{j\in \N} f^{-1}(B_j)=f^{-1}\left( \bigcup_{j\in \N} B_j \right) 
			 .\] 
	 \end{enumerate}
		 \item Falsch. Sei $a\in Y$ und $f$ die konstante Abbildung $f(x)=a\forall x\in X$. Dann gilt
			 \[
			 f(\mathcal A)=\left\{ \varnothing, \left\{ a \right\}  \right\} 
			 \] 
			 was keine $\sigma$-Algebra ist, solange $Y\neq \left\{ a \right\} $.
	\end{parts}
\end{proof}
\begin{Problem}
	\begin{parts}
		
	\item Sei $X := \Q$ und $\mathcal A_\sigma (M)$ die von $M := \{(a, b] \cap Q | a, b \in \Q, a < b\}$ erzeugte $\sigma$-Algebra. Zeigen Sie, dass $\mathcal A_\sigma (M ) = \mathcal{P}(\Q)$ gilt.

 \item Seien $X, Y$ nichtleere Mengen und $f:X\to Y$ eine Abbildung. Zeigen Sie: F\"{u}r $\mathcal M\subseteq \mathcal{P}(Y)$ gilt
	 \[
		 f^{-1}\left( A_\sigma(\mathcal M) \right) =\mathcal A_\sigma\left( f^{-1}(\mathcal M) \right) 
	 .\]
	 Das Urbild von $\mathcal M$ ist hierbei analog zum Urbild einer $\sigma$-Algebra definiert durch
	 \[
		 f^{-1}(\mathcal M):=\left\{ f^{-1}(B)\subseteq X|B\in \mathcal M \right\} 
	 .\] 
	\end{parts}
\end{Problem}

\begin{proof}
	\begin{parts}
	\item $\left\{ q \right\} \in \mathcal A_\sigma(\mathcal M)\forall q\in \Q$, weil
		\[
			\left\{ q \right\} =\bigcap_{n=1}^\infty \left(q-\frac{1}{n}, q\right]\in \mathcal A_\sigma(M) 
		.\] 
		Weil $\Q$ abz\"{a}hlbar ist, sind alle Teilmenge $A\in \mathcal P(\Q)$ abz\"{a}hlbar, daher 
		\[
		\mathcal P(\Q)\subseteq \mathcal A_\sigma\left( \left\{ \left\{ q \right\} | q\in \Q \right\}  \right) \subseteq \mathcal A_\sigma(M)
	\]
		Es ist klar, dass
		\[
		\mathcal A_\sigma(M)\subseteq \mathcal P(\Q)
		.\]
	\item 
	
		Sei $ P=\left\{ \mathcal A| \mathcal A\text{ ist eine }\sigma\text{-Algebra},\mathcal M \subseteq \mathcal A \right\} $. Per Definition ist $\mathcal A_\sigma(\mathcal M)=\bigcap_{\mathcal A\in P} \mathcal A$. Dann ist es zu beweisen: 
		\[
			f^{-1}\left( \bigcap_{\mathcal A\in P} \mathcal A \right) =\bigcap_{\mathcal A\in P} f^{-1}(\mathcal A)\overset{?}{=}\mathcal A_\sigma\left( f^{-1}\left( \mathcal M \right)  \right) 
		.\]
		Jeder $\sigma$-Algebra $f^{-1}\left( \mathcal A \right) $ enthält $f^{-1}\left( \mathcal M \right) $. Daraus folgt, dass
		\[
			\mathcal A_\sigma \left( f^{-1}\left( \mathcal M \right)  \right) \subseteq \bigcap_{\mathcal A\in P} f^{-1}(\mathcal A)
		.\] 

	Jetzt betrachten wir
	\[
		\mathcal{M}':=f_*\left( \mathcal{A}_\sigma \left( f^{-1}(\mathcal{M}) \right)  \right) 
	.\]
	Es ist schon in der Vorlesung bewiesen, dass $\mathcal{M}'$ eine $\sigma$-Algebra ist, die $\mathcal{M}$ und daher auch $\mathcal{A}_\sigma \left( \mathcal{M} \right) $ enth\"{a}lt. Weil $f^{-1}(\mathcal{M}')$ eine $\sigma$-Algebra ist, ist $f^{-1}\left( \mathcal{M}' \right)=\mathcal{A}_\sigma\left( f^{-1}\left( \mathcal{M} \right)  \right)  $. Daraus folgt:
	\[
		f^{-1}\left( \mathcal{A}_\sigma \left( \mathcal{M} \right)  \right) \subseteq f^{-1}\left( \mathcal{ M}' \right) =\mathcal{A}_\sigma\left( f^{-1}\left( \mathcal{ M} \right)  \right) . \qedhere\]
	\end{parts}
\end{proof}

\begin{Problem}
	Wir betrachten $\R^n$ mit der Standardmetrik, also ausgestattet mit der Euklidischen Norm $\|\cdot\|$. F\"{u}r $re>0$ und $x\in \R^n$ sei $B_r(x):=\left\{ y\in \R^n |\|x-y\|<r \right\} $. Definiere außerdem $B_\Q := \left\{ B_r(q)\subseteq \R^n | \Q \ni r > 0, q \in \Q^n \right\} $ und $B_\R :=\left\{ B_r(x)\subseteq \R^n | r > 0, x \in \R^n \right\} $ 
	\begin{parts}
	\item Zeigen Sie: F\"{u}r jeder offene Menge $A\subseteq \R^n$ gilt $A=\cup_{B_r(q)\in M}B_r(q)$ mit
		\[
		M:= \left\{ B_r(q)\in B_\Q| B_r(q)\subseteq A \right\} 
		.\]
	\item Folgern Sie nun $\mathcal A_\sigma (B_\Q)=\mathcal A_\sigma(B_\R)=\mathcal{B}^n$
	\end{parts}
\end{Problem}

\begin{proof}
\begin{parts}
\item Es genügt zu beweisen, dass jeder offene Ball eine Vereinigung von $\Q$-Bälle sind.

	Sei $B_p(x), p\in \R, x\in \R^n$ eine offene Ball. Sei auch $(a_i), a_i\in \Q^n$ eine Folge, für die gilt
\begin{gather*}
	\|x-a_i\|<r \forall i\\
	\lim_{i \to \infty} a_i=x
\end{gather*}
Sei dann
\[
	M_i=B_{r-\|x-a_i\|}(a_i)\in B_\Q
.\] 
Es ist klar, dass jeder $M_i\subseteq B_r(x)$ ist. Wir beweisen auch, dass $\bigcup_{i=1} ^\infty M_i=B_r(x)$.

Sei $y\in B_r(x)$. Es gilt $\|y-x\|=r_0<r$. Sei $\xi=r-r_0$. Weil $\lim_{n \to \infty} a_n=x$, gibt es ein Zahl $a_k$, wof\"{u}r gilt
\[
\|a_k-x\|<\frac{\xi}{2}
.\] 
(Eigentlich existiert unendlich viel, aber die brauchen wir nicht). Es gilt dann
\[
\|y-a_k\|\le \|y-x\|+\|x-a_k\|\le r_0+\frac{\xi}{2}<r-\frac{\xi}{2}<r-\|x-a_i\|
,\] 
also $y\in B_{r-\|x-a_k\|}(a_k)$. Jetzt ist die Ergebnis klar: Weil jeder offene Menge eine Vereinigung von offene B\"{a}lle ist, gilt
\[
A=\bigcup B_p(x)= \bigcup \bigcup B_r(q)
,\] 
wobei $ p\in \R, x\in \R^n$ und $r\in \Q, q\in \Q^n$

\item $\mathcal A_\sigma\left( B_\R \right) =\mathcal B^n$ per Definition. 

	Aus $B_\Q\subseteq B_\R$ folgt $\mathcal A_\sigma\left( B_\Q \right)  \subseteq \mathcal A_\sigma\left( B_\R \right) $

Aus (a) folgt, dass
\[
B_\R\subseteq \mathcal A_\sigma\left( B_\Q \right) 
.\] 
Dann
\[
\mathcal A_\sigma\left( B_\R \right) \subseteq \mathcal A_\sigma\left( \mathcal A_\sigma\left( B_\Q \right)  \right) =\mathcal A_\sigma\left( B_\Q \right) 
.\]
Deswegen
\[
	\mathcal A_\sigma\left( B_\Q \right) =\mathcal A_\sigma\left( B_\R \right) =\mathcal B^n
.\qedhere\] 
\end{parts}
\end{proof}

\begin{Problem}
	Sei $X$ eine Menge, $\mathcal A$ eine $\sigma$-Algebra über $X$ und $\mu : A \to [0, \infty]$ eine Mengenfunktion.
	\begin{parts}
	\item  Sei $\mu$ $\sigma$-subadditiv, $B \in \mathcal A$ und definiere $\mu_B : \mathcal A \to [0,\infty ], \mu_B (A) := \mu(A \cap B)$. Zeigen Sie, dass $\mu_B$ wohldefiniert und eine $\sigma$-subadditive Mengenfunktion ist.
	\item $\mu$ erf\"{u}lle die beiden Eigenschaften
		\begin{enumerate}[label=(\arabic*)]
			\item $\mu(A\cup B)=\mu(A)+\mu(B)$ f\"{u}r alle $A,B\in \mathcal A$ mit $A\cap B=\varnothing$.
			\item $\lim_{n \to \infty} \mu\left( A_n \right) =\mu\left( \bigcup_{n=1} ^\infty A_n \right) $ f\"{u}r alle $(A_n)\subseteq A$ mit $A_1\subseteq A_2\subseteq \dots$.
		\end{enumerate}
		Zeigen Sie, dass $\mu$ $\sigma$-additiv ist.
	\end{parts}
\end{Problem}
\begin{proof}
	\begin{parts}
	\item Weil $B\in\mathcal A$, ist $B\cap A\in\mathcal A \forall A\in\mathcal A$. $\mu_B$ ist daher wohldefiniert. 

	Sei $(A_j), A_j\in\mathcal A$, $\bigcup_{j=1}^\infty A_j\in\mathcal A$. Sei auch $B_j=A_j\cap B\in \mathcal A$. Dann gilt
	\begin{gather*}
		\mu_B\left( \bigcup_{j=1} ^\infty A_j \right) =\mu\left( B\cap \bigcup_{j=1} ^\infty A_j \right)=\mu\left( \bigcup_{j=1} ^\infty B_j \right) \le \sum_{j=1}^\infty \mu(B_j)=\sum_{j=1}^\infty \mu_B(A_j) 
	\end{gather*}
	\item Sei $(A_j), A_j\in\mathcal A$ paarweise disjunkter Menge. Dann definiere $B_j=\bigcup_{i=1}^j A_j$. F\"{u}r $k$ endlich ist es klar,
		\[
			\mu(B_k)=\sum_{i=1}^k A_i
		.\]
		Weil $B_i\subseteq B_{i+1}$, (2) gilt auch:
		\[
			\lim_{n \to \infty} \mu(B_n)=\lim_{n \to \infty} \sum_{i=1}^n A_i=\sum_{i=1}^\infty A_i=\mu\left( \bigcup_{n=1} ^\infty B_n \right) =\mu\left( \bigcup_{n=1} ^\infty A_n \right) 
		.\qedhere\] 
	\end{parts}
\end{proof}
