\begin{Problem}
	(Maß über $\mathcal{P}(\N)$) Für $\lambda \in \R$ definiere
	\[
	\mu_\lambda:\mathcal{P}(\N)\to \overline{\R}, \mu_\lambda(A):=\sum_{k\in A} \exp(\lambda) \frac{\lambda^k}{k!}
	.\] 
	Bestimmen Sie jeweils alle $\lambda\in \R$, f\"{u}r die
	\begin{parts}
		\item $(\N, \mathcal{P}(\N), \mu_\lambda)$ ein Maßraum ist.
		\item $\mu_\lambda$ ein endliches Maß ist.
		\item $\mu_\lambda$ ein Wahrscheinlichkeitsmaß ist.
	\end{parts}
\end{Problem}
\begin{proof}
	\begin{parts}
	\item $\mu_\lambda$ ist auf jedem Fall f\"{u}r endliche Teilmengen von $\N$ wohldefiniert. Weil $\sum_{k=0}^\infty \frac{\lambda^k}{k!}$ konvergiert (absolut) f\"{u}r alle $\lambda\in\R$, konvergiert absolut alle Teilfolge. 

		$\mu_\lambda$ ist auch trivialweise additiv.
	\item Das passt f\"{u}r $\lambda \in \R$ 
	\item Wir brauchen 
		\[
		\sum_{k=1}^{\infty} \exp(\lambda) \frac{\lambda^k}{k!}=\exp(\lambda)(\exp(\lambda)-1)=1
		,\] 
		oder 
		\[
		\exp(\lambda)=\frac{1}{2}(1\pm \sqrt{5} )
		.\]
		Weil $\exp(\lambda), \lambda \in \R$ immer positiv ist, gibt es nur eine reelle Lösung:
		\[
		\lambda=\ln\left[ \frac{1}{2}\left( 1+\sqrt{5}  \right)  \right] 
		.\] 
	\end{parts}
\end{proof}
\begin{Problem}
	(vollständiger Maßraum) Sei $X$ eine nichtleere Menge, $(X, \mathcal{A}, \mu)$ ein Maßraum und $B \in \mathcal{A}$. Definiere $\mu_B : \mathcal{A} \to [0, \infty]$, $\mu_B (A) := \mu(A \cap B)$.
	\begin{parts}
	\item Zeigen Sie, dass $\mu_B$ ein Maß über $\mathcal{A}$ ist.
	\item  Beweisen oder widerlegen Sie: Ist $(X, \mathcal{A}, \mu_B)$ ein vollständiger Maßraum, dann auch $(X, \mathcal{A}, \mu)$.
	\item  Beweisen oder widerlegen Sie: Ist $(X, \mathcal{A}, \mu)$ ein vollständiger Maßraum, dann auch $(X, \mathcal{A}, \mu_B)$.
	\end{parts}
\end{Problem}
\begin{proof}
	\begin{parts}
	\item In der Übungsblatt 1. haben wir schon bewiesen, dass es wohldefiniert ist.

		Sei dann $(A_j), A_j\in \mathcal{A}$ eine Folge disjunkte Menge. Es gilt
	\begin{align*}
		&\mu_B\left( \bigcup_{i\in\N} A_i \right) =\mu\left( B\cap \bigcup_{i\in \N} A_i \right)\\
		&=\underbrace{\mu\left( \bigcup_{i\in \N} \left( B \cap A_i \right)  \right)=\sum_{i\in \N}\mu(B\cap A_i)}_{\sigma-\text{additivität von }\mu}=\sum_{i\in \N}\mu_B(A_i)
	\end{align*}
		$\mu_B$ ist dann $\sigma$-Additiv, und daher Maß.
	\item Ja. Sei $\mu(A)=0$. Weil $A\cap B\subseteq A$ ist, gilt auch $\mu_B(A)=0$. Weil $(X,\mathcal{A},\mu_B)$ vollständig ist, ist jede Teilmenge $\mathcal{A}\ni A' \subseteq A$. $(X, \mathcal{A},\mu)$ ist dann vollständig.
	\item Nein. Sei $X=\left\{ a,b,c \right\},B=\left\{ a \right\}, \mathcal{A}=\left\{\varnothing, \left\{ a \right\} , \left\{ b,c \right\}  \right\} $.

		Sei auch $\mu(\left\{ b,c \right\} \neq 0, \mu(X)\neq 0, \mu(\left\{ a \right\} )\neq 0$. Dann ist $(X, \mathcal{A},\mu)$ trivialweise vollständig (es gibt keine Nullmenge), aber $\mu_B(\left\{ b,c \right\} )=\mu(\left\{ a \right\} \cap \left\{ b,c \right\} )=\mu(\varnothing)=0$. Deswegen ist $\left\{ b,c \right\} $ eine Nullmenge in $(X,A,\mu_B)$, aber $\left\{ b \right\} \subseteq \left\{b ,c \right\} \not\in \mathcal{A}$
	\end{parts}
\end{proof}
\begin{Problem}
	\begin{parts}
\item	Seien $K_1 , K_2 \subseteq \mathcal{P}(\R^n)$ mit $\varnothing \in K_i$ für $i = 1, 2$ und $\nu_i : K_i \to [0, \infty]$ mit $\nu_i (\varnothing) = 0$ für $i = 1, 2$. Bezeichne nun mit $\mu^*_i$ die analog zu Satz 1.37 von $\nu_i$ induzierten äußeren Maße. Es existiere ein $\alpha > 0$, so dass
	\[
		\forall I_1\in K_1\exists I_2\in K_2: I_1\subseteq I_2\text{ und }\alpha \nu_2(I_2)\le \nu_1(I_1)
	.\] 
	Zeigen Sie: F\"{u}r alle $A\subseteq \R^n$ gilt $\alpha\mu_2^*(A)\le \mu_1^*(A)$.
\item Vervollständigen Sie den Beweis zu Satz 1.55: Zeigen Sie, dass
	\[
		\lambda_a^*(A)\le \lambda_l^*(A)\le \lambda_n^*(A)\text{ und }\lambda_a^*(A)\le \lambda_r^* (A)\le \lambda_n^*(A)\]
		f\"{u}r alle $A\subseteq \R^n$ gilt.
	\end{parts}
\end{Problem}
\begin{proof}
	\begin{parts}
	\item F\"{u}r alle $\epsilon>0$ gibt es eine Überdeckung $(A_{1,j}),A_{1,j}\in K_1$, f\"{u}r die gilt
		 \begin{align*}
			 \bigcup_{k=1}^\infty A_{1,k}\supseteq& A\\
			 \sum_{k=1}^{\infty} \nu_1(A_{1,k})\le& \mu_1^*(A)+\epsilon 
		\end{align*}
		Es gibt auch per Hypothese eine Folge $(A_{2,k}), A_{2,k}\in K_2, A_{2,k}\supseteq A_{1,k}, \alpha\nu_2(A_{2,k})\le \nu_2(A_{1,k})$. Dann gilt
		\begin{align*}
			\bigcup_{k=1} ^\infty A_{2,k}\supseteq& A\\
			\sum_{k=1}^{\infty} \alpha\nu_2(A_{2,k})\le& \sum_{k=1}^{\infty} \nu_1(A_{2,k})<\mu_1^*(A)+\epsilon
		\end{align*}
		Weil das f\"{u}r alle $\epsilon$ gilt, ist $\alpha\mu_2^*(A)\le \mu_1^*(A)$
	\item Es existiert, f\"{u}r jede Elemente $(a,b)=J\in \mathbb{J}(n)$, ein Element $[a,b)\in\mathbb{J}_l(n)\supseteq (a,b)$, und es gilt  $\text{vol}_n([a,b))\le \text{vol}_n((a,b))$. F\"{u}r jede Elemente  $[a,b)\in\mathbb{J}_l$ existiert auch $\overline{\mathbb{J}}(n)\ni [a,b]\supseteq [a,b)$, f\"{u}r die gilt $\text{vol}_n([a,b])\le \text{vol}_n([a,b))$. Daraus folgt die Behauptung:
		 \[
			 \lambda_a^*(A)\le \lambda_l^*(A)\le \lambda_n^*(A)\text{ f\"{u}r alle }A\subseteq \R^n
		.\] 

		Ähnlich folgt die andere Teil. Man muss nur $(a,b]$ statt $[a,b)$ einsetzen, und alle Aussagen bleiben richtig.\qedhere
	\end{parts}
\end{proof}
\begin{Problem}
	Zeigen Sie folgende Aussagen über das äußere Lebesgue-Maß:
	\begin{parts}
	\item Es gilt $\lambda_n^*(A)=0$ f\"{u}r alle abz\"{a}hlbaren Mengen $A\subseteq \R^n$.
	\item Seien  $A,B\subseteq \R^n$mit  $\lambda_n^*(B)=0$. Dann gilt $\lambda_n^*(A\cup B)=\lambda_n^*(A)$.
	\item Es ist $\lambda_1^*\left( [0,1]\backslash \mathbb{Q} \right) =1$.
\item Es ist $\lambda_2^*\left( \R\times \left\{ 0 \right\}  \right)$ =0
	\end{parts}
\end{Problem}
\begin{proof}
	\begin{parts}
	\item	Sei $A=\left\{ x_1,x_2,\dots \right\} $. Sei auch
		\[
			M_\epsilon=\left\{\left( x_i-\frac{\epsilon}{2^{i+1}},x_i+\frac{\epsilon}{2^{i+1}}\right)|i\in\N  \right\} 
		.\] 
		F\"{u}r jede $\epsilon>0$ ist $M_\epsilon$ eine \"{U}berdeckung von $A$. Es gilt auch:
		\begin{align*}
			\sum_{B \in M_\epsilon} \text{vol}_n(B)=& \sum_{i=1}^\infty \text{vol}_n\left( \left( x_i-\frac{\epsilon}{2^{i+1}},x_i+\frac{\epsilon}{2^{i+1}} \right)  \right) \\
			=&\sum_{i=1}^{\infty} \frac{\epsilon}{2^i}\\
			=&\epsilon
		\end{align*}
		Weil dann $\lambda_n^*(A)\le \epsilon\forall \epsilon>0$, ist $\lambda_n^*(A)=0$.
	\item Weil äußere Maße $\sigma$-subadditiv sind, gilt
		\[
			\lambda_n^*(A\cup B)\le \lambda_n^*(A)+\lambda_n^*(B)=\lambda_n^*(A)
		.\] 
		Weil $A\subseteq A\cup B$, gilt $\lambda_n^*(A)\le \lambda_n^*(A\cup B)$. Daraus folgt:
		\[
		\lambda_n^*(A\cup B)=\lambda_n^*(A)
		.\] 
	\item Weil $\left\{ [0,1] \right\} $ eine \"{U}berdeckung von $[0,1]\backslash \Q$ ist, ist $\lambda_1^*\left( [0,1]\backslash\Q \right) \le 1$. Wegen subadditivität gilt $\lambda_1^*([0,1])=1\le \lambda_1^*([0,1])+\lambda_1^*(\Q\cap [0,1])\le \lambda_1^*([0,1])+\lambda_1^*(\Q)$.Weil $\Q$ abz\"{a}hlbar ist, gilt $\lambda_1^*(\Q)=0$ und daraus folgt $1\le \lambda_1^*\left( [0,1] \right) $. Daher gilt $\lambda_1^*\left( [0,1] \right) =1$
	\item Sei f\"{u}r jeder $\epsilon>0$ $M_\epsilon$ die \"{U}berdeckung. 
		\[
		M_\epsilon=\left\{ \left( \frac{n}{2}-1,\frac{n}{2} \right) \times \left( -\frac{\epsilon}{2^n},\frac{\epsilon}{2^n} \right)|n\in \N  \right\} \bigcup \left\{ \left( -\frac{n}{2},-\frac{n}{2}+1 \right) \times \left( -\frac{\epsilon}{2^n},\frac{\epsilon}{2^n} \right)|n\in \N  \right\} 
		.\] 
		Es gilt $\sum_{A\in M_\epsilon} \text{vol}_2(A)=2\epsilon$. Weil $M_\epsilon$ f\"{u}r alle $\epsilon>0$ eine \"{U}berdeckung von $\R\times \left\{ 0 \right\} $ ist, gilt $\lambda_2^*\left( \R\times \left\{ 0 \right\}  \right) =0$.
	\end{parts}
\end{proof}
