\begin{Problem}
	Sei $(X,\mathcal{A},\mu)$ ein Maßraum und $f:X\to \R$ integrierbar. Zeigen Sie:
	\begin{parts}
		\item F\"{u}r $(E_j)\subseteq \mathcal{A}$ paarweise disjunkt mit $E:=\bigcup_{j=1}^\infty E_j$ gilt
			\[
				\int_E f\dd{\mu}=\sum_{j=1}^\infty \int_{E_j} f\dd{\mu}
			.\] 
		\item Sei nun $X:=\R$ und $A_n:=\left\{ x\in \R| |x|\ge n \right\} =(-\infty,n] \cup [n,\infty)$. F\"{u}r alle $\epsilon>0$ existiert ein $N\in \N$, sodass
			\[
				\left| \int_{A_n}f\dd{\mu} \right| <\epsilon
			\]
			f\"{u}r alle $n\ge N$ gilt.
	\end{parts}
\end{Problem}
\begin{proof}
	\begin{parts}
	\item Wir wissen (Satz 2.39), dass $|f|$ integrierbar ist mit Integral $\int |f|\dd{\mu}<\infty$. 

		Wir betrachten dann die Funktionfolge
		\[
			f_n=\sum_{j=1}^n \chi_{E_j}f
		.\] 
		$f_n$ konvergiert gegen $f$, und es gilt $|f_n(x)|\le |f(x)|$ f\"{u}r alle $x$, also die Folge ist durch $|f(x)|$ dominiert. Es folgt:
		\begin{align*}
			\int_E f\dd{\mu}=&\lim_{n \to \infty} \left[ \int_E f_n \dd{\mu} \right] \\
			=& \lim_{n \to \infty}\left[ \int_{E_j}\sum_{j=1}^n f\dd{\mu} \right] \\
			=&\lim_{n \to \infty} \sum_{j=1}^n\int_{E_i} f\dd{\mu}\\
			=&\sum_{j=1}^\infty \int_{E_j}f\dd{\mu}.
		\end{align*}
	\item Es gilt $\left| \int_{A_n}f\dd{\mu} \right| \le \int_{A_n}|f|\dd{\mu}$, also wir müssen es nur f\"{u}r $|f|$ beweisen. Wir betrachten die Funktionfolge $g_n=\chi_{A_n}|f|$. $g_n$ konvergiert gegen $0$ f\"{u}r alle $x$. Außerdem gilt $|g_n|\le f$ f\"{u}r alle $n$. Wir verwenden dann den dominierte konvergenz Satz:
		\[
			\lim_{n \to \infty} \int_{A_n}|f|\dd{\mu}=\lim_{n \to \infty} \int g_n\dd{\mu}=\int 0\dd{\mu}=0
		.\]
		Aus dem Definition von Konvergenz einer Folge bekommen wir f\"{u}r jedes $\epsilon>0$ eine ganze Zahl $N\in \N$, so dass
		\[
			\int_{A_n}|f|\dd{\mu}<\epsilon
		\]
		f\"{u}r alle $n\ge N$.\qedhere
	\end{parts}
\end{proof}
\begin{Problem}
	Sei $(X,\mathcal{A},\mu)$ ein endlicher Maßraum, $f_k:X\to \R$ eine Folge integrierbare Funktionen, die gleichmäßig gegen ein Funktion $f:X\to \R$ konvergiert.
	\begin{parts}
	\item Zeigen Sie, dass $f$ integrierbar ist mit
		\[
			\int f\dd{\mu}=\lim_{k \to \infty} \int f_k\dd{\mu}
		.\] 
	\item Zeigen Sie, dass auf Voraussetzung $\mu(X)<\infty$ im Allgemein nicht verzichtet werden kann.
	\end{parts}
\end{Problem}
\begin{proof}
	\begin{parts}
	\item $f$ ist messbar (Folgerung 2.25).

	Sei $\epsilon>0$ beliebig. Wir bezeichnen mit $\epsilon$ gleichzeitig eine Zahl und die konstante Funktion $\epsilon(x)=\epsilon~\forall x\in X$. 

	$\epsilon$ ist integrierbar, weil $\int |\epsilon|\dd{\mu}=|\epsilon|\mu(X)<\infty$. Dann ist $|f|+\epsilon:= g$ integrierbar. Weil $f_k$ gleichmäßig konvergiert, gibt es $N\in \N$, so dass $|f(x)-f_n(x)|<\epsilon$ f\"{u}r alle $n\ge N$. Das heißt, dass
	\[
	|f_n(x)|\le |f(x)|+\epsilon
	.\] 
	Dann ist die Funktionfolge f\"{u}r alle $n\ge N$ durch $g$ dominiert: $|f_n|\le g$. Weil nur das Verhalten f\"{u}r $n$ groß wichtig f\"{u}r Konvergenz ist, können wir den Satz von dominierte Konvergenz verwenden, also 
	\[
		\int f\dd{\mu}=\lim_{n \to \infty} \int f_n\dd{\mu}
	.\] 
\item Wir betrachten $(1,\infty)$ mit den eingeschränkten Lebesgue $\sigma$-Algebra und Maß. Sei $\epsilon_j$ eine Folge, $\epsilon_j\in \R,~\epsilon_j\searrow 0$ und die Funktionfolge
	\[
		f_j(x)=\frac{1}{x^{1+\epsilon_j}}
	.\] 
	Die Folge $f_j$ konvergiert gegen $f(x) = 1 / x$. Wir zeigen die Eigenschaften:
	\begin{enumerate}[label=(\roman*)]
		\item Die Funktionfolge konvergiert gleichmäßig.

			Wir berechnen den Fehler f\"{u}r beliebiges $\epsilon>0$ :
			\begin{align*}
				\Delta(x)=&\frac{1}{x}-\frac{1}{x^{1+\epsilon}}\\
				=&\frac{x^{\epsilon}-1}{x^{1+\epsilon}}\\
				\dv{\Delta(x)}{x}=&-x^{-2}+(1+\epsilon)x^{-(2+\epsilon)}=0\\
				-1+(1+\epsilon)x^{-\epsilon}=&0\\
				x^{-\epsilon}=&\frac{1}{1+\epsilon}\\
				x^{\epsilon}=&1+\epsilon\\
				x=&(1+\epsilon)^{1 / \epsilon}\\
				\Delta(x)\le& \frac{\epsilon}{(1+\epsilon)^{\frac{1 + \epsilon}{\epsilon}}}\\
				=& \frac{\epsilon}{(1+\epsilon)^{1+1 /\epsilon}},
			\end{align*}
			also der Fehler ist durch eine streng monoton fallende Funktion eingeschränkt, und die Folge konvergiert gleichmäßig.
		\item $f_j$ ist integrierbar f\"{u}r alle $j\in \N$.

			Dies folgt nicht aus Satz 2.61, weil wir nicht über einem kompakten Interval integrieren. Wir wissen nur, dass das Riemann-integral konvergent ist. Sei $j$ fest und definiere
			\[
				f_{j,n}=\chi_{[1,n]}f_h
			.\] 
			Dann stimmt das Lebesgue-integral mit das Riemann-Integral f\"{u}r alle $n\in \N\cup \left\{ 0 \right\} $. Die Funktionfolge ist auch durch $f_j$ dominiert, also
		\begin{align*}
			\int f_j \dd{\mu}=& \lim_{n \to \infty} \int f_{j,n}\dd{\mu} & \text{dominierte Konvergenz}\\
			=&\lim_{n \to \infty} R-\int_1^n f_j(x)\dd{x} & \text{Satz 2.61}\\
			=&R-\int_1^\infty f_j(x)\dd{x} & \text{Definition}
		\end{align*} 
		Weil wir wissen, dass das uneigentliche Riemann-Integral existiert, ist $f_j$ auch integrierbar f\"{u}r alle $j\in \N$.
		\item Ähnlich haben wir
			\[
				\int \frac{1}{x}\dd{\mu}=R-\int_1^\infty \frac{1}{x}\dd{x}
			.\] 
			Weil das Riemann-Integral auf der rechten Seite nicht existiert, ist $\frac{1}{x}$ auch nicht Lebesgue-integrierbar.\qedhere
	\end{enumerate}
	\end{parts}
\end{proof}
\begin{Problem}
	\begin{parts}
	\item Sei $(X,\mathcal{A},\mu)$ ein $\sigma$-endlicher Maßraum mit $\mu(X)>0$. Zeigen Sie, dass dann eine messbare Funktion $f:X\to \overline{R}$ existiert mit $f>0$ auf $X$ und $0<\int |f|\dd{\mu}<\infty$.
\item Geben Sie einen Maßraum an, f\"{u}r den $\mathcal{L}^1(\mu)=\left\{ 0 \right\} $ gilt.
	\end{parts}
\end{Problem}
\begin{proof}
	\begin{parts}
	\item Per Definition gibt es eine Folge $(A_j),A_j\in \mathcal{A}$, so dass alle $A_j$ endliche Maß haben und $\bigcup_{j=1} ^\infty A_j=X$. Wir definieren $f=\sum_{j=1}^\infty \frac{2^{-j}}{\mu(A_i)}\chi_{A_i}$. Per Definition als eine Reihe von einfache Funktionen gilt ($f$ ist positiv, also wir müssen kein Betrag schreiben):
		\begin{align*}
			\int f\dd{\mu}=\sum_{j=1}^\infty \frac{2^{-j}}{\mu(A_i)}\mu(A_i)=\sum_{j=1}^\infty 2^{-j}.
		\end{align*}
		Die Summe konvergiert und ist endlich.
	\item $(\R, \mathcal{P}(\R), \mu)$, wobei
		\[
		\mu(X)=\begin{cases}
			0 & X=\varnothing\\
			\infty & \text{sonst.}
		\end{cases}
		.\] 
		Sei $f\neq 0$, also es gibt ein Punkt $x_0\in \R$,so dass $|f(x_0)|>0$. Es gilt $|f|\ge |f(x_0)|\chi_{\left\{ x_0 \right\} }$. Es gilt auch
		\[
			\int |f(x_0)|\chi_{\left\{ x_0 \right\} }\dd{\mu}=|f(x_0)|\mu(\left\{ x_0 \right\} )=\infty
		,\]
		also $f$ ist nicht in $\mathcal{L}^1(\mu)$.\qedhere 
	\end{parts}
\end{proof}
\begin{Problem}
\begin{parts}
\item Sei $(X,\mathcal{A},\mu)$ ein Maßraum und $A,B\in \mathcal{A}$ mit $\mu(A)<\infty$ und $\mu(B)<\infty$. Zeigen Sie, dass dann $|\mu(A)-\mu(B)|\le\mu(A\triangle B)$ gilt.
\item Seien $(X,\mathcal{A})$ und $(Y,\mathcal{B})$ messbare Räume und $f:X\times Y\to \overline{R}$ sei $\mathcal{A}\otimes \mathcal{B}$-messbar. Zeigen Sie, dass dann f\"{u}r jedes $x\in X$ die Funktion $f_x(y):=f(x,y)$ $\mathcal{B}$-messbar ist.
\end{parts}
\end{Problem}
\begin{proof}
	\begin{parts}
	\item Es gilt $\mu(A\triangle B)=\mu(A)+\mu(B)-2\mu(A\cap B)$. Wir nehmen oBdA an, dass $\mu(A)\ge \mu(B)$. Dann gilt
		\begin{align*}
			|\mu(A)-\mu(B)|\le& \mu(A)-\mu(B)\\
			=&\mu(A)+\mu(B)-2\mu\left( B \right) \\
			\le& \mu(A)+\mu(B)-2\mu(A\cap B)
		\end{align*}
		da $A\supseteq A\cap B\subseteq B$ und daher $\mu(A)\ge \mu(A\cap B)\le \mu(B)$.
	\item Wir betrachten $C=\left\{ f\le k \right\} \in \mathcal{A}\otimes \mathcal{B}$ (weil $f$ messbar ist). Sei $x\in X$ beliebig. Es gilt
		\[
		\left\{ f_x\le k \right\} =\left\{ y|f(x,y)\le k \right\}=\left\{ y|(x,y)\in C \right\}  
		.\] 
		Aber wir wissen aus Übungsblatt 4, Aufgabe 1(a), dass die Menge $\left\{ y|(x,y)\in C \right\} $ $\mathcal{B}$-messbar ist, also $f_x$ ist messbar f\"{u}r alle $x\in X$.\qedhere
	\end{parts}
\end{proof}
