\begin{Problem}
	Seien $M,N\subseteq \R^n$ $k$-dimensionale Untermannigfaltigkeiten der Klasse $C^\alpha$ sowie $P\subseteq \R^m$ eine $l$-dimensionale Untermannigfaltigkeit der Klasse $C^\alpha$. Zeigen Sie:
	\begin{parts}
		\item $M\times P\subseteq \R^{n+m}$ ist eine $(k+l)$-dimensionale Untermannigfaltigkeit der Klasse $C^\alpha$.
		\item Gilt $M\cap \overline{N}=\varnothing = \overline{M}\cap N$, so ist $M\cup N$ eine $k$-dimensionale Untermannigfaltigkeit der Klasse $C^\alpha$.
		\item Die Mengen
			\begin{align*}
				A:=& \{(x,y)\in \R^2|x\in (-1,1),y=x^2\} ,\\
				B:=& \{(x,y)\in \R^2|x\in (-1,0)\cup (0,1),y=-|x|\} ,
			\end{align*}
			sind jeweils 1-dimensionale Untermannigfaltigkeiten der Klasse $C^1$.
		\item Die Aussage aus (b) ist unter der schwächeren Voraussetzung $M \cap N = \varnothing$ im Allgemeinen nicht richtig.
		\end{parts}
\end{Problem}
\begin{proof}
	\begin{parts}
	\item Sei $(m,p)\in M\times P$. Per Definition gibt es offene Mengen $U\subseteq \R^n,~V\subseteq \R^m$, $f$ und $g$ $\alpha$-mal differenzierbare Funktionen $f:U\to \R^{n-k},~g:V\to \R^{m-l}$, so dass
		\begin{align*}
			&m\in U,~p\in V\\
			&M\cap U=\{x\in U:f(x)=0\}\\
			&\text{Rang}(f'(m))=n-k\\
			&P\cap V=\{x\in V:g(x)=0\} \\
			&\text{Rang}(g'(p))=m-l
		\end{align*}
		Dann ist $U\times V\subseteq \R^{k+l}$ offen Sei außerdem $h:U\times V\to \R^{n+m-(n+k)}$ definiert durch $h(x,y)=(f(x),g(y))$, wobei $x\in \R^{n}$ und $y\in \R^m$. 

Dann ist $h(x,y)=0$ genau dann, wenn $f(x)=0$ und $g(y)=0$. Außerdem ist
\[
	h'=\begin{pmatrix} f' & 0 \\ 0 & g' \end{pmatrix} 
.\] 
Da $h'$ eine Blockmatrix ist, ist $\text{Rang}(h'(m,p))=\text{Rang}(f'(m))+\text{Rang}(g'(p))$. (Man kann das beweisen, indem man das Gauss-Algorithismus durchführt, bis $f'$ und $g'$ in Zeilenstufenform sind.)

Weil $f$ und $g$ $\alpha$-mal stetig differenzierbar sind, ist $h$ auch $\alpha$-mal stetig differenzierbar

Es gilt dann
\begin{align*}
	&(U\times V)\cap (M\times P)=\{x\in U\times V:h(x)=0\} \\
	&\text{Rang}(h'(m,p))=n+m-(k+l)
\end{align*}
\item Sei $x\in M\cup N$, also $x\in M$ oder $x\in N$. OBdA betrachten wir den Fall, $x\in M$. Per Definition gibt es $U\in \R^n$ offen und $f:U\to \R^{n-k}$ $\alpha$-mal stetig differenzierbar, so dass
	\begin{align*}
		M\cap U=&\{y\in U:f(y)=0\} \\
		\text{Rang}(f'(x))=&n-k
	\end{align*}
	Da $M\cap \overline{N}=\varnothing$, ist $M\subseteq \overline{N}^c$. Per Definition ist $\overline{N}^c$ offen. Seien $V:=U\cap \overline{N}^c$ und $g:=f|_V$. Weil $f$ $\alpha$-mal stetig differenzierbar ist, ist $g$ auch $\alpha$-mal stetig differenzierbar. Es gilt
	\begin{align*}
		(M\cup N)\cap V=&M\cap V=\{y\in V:g(y)=0\} \\
		\text{Rang}(g'(x))=&n-k
	\end{align*}
	Ähnlich gilt f\"{u}r $x\in N$. Dann ist $M\cup N$ eine $k$-dimensionale Untermannigfaltigkeit der Klasse $C^\alpha$.
\item Sei $f:\R^2\to \R, f(x,y)=x^2-y$. Sei $p\in A$ und $U$ offen, so dass $p\in U$. Per Definition ist
	\[
	A\cap U=\{x\in U: f(x)=0\} 
	.\] 
	Außerdem ist $f$ mindestens einmal stetig differenzierbar, mit Ableitung $f'=(2x,-1)$. Da $f'$ eine $1\times 2$-Matrix ist, ist $f$ vom höchstens Rang $1$. Weil die zweite Komponente konstant $-1\neq 0$ ist, ist $f$ nie von Rang $0$. Dann ist $f$ immer vom Rang $1$, also $A$ ist eine $1$-dimensionale Untermannigfaltigkeit der Klasse $C^1$.

	Sei jetzt $p\in B$. Wir betrachten den Fall, $\pi_1(p)>0$, wobei $\pi_1((x,y))=x$. Sei $f:\R^2\to \R, f((x,y))=x+y$ und $U$ offen mit $p\in U$. OBdA ist $\pi_1(U)\subseteq (0,\infty)$, sonst ist $(0,\infty)\times \R \cap U$ offen mit gleiche Eigenschaften.

	Dann ist
	\begin{align*}
		U\cap B=&\{x\in U:f(x)=0\} \\
		f'(p)=&(1,1)\\
		\text{Rang}(f'(p))=&1
	\end{align*}
	und analog f\"{u}r $p$ mit $\pi_1(p)<0$. Dann ist $B$ eine Untermannigfaltigkeit der Klasse $C^1$.
\item Wir betrachten $A\cap B$. Es gilt $A\cap B=\varnothing$, weil $x^2=|x|$ nur wenn $|x|=0$ oder $|x|=1$, aber die beide Fälle sind ausgeschlossen.
	\end{parts}
\end{proof}
\begin{Problem}
	Sei $a<b,\alpha\in \N$ und $r:(a,b)\to \R$ sei $\alpha$-mal stetig differenzierbar mit $r(z)>0$ f\"{u}r alle $z\in (a,b)$. Definiere
	\[
	R:=\left\{ (x,y,z)\in \R^3|z\in (a,b), \sqrt{x^2+y^2} =r(z) \right\} 
	.\] 
	Dann ist $R$ durch die Abbildung
	\[
	\varphi:(a,b)\times (0,2\pi)\to \R^3,~\varphi(z,\alpha):=\begin{pmatrix} r(z)\cos\alpha \\ r(z)\sin\alpha \\ z \end{pmatrix} 
\]
parametrisiert.
\begin{parts}
\item Zeigen Sie, dass $R$ eine 2-dimensionale Untermannigfaltigkeit der Klasse $C^\alpha$ ist.
\item Zeigen Sie, dass $R$ eine $\lambda_3$-Nullmenge ist.
\item Bestimmen Sie das Integral
	\[
		I:=\int_{(a,b)\times (0,2\pi)} \sqrt{\text{det}(\varphi'^T\varphi'} \dd{\lambda_2(z,\alpha)}
	\]
	in Abhängigkeit der Funktion $r$.
\item Bestimmen Sie das Integral $I$ in (c) f\"{u}r den Fall $r(z):=\cosh(z)$ und $(a,b):=(0,1)$.
\end{parts}
\end{Problem}
\begin{proof}
	\begin{parts}
	\item Sei $p\in R$. 
\begin{tcolorbox}[title=Lemma]
	\begin{Lemma}
		Ist $p\in \{(0,0,z)|z\in \R\} $, so ist $p\not\in R$.
	\end{Lemma}
	\begin{proof}
		Es gälte dann $\sqrt{x^2+y^2} =0$, also $0>r(z)$, was unmöglich ist, weil $r(z)>0$ per Definition.
	\end{proof}
\end{tcolorbox}
Sei jetzt $f:\R^{3}\to \R,f((x,y,z))=\sqrt{x^2+y^2} -r(z)$. Sei jetzt $p\in R$ und $U\subseteq \R^3$ offen mit $p\in U$. Per Definition ist
\begin{equation}\label{eq:advanalblatt12-1}
R\cap U=\{q\in U:f(q)=0\} 
.\end{equation}
Außerdem ist $f$ stetig differenzierbar mit
 \[
f'=\left( \frac{x}{\sqrt{x^2+y^2} }, \frac{y}{\sqrt{x^2+y^2} },r'(z) \right) 
,\]
solange $(x,y,z) \not\in \{(0,0,z)|z\in \R\} $. Dies ist aber kein Problem wegen des Lemmas. Als Verkettung von elementäre Funktionen sind die ersten zwei Komponenten unendlich mal stetig differenzierbar. $r(z)$ ist bekanntermaßen $\alpha$-mal stetig differenzierbar.

Wenn $f'$ Null ist, muss $x=y=0$ gelten, da $(x^2+y^2)^{-1 / 2}>0$. Dies ist noch einmal wegen des Lemmas ausgeschlossen, also $f'$ ist immer vom Rang $1$.

Zusammen mit Eq.~\eqref{eq:advanalblatt12-1} ist $R$ eine $2$-dimensionale Untermannigfaltigkeit der Klasse $C^\alpha$.
\item $R$ ist messbar, weil $R$ abgeschlossen (und daher eine Borelmenge) ist.

	Da $(\R^3,\lambda_3,\mathcal{L}(3))$ $\sigma$-endlich ist (oder weil es in der Vorlesung beweisen wurde), schreiben wir das Maß als Integral.
\[
	\lambda_3(R)=\int_a^b \lambda_2(R_z)\dd{z}
.\] 
Jetzt betrachten wir $R_z$ und schreiben das Maß aus dem gleichen Grund noch einmal als Integral.
\[
	R_z=\{(x,y)|\sqrt{x^2+y^2} =r(z)\} \subseteq \R^2\\
.\]
Weil $\sqrt{\cdot} $ monoton wachsend ist, muss $|x|<r(z)$ gelten, sonst kann die Bedingung nicht erfüllt werden. Sei jetzt $x$ fest. Es gilt $y^2=r(z)^2-x^2$, oder
\[
y=\pm \sqrt{r(z)^2-x^2} 
.\] 
also $(R_z)_x=\{(x,\sqrt{r(z)^2-x^2} ,z), (x,-\sqrt{r(z)^2-x^2} ,z)\} $. Als endliche Menge ist $\lambda_1((R_z)_x)=0$. Daraus folgt:
\begin{align*}
	\lambda_3(R)=&\int_a^b \lambda_2(R_z)\dd{z}\\
	=&\int_a^b \int_{-r(z)}^{r(z)} \lambda_1((R_z)_x)\dd{x}\dd{z}\\
	=&\int_a^b \int_{-r(z)}^{r(z)} 0\dd{x}\dd{z}\\
	=&0
\end{align*}
	\end{parts}
\end{proof}
