\begin{Problem}
	Sei $\lambda^*_n$ das äußere Lebesgue-Maß und $A \subseteq \R^n$. Zeigen Sie, dass folgende Aussagen äquivalent sind:
	\begin{parts}
		\item $A$ ist $\lambda_n^*$ messbar.
		\item Es gilt $\lambda_n^*\left( A\cap Q \right) +\lambda_n^*\left( A^c\cap Q \right) =\lambda_n^*(Q)$ f\"{u}r alle $Q\in \mathbb{J}(n)$.
	\end{parts}
\end{Problem}
\begin{proof}\noindent 
	\begin{tcolorbox}
	\begin{Definition}
	Sei $\mu^*$ ein äußeres Maß auf $X$. Eine Menge $A\subseteq X$ heißt $\mu^*$-messbar, falls gilt
	\[
		\mu^*(D)=\mu^*(A\cap D)+\mu^*(A^c\cap D)\qquad \forall D\subseteq X
	.\] 
	\end{Definition}
\end{tcolorbox}
	Weil alle Teilmengen $I\in \mathbb{J}(n)$ solche Teilmengen $D\subseteq X$ sind, gilt natürlich (a)$\implies$(b). Jetzt bleibt (b)$\implies$(a) zu zeigen. Es gibt, f\"{u}r jede $\epsilon>0$, eine abzählbare Überdeckung $M=\left\{ Q_i, i\in \N \right\} \subseteq \mathbb{J}$ aus offene Intervale von $D$, f\"{u}r die gilt $\sum_{i=1}^{\infty} \lambda_n^*(Q_i)=\lambda_n^*(D)+\epsilon$. F\"{u}r jede $Q_i\in M$ gilt
	\[
		\lambda_n^*\left( A\cap Q_i \right) +\lambda_n^*\left( A^c\cap Q_i \right) =\lambda_n^*(Q_i)
	.\] 
	Außerdem gilt
	\[
	\sum_{i=1}^{\infty} \lambda_n^*\left( A\cap Q_i \right) +\sum_{i=1}^{\infty} \lambda_n^*\left( A^c\cap Q_i \right) =\sum_{i=1}^{\infty} \lambda_n^*(Q_i)=\lambda_n^*(D)+\epsilon
	.\] 
	Weil $A\cap Q_i$ bzw. $A^c\cap Q_i$ eine abzählbare Überdeckung von $A$ bzw. $A^c$ ist, gilt f\"{u}r $Q=\bigcup_{i=1}^\infty Q_i$:
	\[
	\lambda_n^*(A\cap D)\le \lambda_n^*(A \cap Q)\le \sum_{i=1}^{\infty} \lambda_n^*\left( A\cap Q_i \right) 
	,\]
	und ähnlich
	\[
	\lambda_n^*(A^c\cap D)\le \lambda_n^*(A^c\cap Q)\le \sum_{i=1}^{\infty} \lambda_n^*(A^C\cap Q_i)
	.\] 
	Daraus folgt
	\begin{align*}
		\lambda_n^*(A\cap D)+\lambda_n^*(A^c\cap D)&\le \sum_{i=1}^{\infty} \lambda_n^*(A\cap Q_i)+\sum_{i=1}^{\infty} \lambda_n^*(A^c\cap Q_i)\\
	&=\sum_{i=1}^{\infty} \lambda_n^*(Q_i)\le\lambda_n^*(D)+\epsilon
\end{align*}
Weil $\epsilon>$ beliebig war, ist
\[
	\lambda_n^*(A\cap D)+\lambda_n^*(A^c\cap D)\le \lambda_n^*(D)
.\qedhere\] 
\end{proof}

\begin{Problem}
	 Sei $(X, \mathcal{A}, \nu)$ ein Maßraum und $\mu^*$ das von $(\mathcal{A}, \nu)$ induzierte äußere Maß auf $X$, d.h. in Satz 1.37 ist $K = \mathcal{A}$ und $\nu=\nu$. Nach Satz 1.59 induziert $\mu^*$ ein Maß $\mu := \mu^* |A(\mu^*)$ auf der $\sigma$-Algebra $\mathcal{A}(\mu^*)$.
	 \begin{parts}
	 \item Zeigen Sie, dass $\mu$ eine sogenannte Erweiterung von $\nu$ ist, also dass
		 \begin{enumerate}[label=(\arabic*)]
			 \item $\mathcal{A}\subseteq \mathcal{A}(\mu^*)$ und
			 \item $\mu(A)=\nu(A)$ f\"{u}r alle $A\in \mathcal{A}$ gilt.
		 \end{enumerate}
	 \item Gilt sogar $\mu=\nu$, also $\mathcal{A}=\mathcal{A}(\mu^*)$?
	 \end{parts}
\end{Problem}
\begin{proof}
	\begin{parts}
	\item Wir beweisen zuerst $\mu(A)=\nu(A)$ f\"{u}r alle $A\in \mathcal{A}$. Es genugt zu beweisen, dass $\nu(A)=\mu^*(A)$. Es ist klar, dass $\{A\}$ eine abzählbare Überdeckung von $A$ ist, und daher $\mu^*(A)\le \nu(A)$. Wir betrachten dann eine abzählbare Überdeckung $(Q_i),Q_i\in \mathcal{A},\bigcup_{i=1}^\infty Q_i\supseteq A $. $\mu^*(A)$ ist die Infinum von solchen Folgen von Mengen. Es gilt wegen der Monotonie von $\mu^*$ und der $\sigma$-Additivität von $\nu$: $\sum_{i=1}^{\infty} \nu(Q_i)\ge \nu(A)$. Daraus folgt
		\[
			\mu(A)=\nu(A)\text{ f\"{u}r alle }A\in \mathcal{A}
		.\] 
		Jetzt beweisen wir (1). Sei $A\in \mathcal{A}$. Wir müssen zeigen, das f\"{u}r alle $D\subseteq X$, gilt
		\[
			\mu^*(A\cap D)+\mu^*(A^c\cap D)=\mu^*(D)
		.\] 
		Sei $(Q_i),Q_i\in\mathcal{A}$ eine abzählbare Überdeckung von $D$, f\"{u}r die gilt $\sum_{i=1}^{\infty} \nu(Q_i)\le \mu^*(D)+\epsilon, \epsilon>0$ beliebig.  Betrachten Sie
		\[
			\mu^*(A\cap Q_i)+\mu^*(A^c\cap Q_i)
		.\] 
		Weil sowohl $A$ als auch  $Q_i$ in $\mathcal{A}$ sind, gilt
		\[
			\mu^*(A\cap Q_i)+\mu^*(A^c\cap Q_i)=\mu^*(Q_i)
		.\] 
		Daraus folgt
	\begin{align*}
		\mu^*(A\cap D)+\mu^*(A^c\cap D)\le& \mu^*(A\cap Q)+\mu^*(A^c\cap Q)\\
		\le& \sum_{i=1}^{\infty} \left( \mu^*(A\cap Q_i)+\mu^*(A^c\cap Q_i) \right)\\
		=&\sum_{i=1}^{\infty} \mu^*(Q_i)\le\mu^*(D)+\epsilon
		\end{align*}
		Weil $\epsilon>0$ beliebig war, gilt
		\[
		\mu^*(A\cap D)+\mu^*(A^c\cap D)\le \mu^*(D)
		,\] 
		also $A$ ist messbar.
	\item Nein. Sei zum Beispiel $\mathcal{A}=\mathcal{A}_\sigma\left(\mathbb{J}(n)\right)$, und $\nu:\mathcal{A}\to [0,\infty]$ das eingeschränkte Lebesgue-Maß. Dann ist $\mu^*=\lambda_n^*$, und daher $\mu$ das Lebesgue-Maß. Es gilt aber
		\[
			\left\{ q \right\} \not\in \mathcal{A}_\sigma\left( \mathbb{J}(n) \right), \qquad q\in \R
		,\] 
		obwohl jede Punktmenge $\lambda_n^*$ messbar ist.\qedhere
	\end{parts}
\end{proof}
