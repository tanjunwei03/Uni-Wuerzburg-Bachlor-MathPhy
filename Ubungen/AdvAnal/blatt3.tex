\begin{Problem}
	Sei $\lambda^*_n$ das äußere Lebesgue-Maß und $A \subseteq \R^n$. Zeigen Sie, dass folgende Aussagen äquivalent sind:
	\begin{parts}
		\item $A$ ist $\lambda_n^*$ messbar.
		\item Es gilt $\lambda_n^*\left( A\cap Q \right) +\lambda_n^*\left( A^c\cap Q \right) =\lambda_n^*(Q)$ f\"{u}r alle $Q\in \mathbb{J}(n)$.
	\end{parts}
\end{Problem}
\begin{proof}
	\begin{Definition}
	Sei $\mu^*$ ein äußeres Maß auf $X$. Eine Menge $A\subseteq X$ heißt $\mu^*$-messbar, falls gilt
	\[
		\mu^*(D)=\mu^*(A\cap D)+\mu^*(A^c\cap D)\qquad \forall D\subseteq X
	.\] 
	\end{Definition}
Weil alle Teilmengen $I\in \mathbb{J}(n)$ solche Teilmengen $D\subseteq X$ sind, gilt natürlich (a)$\implies$(b).
\end{proof}

\begin{Problem}
	 Sei $(X, \mathcal{A}, \nu)$ ein Maßraum und $\mu^*$ das von $(\mathcal{A}, \nu)$ induzierte äußere Maß auf $X$, d.h. in Satz 1.37 ist $K = \mathcal{A}$ und $\nu=\nu$. Nach Satz 1.59 induziert $\mu^*$ ein Maß $\mu := \mu^* |A(\mu^*)$ auf der $\sigma$-Algebra $\mathcal{A}(\mu^*)$.
	 \begin{parts}
	 \item Zeigen Sie, dass $\mu$ eine sogenannte Erweiterung von $\nu$ ist, also dass
		 \begin{enumerate}[label=(\arabic*)]
			 \item $\mathcal{A}\subseteq \mathcal{A}(\mu^*)$ und
			 \item $\mu(A)=\nu(A)$ f\"{u}r alle $A\in \mathcal{A}$ gilt.
		 \end{enumerate}
	 \item Gilt sogar $\mu=\nu$, also $\mathcal{A}=\mathcal{A}(\mu^*)$?
	 \end{parts}
\end{Problem}
\begin{proof}
	\begin{parts}
	\item Sei $A\in \mathcal{A}$ 
	\item Nein. Sei zum Beispiel $\mathcal{A}=\mathcal{A}_\sigma\left(\mathbb{J}(n)\right)$, und $\nu:\mathcal{A}\to [0,\infty]$ das eingeschränkte Lebesgue-Maß. Dann ist $\mu^*=\lambda_n^*$, und daher $\mu$ das Lebesgue-Maß. Es gilt aber
		\[
			\left\{ q \right\} \not\in \mathcal{A}_\sigma\left( \mathbb{J}(n) \right), \qquad q\in \R
		,\] 
		obwohl jeder Punktmenge $\lambda_n^*$ messbar ist.
	\end{parts}
\end{proof}
