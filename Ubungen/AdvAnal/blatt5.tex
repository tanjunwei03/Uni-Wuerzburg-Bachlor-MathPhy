
\begin{Problem}
	Sei $(X,\mathcal{A},\mu)$ ein Maßraum und $f:X\to \R$ nicht-negativ und messbar. Es existiere eine Menge $A\in \mathcal{A}$ mit $\mu(A)>0$ und $f(x)>0$ f\"{u}r alle $x\in A$. Zeigen Sie, dass ein $\epsilon>0$ und eine Menge $B\in A$ mit $\mu(B)>0$ existieren, sodass $f(x)>\epsilon$ f\"{u}r alle $x\in B$ gibt.
\end{Problem}

\begin{proof}
	Wir beweisen es per Widerspruch. Wir nehmen an, dass die Behauptung falsch ist. Dann f\"{u}r jedes $\epsilon>0$ ist $\left\{ f>\epsilon \right\} $ eine Nullmenge. (Wir wissen, dass die Menge messbar ist, weil $f$ bessbar ist.)

	Insbesondere gilt es f\"{u}r alle $\Q\ni \epsilon >0$. Wir bezeichnen $\Q^+:=\left\{ x|x\in \Q, x>0 \right\} $. Es gilt
	\[
	A\subseteq \left\{ f>0 \right\} =\bigcup_{x\in \Q^+} \left\{ f>x \right\} 
,\] 
also
\begin{align*}
	\mu(A)\le& \mu(\left\{ f>0 \right\} )\\
	\le& \sum_{x\in \Q^+} \left\{ f>x \right\} \\
	=& \sum_{x\in \Q^+} 0\\
	=& 0
\end{align*}
also $A$ ist eine Nullmenge, ein Widerspruch zu die Annahme, dass $\mu(A)>0$.
\end{proof}
\begin{Problem}
	Sei $(X,\mathcal{A},\mu)$ ein vollst\"{a}ndiger Maßraum und $f_n:X\to \R$ messbar f\"{u}r alle $n\in \N$. Außerdem sei $(f_n)$ punktweise $\mu$-fast überall konvergent, d.h. es existiert ein $\mu$-Nullmenge $N\in \mathcal{A}$ und eine Funktion $f:X\to \R$, sodass $\lim_{n \to \infty} f_n(x)=f(x)$ f\"{u}r alle $x\in X\backslash N$ gilt. Zeigen Sie, dass $f$ messbar ist.
\end{Problem}
\begin{proof}
	Wir wissen, dass sowohl
	\begin{align*}
		& f_s=\limsup_{n\to\infty} f_n\text{ als auch}\\
		& f_l=\liminf_{n\to\infty} f_n
	\end{align*}
	messbar sind. Es gilt auch $f_s(x)=f_l(x),~\forall x\in X\backslash N$. Sei jetzt $U\subseteq \R$ messbar. Es gilt $f^{-1}(U)=\left[f^{-1}(U)\cap N^c\right]\cup \left[f^{-1}(U)\cap N\right]$. 

	Es ist $f^{-1}(U)\cap N\in \mathcal{A}$, weil $f^{-1}(U)\cap N\subset N$, $N$ ist eine Nullmenge und $(X,\mathcal{A},\mu)$ ist vollständig. Es gilt auch
	\[
		f^{-1}(U)\cap N^c=f_{l / s}^{-1}(U)\cap N^c
	,\] 
	wobei $f_{l / s}$ bedeutet, dass entweder $f_l$ oder $f_s$ funktionert (hier möchten wir betonen, dass die beide Funktionen auf $X \backslash N$ gleich sind). Aber weil $f_{l / s}$ messbar sind, ist
	\[
		f^{-1}(U)\cap N^c=\underbrace{f_{l / s}^{-1}(U)}_{\in \mathcal{A}}\cap \underbrace{N^c}_{\in \mathcal{A}}\in \mathcal{A}
	.\] 
	Es ist dann
	\[
		f^{-1}(U)\in \mathcal{A}~\forall U\subseteq \R\text{ messbar}
	,\] 
	also $f$ ist messbar.
\end{proof}
\begin{Problem}
	Sei $A\subsetneq \R$ nichtleer. Bestimmen Sie alle Funktionen $f:\R\to \R$, welche bezüglich der $\sigma$-Algebra $\mathcal{A}:=\left\{ \varnothing, A,A^c, \R \right\} $ messbar sind.
\end{Problem}
\begin{proof}
	Es gilt immer $f^{-1}(\varnothing)=\varnothing$. Außerdem wissen wir aus der Mengentheorie, dass $f^{-1}(A)\cap f^{-1}(A^c)=\varnothing$, und das Bild von einer nichtleeren Menge ist immer nichtleer. Die unterschiedliche Funktionen sind
	\begin{align*}
		& f(A)=A & f(A^c)=A^c & & f(\R)=\R, \\
		& f(A)=A^c & f(A^c)=A & & f(\R)=\R. & \qedhere
	\end{align*}
\end{proof}
\begin{Problem}
	Der Beweis von Lemma 1.83 funktionert anstelle von $[0,1]$ auch analog f\"{u}r jede beliebige Lebesgue-messbare Menge in $\R$, die keine Nullmenge ist. Das heißt f\"{u}r jede solche Menge existiert eine Teilmenge, die nicht Lebesgue-messbar ist.
	Sei nun $f$ die Cantor-Funktion aus Präsenzübung 3 und definiere $g:[0,1]\to [0,2],x\to x+f(x)$.
	\begin{tcolorbox}
		Wir definieren die Funktionenfolge $(f_n)$ durch $f_1:[0,1]\to \R$, $x\to x$ und
		\[
			f_{n+1}:[0,1]\to \R,\qquad x\to \begin{cases}
				\frac{1}{2}f_n(3x) & 0 \le x \le \frac{1}{3},\\
				\frac{1}{2}, \frac{1}{3}<x<\frac{2}{3},\\
				\frac{1}{2}+\frac{1}{2}f_n(3x-2) & \frac{2}{3}\le x \le 1.
			\end{cases}
		.\] 
		Dann konvergiert $f_n$ gleichmäßig gegen eine Grenzfunktion $f$.
	\end{tcolorbox}
	\begin{parts}
		\item Zeigen Sie, dass $g$ bijektiv und die Umkehrfunktion $g^{-1}$ messbar ist.
		\item Zeigen Sie nun, dass der Maßraum $(\R, \mathcal{B}^1, \lambda_1|_{\mathcal{B}^1})$ nicht vollständig ist.
			
			{\footnotesize \emph{Hinweis: Nutzen Sie eine geeignete nicht Lebesgue-messbare Menge.}}
	\end{parts}
\end{Problem}
\begin{proof}
	\begin{parts}
	\item Wir wissen, dass $f$ stetig und monoton wachsend ist.  Daraus folgt, das s $x+f(x)$ stetig und \emph{streng} monoton wachsend ist. Es gilt auch $g(0)=0$ und $g(1)=2$. 

		Das $g$ surjektiv ist folgt aus dem Zwischenwertsatz. $g$ ist auch injektiv. Sei $x>y$. Dann gilt $g(x)>g(y)$, insbesondere $g(x)\neq g(y)$.

		Es gilt $(g^{-1})^{-1}=g$, also wir betrachten eine messbare Teilmenge von $[0,1]$.  
	\end{parts}
\end{proof}
