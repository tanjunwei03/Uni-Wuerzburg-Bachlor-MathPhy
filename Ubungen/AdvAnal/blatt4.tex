\begin{Problem}
	\begin{parts}
		\item Seien $(X,\mathcal{A}), (Y,\mathcal{B})$ messbare R\"{a}ume, $C\in \mathcal{A}\otimes \mathcal{B}$ und $a\in X$. Zeigen Sie, dass
			\[
			\left\{ y\in Y|(a,y)\in C \right\} \in \mathcal{B}
			.\] 
		\item Sei $K\subseteq \R^m$ kompakt und $N\subseteq \R^n$ eine $\lambda_n$-Nullmenge. Zeigen Sie, dass dann $K\times N$ eine $\lambda_{m+n}$-Nullmenge ist.
		\item Sei $M\subseteq \R^m$ eine $\lambda_m$-Nullmenge und $N\subseteq \R^n$ eine $\lambda_n$-Nullmenge. Zeigen Sie, dass dann $M \times N$ eine $\lambda_{m+n}$-Nullmenge ist.
		\item Zeigen Sie Bemerkung 1.71, also dass $\mathcal{L}(m)\otimes \mathcal{L}(n)\subsetneq \mathcal{L}(m+n)$.

			{\footnotesize Hinweis: Sie dürfen hierfür annehmen, dass $B\not\in \mathcal{L}(n)$ tatsächlich existiert.}
	\end{parts}
\end{Problem}
\begin{proof}
	\begin{parts}
	\item Zuerst betrachten wir dem Fall, in dem $C=A\times B\in \mathcal{A}\times \mathcal{B}$. Es gilt dann
		\[
		\left\{ y\in Y|(a,y)\in C \right\} =\begin{cases}
			B & x\in A\\
			\varnothing & x \not\in A
		\end{cases}
		.\]
		Weil sowohl $\varnothing$ als auch $B$ Elemente von $\mathcal{B}$ sind, gilt die Aussage f\"{u}r solche Mengen $C\in A \times B$.

		Jetzt beweisen wir die Aussage im Allgemein: Sei $C\in A\otimes B$. Dann gilt entweder:
		\begin{enumerate}[label=(\roman*)]
			\item C ist eine abzählbare Vereinigung von Mengen $C=\bigcup_{i=1}^\infty A_i\times B_i$, wobei $A_i\times B_i\in \mathcal{A}\times \mathcal{B}$.\label{enum:advanalblatt4-1}
			\item $C$ ist eine abzählbare Schnitt von Mengen $C=\bigcap_{i=1}^\infty A_i\times B_i$, wobei $A_i\times B_i\in \mathcal{A}\times \mathcal{B}$. \label{enum:advanalblatt4-2}
		\end{enumerate}
Sei dann $M\subseteq \N$ die Zahlen, sodass $a\in A_i \iff i\in M$. Es gilt, in der beiden Fälle
\[
\left\{ y\in Y|(a,y)\in C \right\}=\begin{cases}
	\bigcup_{i\in M} B_i & \ref{enum:advanalblatt4-1},\\
	\bigcap_{i\in M} B_i & \ref{enum:advanalblatt4-2}.
\end{cases}
\] 
was immer ein Element von $\mathcal{B}$ ist.
	\item Sei $\epsilon>0$ gegeben, und $\left\{ A_1,A_2,\dots,A_n \right\},A_i\in\mathbb{J}(m)$ eine endliche Überdeckung von $K$, wobei jedes $\lambda_m(A_i)<\infty$. 
		\begin{tcolorbox}
			Das ist immer möglich, weil $K$ kompakt ist. Wir können dann eine Überdeckung $(A_i)$ von $K$ wählen, und die Kompaktheit von $K$ liefert eine endliche Teilüberdeckung. 
		\end{tcolorbox}
		Sei dann $A=\text{max}\left( \lambda_m(A_1),\lambda_m(A_2),\dots, \lambda_m(A_n) \right)$. Per Definition (N ist eine $\lambda_n$-Nullmenge) gibt es eine abzählbare Überdeckung $(B_j),B_j\subseteq \R^n$, sodass
		 \[
		\sum_{i=1}^{\infty} \lambda_n(B_i)<\frac{\epsilon}{nA}
		.\] 
		Wir betrachten dann
		\[
		M=\bigcup_{j=1}^n\left\{ A_j\times B_i|i\in \N \right\}  
		,\]
		was eine abzählbare Überdeckung von $K\times N$ ist, aber $\sum_{C\in M}\lambda_{n+m}(C)<\epsilon$, also $K\times N$ ist eine $\lambda_{m+n}$-Nullmenge.
	\item Wir haben, f\"{u}r alle Würfel $I_m\in \R^m$ und $I_n\in \R^n$, dass
		\[
			\lambda_{n+m}^*(I_n\times I_m) =\lambda_n^*(I_n)\lambda_m^*(I_m)
		.\] 
	Sei $\epsilon>0$ gegeben. Per Definition gibt es eine Überdeckung $(B_i),B_i\in \R^n$ von $N$, sodass
	\[
	\bigcup_{i=1} ^\infty B_i\supseteq N\qquad \sum_{i=1}^{\infty} \lambda_n^*(B_i)\le 1
	.\] 
	Wir haben daher
\[
\lambda_n^*\left( \bigcup_{i=1} ^\infty B_i \right) \le \sum_{i=1}^{\infty} \lambda_n^*(B_i)\le 1
.\] 
Sei $B=\bigcup_{i=1}^\infty B_i$. Es ist also $\lambda_n(B)\le 1$. Sei $(A_i),A_i\in \R^m$ eine Überdeckung von $M$, f\"{u}r die gilt
	\[
	\sum_{i=1}^{\infty} \lambda_m^*(A_i)<\epsilon
	.\] 
	Jetzt ist $A_i\times B$ eine abzählbare Überdeckung von $N\times M$, und
	\[
		\sum_{i=1}^\infty\lambda_{n+m}^*(A_i\times B)\le \sum_{i=1}^\infty \lambda_n^*(A_i)<\epsilon
	,\]
	also $M\times N$ ist eine $\lambda_{n+m}$-Nullmenge.
\item Wie im Skript betrachten wir $\left\{ 0 \right\} \times B, B\not\in  \mathcal{L}(n)$. Es gilt $\left\{ 0 \right\} \times B\in \mathcal{L}(m+n)$, weil $\left\{ 0 \right\} \times B$ eine $\lambda_{n+m}$-Nullmenge ist.

	Jetzt versuchen wir, $\left\{ 0 \right\} \times B$ als eine abzählbare Schnitt
	\[
		\left\{ 0 \right\} \times B\overset{?}{=}\bigcap_{i=1}^\infty \underbrace{A_i\times B_i}_{\in \mathcal{L}(m)\times \mathcal{L}(n)}
	\] 
	darzustellen. Aber
	\[
	\bigcap_{i=1}^\infty A_i\times B_i=\left( \bigcap_{i=1} ^\infty A_i \right) \times \left(\bigcap_{i=1}^\infty B_i\right)
	,\] 
	und wir wissen schon, dass es keine Folgen von Mengen $(B_i),B_i\in \mathcal{B}$ existiert, sodass $\bigcap_{i=1}^\infty B_i=B$, also $\left\{ 0 \right\} \times B\not\in \mathcal{M}\otimes \mathcal{N}$.\qedhere
	\end{parts}
\end{proof}
\begin{Problem}
	Sei $A\in \mathcal{L}(n)$. Beweisen oder widerlegen Sie:
	\begin{parts}
	\item Es gilt
		\[
			\lambda_n(A)=\text{inf}\left\{ \lambda_n(K)|K\supseteq A,~K\text{ kompakt} \right\} 
		.\] 
	\item Es gilt
		\[
			\lambda_n(A)=\text{sup}\left\{ \lambda_n(O)|O\subseteq A,~O\text{ offen} \right\} 
		.\] 
	\end{parts}
\end{Problem}
\begin{proof}
	\begin{parts}
	\item Falsch. Betrachten Sie $A=\Q$. Weil $\Q$ nicht beschränkt ist, gibt es keine kompakte Mengen $K$ mit $K\supseteq A$. Deswegen ist 
		\[0=\lambda_n(\Q)\neq\text{inf}\left\{ \lambda_n(K)|K\supseteq A,~K\text{ kompakt} \right\}=\infty.\] 
		\begin{tcolorbox}[title=Bemerkung]
			Erinnern Sie sich daran, dass wir $\text{inf}(\varnothing)=\infty$ definieren. Sonst kann man $\lambda_n(\R)$ betrachten.
		\end{tcolorbox}
	\item Falsch. Die Intuition dafür ist, dass
		\[\lambda_n(A)=\sup\left\{K\subseteq A, K\text{ kompakt}  \right\} \]
		Wenn wir eingeschränkte Mengen betrachten, gilt dass, nur wenn $K$ abgeschlossen ist. Die meistens abschlossenen Mengen $K$ sind der Abschluss eine offene Menge $K=\overline{U}=U+\partial U$. Wir müssen daher nur eine offene Menge finden, sodass $\partial U$ Maß $>0$ hat.

		Konkretes Gegenbeispiel: Smith-Volterra-Cantor-Menge.

		Wir definieren $S_0=[0,1]$ und weiter induktiv
\[
	S_n=\bigcup_{k=1}^{2^{n-1}}\left(\left[a_k,\frac{a_k+b_k}{2}-\frac{1}{2^{2n+1}}\right]\cup \left[ \frac{a_k+b_k}{2}+\frac{1}{2^{2n+1}},b_k \right]   \right) 
,\] 
wobei $0<a_1<b_1<\dots, a_{2^{n-1}},b_{2^{n-1}}$ durch
\[
	S_{n-1}=\bigcup_{k=1}^{2^{n-1}}[a_k,b_k]
\] 
definiert sind. Sei jetzt $S=\bigcap_{i=1}^\infty S_n$. Wir beweisen jetzt die Eigenschaften der Menge:
\begin{enumerate}[label=(\roman*)]
	\item Alle Intervalle, aus den $S_n$ besteht, haben die gleiche Länge.

		Wir beweisen es per Induktion. F\"{u}r $S_0$ ist es trivial, weil es nur ein Intervall gibt. Jetzt nehmen wir an, das es f\"{u}r beliebige $n\in \N$ gilt, also $b_k-a_k=\Delta_n~\forall k\in 2^{n-1}$. Es gilt:
		\[
			\frac{a_k+b_k}{2}-\frac{1}{2^{2n+1}}=a_k+\frac{\Delta_{n-1}}{2}-\frac{1}{2^{2n+1}}
		,\] 
		und auch
		\[
			\frac{a_k+b_k}{2}+\frac{1}{2^{2n+1}}=a_k+\frac{\Delta_{n-1}}{2}+\frac{1}{2^{2n+1}}
		.\] 
Also alle Intervalle haben die Länge
\[
	\Delta_{n}=\frac{\Delta_{n-1}}{2}-\frac{1}{2^{2n+1}}
.\] 
Die Lösung ist
\[
	\Delta_n=2^{-(2n+1)}(1+2^n)
.\] 
\item Das Innere von $S$ ist leer. 

	Wir nehmen an, dass eine offene Intervall $(a,b)\subseteq S$ gibt. Das Intervall hat Länge $b-a$. Aber, weil $\lim_{n \to \infty} \Delta_n=0$ gilt, muss es ein $S_n$ geben, f\"{u}r das gilt, dass die Intervalle Länge $<b-a$ haben, also $(a,b)\not\subseteq S$. 
\item Also $\left\{U| U\subseteq S~U\text{ offen} \right\} =\left\{ \varnothing \right\} $.
\item $\sup\left\{ \lambda_n(O)|O\subseteq S, O\text{ offen} \right\} =\sup\left\{ 0 \right\} =0$.
\item $\lambda_n(S) \neq 0$. In jedem Schritt nehmen wir Intervalle mit Maß
	\[
		\frac{2^{n}}{2^{2n+2}}
	,\] 
	also ingesamt nehmen wir Maß
	\[
		\sum_{n=0}^{\infty} \frac{2^n}{2^{2n+2}}=\frac{1}{2}
	\]
	raus. Daraus folgt $\lambda_n(S)=\frac{1}{2}\neq 0$.\qedhere
\end{enumerate}
	\end{parts}
\end{proof}
\begin{Problem}
	(Maße von Matrixbildern) 
	\begin{parts}
	\item Sei $S \in \R^{n\times n}$ eine invertierbare Matrix und $\mu : \mathcal{B}^n \to [0, \infty]$ ein Maß. Zeigen Sie, dass $\mu_S : \mathcal{B}_n\to [0,\infty],\mu_S(A):=\mu(SA)$ wohldefiniert und ein Maß ist.
	\item Sei $S \in \R^{n\times n}$ nicht invertierbar. Zeigen Sie, dass $\lambda_n(SA)=0$ für alle $A \in L(n)$ gilt.
	\end{parts}
\end{Problem}
\begin{proof}
\begin{parts}
\item 
	\begin{enumerate}[label=(\roman*)]
		\item (Wohldefiniert) Wir müssen nur zeigen, dass $SA$ offen ist. Wir definieren die übliche Norm auf Matrizen $\|A\|=\text{sup}_{|x|=1}|Ax|$. Weil $S$ invertierbar ist, muss $\|A\|>0$ gelten. Sei $x\in A$ und eine $r$, sodass $B_r(x)\subseteq A$. Sei $y\in B_r(x)$. Es gilt
			\[
			|Ay-Ax|\le \|A\| |y-x|\le \|A\|r
		\]
		also $B_{\|A\|r}(Sx)\subseteq SA$. Daraus folgt, dass $SA$ offen ist, und $\mu_S$ wohldefiniert ist.
	\item Wir haben auch, $\mu_S(\varnothing)=\mu(S\varnothing)=\mu(\varnothing)=0$.
	\item Sei  $(A_j), A_j\in \mathcal{L}(n)$ eine Folge von messbare Mengen. Wir betrachten
		\begin{align*}
			\mu_S\left(\bigcup_{i=1} ^\infty A_i\right)=&\mu\left( S\bigcup_{i=1}^\infty A_i\right) \\
			=& \sum_{i=1}^{\infty} \mu(SA_i) &(\sigma\text{-Additivität von $\mu$)}\\
			=&  \sum_{i=1}^{\infty} \mu_S(A_i)
		\end{align*}
		Hier haben wir verwendet, dass $SA_i$ noch paarweise disjunkt ist, weil $S$ bijektiv (insbesondere injektiv) ist.
	\end{enumerate}
\item $\mu_S(A):=\mu(SA)$ ist noch Maß (vorherige Beweis gilt noch): Weil $S$ nicht invertierbar ist, gibt es einen nichttrivialen Kern. Wir entscheiden uns f\"{u}r ein Vektor $v_1\in \text{ker}(S)$. Dann machen wir eine Basisergänzung, um eine Basis zu konstruieren:
	\[
	B=\left\{ v_1,v_2,\dots, v_n \right\} 
	.\] 
	Wir bezeichnen $[0,1)^n\in \mathcal{L}(n)$ als
	\[
	[0,1)^n=\left\{ a_1v_1+a_2v_2+\dots+a_nv_n|(a_1,a_2,\dots,a_n)\in M\subseteq \R^n \right\} 
	.\] 
	f\"{u}r eine Menge $M$. Es gilt dann
	\begin{align*}
		S[0,1)^n=&\left\{ a_1S(v_1)+a_2S(v_2)+\dots +a_nS(v_n)|(a_1,a_2,\dots, a_n)\in M \right\}\\
		=&\left\{ a_2S(v_2)+\dots+a_nS(v_n)|(a_1,a_2,\dots a_n)\in M \right\} 
\end{align*}
	$\left\{ S(v_1),S(v_2),\dots, S(v_n) \right\}$ ist keine Basis, weil der Nullvektor darin vorkommt (vielleicht mehr als einmal). Wir können jedoch eine Basisergänzung machen, die Nullvektoren einzusetzen. Sei die neue Basis
	\[
	\left\{ u_1,u_2,\dots, u_n \right\} 
	.\] 
	Wir nehmen hier zum Beispiel an, dass $u_i=S(v_i)\forall i\ge 2$, also wir müssen nur $S(v_1)=0$ einsetzen. Das Argument ist gleich, wenn das nicht stimmt. Sei $M=(u_1,u_2,\dots, u_n)$. Dann ist $\mu'(A):=\mu(M^{-1}SA)$ ein Maß, und
	\[
		\mu'([0,1)^n)=\mu(\left\{\left\{ 0 \right\} \times (a_2,a_3,\dots, a_n)|(a_1,a_2,\dots, a_n)\in M\right\})=0
	.\] 
	Also $\mu'$ ist ein Bewegungsinvariantes Maß mit $\mu'([0,1)^n)=0$. Daraus folgt, dass
	\[
	\mu(SA)=0~\forall A\in \mathcal{L}(n)
	.\qedhere\] 
\end{parts}	
\end{proof}
