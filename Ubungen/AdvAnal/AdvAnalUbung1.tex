\documentclass[prb,12pt]{revtex4-2}

\usepackage{amsmath, amssymb,physics,amsfonts,amsthm}
\usepackage{enumitem}
\usepackage{cancel}
\usepackage{booktabs}
\usepackage{tikz}
\usepackage{hyperref}
\usepackage{enumitem}
\usepackage{transparent}
\usepackage{float}
\usepackage{multirow}
\newtheorem{Theorem}{Theorem}
\newtheorem{Proposition}{Theorem}
\newtheorem{Lemma}[Theorem]{Lemma}
\newtheorem{Corollary}[Theorem]{Corollary}
\newtheorem{Example}[Theorem]{Example}
\newtheorem{Remark}[Theorem]{Remark}
\theoremstyle{definition}
\newtheorem{Problem}{Problem}
\theoremstyle{definition}
\newtheorem{Definition}[Theorem]{Definition}
\newenvironment{parts}{\begin{enumerate}[label=(\alph*)]}{\end{enumerate}}
%tikz
\usetikzlibrary{patterns}
% definitions of number sets
\newcommand{\N}{\mathbb{N}}
\newcommand{\R}{\mathbb{R}}
\newcommand{\Z}{\mathbb{Z}}
\newcommand{\Q}{\mathbb{Q}}
\newcommand{\C}{\mathbb{C}}
\begin{document}
	\title{Lineare Algebra 1 Hausaufgabenblatt Nr. 1}
	\author{Jun Wei Tan}
	\email{jun-wei.tan@stud-mail.uni-wuerzburg.de}
	\affiliation{Julius-Maximilians-Universit\"{a}t W\"{u}rzburg}
	\date{\today}
	\maketitle

\begin{Problem}
	 Seien $X, Y$ nichtleere Mengen, $f : X \to Y$ eine Abbildung und $\mathcal A, \mathcal S$ $\sigma$-Algebren über $X$ sowie $B$ eine $\sigma$-Algebra über $Y$. Beweisen oder widerlegen Sie:
	 \begin{parts}
		 \item $\mathcal A \cup \mathcal S$ ist eine $\sigma$-Algebra \"{u}ber $X$.
		 \item $\mathcal A \cap \mathcal S$ ist eine $\sigma$-Algebra \"{u}ber $X$. 
		 \item $\mathcal A \backslash \mathcal S$ ist eine $\sigma$-Algebra \"{u}ber $X$.
		 \item $f ^{-1}(\mathcal B)=\left\{ f^{-1}(B)\subseteq X | B\in\mathcal B \right\}$ ist eine $\sigma$-Algebra \"{u}ber $X$.
		 \item $f(\mathcal A)=\left\{ f(A)\subseteq Y|A\in \mathcal A \right\} $ ist eine $\sigma$-Algebra \"{u}ber $Y$.
	 \end{parts}
\end{Problem}
\begin{proof}
	\begin{parts}
	\item Falsch. Sei
		\begin{align*}
			X=&\left\{ a,b,c \right\} \\
			\mathcal A=&\left\{ \varnothing, \left\{ a,b \right\} , \left\{ c \right\} , X \right\} \\
			\mathcal S=&\left\{ \varnothing, \left\{ a \right\} , \left\{ b,c \right\} , X \right\}\\
		\end{align*}

		Dann ist 
		\[
		A\cup S=\left\{ \varnothing, \left\{ a \right\} ,\left\{ a,b \right\} , \left\{ c \right\} , \left\{ b,c \right\}, X  \right\} 
		.\] 
		keine $\sigma$-Algebra, weil
		\[
		\left\{ a,b \right\} \cap \left\{ b,c \right\} =\left\{ b \right\} \not\in \mathcal A \cup \mathcal S
		.\] 

	\item Richtig.
		\begin{enumerate}[label=(\arabic*)]
			\item $X\in \mathcal A, X\in \mathcal S\implies X \in\mathcal A \cap \mathcal S$
			\item Sei $A\in \mathcal A\cap \mathcal S$. Dann  $A\in \mathcal A$ und $A\in\mathcal S$. 

				Daraus folgt: $A^c\in\mathcal A$ und $A^c\in \mathcal S$. Deswegen ist $A^c\in \mathcal A \cap \mathcal S$.

			\item Sei $(A_j), A_j\in\mathcal A\cap \mathcal S$. Dann gilt:
\begin{align*}
	\bigcup_{j=1} ^\infty A_j\in& \mathcal A\\
	\bigcup_{j=1} ^\infty A_j\in& \mathcal S
\end{align*}
Daraus folgt
\[
	\bigcup_{j=1} ^\infty A_j\in \mathcal A\cap \mathcal S
.\] 
		\end{enumerate}
	\item Falsch. $X\in \mathcal A$, $X\in \mathcal S\implies X\not\in \mathcal A \backslash\mathcal S$
	\item Richtig.
		 \begin{enumerate}[label=(\arabic*)]
			 \item $f^{-1}(Y)=X\in f^{-1}\mathcal{B}$ 
			 \item Sei $A=f^{-1}(B)$
				 \[
					 X-A=f^{-1}(\underbrace{Y-B}_{\in \mathcal B})\in f^{-1}(\mathcal B)
			 .\] 

		 \item Es folgt aus
			 \[
				 \bigcup_{j\in \N} f^{-1}(B_j)=f^{-1}\left( \bigcup_{j\in \N} B_j \right) 
			 .\] 
	 \end{enumerate}
		 \item Falsch. Sei $a\in Y$ und $f$ die konstante Abbildung $f(x)=a\forall x\in X$. Dann gilt
			 \[
			 f(\mathcal A)=\left\{ \varnothing, \left\{ a \right\}  \right\} 
			 \] 
			 was keine $\sigma$-Algebra ist
	\end{parts}
\end{proof}

\end{document}
