\begin{Problem}[Problem 1]
	Consider two reservoirs for electrons at thermal equilibrium. Refer to them as left $L$ and right $R$ leads which extend in the $x$ direction. Imagine we apply small voltages to the two leads which tunes their chemical potential by $eV_L$ and $eV_R$.
	
	\begin{itemize}
		\item Starting from the scattering matrix approach, relate the states entering and leaving each lead at the interface.
		
		\item Within the second quantization formalism, write the formula for the electrical current. \textbf{Hint:} it will be useful later to write it as an energy integration.
		
		\item Taking advantage of the elements of the scattering matrix, identify in the previous formula the transmission function.
		
		\item Identify the equilibrium conductance
	
	\[
		G = \frac{e^2}{h} \, \mathrm{Tr}\!\left[ \mathbf{t}^\dagger(E_F)\, \mathbf{t}(E_F) \right],
	\tag{1}
	\]
	
	where $t(E_F)$ is the transmission matrix of our system evaluated at the Fermi energy $E_F$. What is the meaning of the eigenvalues of the hermitian matrix 
	$\mathbf{t}^\dagger(E_F)\mathbf{t}(E_F)$?
		\end{itemize}
\end{Problem}
\begin{proof}
	\begin{parts}
	\item We take each lead to have $n$ channels, each with a quantum number $\pm$ denoting whether a certain wave is moving right ($+$) or left ($-$). Thus, the outgoing waves for the $L$ ($R$) lead are those going in the $-$ ($+$) direction, and the incoming waves are those going in the $+$ ($-$) direction.

		Then, denoting the amplitudes for each as $\va{a}_{L / R}^{\pm}$, where $\va{a}$ is a vector denoting the $n$ channels, the $S$-matrix relates the amplitudes of the ingoing and outgoing waves as
		\[
			\begin{pmatrix} \va{a}_L^- \\ \va{a}_R^+ \end{pmatrix} = \underbrace{\begin{pmatrix} \mathbf{r} & \mathbf{t}' \\ \mathbf{t} & \mathbf{r}' \end{pmatrix} }_{S}\begin{pmatrix} \va{a}_L^+ \\ \va{a}_R^- \end{pmatrix} 
		.\] 
		Due to this constraint, the Hilbert space dimension is $2n$ instead of the $4n$ that we would get from $4n$ independent modes. We can write creation operators for the $2n$ eigenstates as $b_{\lambda, R / L}^\dagger$, where $\lambda$ denotes the channel and
		\begin{align*}
			b_{\lambda, L}^\dagger &= (c_{\lambda, L}^+)^\dagger + \sum_{\lambda'=1}^n \left[r_{\lambda'\lambda} (c_{\lambda',L}^-)^\dagger + t_{\lambda'\lambda} (c_{\lambda',R}^+)^\dagger\right] \\
			b_{\lambda, R}^\dagger &= (c_{\lambda, R}^-)^\dagger + \sum_{\lambda'=1}^{n} \left[ (t')_{\lambda'\lambda}(c_{\lambda', L}^-)^\dagger + (r')_{\lambda'\lambda} (c_{\lambda', R}^+)^\dagger \right] 
		.\end{align*}
	\item In general, the electrical current can be determined through the derivative $j^\mu = \fdv{S[A]}{A^\mu}$, and leads to the expression for free particles coupled to an electric field
		\begin{align*}
			\va{J}&= \va{J}^{\grad} + \va{J}^{\va A} \\
			\va{J}^{\grad} &= \frac{\hbar}{2mi}\left[ \psi^\dagger \grad{\psi} - (\grad{\psi^\dagger})\psi \right]  \\
			\va{J}^{\va A}&=-\frac{q}{m}\va{A}\psi^\dagger\psi
		\end{align*}
		In the case of a DC current, we usually have an electric field but no magnetic field. Hence, we can set $\va{A}=0$. We can write this as the current operator
		\[
			\mathbf{J}=\frac{\hbar}{2mi}\left[ \vec{\partial} + \backvec{\partial} \right] 
		\] 
		where the very dumb arrow denotes that the derivative acts to the left or to the right, because we have to write it as a matrix before second quantising it :((

		We second quantise it in the typical fashion by taking the matrix elements
		\[
			\mathbf{\hat{J}} = \sum_{\lambda\lambda'} \mel{\lambda}{\mathbf{J}}{\lambda'}b^\dagger_\lambda b_{\lambda'} 
		\]
		where we have cheated by letting $\lambda$ include the quantum number $L / R$ because this is a formal expression anyway. In the case of a continuous spectrum, we replace the sum by an integral
		\[
			\mathbf{\hat{J}} = \iint_{\lambda\lambda'}\mel{\lambda}{\mathbf{J}}{\lambda'}b_\lambda^\dagger b_{\lambda'}\dd{\lambda}\dd{\lambda'}
		.\] 
		Now, we assume that the leads are coupled to thermal baths, i.e that the correlation functions are given by, with $\eta\in \{L,R\} $
		\[
			\langle b^\dagger_{\lambda, \eta}b_{\lambda', \eta'} \rangle = \delta_{\eta\eta'}\delta_{\lambda\lambda'} f_{\mu_\eta}(E_\lambda)
		.\] 
		The function $f_\mu$ is the fermi function
		\[
			f_\mu(x) = \frac{1}{e^{\beta(x-\mu)}+ 1}
		\] 
	and the chemical potentials are
	\[
		\mu_{L / R} = eV_{L / R}
	.\] 
	\end{parts}
\end{proof}


\begin{Problem}[Problem 2]
	Consider a wide conductor along $x$, e.g.\ width in $y$ direction is large $W$ while the height in $z$ direction is small. The length in the $x$ direction is large $L$.
	
	Given information: The density of states of a 2D spin degenerate system is
	\[
	\mathcal{D}_0 = \frac{m}{\pi \hbar^2}.
	\]
	The transmission through a wire with length $L$ is
	\[
	T = \frac{L_0}{L + L_0},
	\]
	where $L_0$ is the mean free path. The conductivity is
	\[
	\sigma = e^2 \mathcal{D}_0 D,
	\]
	where
	\[
	D = \frac{v_F L_0}{\pi}
	\]
	is the diffusion coefficient and $v_F$ is the Fermi velocity.
	
	Using this information, relate the previous results for conductance $G$ with Ohm's law. I.e., does the relationship between $G$ and conductivity $\sigma$ correspond to Ohm’s law?
\end{Problem}
