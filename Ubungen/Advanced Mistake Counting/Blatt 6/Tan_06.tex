\documentclass[prb,12pt]{revtex4-2}

\usepackage{amsmath, amssymb,physics,amsfonts,amsthm}
\usepackage[most]{tcolorbox}
\usepackage{enumitem}
\usepackage{cancel}
\usepackage{booktabs}
\usepackage{polynom}
\usepackage{tabularx}
\usepackage{tikz}
\usepackage{hyperref}
\usepackage{enumitem}
\usepackage[normalem]{ulem}
\usepackage{transparent}
\usepackage{caption}
\usepackage{float}
\usepackage{multirow}
\newtheorem{Theorem}{Theorem}
\newtheorem{Proposition}{Theorem}
\newtheorem{Lemma}[Theorem]{Lemma}
\newtheorem{Corollary}[Theorem]{Corollary}
\newtheorem{Example}[Theorem]{Example}
\newtheorem{Remark}[Theorem]{Remark}
\theoremstyle{definition}
\newtheorem{Problem}{Problem}
\theoremstyle{definition}
\newtheorem{Definition}[Theorem]{Definition}
\newenvironment{parts}{\begin{enumerate}[label=(\alph*)]}{\end{enumerate}}
%tikz
\usetikzlibrary{patterns}
\usetikzlibrary{matrix}
%tcolorbox
\tcbset{breakable=true,toprule at break = 0mm,bottomrule at break = 0mm}
% definitions of number sets
\newcommand{\N}{\mathbb{N}}
\newcommand{\R}{\mathbb{R}}
\newcommand{\Z}{\mathbb{Z}}
\newcommand{\Q}{\mathbb{Q}}
\newcommand{\C}{\mathbb{C}}
\allowdisplaybreaks
\setlength{\parindent}{0cm}
\captionsetup[table]{name=Tabelle}

\begin{document}
\title{Fortgeschrittene Fehlerrechnung Übungsblatt 6}
	\author{Jun Wei Tan}
	\email{jun-wei.tan@stud-mail.uni-wuerzburg.de}
	\affiliation{Julius-Maximilians-Universit\"{a}t W\"{u}rzburg}
	\date{\today}
	\maketitle

\section{Bereingung der Untergrund}
Wir führen eine Messung durch und erhalten die folgenden Messwerte für die Zählereignisse eines befüllten Behälters und die Zählereignisse eines \emph{leeren} Behälters.

\setlength{\tabcolsep}{12pt}

\begin{center}
	\begin{tabular}{p{3cm}p{5cm}p{5cm}}
		\toprule
		\textbf{Winkel ($^\circ$)} & \textbf{Zählereignisse befülltes Behälters ($y_f$)} & \textbf{Zählereignisse leeres Behälters ($y_l$)}\\\midrule
		 155 & 184 & 5 \\\midrule
		135 & 134 & 4 \\\midrule
		120 & 99 & 4 \\\midrule
		100 & 49 & 1 \\\midrule
		90 & 53 & 3 \\\midrule
		75 & 55 & 1 \\\midrule
		65 & 70 & 4 \\\midrule
		55 & 81 & 9 \\\midrule
		40 & 130 & 8 \\\midrule
		20 & 216 & 7 \\\bottomrule
	\end{tabular}
\end{center}

Zur Bereinigung der Untergrund müssen wir die Zählereignisse bei einem leeren Behälter vom Zählereignisse bei einem befüllten Behälter abziehen. Es ist allerdings dabei zu beachten, dass die Anzahl der Teilchen in dieser Messung eine Hälfe die Anzahl beim Versuch mit einem befüllten Target, also wir müssen \emph{zweimal} $y_l$ vom $y_f$ abziehen. Wir bezeichnen die Anzahl der Ereignisse ohne Untergrund als $y$.
\begin{equation}
	y=y_f-2y_l
\end{equation}
Der Fehler kann nach Gauss fortgepflanzt werden:
\[\Delta y = \sqrt{(\Delta y_f)^2 + 4(\Delta y_l)^2}\]
	Wir nehmen an, dass die Anzahl der Ereignisse poissonverteilt ist. Daher ist der Fehler in $y_f$ $\Delta y_f=\sqrt{y_f}$ und analog f\"{u}r $\Delta y_l=\sqrt{y_l}$. Daraus ergibt sich
	\begin{equation}
		\Delta y = \sqrt{y_f+4 y_l}
	\end{equation}

\begin{center}
	\begin{tabular}{ccc}
	\toprule
	\textbf{Winkel ($^\circ$)} & $\cos\theta$ & \textbf{Zahlereignisse ohne Untergrund ($y$)}\\\midrule
		155 & -0.906308 & $174 \pm 14$ \\\midrule
135 & -0.707107 & $126 \pm 12$ \\\midrule
120 & -0.5 & $91 \pm 11$ \\\midrule
100 & -0.173648 & $47,0 \pm 7,3$ \\\midrule
90 & 0. & $47,0 \pm 8,1$ \\\midrule
75 & 0.258819 & $53,0 \pm 7,7$ \\\midrule
65 & 0.422618 & $62,0 \pm 9,3$ \\\midrule
55 & 0.573576 & $63 \pm 11$ \\\midrule
40 & 0.766044 & $114 \pm 13$ \\\midrule
20 & 0.939693 & $202 \pm 16$ \\\bottomrule
\end{tabular}
\end{center}

\section{Regression}
Wir suchen die Parameter $a_1,~a_2,~a_3\in \R$, sodass die folgende Funktion
\begin{equation}
	y(x)=a_1 P_0(x)+a_2 P_1(x)+a_3 P_2(x)
\end{equation}
die beste Anpassung an die Daten ist. Dabei sind $P_i,~i\in \{0,1,2\}$ die \emph{Legendre-Polynomen}
\begin{align*}
	P_0(x)&=1\\
	P_1(x)&=x\\
	P_2(x)&=\frac 12 (3x^2-1)
\end{align*}
Die Regressionskoeffizienten ergeben sich durch
\begin{align*}
	a_1&=\frac 1\Delta\begin{vmatrix}
		\sum y_i \frac{f_1(x_i)}{\sigma_i^2} & \sum \frac{f_1(x_i)f_2(x_i)}{\sigma_i^2} & \sum \frac{f_1(x_i)f_3(x_i)}{\sigma_i^2} \\
		\sum y_i \frac{f_2(x_i)}{\sigma_i^2} & \sum \frac{f_2(x_i)f_2(x_i)}{\sigma_i^2} & \sum \frac{f_2(x_i)f_3(x_i)}{\sigma_i^2} \\
		\sum y_i \frac{f_3(x_i)}{\sigma_i^2} & \sum \frac{f_3(x_i)f_2(x_i)}{\sigma_i^2} & \sum \frac{f_3(x_i)f_3(x_i)}{\sigma_i^2}
	\end{vmatrix}\\
	a_2&=\frac 1\Delta\begin{vmatrix}
	\sum \frac{f_1(x_i)f_1(x_i)}{\sigma_i^2} & \sum y_i \frac{f_1(x_i)}{\sigma_i^2} & \sum \frac{f_1(x_i)f_3(x_i)}{\sigma_i^2} \\
	\sum \frac{f_2(x_i)f_1(x_i)}{\sigma_i^2} & \sum y_i \frac{f_2(x_i)}{\sigma_i^2} & \sum \frac{f_2(x_i)f_3(x_i)}{\sigma_i^2} \\
	\sum \frac{f_3(x_i)f_1(x_i)}{\sigma_i^2} & \sum y_i \frac{f_3(x_i)}{\sigma_i^2} & \sum \frac{f_3(x_i)f_3(x_i)}{\sigma_i^2}
\end{vmatrix}\\
	a_3&=\frac 1\Delta\begin{vmatrix}
	\sum \frac{f_1(x_i)f_1(x_i)}{\sigma_i^2} & \sum \frac{f_1(x_i)f_2(x_i)}{\sigma_i^2} &  \sum y_i \frac{f_1(x_i)}{\sigma_i^2} \\
	\sum \frac{f_2(x_i)f_1(x_i)}{\sigma_i^2} & \sum \frac{f_2(x_i)f_2(x_i)}{\sigma_i^2} &  \sum y_i \frac{f_2(x_i)}{\sigma_i^2} \\
	\sum \frac{f_3(x_i)f_1(x_i)}{\sigma_i^2} & \sum \frac{f_3(x_i)f_2(x_i)}{\sigma_i^2} & \sum y_i \frac{f_3(x_i)}{\sigma_i^2}
\end{vmatrix}\\
	\Delta&=\begin{vmatrix}
		\sum \frac{f_1(x_i)f_1(x_i)}{\sigma_i^2} & \sum \frac{f_1(x_i)f_2(x_i)}{\sigma_i^2} & \sum \frac{f_1(x_i)f_3(x_i)}{\sigma_i^2} \\
		\sum \frac{f_2(x_i)f_1(x_i)}{\sigma_i^2} & \sum \frac{f_2(x_i)f_2(x_i)}{\sigma_i^2} & \sum \frac{f_2(x_i)f_3(x_i)}{\sigma_i^2} \\
		\sum \frac{f_3(x_i)f_1(x_i)}{\sigma_i^2} & \sum \frac{f_3(x_i)f_2(x_i)}{\sigma_i^2} & \sum \frac{f_3(x_i)f_3(x_i)}{\sigma_i^2}
	\end{vmatrix}
\end{align*}
Wir berechnen den Fehler analog wie Blatt 5. Daraus ergeben sich die Bestwerte
\begin{align*}
	a_1=&93,4\pm 3,4\\
	a_2=&-11,9\pm 6,6\\
	a_3=&104,3\pm 8,0
\end{align*}
\section{Güte der Anpassung}
Für eine qualitative Schätzung der Güte der Messung verwenden wir ein Residuenplot

\begin{center}
	\includegraphics[width=0.8\textwidth]{plt2.pdf}
\end{center}

Insgesamt ist die Streuung ordentlich, also es gibt ungefähr die gleiche Anzahl von Punkte, die oberhalb und unterhalb der Nulllinie sind. Für eine quantitative Schätzung der Güte nutzen wir das $\chi^2$-Test. Das $\chi^2$-Statistik ist gegeben durch
\begin{align*}
	\chi^2=&\sum \frac{(y_i - a_1 f_1(x_i) - a_2 f_2(x_i) - a_3 f_3(x_i))^2}{a_1 f_1(x_i) + a_2 f_2(x_i) + a_3 f_3(x_i)}\\
	=&15,49604105544404
\end{align*}
Die Anzahl der Zwangsbedingungen ist 3 ($a_i,~i\in \{1,2,3\}$). Daher gibt es $7$ Freiheitsgrade. Der $p$-Wert ist daher
\begin{align*}
	p=&\int_{15,49604105544404}^\infty f_{\chi^2(7)}(x)\dd{x}\\
	\approx&0,03014127140085254
\end{align*}
Da der $p$-Wert großer als ein Irrtumsniveau von $0,05$ ist, ist die Streuung mit einer Irrtumswahrscheinlichkeit von 5\% \uline{nicht} durch unsere Anpassung beschrieben. Die Übereinstimmung ist also schlecht. Das kann daran liegen, dass das erste Messpunkt (Abstrahlwinkel = 20$^\circ$) sehr weit weg vom zu erwartenden Wert liegt. Eine Möglichkeit ist, dass es einen systematischen Fehler gibt, der stärker bei höhere Streuungswinkel ist, z.B. die Justierung des Messinstruments.

\begin{center}
	\includegraphics[width=0.8\textwidth]{plt1.pdf}
\end{center}
\end{document}
