\documentclass[prb,12pt]{revtex4-2}

\usepackage{amsmath, amssymb,physics,amsfonts,amsthm}
\usepackage[most]{tcolorbox}
\usepackage{enumitem}
\usepackage{cancel}
\usepackage{booktabs}
\usepackage{polynom}
\usepackage{tikz}
\usepackage{hyperref}
\usepackage{enumitem}
\usepackage{transparent}
\usepackage{float}
\usepackage{multirow}
\newtheorem{Theorem}{Theorem}
\newtheorem{Proposition}{Theorem}
\newtheorem{Lemma}[Theorem]{Lemma}
\newtheorem{Corollary}[Theorem]{Corollary}
\newtheorem{Example}[Theorem]{Example}
\newtheorem{Remark}[Theorem]{Remark}
\theoremstyle{definition}
\newtheorem{Problem}{Problem}
\theoremstyle{definition}
\newtheorem{Definition}[Theorem]{Definition}
\newenvironment{parts}{\begin{enumerate}[label=(\alph*)]}{\end{enumerate}}
%tikz
\usetikzlibrary{patterns}
\usetikzlibrary{matrix}
%tcolorbox
\tcbset{breakable=true,toprule at break = 0mm,bottomrule at break = 0mm}
% definitions of number sets
\newcommand{\N}{\mathbb{N}}
\newcommand{\R}{\mathbb{R}}
\newcommand{\Z}{\mathbb{Z}}
\newcommand{\Q}{\mathbb{Q}}
\newcommand{\C}{\mathbb{C}}
\allowdisplaybreaks
\begin{document}
\title{Fortgeschrittene Fehlerrechnung Übungsblatt 2}
	\author{Jun Wei Tan}
	\email{jun-wei.tan@stud-mail.uni-wuerzburg.de}
	\affiliation{Julius-Maximilians-Universit\"{a}t W\"{u}rzburg}
	\date{\today}
	\maketitle


\section{Nullhypothese}
Nullhypothese: Die Ereignisse sind nach einer Poisson-Verteilung mit Mittelwert $\mu=2,148$ verteilt.
\begin{table}[h]
	{\scriptsize
\begin{tabular}{cccccccccc}
	\toprule
\textbf{Ereignisse} & 0 & 1 & 2 & 3 & 4 & 5 & 6 & 7 & $\ge$8 \\\midrule
H\"{a}ufigkeit & 40 & 85 & 92 & 62 & 25 & 19 & 7 & 4 & 2 \\\midrule
Poisson-Wahrscheinlichkeit & 0,116717 & 0,250709 & 0,269261 & 0,192791 & 0,103529 & 0,044476 & 0,0159224 & 0,0048859 & 0,00170816 \\\midrule
Poisson-Häufigkeit  & 39,217 & 84,2382 & 90,4718 & 64,7778 & 34,7857 & 14,9439 & 5,34993 & 1,64166 & 0,573941 \\\bottomrule
\end{tabular}
}
\end{table}

Beobachtung: Die letzte $2$ Klassen haben theoretische Häufigkeit, die kleine als 5 ist. Wir fassen deswegen die 3 letzte Klassen zusammen.
\begin{table}[h]
	{\footnotesize
		\setlength{\tabcolsep}{4.5pt}
		\begin{tabular}{cccccccc}
			\toprule
\textbf{Ereignisse} &	0 & 1 & 2 & 3 & 4 & 5 & $\ge$6 \\\midrule
H\"{a}ufigkeit &	40 & 85 & 92 & 62 & 25 & 19 & 13 \\\midrule
Poisson-Wahrscheinlichkeit &	0,116717 & 0,250709 & 0,269261 & 0,192791 & 0,103529 & 0,044476 & 0,0225165 \\\midrule
Poisson-Häufigkeit  & 	39,217 & 84,2382 & 90,4718 & 64,7778 & 34,7857 & 14,9439 & 7,56553 \\\bottomrule
		\end{tabular}
	}
\end{table}

$\chi^2$ Statistik:
\begin{align*}
	\chi^2 =& \frac{(40 - 39.217)^2}{39.217}+\frac{(85 - 84.2382)^2}{84.2382}\\
	&+\frac{(92 - 90.4718)^2}{90.4718}+\frac{(62 - 64.7778)^2}{64.7778}\\
	&+\frac{(25 - 34.7857)^2}{34.7857}+\frac{(19 - 14.9439)^2}{14.9439}\\
	&+\frac{(13 - 7.56553)^2}{7.56553}\approx 7,92488
\end{align*}
Bestimmung der Anzahl der Freiheitzgrade
\begin{align*}
\text{Anzahl der Klassen}:&~7\\
\text{Zwangsbedingungen}:&~2\\
\text{Freiheitsgrade}:&~7-2=5
\end{align*}
$p$-Wert
\[p=\int_{7,92488}^\infty f_{\chi^2(5)}(x) \dd{x}\approx 0,160425\]
Da der $p$-Wert großer als das Irrtumsniveau (=0,05) ist, ist die Poisson-Verteilung mit einer Irrtumswahrscheinlichkeit von 5\% Poisson verteilt mit Mittelwert $2,148$
\end{document}
