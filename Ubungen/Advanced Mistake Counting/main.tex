\documentclass[prb,12pt]{revtex4-2}

\usepackage{amsmath, amssymb,physics,amsfonts,amsthm}
\usepackage[most]{tcolorbox}
\usepackage{enumitem}
\usepackage{cancel}
\usepackage{booktabs}
\usepackage{polynom}
\usepackage{tikz}
\usepackage{hyperref}
\usepackage{enumitem}
\usepackage{transparent}
\usepackage{float}
\usepackage{multirow}
\newtheorem{Theorem}{Theorem}
\newtheorem{Proposition}{Theorem}
\newtheorem{Lemma}[Theorem]{Lemma}
\newtheorem{Corollary}[Theorem]{Corollary}
\newtheorem{Example}[Theorem]{Example}
\newtheorem{Remark}[Theorem]{Remark}
\theoremstyle{definition}
\newtheorem{Problem}{Problem}
\theoremstyle{definition}
\newtheorem{Definition}[Theorem]{Definition}
\newenvironment{parts}{\begin{enumerate}[label=(\alph*)]}{\end{enumerate}}
%tikz
\usetikzlibrary{patterns}
\usetikzlibrary{matrix}
%tcolorbox
\tcbset{breakable=true,toprule at break = 0mm,bottomrule at break = 0mm}
% definitions of number sets
\newcommand{\N}{\mathbb{N}}
\newcommand{\R}{\mathbb{R}}
\newcommand{\Z}{\mathbb{Z}}
\newcommand{\Q}{\mathbb{Q}}
\newcommand{\C}{\mathbb{C}}
\allowdisplaybreaks
\begin{document}
\title{Fortgeschrittene Fehlerrechnung Übungsblatt 2}
	\author{Jun Wei Tan}
	\email{jun-wei.tan@stud-mail.uni-wuerzburg.de}
	\affiliation{Julius-Maximilians-Universit\"{a}t W\"{u}rzburg}
	\date{\today}
	\maketitle


\section{Nullhypothese}
Nullhypothese: 
\begin{table}[h]
	{\scriptsize
\begin{tabular}{cccccccccc}
	\toprule
	\textbf{Ereignisse} & 0 & 1 & 2 & 3 & 4 & 5 & 6 & 7 & 8 \\\midrule
H\"{a}ufigkeit & 40 & 85 & 92 & 62 & 25 & 19 & 7 & 4 & 2 \\\midrule
Relative H\"{a}ufigkeit & 0,119048 & 0,252976 & 0,27381 & 0,184524 & 0,0744048 & 0,0565476 & 0,0208333 & 0,0119048 & 0,00595238 \\\midrule
Poisson-Wahrscheinlichkeit & 0,116717 & 0,250709 & 0,269261 & 0,192791 & 0,103529 & 0,044476 & 0,0159224 & 0,0048859 & 0,00131187 \\\bottomrule
\end{tabular}
}
\end{table}

$\chi^2$ Statistik:
\begin{align*}
	\chi^2 =& \frac{(0,119048 - 0,116717)^2}{0,116717}+\frac{(0,252976 - 0,250709)^2}{0,250709}\\&+\frac{(0,27381 - 0,269261)^2}{0,269261}+\frac{(0,184524 - 0,192791)^2}{0,192791}\\&+\frac{(0,0744048 - 0,103529)^2}{0,103529}+\frac{(0,0565476 - 0,044476)^2}{0,044476}\\&+\frac{(0,0208333 - 0,0159224)^2}{0,0159224}+\frac{(0,0119048 - 0,0048859)^2}{0,0048859}\\&+\frac{(0,00595238 - 0,00131187)^2}{0,00131187}\approx 0,0399806
\end{align*}
Bestimmung der Anzahl der Freiheitzgrade
\begin{align*}
\text{Anzahl der Klassen}:&~9\\
\text{Zwangsbedingungen}:&~2\\
\text{Freiheitsgrade}:&~9-2=7
\end{align*}
$p$-Wert
\[p=\int_{0,0399806}^\infty f_{\chi^2(7)}(x) \dd{x}\approx 1\]
\end{document}
