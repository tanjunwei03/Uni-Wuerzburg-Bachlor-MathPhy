\documentclass[prb,12pt]{revtex4-2}

\usepackage{amsmath, amssymb,physics,amsfonts,amsthm}
\usepackage[most]{tcolorbox}
\usepackage{enumitem}
\usepackage{cancel}
\usepackage{booktabs}
\usepackage{polynom}
\usepackage{tabularx}
\usepackage{tikz}
\usepackage{hyperref}
\usepackage{enumitem}
\usepackage{ulem}
\usepackage{transparent}
\usepackage{float}
\usepackage{multirow}
\newtheorem{Theorem}{Theorem}
\newtheorem{Proposition}{Theorem}
\newtheorem{Lemma}[Theorem]{Lemma}
\newtheorem{Corollary}[Theorem]{Corollary}
\newtheorem{Example}[Theorem]{Example}
\newtheorem{Remark}[Theorem]{Remark}
\theoremstyle{definition}
\newtheorem{Problem}{Problem}
\theoremstyle{definition}
\newtheorem{Definition}[Theorem]{Definition}
\newenvironment{parts}{\begin{enumerate}[label=(\alph*)]}{\end{enumerate}}
%tikz
\usetikzlibrary{patterns}
\usetikzlibrary{matrix}
%tcolorbox
\tcbset{breakable=true,toprule at break = 0mm,bottomrule at break = 0mm}
% definitions of number sets
\newcommand{\N}{\mathbb{N}}
\newcommand{\R}{\mathbb{R}}
\newcommand{\Z}{\mathbb{Z}}
\newcommand{\Q}{\mathbb{Q}}
\newcommand{\C}{\mathbb{C}}
\allowdisplaybreaks
\begin{document}
\title{Fortgeschrittene Fehlerrechnung Übungsblatt 3}
	\author{Jun Wei Tan}
	\email{jun-wei.tan@stud-mail.uni-wuerzburg.de}
	\affiliation{Julius-Maximilians-Universit\"{a}t W\"{u}rzburg}
	\date{\today}
	\maketitle

\section{Messung}
Zur Berechnung der Fehler nehmen wir an, dass die Streuung durch eine Poissonverteilung gut beschrieben ist. Der Fehler ist also der quadratische Wurzel des Messwerts. Wir bezeichnen die Anzahl der Zerfälle mit $r$.
\begin{center}
	\begin{tabular}{p{3cm}p{3cm}p{3cm}}
		\toprule
		\textbf{Zeit (s)} & \textbf{Anzahl Zerfälle} & \textbf{Poissonfehler} \\\midrule
	20 & 33 & 5,7 \\\midrule
40 & 30 & 5,5 \\\midrule
60 & 27 & 5,2 \\\midrule
80 & 26 & 5,1 \\\midrule
100 & 23 & 4,8 \\\midrule
120 & 20 & 4,5 \\\midrule
140 & 19 & 4,4 \\\midrule
160 & 17 & 4,1 \\\midrule
180 & 18 & 4,2 \\\midrule
200 & 12 & 3,5 \\\bottomrule
	\end{tabular}
\end{center}
Radioaktive Zerfall ist durch das Zerfallgesetz beschrieben
\begin{equation}\label{eq:1}
	N(t)=N_0e^{-\lambda t}.
\end{equation}
Deren Ableitung ist
\[N'(t)=-\lambda N_0e^{-\lambda t}\]
und die Anzahl der Zerfälle ist damit proportional zu $\lambda e^{-\lambda t}$. Wir schreiben die Anzahl der Zerfälle als $r(t)=k\lambda e^{-\lambda t}$. Es gilt also
\begin{equation}\label{eq:2}
\ln r = \ln (k\lambda) -\lambda t
\end{equation}
Wir plotten $\ln r$ in Abhängigkeit von der Zeit, und die Steigung ist $-\lambda$. Man kann daraus die Halbwertzeit bestimmen, indem man $N(t) = N_0/2$ in Gl.~\eqref{eq:1} einsetzt.
\[
	\frac{N_0}{2} = N_0e^{-\lambda t_{1/2}}
\]
mit Lösung
\begin{equation}
	t_{1/2}=\frac{\ln 2}{\lambda}
\end{equation}
Wir bestimmen also das natürliche Logarithmus der Zerfallanzahl. Dessen Fehler ist durch
\[\Delta(\log r)= \frac{\Delta r}{r}\]
bestimmt. Weiterhin bezeichnen wir die Zeit mit $x$ und $\ln r$ mit $y$.
\begin{center}
	\begin{tabular}{p{2cm}p{2cm}p{2cm}}
		\toprule
\textbf{Zeit (s)} & $\ln r$ & $\Delta(\ln r)$\\\midrule
 20 & 3,50 & 0,17 \\\midrule
40 & 3,40 & 0,18 \\\midrule
60 & 3,30 & 0,19 \\\midrule
80 & 3,26 & 0,20 \\\midrule
100 & 3,14 & 0,21 \\\midrule
120 & 3,00 & 0,22 \\\midrule
140 & 2,94 & 0,23 \\\midrule
160 & 2,83 & 0,24 \\\midrule
180 & 2,89 & 0,24 \\\midrule
200 & 2,48 & 0,29 \\\bottomrule
	\end{tabular}
\end{center}
\section{Ungewichtete Lineare Regression}
Wir finden eine Gerade
\[y=a+bx\]
sodass der Fehler so klein wie möglich ist. Die benötigen Größen sind
\begin{align*}
	N=&10\\
	\sum x_i=&1100\\
	\sum y_i=&30,73579556754441\\
	\sum x_i^2=&154000\\
	\sum x_i y_i=&3220,197114428394
\end{align*}
Dann berechnen wir
\begin{align*}
	\Delta =& N\sum x_i^2- (\sum x_i)^2\\
	=&330000\\
	a=&\frac 1\Delta(\sum x_i^2 \sum y_i-\sum x_i \sum x_i y_i)\\
	=& 3,609380883426078\\
	b=&\frac 1\Delta (N \sum x_i y_i - \sum x_i \sum y_i)\\
	=&-0,004870921151560359
\end{align*}
\end{document}
