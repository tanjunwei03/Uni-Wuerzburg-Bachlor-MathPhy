\documentclass[prb,12pt]{revtex4-2}

\usepackage{amsmath, amssymb,physics,amsfonts,amsthm}
\usepackage[most]{tcolorbox}
\usepackage{enumitem}
\usepackage{cancel}
\usepackage{booktabs}
\usepackage{polynom}
\usepackage{tabularx}
\usepackage{tikz}
\usepackage{hyperref}
\usepackage{enumitem}
\usepackage{ulem}
\usepackage{transparent}
\usepackage{float}
\usepackage{multirow}
\newtheorem{Theorem}{Theorem}
\newtheorem{Proposition}{Theorem}
\newtheorem{Lemma}[Theorem]{Lemma}
\newtheorem{Corollary}[Theorem]{Corollary}
\newtheorem{Example}[Theorem]{Example}
\newtheorem{Remark}[Theorem]{Remark}
\theoremstyle{definition}
\newtheorem{Problem}{Problem}
\theoremstyle{definition}
\newtheorem{Definition}[Theorem]{Definition}
\newenvironment{parts}{\begin{enumerate}[label=(\alph*)]}{\end{enumerate}}
%tikz
\usetikzlibrary{patterns}
\usetikzlibrary{matrix}
%tcolorbox
\tcbset{breakable=true,toprule at break = 0mm,bottomrule at break = 0mm}
% definitions of number sets
\newcommand{\N}{\mathbb{N}}
\newcommand{\R}{\mathbb{R}}
\newcommand{\Z}{\mathbb{Z}}
\newcommand{\Q}{\mathbb{Q}}
\newcommand{\C}{\mathbb{C}}
\allowdisplaybreaks
\begin{document}
\title{Fortgeschrittene Fehlerrechnung Übungsblatt 3}
	\author{Jun Wei Tan}
	\email{jun-wei.tan@stud-mail.uni-wuerzburg.de}
	\affiliation{Julius-Maximilians-Universit\"{a}t W\"{u}rzburg}
	\date{\today}
	\maketitle

\section{Messung}
\begin{center}
\begin{tabular}{cc}
	\toprule
	\textbf{Temperatur ($^{\circ}$C)} & \textbf{Luftfeuchte (g/m$^3$)} \\\midrule 
-15 & 0,47 \\\midrule
-11 & 0,54 \\\midrule
-7 & 0,79 \\\midrule
-3 & 1,32 \\\midrule
1 & 1,13 \\\midrule
5 & 1,26 \\\midrule
9 & 3,02 \\\midrule
13 & 2,99 \\\midrule
17 & 3,15 \\\midrule
21 & 3,90 \\\midrule
25 & 5,04 \\\midrule
29 & 7,72 \\\midrule
33 & 11,02 \\\midrule
37 & 13,42 \\\midrule
41 & 20,85 \\\bottomrule
\end{tabular}
\end{center}
Im Zukunft werden wir die Temperatur mit entweder $T$ oder $x$ bezeichnen und Luftfeuchte mit $\rho$ bezeichnen.
\section{Kovarianz und Korrelationskoeffizient}
\begin{align*}
	N=&~15\\
	\bar{T}=&~13~{}^{\circ}\text{C}\\
	\bar{\rho}=&~5,108~\text{gm}^{-3}\\
	\sigma_T=&~17,282~\text{K}\\
	\sigma_\rho=&~5,6499~\text{gm}^{-3}\\
	\text{Kovarianz }\sigma_{T\rho}=&~83,688~\text{Kgm}^{-3}\\
	\text{Korrelationskoeffizient }r=&0,857095
\end{align*}
\end{document}
