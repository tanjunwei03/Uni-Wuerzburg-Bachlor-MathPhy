\begin{Problem}
	Entscheiden Sie, welche der folgenden Mengen ein Unterraum des jeweils angegebenen Vektorraums sind.
	\begin{parts}
		\item $\left\{ p\in \R[t]|\text{deg}(p)\le 4 \right\} \subseteq \R[t]$.
		\item $\R\backslash \Q$ mit $\R$ als $\Q$-Vektorraum.
		\item $B=\left\{ z\in \C| |z|\le 1 \right\} $ mit $\C$ als $\R$-Vektorraum.
		\item $\left\{ p\in \Q[t]|\forall a\in \Q:\tilde{p}(-a)=\tilde{p}(a) \right\} $, wobei $\tilde{p}$ die zu $p$ gehörige Polynomfunktion bezeichnet, als Teilmenge von $\Q[t]$.
		\item $\left\{ (1,2,3)+t\cdot (4,5,6)|t\in \R \right\} \cup \left\{ (0,0,0) \right\} \subseteq \R^3$.
	\end{parts}
\end{Problem}
\begin{proof}
	\begin{parts}
	\item Ja. Sei $p_1=a_0+a_1t+a_2t^2+a_3t^3+a_4t^4$ und $p_2=b_0+b_1t+b_2t^2+b_3t^3+b_4t^4$. Es gilt
		\begin{align*}
			p_1+p_2=&(a_0+b_0)+(a_1+b_1)t\\
				&+(a_2+b_2)t^2+(a_3+b_3)t^3+(a_4+b_4)t^3
	\end{align*}
	was auch ein Polynom mit Grad $\le 4$ ist.

	Sei $x\in \R$. Dann ist
	\[
	x p_1=xa_0+xa_1t+xa_2t^2+xa_3t^3+xa_4t^4
	,\] 
	noch ein Polynom mit Grad $\le 4$.
\item Nein. Es ist sogar kein Vektorraum, weil $0\not\in \R\backslash \Q$. 
\item Nein. Sei $z\in B$ mit $|z|=a<1$. Dann ist
	\[
	\left| \frac{1}{a^2}z \right| =\frac{1}{a^2}|z|=\frac{1}{a}>1
	,\] 
	also $B$ ist nicht unter Skalarmultiplikation abgeschlossen.
\item Ja. Sei $p_1,p_2$ beliebige Elemente von der Menge, und $x\in \R$ beliebig. Es gilt $\tilde{p_1+p_2}=\tilde{p}_1+\tilde{p}_2$, und per Definition
	\begin{align*}
		\tilde{p}_1(a)=&\tilde{p}_1(-a)\\
		\tilde{p}_2(a)=&\tilde{p}_2(-a)
	\end{align*}
Daraus folgt
\[
	\tilde{p_1+p_2}(a)=\tilde{p}_1(a)+\tilde{p}_2(a)=\tilde{p}_1(-a)+\tilde{p}_2(-a)=\tilde{p_1+p_2}(-a)
.\] 
Außerdem ist $\tilde{x p_1}=x\tilde{p}_1$, und
\[
	\tilde{x p_1}(a)=x\tilde{p}_1(a)=x\tilde{p}_1(-a)=\tilde{x p_1}(-a)
.\] 
\item Nein. Wir wissen, dass $(1,2,3)$ ein Element von unserer Menge ist. Dann sollte $2(1,2,3)=(2, 4, 6)$ auch ein Element sein. Wir würden dann schreiben
	\[
		(1,2,3)+t(4,5,6)=(2,4,6)
	,\] 
	also
	\[
	t(4,5,6)=(1,2,3)
	.\] 
	Aus $6t=3$ haben wir $t=\frac{1}{2}$. Dann ist $5t=\frac{5}{2}\neq 2$, ein Widerspruch, also es ist kein Untervektorraum.\qedhere
	\end{parts}
\end{proof}
\begin{Problem}
	Entscheiden Sie, welche der folgenden Familien im jeweiligen Vektorraum linear unabhängig sind
	\begin{parts}
	\item $((1, 2, 3), (5, 4, 3), (6, 6, 6))$ in $\R^3$ als $\R$-Vektorraum.
	\item $((1, i), (i, -1))$ im $\R$-Vektorraum $\C^2$.
	\item $((1, i), (i, -1))$ im $\C$-Vektorraum $\C^2$.
	\item $(1,1+t,1+t+t^2)$ im $\Z / 2\Z$-Vektorraum $\Z / 2\Z[t]$.
	\item $((1, 2, 3), (5, 4, 3), (6, 6, 7))$ in $\R^3$ als $\Q$-Vektorraum.
	\end{parts}
\end{Problem}
\begin{proof}
	\begin{parts}
	\item Nein, weil
		\[
			(1,2,3)+(5,4,3)+(-1)(6,6,6)=0
		.\] 
	\item Ja. Sei $a,b\in \R$. Es gilt
		\[
		a(1,i)+b(i,-1)=(a+bi,ai-b)
	.\]
	Wir würden $a+bi=0$ und $ai-b=0$. Aber wir wissen, dass das nur möglich ist, wenn $a=b=0$, also $(1,i)$ und $(i,-1)$ sind linear unabhängig.
\item Nein, weil
	\[
	i(1,i)+(-1)(i,-1)=(i,-1)-(i,-1)=0
	,\]
	und wir haben $i\neq 0$.
\item Ja. Sei $a,b,c\in \Z / 2\Z$, sodass
	\[
	a(1+t+t^2)+b(1+t)+c(1)=at^2+(a+b)t+(a+b+c)=0
	.\] 
	Per Defintion der Nullpolynom gilt $a=0$. Dann ist $b+a=0$, oder $b=0$. Zuletzt gilt $a+b+c=0$, also $c=0$ weil $a=b=0$.
\item  Ja. Sei $a,b,c\in \R$, sodass $a(1,2,3)+b(5,4,3)+c(6,6,7)=0$. Es gilt
\begin{align*}
	-6c=&a+5b\\
	-6c=&2a+4b\\
	-7c=&3a+3b
\end{align*}
Aus den zweiten und ersten Gleichungen folgt
\[
a+5b=2a+4b
,\]
also $a=b$. Aus der ersten oder zweiten Gleichung folgt $c=-a$. Aus der dritten Gleichung folgt
\[
-7(-a)=7a=3a+3b=6a
.\]
Weil $7a=6a$ muss $a=0$ sein. Dann ist $b=c=0$. Also die Vektoren sind linear unabhängig.
	\end{parts} 
\end{proof}
\begin{Problem}
	Entscheiden Sie, welche der folgenden Familien ein Erzeugendensystem des jeweiligen Vektorraums sind
	\begin{parts}
	\item $((1, 2, 3), (5, 4, 3), (6, 6, 6))$ in $\R^3$ als  $\R$-Vektorraum.
	\item $(\exp it)_{t\in \Q}$ in $\C$ als $\R$-Vektorraum.
	\item $(1+t^n)_{n\in \N_0}$ in $\Q[t]$ als $\Q$-Vektorraum. 
	\item $(\tilde{p})_{p\in \R[t]}$ f\"{u}r den Vektorraum aller Funktionen $\R\to \R$.
	\item $((1, 2, 3), (5, 4, 3), (6, 6, 7))$ in $\R^3$ als $\Q$-Vektorraum. 
	\end{parts}
\end{Problem}
\begin{proof}
	\begin{parts}
	\item Nein. Wir betrachten $(6,6,5)$. Wir hoffen, dass es als Summe
		\[
			(6,6,5)=a(1,2,3)+b(5,4,3)+c(6,6,6)
		\]
		dargestellt werden kann, wobei $a,b,c\in \R$. Weil $(6,6,6)$ als linear Kombination der anderen $2$ Vektoren geschrieben werden kann, können wir oBdA schreiben
		\[
			(6,6,5)=a'(1,2,3)+b'(5,4,3)+(6,6,6)
		,\]
		oder einfach
\[
	(0,0,-1)=a'(1,2,3)+b'(5,4,3)
.\] 
		wobei $a',b'\in \R$. Dann ist $a'+5b'=0$, oder $a=-5b'$. Es gilt also
		\begin{align*}
			0=&2a'+4b'\\
			=&2(-5b')+4b'\\
			=& -6b'
		\end{align*}
		also $b'=0$. Daraus folgt $a'=0$. Aber
		\[
		0(1,2,3)+0(5,4,3)=(0,0,0)\neq (0,0,-1)
		.\] 
	\item Nein. 
	\item Ja. Sei $v_n=1+t^n~n\in \N_0$ der Erzeugendensystem. Es ist $v_0=1+1=2$. Dann ist $v_n-\frac{1}{2}v_0=t^n~n\in \N$. Weil wir wissen, dass $t^n$ ein Erzeugendensystem ist, ist dann auch $(1+t^n)_{n\in \N_0}$ ein Erzeugendensystem.
	\item Nein. Alle solche Funktionen $\tilde{p}$ sind stetig, also lineare Kombinationen davon $a_1\tilde{p}_1+a_2\tilde{p}_2+\dots$ sind auch stetig. Aber es gibt unstetige Funktionen $\R\to \R$, z.B.
		\[
		f(x)=\begin{cases}
			1 & x \le 0\\
			-1 & x > 0
		\end{cases}
		.\] 
	\item Ja. Es gilt
	\begin{align*}
		(0,0,1)=&(6,6,7)-(5,4,3)-(1,2,3)\\
		(1,0,0)=&-2(1,2,3)+(5,4,3)+3(0,0,1)\\
		(0,1,0)=&\frac{1}{2}\left[ (1,2,3)-(1,0,0)-3(0,0,1) \right] 
	\end{align*}
	Weil alle Elemente der Standardbasis als Linearkombination von unserem System geschrieben werden können, können alle Elemente in $\R^3$ auch. Dann ist es ein Erzeugendensystem. Weil es linear unabhängig ist, ist es eine Basis.\qedhere
	\end{parts}
\end{proof}
\begin{Problem}
	\begin{parts}
	\item Zeigen Sie: Ist $(b_i)_{i\in B}$ eine Basis des $\R$-Vektorraums $V$ und $i_0 \in B$, dann ist auch $(b_i - 2b_{i_0} )_{i\in B}$ eine Basis von $V$.
	\item Zeigen Sie: Ist $K$ ein unendlicher Körper und sind $(x_i)_{i\in \N_0}$ Elemente von $K$ mit $x_i\neq x_i$ f\"{u}r $i\neq j$, so ist $(b_k)_{k\in \left\{ 0,1,2,3,\dots,n \right\} }$ mit
		\[
			b_k=\frac{\left(\prod_{i=0}^{k-1}(t-x_i)\right) \left(\prod_{i=k+1}^{n}(t-x_i) \right) }{\left(\prod_{i=0}^{k-1}(x_k-x_i)\right) \left(\prod_{i=k+1}^{n}(x_k-x_i) \right) }
		\] 
		f\"{u}r jedes $n\in \N_0$ eine Basis von $K_{\le n}[t]=\left\{ p(t)\in K[t]|\text{deg}(p)\le n \right\} $.
	\item Geben Sie mit Hilfe der Basis aus der vorigen Teilaufgabe ein rationales Polynom vom Grad höchstens 4 an, dessen Polynomfunktion die Funktionswerte $f(1)=1,~f(2)=4,~f(3)=9,~f(4)=17,~f(5)=6000$ hat. 
	\end{parts}
\end{Problem}
\begin{proof}
	\begin{parts}
	\item $(b_i-2b_{i_0})_{i\in B}$ ist\ldots
		\begin{enumerate}[label=(\roman*)]
			\item Linear unabhängig

				Sei $(a_k)_{k\in B},a_k\in R$, nicht alle $0$, sodass
				\[
					\sum_{i\in B}a_i(b_i-2b_{i_0})=0
				.\] 
				Es ist dann
				\[
					\sum_{B\ni i\neq i_0}a_i b_i+2\left( \sum_{B\ni i\neq i_0}a_i \right) b_{i_0}-a_{i_0}b_{i_0}=0
				.\] 
				Die Koeffizienten sind nicht alle null, also $(b_i)_{i\in B}$ ist nicht linear unabhängig, ein Widerspruch, weil es ist dann kein Basis.
			\item Ein Erzeugendesystem

				Sei $v\in V$ beliebig und $a_i\in \R$, sodass
				\[
					v=\sum_{i\in B} a_i b_i
				,\] 
				was immer möglich ist, weil $(b_i)_{i\in B}$ eine Basis ist. Sei dann 
				\[
				c_i=\begin{cases}
					a_i & i \neq i_0\\
					-2\sum_{B\ni i\neq i_0} a_i-a_{i_0} & i=i_0
				\end{cases}
				.\] 
				Es gilt dann
				\begin{align*}
					\sum_{i\in B} c_i (b_i-2b_{i_0})=&\sum_{B\ni i\neq n_0}c_i (b_i-2b_{i_0}) + c_{i_0}(b_{i_0}-2b_{i_0})\\
						=&\sum_{B\ni i\neq 0}c_i b_i - 2\sum_{B\ni i\neq i_0} b_{i_0}-c_{i_0}b_{i_0}\\
						=&\sum_{B\ni i \neq i_0} a_ib_i+a_{i_0}b_{i_0}\\
						=&\sum_{i\in B }a_i b_i\\
						=& v
				\end{align*}
				Weil $v$ beliebig war, haben wir einen Erzeugendensystem. Die Behauptung folgt.
		\end{enumerate}
	\item Wir betrachten die Wirkung der Polynomfunktion $\tilde{b}_k$ auf ein Punkt $x_i$. Es gilt, f\"{u}r $i\neq k$, $\tilde{b}_k(x_i)=0$. F\"{u}r $i=k$ ist
		\[
			\tilde{b}_k(x_k)=\frac{\left(\prod_{i=0}^{k-1}(x_k-x_i)\right) \left(\prod_{i=k+1}^{n}(x_k-x_i) \right) }{\left(\prod_{i=0}^{k-1}(x_k-x_i)\right) \left(\prod_{i=k+1}^{n}(x_k-x_i) \right) }=1
		.\]
		Sei $p\in K[t]$ ein Polynom mit Grad $\le n$. Wir wissen, dass nachdem wir $\tilde{p}(x_i)$ f\"{u}r alle $1\le i\le n$ wissen, ist das Polynom eindeutig. Sei jetzt $p(x_i)=a_i$ f\"{u}r alle $i$. Es ist $p'=\sum_{i=1}^n a_i b_i$. Dann gilt $\tilde{p}'(x_i)=a_i$ f\"{u}r alle $i$, also $p'=p$, also $(b_k)$ ist ein Erzeugendensystem.

	Weil wir genau $\text{dim}(K_{\le n}[t])=n$ $b_k$ haben, ist es ein Basis, sonst wäre die Dimension nicht eindeutig.
\item Sei $x_0=1,x_1=2,x_2=3,x_3=4,x_4=5$. Wir betrachten zuerst
	\[
c_k=\left(	\prod_{i=0}^{k-1} (x_k-x_i)\right)\left( \prod_{i=k+1}^{n} (x_k-x_i)  \right)  
\]
f\"{u}r $k=$ 
\begin{enumerate}[label=(\arabic*),start=0]
	\item $c_0=24$
	\item $c_1=-6$
	\item $c_2=4$
	\item $c_3=-6$
	\item $c_4=24$
\end{enumerate}
Dann betrachten wir die Polynomfunktion:
\[
	\tilde{p}=\sum_{i=0}^4\left( \frac{1}{c_i}\prod_{i\neq k = 0}^4 (t-x_i)  \right) 
.\] 
$p$ ist dann ein Polynom mit die gewünschte Nullstellen. Nach Vereinfachung ist
\[
\frac{5971 x^4}{24}-\frac{9951 x^3}{4}+\frac{208985 x^2}{24}-\frac{49751 x}{4}+5970
.\] 
	\end{parts}
\end{proof}
