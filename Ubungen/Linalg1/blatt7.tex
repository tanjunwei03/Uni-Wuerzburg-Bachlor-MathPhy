\begin{Problem}
	Wir betrachten den Unterraum $U$ von $\Z / 101\Z[t]$, der von
	\[
	B=\left( \overline{1},t^2+t,t^3-t^2,t+t^3,t^7,t^6+t^5+t^3,t^4-\overline{5} \right)\] erzeugt wird.
	\begin{parts}
		\item Entscheiden Sie, ob $B$ eine Basis von $U$ ist. Finden Sie andernfalls eine Basis $B'$ von $U$, die nur aus Elementen von $B$ besteht.
		\item Zeigen Sie: Die Polynome
			\[
				(t^7-t^3-t-\overline{5}, \overline{2}t^7-\overline{27},t^6+t^5-t^4+t^3+\overline{5})\] liegen alle in $U$ und sie sind linear unabhängig.
			\item Bestimmen Sie eine Umnummerierung von $B'$, die die Aussage des Austauschsatzes 2.6.8 mit $(w_1,w_2,w_3)=(t^7-t^3-t-\overline{5},\overline{2}t^7-\overline{27},t^6+t^5-t^4+t^3+\overline{5})$ erfüllt.
	\end{parts}
\end{Problem}
\begin{proof}
	\begin{parts}
	\item Nein. Per Definition ist $B$ ein Erzeugendensystem. Es bleibt nur zu zeigen, dass es linear unabhängig ist.

		Es ist aber nicht linear unabhängig. Es gilt
		\[
			-(t^3+t)+(t^2+t)+(t^3-t^2)=0
		,\] 
		also es ist nicht linear unabhängig. Eine Basis $B'$ ist
		\[
		B'=\left\{ \overline{1},t^2+t,t^3+t,t^7,t^6+t^5+t^3,t^4-\overline{5} \right\} 
		\] 
		(also $B$ ohne $t^3-t^2$.)
	\item Es gilt
		\begin{align*}
			t^7-t^3-t-\overline{5}=&t^7-(t^3+t)-\overline{5}(\overline{1})\\
			\overline{2}t^7-\overline{27}=&\overline{2}t^7-\overline{27}(\overline{1})\\
			t^6+t^5-t^4+t^3+\overline{5}=&(t^6+t^5+t^3)-(t^4-\overline{5})
		\end{align*}
		also die liegen alle in $U$. Sei $x_1,x_2,x_3\in \Z / 101\Z$, so dass
		\begin{align*}
			&x_1(t^7-t^3-t-\overline{5})+x_2(\overline{2}t^7-\overline{27})\\
			&+x_3(t^6+t^5-t^4+t^3+\overline{5})=0
		\end{align*}
		Aus Vergleich des Koeffizienten von $t$ gilt $x_1=\overline{0}$. Dann vergleich wir den Koeffizient von $t^7$ und erhalten $x_2=\overline{0}$. Dann muss $x_3=\overline{0}$, also sie sind linear unabhängig.
	\item 
		\[
			B'=\left\{ t^3+t,t^7,t^4-\overline{5},\overline{1},t^2+t,t^6+t^5+t^3 \right\} 
		.\qedhere\] 
	\end{parts}
\end{proof}
\begin{Problem}
Bestimmen Sie die Dimension der folgenden Vektorräume.	
\begin{parts}
	\item $V_1=\left\{ p\in \Q[x]|p(0)=0\wedge \text{deg}(p)\le 6 \right\} $ 
	\item $V_2=\left\{ p\in \Q[x]|\forall a\in \Q:p(-a)=-p(a) \right\} $ 
	\item $V_3=\left\{ v\in \R^n|\forall i\in \left\{ 1,2,3,\dots, n \right\} :v_i-v_{n-i+1}=0 \right\} $
\end{parts}
\end{Problem}
\begin{proof}
	\begin{parts}
	\item Sei $p=a_0+a_1x+a_2x^2+a_3x^3+a_4x^4+a_5x^5+a_6x^5$. Es gilt $p(0)=a_0$. Also wir müssen $a_0=0$. Dann ist $V_1$ gespannte durch
		\[
		B_1=\left\{ t,t^2,t^3,t^4,t^5,t^6 \right\} 
	.\]
	Die Vektoren sind linear unabhängig, also ist $\text{dim}(V_1)=6$.
\item Ein Polynom in $V_2$ muss dan nur gerade Potenzen enthalten. Dann ist ein Basis for $V_2$ 
	\[
		B_2=\left\{ x^{2n}, n\in \N\cup \left\{ 0 \right\}  \right\} 
	\]
	also $\text{dim}(B_2)$ ist abzählbar.
\item Ein Basis f\"{u}r $V_3$ ist
	\end{parts}
\end{proof}
\begin{Problem}
	Gegeben sei die Matrix
	\[
		A=\begin{pmatrix} \overline{1} & \overline{2} & \overline{3} & \overline{4}\\ \overline{5} & \overline{6} & \overline{7} & \overline{8} \\ \overline{4}& \overline{3} & \overline{2} & \overline{1} \end{pmatrix} \in (\Z / 11\Z)^{3\times 4}
	.\] 
	\begin{parts}
	\item Bringen Sie $A$ mit elementaren Zeilenoperationen auf Zeilenstufenform. Lesen Sie anschließend den Zeilenrang von $A$ ab.  
	\item Bringen Sie $B=A^T$ mit elementaren Zeilenoperationen auf Zeilenstufenform. Lesen Sie anschließend den Zeilenrang von B ab. 
	  \end{parts}
\end{Problem}
\begin{proof}
	\begin{parts}
	\item
		\begin{gather*}
		\left(
\begin{array}{cccc}
 \overline{1} & \overline{2} & \overline{3} & \overline{4} \\
 \overline{5} & \overline{6} & \overline{7} & \overline{8} \\
 \overline{4} & \overline{3} & \overline{2} & \overline{1} \\
\end{array}
\right) \xrightarrow{R_2+\overline{6}R_1} \left(
\begin{array}{cccc}
 \overline{1} & \overline{2} & \overline{3} & \overline{4} \\
 \overline{0} & \overline{7} & \overline{3} & \overline{10} \\
 \overline{4} & \overline{3} & \overline{2} & \overline{1} \\
\end{array}
\right) \xrightarrow{R_3+\overline{7}R_1} \\\left(
\begin{array}{cccc}
 \overline{1} & \overline{2} & \overline{3} & \overline{4} \\
 \overline{0} & \overline{7} & \overline{3} & \overline{10} \\
 \overline{0} & \overline{6} & \overline{1} & \overline{7} \\
\end{array}
\right) \xrightarrow{R_2\times \overline{8}} \left(
\begin{array}{cccc}
 \overline{1} & \overline{2} & \overline{3} & \overline{4} \\
 \overline{0} & \overline{1} & \overline{2} & \overline{3} \\
 \overline{0} & \overline{6} & \overline{1} & \overline{7} \\
\end{array}
\right) \xrightarrow{R_3+5R_2} \left(
\begin{array}{cccc}
 \overline{1} & \overline{2} & \overline{3} & \overline{4} \\
 \overline{0} & \overline{1} & \overline{2} & \overline{3} \\
 \overline{0} & \overline{0} & \overline{0} & \overline{0} \\
\end{array}
\right)	
		\end{gather*}
		also der Zeilenrang von $A$ ist $2$.
	\item 
		\begin{gather*}
		\left(
\begin{array}{ccc}
 \overline{1} & \overline{5} & \overline{4} \\
 \overline{2} & \overline{6} & \overline{3} \\
 \overline{3} & \overline{7} & \overline{2} \\
 \overline{4} & \overline{8} & \overline{1} \\
\end{array}
\right) \xrightarrow{R_2+\overline{9}R_1} \left(
\begin{array}{ccc}
 \overline{1} & \overline{5} & \overline{4} \\
 \overline{0} & \overline{7} & \overline{6} \\
 \overline{3} & \overline{7} & \overline{2} \\
 \overline{4} & \overline{8} & \overline{1} \\
\end{array}
\right) \xrightarrow{R_3+\overline{8}R_1}\\ \left(
\begin{array}{ccc}
 \overline{1} & \overline{5} & \overline{4} \\
 \overline{0} & \overline{7} & \overline{6} \\
 \overline{0} & \overline{3} & \overline{1} \\
 \overline{4} & \overline{8} & \overline{1} \\
\end{array}
\right) \xrightarrow{R_4+\overline{7}R_1} \left(
\begin{array}{ccc}
 \overline{1} & \overline{5} & \overline{4} \\
 \overline{0} & \overline{7} & \overline{6} \\
 \overline{0} & \overline{3} & \overline{1} \\
 \overline{0} & \overline{10} & \overline{7} \\
\end{array}
\right) \xrightarrow{R_2\times \overline{8}}\\ \left(
\begin{array}{ccc}
 \overline{1} & \overline{5} & \overline{4} \\
 \overline{0} & \overline{1} & \overline{4} \\
 \overline{0} & \overline{3} & \overline{1} \\
 \overline{0} & \overline{10} & \overline{7} \\
\end{array}
\right) \xrightarrow{R_3+\overline{8}R_2} \left(
\begin{array}{ccc}
 \overline{1} & \overline{5} & \overline{4} \\
 \overline{0} & \overline{1} & \overline{4} \\
 \overline{0} & \overline{0} & \overline{0} \\
 \overline{0} & \overline{10} & \overline{7} \\
\end{array}
\right) \xrightarrow{R_4+R_2} \left(
\begin{array}{ccc}
 \overline{1} & \overline{5} & \overline{4} \\
 \overline{0} & \overline{1} & \overline{4} \\
 \overline{0} & \overline{0} & \overline{0} \\
 \overline{0} & \overline{0} & \overline{0} \\
\end{array}
\right)	
		\end{gather*}
		also der Zeilenrank von $A^T$ ist $2$.\qedhere
	\end{parts}
\end{proof}
\begin{Problem}
	Wir betrachten $U=\text{span}\left( \left( 1,2,5,6 \right) , \left( 5,4,3,2 \right) , \left( 4,3,6,5 \right)  \right) \subseteq \Q^4$.
	\begin{parts}
	\item Bringen Sie
		\[
			A=\begin{pmatrix} 1 & 2 &5 & 6 \\ 5 & 4 & 3 & 2 \\ 4 & 3 & 6 & 5 \end{pmatrix} 
		\]
mit elementaren Zeilenoperationen auf Zeilenstufenform. Die entstandenen Zeilenvektoren nennen wir ab jetzt $b_1,b_2,b_3$.
\item Begründen Sie, dass $b_1,b_2,b_3$ linear unabhängig sind.
\item Begründen Sie, dass $b_1,b_2,b_3$ in $U$ liegen.
\item Folgern Sie mit Hilfe der vorigen beiden Aussagen, dass sowohl $((1, 2, 4, 5), (5, 4, 3, 2), (4, 3, 6, 5))$ als auch $(b_1,b_2,b_3)$ eine Basis von $U$ bilden.
\item Nutzen Sie die Basis $(b_1,b_2,b_3)$, um herauszufinden, welche der Vektoren $u_1=(1,1,1,1),u_2=(1,1,0,0)$ bzw. $u_3=(1,2,3,4)$ in $U$ enthalten sind. Geben Sie in diesem Fall zudem die Koeffizienten der Linearkombination
	\[
	\lambda_i(1,2,5,6)+\mu_i(5,4,3,2)+\tau_i(4,3,6,5)=u_i\] an.
	\end{parts}
\end{Problem}
\begin{proof}
\begin{parts}
\item 
	\begin{gather*}
	\left(
\begin{array}{cccc}
 1 & 2 & 5 & 6 \\
 5 & 4 & 3 & 2 \\
 4 & 3 & 6 & 5 \\
\end{array}
\right) \xrightarrow{R_2-5R_1} \left(
\begin{array}{cccc}
 1 & 2 & 5 & 6 \\
 0 & -6 & -22 & -28 \\
 4 & 3 & 6 & 5 \\
\end{array}
\right) \xrightarrow{R_3-4R_1}\\ \left(
\begin{array}{cccc}
 1 & 2 & 5 & 6 \\
 0 & -6 & -22 & -28 \\
 0 & -5 & -14 & -19 \\
\end{array}
\right) \xrightarrow{R_2\times -\frac{1}{6}} \left(
\begin{array}{cccc}
 1 & 2 & 5 & 6 \\
 0 & 1 & \frac{11}{3} & \frac{14}{3} \\
 0 & -5 & -14 & -19 \\
\end{array}
\right) \xrightarrow{R_3+5R_2} \left(
\begin{array}{cccc}
 1 & 2 & 5 & 6 \\
 0 & 1 & \frac{11}{3} & \frac{14}{3} \\
 0 & 0 & \frac{13}{3} & \frac{13}{3} \\
\end{array}
\right)	
	\end{gather*}
\item Sei $q_1,q_2,q_3\in \Q$, so dass
	\[
	q_1b_1+q_2b_2+q_3b_3=0
	.\] 
	Wir betrachten zuerst die erste Komponent. Weil die erste Komponent nur in $b_1$ ungleich $0$ ist, muss $q_1=0$. Die zweite Komponent ist nur in $b_2$ ungleich $0$, also $q_2=0$. Daraus folgt, weil $b_3\neq 0$, dass $q_3=0$, also $b_1,b_2,b_3$ sind linear unabhängig.
\item Durch elementare Zeilenoperationen arbeiten wir immer nur mit linear Kombinationen von Zeilen, also das Ergebnis muss eine lineare Kombination sein.

	Wir berechnen es explizit, weil wir es später brauchen werden:
	\begin{align*}
		b_1=&(1,2,5,6)\\
		b_2=&\frac{5}{6}(1,2,5,6)-\frac{1}{6}(5,4,3,2)\\
		b_3=&\frac{1}{6}(1,2,5,6)-\frac{5}{6}(5,4,3,2)+(4,3,6,5)
	\end{align*}
\item Wir wissen aus Satz 2.4.18, dass die Erzeugendensysteme sind. Dann ist es nur zu zeigen: Die Systeme sind linear unabhängig. Wir wissen, dass $b_1,b_2,b_3$ linear unabhängig sind. 

	Wir nehmen an, dass es Zahlen $x,y,z\in \Q$ gibt, nicht alle null, so dass
	\[
	x(1,2,4,5)+y(5,4,3,2)+z(4,3,6,5)=(0,0,0,0)
	.\] 
	Dann können wir $x(1,2,4,5)+y(5,4,3,2)+z(4,3,6,5)$ als Summe von $b_1,b_2,b_3$ schreiben. Dann haben wir ein linear Kombination von $b_1,b_2,b_3$ mit Koeffizienten nicht alle 0, aber das Kombination ist $0$, also $b_1,b_2,b_3$ wären dann nicht linear unabhängig.
\item 
	\begin{enumerate}[label=(\arabic*)]
		\item Es gilt
			 \[
			x_1(1,2,5,6)+y_1(0,1,\frac{11}{3},\frac{14}{3})+z_1(0,0,\frac{13}{3},\frac{13}{3})=(1,1,1,1)
			.\] 
			Daraus folgt: $x_1=1$ und $y_1=-1$, also
			\[
			z_1(0,0,\frac{13}{3},\frac{13}{3})+(1,1,4 / 3, 4 / 3)=(1,1,1,1).\]
Wir entscheiden uns f\"{u}r $z_1=-\frac{1}{13}$ und die Behauptung folgt.
\begin{align*}
	(1,1,1,1)=&(1,2,5,6)-(0,1, 11 /3, 14 / 3)-\frac{1}{13}(0,0,13 / 3, 13 /3)\\
	=&(1,2,5,6)-\left[ \frac{5}{6}(1,2,5,6)-\frac{1}{6}(5,4,3,2) \right] \\
	 &-\frac{1}{13}\left[ \frac{1}{6}(1,2,5,6)-\frac{5}{6}(5,4,3,2)+(4,3,6,5) \right]\\
	=&\frac{2}{13}(1,2,5,6)+\frac{3}{13}(5,4,3,2)-\frac{1}{13}(4,3,2,5)
\end{align*}
\item Es würde gelten
	\[
		(1,1,0,0)=x_2(1,2,5,6)+y_2\left( 0,1,\frac{11}{3},\frac{14}{3} \right) +z_2\left( 0,0,\frac{13}{3},\frac{13}{3} \right) 
	.\] 
	Daraus folgt: $x_2=1$ und $y_2=-1$, also
	\[
		\left( 1,1,\frac{4}{3},\frac{4}{3} \right) +z_2\left( 0,0,\frac{13}{3},\frac{13}{3} \right) =(1,1,0,0)
	.\] 
	Wir entscheiden uns f\"{u}r $z_2=-4 / 13$, und die Gleichung ist erfüllt, also $(1,1,0,0)$ liegt in $U$. Es folgt:
	\begin{align*}
		(1,1,1,1)=&(1,2,5,6)-(0,1, 11 /3, 14 / 3)-\frac{4}{13}(0,0,13 / 3, 13 /3)\\
		=&(1,2,5,6)-\left[ \frac{5}{6}(1,2,5,6)-\frac{1}{6}(5,4,3,2) \right] \\
		&-\frac{4}{13}\left[ \frac{1}{6}(1,2,5,6)-\frac{5}{6}(5,4,3,2)+(4,3,6,5) \right]\\
		=&\frac{3}{26}(1,2,5,6)+\frac{11}{26}(5,4,3,2)-\frac{4}{13}(4,3,2,5)
	\end{align*}
\item Es würde gelten
	\[
		(1,2,3,4)=x_3(1,2,5,6)+y_3\left( 0,1,\frac{11}{3},\frac{14}{3} \right) +z_3\left( 0,0,\frac{13}{3},\frac{13}{3} \right) 
	.\] 
	Daraus folgt: $x_3=1$ und $y_3=0$, also
	\[
		(1,2,3,4)=(1,2,5,6)+z_3\left( 0,0,\frac{13}{3},\frac{13}{3} \right) 
	.\] 
	Dann sei $z_3=-\frac{6}{13}$, und die Gleichung wurde erfüllt, also es liegt in $U$. Dann ist
	\begin{align*}
		(1,2,3,4)=&(1,2,5,6)-\frac{6}{13}\left( 0,0,\frac{13}{3},\frac{13}{3} \right) \\
		=&(1,2,5,6)-\frac{6}{13}\left( \frac{1}{6}(1,2,5,6)\right.\\
		 &\left.-\frac{5}{6}(5,4,3,2)+(4,3,6,5)\right)\\
		=&\frac{12}{13}(1,2,5,6)+\frac{5}{13}(5,4,3,2)-\frac{6}{13}(4,3,6,5).\qedhere
	\end{align*}
	\end{enumerate}
\end{parts}
\end{proof}
