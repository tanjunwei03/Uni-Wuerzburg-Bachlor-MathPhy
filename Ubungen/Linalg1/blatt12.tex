\begin{Problem}
	Es seien
	\[
		A:=\begin{pmatrix} 10 & -31 & -60 & 180 \\ 0 & 3 & -21 & 63 \\ 2 & -8 & 0 & 0 \\ 10 & -31 & -60 & 183 \end{pmatrix},\qquad B := \begin{pmatrix} 1 & 3 & 0 & 0 \\ 2 & 3 & 1 & 0 \\ 1 & 0 & 0 & 2 \\ 0 & 0 & 1 & 0 \end{pmatrix} 
	.\] 
	\begin{parts}
		\item Bestimmen Sie die Determinante von $A$ und $B$ mit elementaren Zeilen- und Spaltenumformungen. 
		\item Bestimmen Sie die Determinante von $A$ und $B$ direkt mit der Leibnizformel. 
		\item Vergleichen Sie den Rechenaufwand der beiden Methoden. Welche würden Sie bevorzugen? Hängt Ihre Antwort von der Struktur der Matrix ab?  
	\end{parts}
\end{Problem}
\begin{proof}
	\begin{parts}
	\item 
		\begin{gather*}
		\left(
\begin{array}{cccc}
 10 & -31 & -60 & 180 \\
 0 & 3 & -21 & 63 \\
 2 & -8 & 0 & 0 \\
 10 & -31 & -60 & 183 \\
\end{array}
\right) \xrightarrow{R_3\times 5} \left(
\begin{array}{cccc}
 10 & -31 & -60 & 180 \\
 0 & 3 & -21 & 63 \\
 10 & -40 & 0 & 0 \\
 10 & -31 & -60 & 183 \\
\end{array}
\right) \\\xrightarrow{R_3-R_1} \left(
\begin{array}{cccc}
 10 & -31 & -60 & 180 \\
 0 & 3 & -21 & 63 \\
 0 & -9 & 60 & -180 \\
 10 & -31 & -60 & 183 \\
\end{array}
\right) \xrightarrow{R_4-R_1} \left(
\begin{array}{cccc}
 10 & -31 & -60 & 180 \\
 0 & 3 & -21 & 63 \\
 0 & -9 & 60 & -180 \\
 0 & 0 & 0 & 3 \\
\end{array}
\right) \\\xrightarrow{R_3+3R_2} \left(
\begin{array}{cccc}
 10 & -31 & -60 & 180 \\
 0 & 3 & -21 & 63 \\
 0 & 0 & -3 & 9 \\
 0 & 0 & 0 & 3 \\
\end{array}
\right)	
		\end{gather*}
		Als obere Dreiecksmatrix hat die Matrix am Ende die Determinante $10(3)(-3)(3)=-270$. Weil wir im ersten Schritt durch $5$ multipliziert haben, ist das genau $5$ mal die gewünschte Determinante, also $\text{det }A=-270 / 5=-54$.
\begin{gather*}
\left(
\begin{array}{cccc}
 1 & 3 & 0 & 0 \\
 2 & 3 & 1 & 0 \\
 1 & 0 & 0 & 2 \\
 0 & 0 & 1 & 0 \\
\end{array}
\right) \xrightarrow{R_2-2R_1} \left(
\begin{array}{cccc}
 1 & 3 & 0 & 0 \\
 0 & -3 & 1 & 0 \\
 1 & 0 & 0 & 2 \\
 0 & 0 & 1 & 0 \\
\end{array}
\right) \\\xrightarrow{R_3-R_1} \left(
\begin{array}{cccc}
 1 & 3 & 0 & 0 \\
 0 & -3 & 1 & 0 \\
 0 & -3 & 0 & 2 \\
 0 & 0 & 1 & 0 \\
\end{array}
\right) \xrightarrow{R_3-R_2} \left(
\begin{array}{cccc}
 1 & 3 & 0 & 0 \\
 0 & -3 & 1 & 0 \\
 0 & 0 & -1 & 2 \\
 0 & 0 & 1 & 0 \\
\end{array}
\right) \\\xrightarrow{R_4+R_3} \left(
\begin{array}{cccc}
 1 & 3 & 0 & 0 \\
 0 & -3 & 1 & 0 \\
 0 & 0 & -1 & 2 \\
 0 & 0 & 0 & 2 \\
\end{array}
\right)	
\end{gather*}
Ähnlich ist die Determinante der Matrix am Ende $1(-3)(-1)(2)=6$. Da wir keine Operationen gemacht haben, die die Determinante verändern, ist das die gewünschte Determinante.
\item Es gilt
	\begin{align*}
		\text{det }A=& A_{1,4} A_{2,3} A_{3,2} A_{4,1}-A_{1,3} A_{2,4} A_{3,2} A_{4,1}\\&-A_{1,4} A_{2,2} A_{3,3} A_{4,1}+A_{1,2} A_{2,4} A_{3,3} A_{4,1}\\&+A_{1,3} A_{2,2} A_{3,4} A_{4,1}-A_{1,2} A_{2,3} A_{3,4} A_{4,1}\\&-A_{1,4} A_{2,3} A_{3,1} A_{4,2}+A_{1,3} A_{2,4} A_{3,1} A_{4,2}\\&+A_{1,4} A_{2,1} A_{3,3} A_{4,2}-A_{1,1} A_{2,4} A_{3,3} A_{4,2}\\&-A_{1,3} A_{2,1} A_{3,4} A_{4,2}+A_{1,1} A_{2,3} A_{3,4} A_{4,2}\\&+A_{1,4} A_{2,2} A_{3,1} A_{4,3}-A_{1,2} A_{2,4} A_{3,1} A_{4,3}\\&-A_{1,4} A_{2,1} A_{3,2} A_{4,3}+A_{1,1} A_{2,4} A_{3,2} A_{4,3}\\&+A_{1,2} A_{2,1} A_{3,4} A_{4,3}-A_{1,1} A_{2,2} A_{3,4} A_{4,3}\\&-A_{1,3} A_{2,2} A_{3,1} A_{4,4}+A_{1,2} A_{2,3} A_{3,1} A_{4,4}\\&+A_{1,3} A_{2,1} A_{3,2} A_{4,4}-A_{1,1} A_{2,3} A_{3,2} A_{4,4}\\&-A_{1,2} A_{2,1} A_{3,3} A_{4,4}+A_{1,1} A_{2,2} A_{3,3} A_{4,4}\\
		=& 0 + 0 -307440 + 0 + 302400\\ &+ 0 + 0 + 0 + 238266 + 0 -234360 \\&+ 0 + 0 + 0 + 65880 + 0 + 234360 \\&-302400 + 0 + 0 -64800 + 0 \\&-234360 + 302400=-54 
	\end{align*}
	Ähnlich für $B$, aber weil in der letzten Zeile eine nicht null Zahl nur im dritten Spalte entsteht, tragt nur Permutationen $\sigma$ mit $\sigma(4)=4$ bei. Dann gibt es nur zwei Terme im Summe, die die Permutationen $(3,4)$ und $(1,2)(3,4)$ entsprechen. 
	\[
		\text{det }B=12-6=6
	.\] 
\item F\"{u}r $A$ habe ich mehr Arbeit gebraucht, durch die Leibnizformel die Determinante zu berechnen. F\"{u}r $B$ ist es anders.

	F\"{u}r Matrizen mit viele null Elemente würde ich daher mit der Leibnizformel die Determinante zu berechnen, sonst würde ich Zeilen- und Spaltenumformungen verwenden.\qedhere
	\end{parts}
\end{proof}

\begin{Problem}
	E sei $L:\R^3\to \R^3$ definiert durch $x\to Ax$ mit
	\[
		A=\begin{pmatrix} 7 & -15 & 5 \\ -96 & 268 & -63 \\ -400 & 1110 & 263 \end{pmatrix} 
\]
sowie $B:=(b_1,b_2,b_3):=((13,6,4),(10,6,10),(6,8,24))$. 
\begin{parts}
\item Zeigen Sie mit Hilfe der Determinante, dass $B$ eine Basis von $\R^3$ ist.
\item Bestimmen Sie die Basiswechselmatrix $C=T_B^E$, wobei $E$ die Standardbasis ist (Notation wie in 3.5.2)
\item Bestimmen Sie $C^{-1}$ und berechnen Sie $CAC^{-1}$.
\item Geben Sie die Darstellungsmatrix von $L$ bezüglich der Basis $B$ an.
\item Bestimmen Sie $\text{det}(A),\text{det}(C),\text{det}(C^{-1})$ und $\text{det}(CAC^{-1})$, ohne den Determinantenmultiplikationssatz zu verwenden. Verifizieren Sie damit für diesen Spezialfall Folgerung 4.3.6 und Satz 4.3.7. 
\end{parts}
\end{Problem}
\begin{proof}
	\begin{parts}
	\item Die Vektoren $(b_1,b_2,b_3)$ sind linear unabhängig genau dann, wenn $\text{det }B\neq 0$. Wir berechnen die Determinante
		\begin{gather*}
		\left(
\begin{array}{ccc}
 14 & 10 & 6 \\
 6 & 6 & 8 \\
 4 & 10 & 24 \\
\end{array}
\right) \xrightarrow{R_2\times \frac{7}{3}} \left(
\begin{array}{ccc}
 14 & 10 & 6 \\
 14 & 14 & \frac{56}{3} \\
 4 & 10 & 24 \\
\end{array}
\right) \xrightarrow{R_2-R_1} \\\left(
\begin{array}{ccc}
 14 & 10 & 6 \\
 0 & 4 & \frac{38}{3} \\
 4 & 10 & 24 \\
\end{array}
\right) \xrightarrow{R_3\times \frac{7}{2}} \left(
\begin{array}{ccc}
 14 & 10 & 6 \\
 0 & 4 & \frac{38}{3} \\
 14 & 35 & 84 \\
\end{array}
\right) \xrightarrow{R_3-R_1} \\\left(
\begin{array}{ccc}
 14 & 10 & 6 \\
 0 & 4 & \frac{38}{3} \\
 0 & 25 & 78 \\
\end{array}
\right) \xrightarrow{R_2\times \frac{25}{4}} \left(
\begin{array}{ccc}
 14 & 10 & 6 \\
 0 & 25 & \frac{475}{6} \\
 0 & 25 & 78 \\
\end{array}
\right) \xrightarrow{R_3-R_2} \left(
\begin{array}{ccc}
 14 & 10 & 6 \\
 0 & 25 & \frac{475}{6} \\
 0 & 0 & -\frac{7}{6} \\
\end{array}
\right)	
		\end{gather*}
		deren Determinante ungleich null ist. Daher ist $\text{det }B\neq 0$, und die Vektoren sind linear unabhängig. Da wir $3$ linear unabhängige Vektoren in $\R^3$ haben, wobei $\text{dim }\R^3=3$, sind die Vektoren eine Basis.
	\item Es gilt
		\[
			T_{E}^B=\begin{pmatrix} 14 & 10 & 6 \\ 6 & 6 & 8 \\ 4 & 10 & 24 \end{pmatrix} =:B
		.\] 
		Da $T_B^E=(T_E^B)^{-1}$, ist
		\begin{gather*}
		\left(
\begin{array}{ccc|ccc}
 14 & 10 & 6 & 1 & 0 & 0 \\
 6 & 6 & 8 & 0 & 1 & 0 \\
 4 & 10 & 24 & 0 & 0 & 1 \\
\end{array}
\right) \xrightarrow{R_1\times \frac{1}{14}} \left(
\begin{array}{ccc|ccc}
 1 & \frac{5}{7} & \frac{3}{7} & \frac{1}{14} & 0 & 0 \\
 6 & 6 & 8 & 0 & 1 & 0 \\
 4 & 10 & 24 & 0 & 0 & 1 \\
\end{array}
\right) \xrightarrow{R_2-6R_1} \\\left(
\begin{array}{ccc|ccc}
 1 & \frac{5}{7} & \frac{3}{7} & \frac{1}{14} & 0 & 0 \\
 0 & \frac{12}{7} & \frac{38}{7} & -\frac{3}{7} & 1 & 0 \\
 4 & 10 & 24 & 0 & 0 & 1 \\
\end{array}
\right) \xrightarrow{R_3-4R_1} \left(
\begin{array}{ccc|ccc}
 1 & \frac{5}{7} & \frac{3}{7} & \frac{1}{14} & 0 & 0 \\
 0 & \frac{12}{7} & \frac{38}{7} & -\frac{3}{7} & 1 & 0 \\
 0 & \frac{50}{7} & \frac{156}{7} & -\frac{2}{7} & 0 & 1 \\
\end{array}
\right) \xrightarrow{R_2\times \frac{25}{6}}\\ \left(
\begin{array}{ccc|ccc}
 1 & \frac{5}{7} & \frac{3}{7} & \frac{1}{14} & 0 & 0 \\
 0 & \frac{50}{7} & \frac{475}{21} & -\frac{25}{14} & \frac{25}{6} & 0 \\
 0 & \frac{50}{7} & \frac{156}{7} & -\frac{2}{7} & 0 & 1 \\
\end{array}
\right) \xrightarrow{R_3-R_2} \left(
\begin{array}{ccc|ccc}
 1 & \frac{5}{7} & \frac{3}{7} & \frac{1}{14} & 0 & 0 \\
 0 & \frac{50}{7} & \frac{475}{21} & -\frac{25}{14} & \frac{25}{6} & 0 \\
 0 & 0 & -\frac{1}{3} & \frac{3}{2} & -\frac{25}{6} & 1 \\
\end{array}
\right) \xrightarrow{R_1-\frac{1}{10}R_2} \\\left(
\begin{array}{ccc|ccc}
 1 & 0 & -\frac{11}{6} & \frac{1}{4} & -\frac{5}{12} & 0 \\
 0 & \frac{50}{7} & \frac{475}{21} & -\frac{25}{14} & \frac{25}{6} & 0 \\
 0 & 0 & -\frac{1}{3} & \frac{3}{2} & -\frac{25}{6} & 1 \\
\end{array}
\right)  \xrightarrow{R_2\times \frac{7}{50}} \left(
\begin{array}{ccc|ccc}
 1 & 0 & -\frac{11}{6} & \frac{1}{4} & -\frac{5}{12} & 0 \\
 0 & 1 & \frac{19}{6} & -\frac{1}{4} & \frac{7}{12} & 0 \\
 0 & 0 & -\frac{1}{3} & \frac{3}{2} & -\frac{25}{6} & 1 \\
\end{array}
\right) \xrightarrow{R_3\times -3} \\\left(
\begin{array}{ccc|ccc}
 1 & 0 & -\frac{11}{6} & \frac{1}{4} & -\frac{5}{12} & 0 \\
 0 & 1 & \frac{19}{6} & -\frac{1}{4} & \frac{7}{12} & 0 \\
 0 & 0 & 1 & -\frac{9}{2} & \frac{25}{2} & -3 \\
\end{array}
\right) \xrightarrow{R_2-\frac{19}{6}R_3} \left(
\begin{array}{ccc|ccc}
 1 & 0 & -\frac{11}{6} & \frac{1}{4} & -\frac{5}{12} & 0 \\
 0 & 1 & 0 & 14 & -39 & \frac{19}{2} \\
 0 & 0 & 1 & -\frac{9}{2} & \frac{25}{2} & -3 \\
\end{array}
\right) \xrightarrow{R_1+\frac{11}{6}R_3} \\\left(
\begin{array}{ccc|ccc}
 1 & 0 & 0 & -8 & \frac{45}{2} & -\frac{11}{2} \\
 0 & 1 & 0 & 14 & -39 & \frac{19}{2} \\
 0 & 0 & 1 & -\frac{9}{2} & \frac{25}{2} & -3 \\
\end{array}
\right) 	
		\end{gather*}
	\end{parts}
\end{proof}
