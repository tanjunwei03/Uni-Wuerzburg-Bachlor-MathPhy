\begin{Problem}
	Es seien
	\[
		A:=\begin{pmatrix} 10 & -31 & -60 & 180 \\ 0 & 3 & -21 & 63 \\ 2 & -8 & 0 & 0 \\ 10 & -31 & -60 & 183 \end{pmatrix},\qquad B := \begin{pmatrix} 1 & 3 & 0 & 0 \\ 2 & 3 & 1 & 0 \\ 1 & 0 & 0 & 2 \\ 0 & 0 & 1 & 0 \end{pmatrix} 
	.\] 
	\begin{parts}
		\item Bestimmen Sie die Determinante von $A$ und $B$ mit elementaren Zeilen- und Spaltenumformungen. 
		\item Bestimmen Sie die Determinante von $A$ und $B$ direkt mit der Leibnizformel. 
		\item Vergleichen Sie den Rechenaufwand der beiden Methoden. Welche würden Sie bevorzugen? Hängt Ihre Antwort von der Struktur der Matrix ab?  
	\end{parts}
\end{Problem}
\begin{proof}
	\begin{parts}
	\item 
		\begin{gather*}
		\left(
\begin{array}{cccc}
 10 & -31 & -60 & 180 \\
 0 & 3 & -21 & 63 \\
 2 & -8 & 0 & 0 \\
 10 & -31 & -60 & 183 \\
\end{array}
\right) \xrightarrow{R_3\times 5} \left(
\begin{array}{cccc}
 10 & -31 & -60 & 180 \\
 0 & 3 & -21 & 63 \\
 10 & -40 & 0 & 0 \\
 10 & -31 & -60 & 183 \\
\end{array}
\right) \\\xrightarrow{R_3-R_1} \left(
\begin{array}{cccc}
 10 & -31 & -60 & 180 \\
 0 & 3 & -21 & 63 \\
 0 & -9 & 60 & -180 \\
 10 & -31 & -60 & 183 \\
\end{array}
\right) \xrightarrow{R_4-R_1} \left(
\begin{array}{cccc}
 10 & -31 & -60 & 180 \\
 0 & 3 & -21 & 63 \\
 0 & -9 & 60 & -180 \\
 0 & 0 & 0 & 3 \\
\end{array}
\right) \\\xrightarrow{R_3+3R_2} \left(
\begin{array}{cccc}
 10 & -31 & -60 & 180 \\
 0 & 3 & -21 & 63 \\
 0 & 0 & -3 & 9 \\
 0 & 0 & 0 & 3 \\
\end{array}
\right)	
		\end{gather*}
		Als obere Dreiecksmatrix hat die Matrix am Ende die Determinante $10(3)(-3)(3)=-270$. Weil wir im ersten Schritt durch $5$ multipliziert haben, ist das genau $5$ mal die gewünschte Determinante, also $\text{det }A=-270 / 5=-54$.
\begin{gather*}
\left(
\begin{array}{cccc}
 1 & 3 & 0 & 0 \\
 2 & 3 & 1 & 0 \\
 1 & 0 & 0 & 2 \\
 0 & 0 & 1 & 0 \\
\end{array}
\right) \xrightarrow{R_2-2R_1} \left(
\begin{array}{cccc}
 1 & 3 & 0 & 0 \\
 0 & -3 & 1 & 0 \\
 1 & 0 & 0 & 2 \\
 0 & 0 & 1 & 0 \\
\end{array}
\right) \\\xrightarrow{R_3-R_1} \left(
\begin{array}{cccc}
 1 & 3 & 0 & 0 \\
 0 & -3 & 1 & 0 \\
 0 & -3 & 0 & 2 \\
 0 & 0 & 1 & 0 \\
\end{array}
\right) \xrightarrow{R_3-R_2} \left(
\begin{array}{cccc}
 1 & 3 & 0 & 0 \\
 0 & -3 & 1 & 0 \\
 0 & 0 & -1 & 2 \\
 0 & 0 & 1 & 0 \\
\end{array}
\right) \\\xrightarrow{R_4+R_3} \left(
\begin{array}{cccc}
 1 & 3 & 0 & 0 \\
 0 & -3 & 1 & 0 \\
 0 & 0 & -1 & 2 \\
 0 & 0 & 0 & 2 \\
\end{array}
\right)	
\end{gather*}
Ähnlich ist die Determinante der Matrix am Ende $1(-3)(-1)(2)=6$. Da wir keine Operationen gemacht haben, die die Determinante verändern, ist das die gewünschte Determinante.
\item Es gilt
	\begin{align*}
		\text{det }A=& A_{1,4} A_{2,3} A_{3,2} A_{4,1}-A_{1,3} A_{2,4} A_{3,2} A_{4,1}\\&-A_{1,4} A_{2,2} A_{3,3} A_{4,1}+A_{1,2} A_{2,4} A_{3,3} A_{4,1}\\&+A_{1,3} A_{2,2} A_{3,4} A_{4,1}-A_{1,2} A_{2,3} A_{3,4} A_{4,1}\\&-A_{1,4} A_{2,3} A_{3,1} A_{4,2}+A_{1,3} A_{2,4} A_{3,1} A_{4,2}\\&+A_{1,4} A_{2,1} A_{3,3} A_{4,2}-A_{1,1} A_{2,4} A_{3,3} A_{4,2}\\&-A_{1,3} A_{2,1} A_{3,4} A_{4,2}+A_{1,1} A_{2,3} A_{3,4} A_{4,2}\\&+A_{1,4} A_{2,2} A_{3,1} A_{4,3}-A_{1,2} A_{2,4} A_{3,1} A_{4,3}\\&-A_{1,4} A_{2,1} A_{3,2} A_{4,3}+A_{1,1} A_{2,4} A_{3,2} A_{4,3}\\&+A_{1,2} A_{2,1} A_{3,4} A_{4,3}-A_{1,1} A_{2,2} A_{3,4} A_{4,3}\\&-A_{1,3} A_{2,2} A_{3,1} A_{4,4}+A_{1,2} A_{2,3} A_{3,1} A_{4,4}\\&+A_{1,3} A_{2,1} A_{3,2} A_{4,4}-A_{1,1} A_{2,3} A_{3,2} A_{4,4}\\&-A_{1,2} A_{2,1} A_{3,3} A_{4,4}+A_{1,1} A_{2,2} A_{3,3} A_{4,4}\\
		=& 0 + 0 -307440 + 0 + 302400\\ &+ 0 + 0 + 0 + 238266 + 0 -234360 \\&+ 0 + 0 + 0 + 65880 + 0 + 234360 \\&-302400 + 0 + 0 -64800 + 0 \\&-234360 + 302400=-54 
	\end{align*}
	Ähnlich für $B$, aber weil in der letzten Zeile eine nicht null Zahl nur im dritten Spalte entsteht, tragt nur Permutationen $\sigma$ mit $\sigma(4)=4$ bei. Dann gibt es nur zwei Terme im Summe, die die Permutationen $(3,4)$ und $(1,2)(3,4)$ entsprechen. 
	\[
		\text{det }B=12-6=6
	.\] 
\item F\"{u}r $A$ habe ich mehr Arbeit gebraucht, durch die Leibnizformel die Determinante zu berechnen. F\"{u}r $B$ ist es anders.

	F\"{u}r Matrizen mit viele null Elemente würde ich daher mit der Leibnizformel die Determinante zu berechnen, sonst würde ich Zeilen- und Spaltenumformungen verwenden.\qedhere
	\end{parts}
\end{proof}

\begin{Problem}
	E sei $L:\R^3\to \R^3$ definiert durch $x\to Ax$ mit
	\[
		A=\begin{pmatrix} 7 & -15 & 5 \\ -96 & 268 & -63 \\ -400 & 1110 & -263 \end{pmatrix} 
\]
sowie $B:=(b_1,b_2,b_3):=((13,6,4),(10,6,10),(6,8,24))$. 
\begin{parts}
\item Zeigen Sie mit Hilfe der Determinante, dass $B$ eine Basis von $\R^3$ ist.
\item Bestimmen Sie die Basiswechselmatrix $C=T_B^E$, wobei $E$ die Standardbasis ist (Notation wie in 3.5.2)
\item Bestimmen Sie $C^{-1}$ und berechnen Sie $CAC^{-1}$.
\item Geben Sie die Darstellungsmatrix von $L$ bezüglich der Basis $B$ an.
\item Bestimmen Sie $\text{det}(A),\text{det}(C),\text{det}(C^{-1})$ und $\text{det}(CAC^{-1})$, ohne den Determinantenmultiplikationssatz zu verwenden. Verifizieren Sie damit für diesen Spezialfall Folgerung 4.3.6 und Satz 4.3.7. 
\end{parts}
\end{Problem}
\begin{proof}
	\begin{parts}
	\item Die Vektoren $(b_1,b_2,b_3)$ sind linear unabhängig genau dann, wenn $\text{det }B\neq 0$. Wir berechnen die Determinante
		\begin{gather*}
		\left(
\begin{array}{ccc}
 14 & 10 & 6 \\
 6 & 6 & 8 \\
 4 & 10 & 24 \\
\end{array}
\right) \xrightarrow{R_2\times \frac{7}{3}} \left(
\begin{array}{ccc}
 14 & 10 & 6 \\
 14 & 14 & \frac{56}{3} \\
 4 & 10 & 24 \\
\end{array}
\right) \xrightarrow{R_2-R_1} \\\left(
\begin{array}{ccc}
 14 & 10 & 6 \\
 0 & 4 & \frac{38}{3} \\
 4 & 10 & 24 \\
\end{array}
\right) \xrightarrow{R_3\times \frac{7}{2}} \left(
\begin{array}{ccc}
 14 & 10 & 6 \\
 0 & 4 & \frac{38}{3} \\
 14 & 35 & 84 \\
\end{array}
\right) \xrightarrow{R_3-R_1} \\\left(
\begin{array}{ccc}
 14 & 10 & 6 \\
 0 & 4 & \frac{38}{3} \\
 0 & 25 & 78 \\
\end{array}
\right) \xrightarrow{R_2\times \frac{25}{4}} \left(
\begin{array}{ccc}
 14 & 10 & 6 \\
 0 & 25 & \frac{475}{6} \\
 0 & 25 & 78 \\
\end{array}
\right) \xrightarrow{R_3-R_2} \left(
\begin{array}{ccc}
 14 & 10 & 6 \\
 0 & 25 & \frac{475}{6} \\
 0 & 0 & -\frac{7}{6} \\
\end{array}
\right)	
		\end{gather*}
		deren Determinante ungleich null ist. Daher ist $\text{det }B\neq 0$, und die Vektoren sind linear unabhängig. Da wir $3$ linear unabhängige Vektoren in $\R^3$ haben, wobei $\text{dim }\R^3=3$, sind die Vektoren eine Basis.
	\item Es gilt
		\[
			C^{-1}:=T_{E}^B=\begin{pmatrix} 14 & 10 & 6 \\ 6 & 6 & 8 \\ 4 & 10 & 24 \end{pmatrix}
		.\] 
		Da $T_B^E=(T_E^B)^{-1}$, berechnen wir die inverse Matrix.
		\begin{gather*}
		\left(
\begin{array}{ccc|ccc}
 14 & 10 & 6 & 1 & 0 & 0 \\
 6 & 6 & 8 & 0 & 1 & 0 \\
 4 & 10 & 24 & 0 & 0 & 1 \\
\end{array}
\right) \xrightarrow{R_1\times \frac{1}{14}} \left(
\begin{array}{ccc|ccc}
 1 & \frac{5}{7} & \frac{3}{7} & \frac{1}{14} & 0 & 0 \\
 6 & 6 & 8 & 0 & 1 & 0 \\
 4 & 10 & 24 & 0 & 0 & 1 \\
\end{array}
\right) \xrightarrow{R_2-6R_1} \\\left(
\begin{array}{ccc|ccc}
 1 & \frac{5}{7} & \frac{3}{7} & \frac{1}{14} & 0 & 0 \\
 0 & \frac{12}{7} & \frac{38}{7} & -\frac{3}{7} & 1 & 0 \\
 4 & 10 & 24 & 0 & 0 & 1 \\
\end{array}
\right) \xrightarrow{R_3-4R_1} \left(
\begin{array}{ccc|ccc}
 1 & \frac{5}{7} & \frac{3}{7} & \frac{1}{14} & 0 & 0 \\
 0 & \frac{12}{7} & \frac{38}{7} & -\frac{3}{7} & 1 & 0 \\
 0 & \frac{50}{7} & \frac{156}{7} & -\frac{2}{7} & 0 & 1 \\
\end{array}
\right) \xrightarrow{R_2\times \frac{25}{6}}\\ \left(
\begin{array}{ccc|ccc}
 1 & \frac{5}{7} & \frac{3}{7} & \frac{1}{14} & 0 & 0 \\
 0 & \frac{50}{7} & \frac{475}{21} & -\frac{25}{14} & \frac{25}{6} & 0 \\
 0 & \frac{50}{7} & \frac{156}{7} & -\frac{2}{7} & 0 & 1 \\
\end{array}
\right) \xrightarrow{R_3-R_2} \left(
\begin{array}{ccc|ccc}
 1 & \frac{5}{7} & \frac{3}{7} & \frac{1}{14} & 0 & 0 \\
 0 & \frac{50}{7} & \frac{475}{21} & -\frac{25}{14} & \frac{25}{6} & 0 \\
 0 & 0 & -\frac{1}{3} & \frac{3}{2} & -\frac{25}{6} & 1 \\
\end{array}
\right) \xrightarrow{R_1-\frac{1}{10}R_2} \\\left(
\begin{array}{ccc|ccc}
 1 & 0 & -\frac{11}{6} & \frac{1}{4} & -\frac{5}{12} & 0 \\
 0 & \frac{50}{7} & \frac{475}{21} & -\frac{25}{14} & \frac{25}{6} & 0 \\
 0 & 0 & -\frac{1}{3} & \frac{3}{2} & -\frac{25}{6} & 1 \\
\end{array}
\right)  \xrightarrow{R_2\times \frac{7}{50}} \left(
\begin{array}{ccc|ccc}
 1 & 0 & -\frac{11}{6} & \frac{1}{4} & -\frac{5}{12} & 0 \\
 0 & 1 & \frac{19}{6} & -\frac{1}{4} & \frac{7}{12} & 0 \\
 0 & 0 & -\frac{1}{3} & \frac{3}{2} & -\frac{25}{6} & 1 \\
\end{array}
\right) \xrightarrow{R_3\times -3} \\\left(
\begin{array}{ccc|ccc}
 1 & 0 & -\frac{11}{6} & \frac{1}{4} & -\frac{5}{12} & 0 \\
 0 & 1 & \frac{19}{6} & -\frac{1}{4} & \frac{7}{12} & 0 \\
 0 & 0 & 1 & -\frac{9}{2} & \frac{25}{2} & -3 \\
\end{array}
\right) \xrightarrow{R_2-\frac{19}{6}R_3} \left(
\begin{array}{ccc|ccc}
 1 & 0 & -\frac{11}{6} & \frac{1}{4} & -\frac{5}{12} & 0 \\
 0 & 1 & 0 & 14 & -39 & \frac{19}{2} \\
 0 & 0 & 1 & -\frac{9}{2} & \frac{25}{2} & -3 \\
\end{array}
\right) \xrightarrow{R_1+\frac{11}{6}R_3} \\\left(
\begin{array}{ccc|ccc}
 1 & 0 & 0 & -8 & \frac{45}{2} & -\frac{11}{2} \\
 0 & 1 & 0 & 14 & -39 & \frac{19}{2} \\
 0 & 0 & 1 & -\frac{9}{2} & \frac{25}{2} & -3 \\
\end{array}
\right) 	
		\end{gather*}
		also
		\[
			C=\begin{pmatrix} -8 & \frac{45}{2} & -\frac{11}{2} \\ 14 & -39 & \frac{19}{2} \\ -\frac{9}{2} & \frac{25}{2} & -3\end{pmatrix} 
		.\] 
	\item $C^{-1}$ wurde schon in (b) gegeben. Durch direkte Rechnung erhalten wir
\[
	CAC^{-1}=\begin{pmatrix}  2 & 0 & 0 \\ 0 & 3 & 0 \\ 0 & 0 & 7 \end{pmatrix} 
.\] 
\item Die Darstellungsmatrix ist genau $CAC^{-1}$.
\item $A$:
	\begin{gather*}
	\left(
\begin{array}{ccc}
 7 & -15 & 5 \\
 -96 & 268 & -63 \\
 -400 & 1110 & -263 \\
\end{array}
\right) \xrightarrow{R_2+\frac{96}{7}R_1} \left(
\begin{array}{ccc}
 7 & -15 & 5 \\
 0 & \frac{436}{7} & \frac{39}{7} \\
 -400 & 1110 & -263 \\
\end{array}
\right) \\\xrightarrow{R_3+\frac{400}{7}R_1} \left(
\begin{array}{ccc}
 7 & -15 & 5 \\
 0 & \frac{436}{7} & \frac{39}{7} \\
 0 & \frac{1770}{7} & \frac{159}{7} \\
\end{array}
\right) \xrightarrow{R_3-\frac{885}{218}R_2} \left(
\begin{array}{ccc}
 7 & -15 & 5 \\
 0 & \frac{436}{7} & \frac{39}{7} \\
 0 & 0 & \frac{21}{218} \\
\end{array}
\right)	
	\end{gather*}
	also $\text{det}(A)=7(436 / 7)(21 / 218)=42$. F\"{u}r $C$ gilt:
	\begin{gather*}
	\left(
\begin{array}{ccc}
 14 & 10 & 6 \\
 6 & 6 & 8 \\
 4 & 10 & 24 \\
\end{array}
\right) \xrightarrow{R_1\times 3} \left(
\begin{array}{ccc}
 42 & 30 & 18 \\
 6 & 6 & 8 \\
 4 & 10 & 24 \\
\end{array}
\right) \xrightarrow{R_2\times 7} \\\left(
\begin{array}{ccc}
 42 & 30 & 18 \\
 42 & 42 & 56 \\
 4 & 10 & 24 \\
\end{array}
\right) \xrightarrow{R_2-R_1} \left(
\begin{array}{ccc}
 42 & 30 & 18 \\
 0 & 12 & 38 \\
 4 & 10 & 24 \\
\end{array}
\right) \xrightarrow{R_3\times \frac{21}{2}} \\\left(
\begin{array}{ccc}
 42 & 30 & 18 \\
 0 & 12 & 38 \\
 42 & 105 & 252 \\
\end{array}
\right) \xrightarrow{R_3-R_1} \left(
\begin{array}{ccc}
 42 & 30 & 18 \\
 0 & 12 & 38 \\
 0 & 75 & 234 \\
\end{array}
\right) \xrightarrow{R_3\times 4} \\\left(
\begin{array}{ccc}
 42 & 30 & 18 \\
 0 & 12 & 38 \\
 0 & 300 & 936 \\
\end{array}
\right) \xrightarrow{R_2\times 25} \left(
\begin{array}{ccc}
 42 & 30 & 18 \\
 0 & 300 & 950 \\
 0 & 300 & 936 \\
\end{array}
\right) \xrightarrow{R_3-R_2} \left(
\begin{array}{ccc}
 42 & 30 & 18 \\
 0 & 300 & 950 \\
 0 & 0 & -14 \\
\end{array}
\right)	
	\end{gather*}
	also $\text{det}(C)=-8$. F\"{u}r $C^{-1}$:
	\begin{gather*}
	\left(
\begin{array}{ccc}
 -8 & \frac{45}{2} & -\frac{11}{2} \\
 14 & -39 & \frac{19}{2} \\
 -\frac{9}{2} & \frac{25}{2} & -3 \\
\end{array}
\right) \xrightarrow{R_2+\frac{7}{4}R_1} \left(
\begin{array}{ccc}
 -8 & \frac{45}{2} & -\frac{11}{2} \\
 0 & \frac{3}{8} & -\frac{1}{8} \\
 -\frac{9}{2} & \frac{25}{2} & -3 \\
\end{array}
\right) \xrightarrow{R_3-\frac{9}{16}R_1} \\\left(
\begin{array}{ccc}
 -8 & \frac{45}{2} & -\frac{11}{2} \\
 0 & \frac{3}{8} & -\frac{1}{8} \\
 0 & -\frac{5}{32} & \frac{3}{32} \\
\end{array}
\right) \xrightarrow{R_3+\frac{5}{12}R_2} \left(
\begin{array}{ccc}
 -8 & \frac{45}{2} & -\frac{11}{2} \\
 0 & \frac{3}{8} & -\frac{1}{8} \\
 0 & 0 & \frac{1}{24} \\
\end{array}
\right)	
	\end{gather*}
	also $\text{det}(C^{-1})=-1 / 8$. Zuletzt ist $\text{det}(CAC^{-1})=2\cdot 3\cdot 7=42$.\qedhere
	\end{parts}
\end{proof}

\begin{Problem}
	In dieser Aufgabe sei stets $D:\text{Mat}(n\times n,K)\to K$ eine Abbildung, die linear in jeder Spalte ist, d.h. Eigenschaft 1 von Definition 4.2.1 erfüllt.
	\begin{parts}
	\item Zeigen Sie: Ist die Charakteristik von $K$ nicht $2$, dann ist $D$ genau dann alternierend, wenn f\"{u}r $D$ die Aussage $3$ von Satz 4.2.2, also
\begin{center}
	``Entsteht $B$ aus $A$ durch Vertauschung zweier Spalten, so ist $D(B)=-D(A)$."
\end{center}
gilt. Diese Eigenschaft nennt man auch ``schiefsymmetrisch"
\item Zeigen Sie: Die Abbildung $D_1:\text{Mat}(2\times 2, \Z / 2\Z)$ mit $\begin{pmatrix} a & b \\ c & d \end{pmatrix} \to ab+cd$ ist linear in jeder Spalte und schiefsymmetrisch, aber nicht alternierend. 
\item Zeigen Sie: F\"{u}r jedes $k\in K$ gibt es genau eine Abbildung $D:\text{Mat}(n\times n,K)\to K$, die linear in jeder Spalte und alternierend ist und zusätzlich $D(E_n)=k$ erfüllt. Diese ist gegeben durch die Abbildungsvorschrift $D(A)=k\text{det}(A)$.
	\end{parts}
\end{Problem}
\begin{proof}
	\begin{parts}
	\item Die Rückrichtung ist trivial. Angenommen $A$ hat zwei gleiche Spalten. $B$ entstehe aus $A$ durch die Vertauschung diese Spalten. Deswegen gilt $B=A$. Aber $D(B)=D(A)=-D(A)$, oder $2D(A)=0$. Da die Charakteristik nicht $2$ ist, impliziert dies $D(A)=0$. 

		Sei jetzt $D$ alternierend. Sei $A\in\text{Mat}(n\times n, K)$. $B$ entstehe aus $A$ durch die Vertauschung zwei Spalten. Sei die Spalten $v_1,v_2$. Da die andere Spalten fest sind, bezeichnen wir $f(v_1,v_2):=D((\dots, v_1, v_2,\dots))$. $f$ ist auch linear in $v_1$ und $v_2$. Außerdem ist $D(A)=f(v_1,v_2)$ und $D(B)=f(v_2,v_1)$. Es gilt
		\begin{align*}
			f(v_1+v_2,v_1+v_2)=& 0 & D\text{ ist alternierend}\\
			=&f(v_1,v_1)+f(v_1,v_2)+f(v_2,v_1)+f(v_2,v_2) & \text{Linearität}\\
			=&f(v_1,v_2)+f(v_2,v_1)
		\end{align*}
		Daraus folgt:
		\[
		D(A)=f(v_1,v_2)=-f(v_2,v_1)=-D(B)
		.\] 
	\item Linearität: Es gilt
		\[
			D_1\left( \begin{pmatrix} k a & b \\ k c & d \end{pmatrix}  \right) = k(ab+cd)=D_1\left( \begin{pmatrix} a & kb \\ c & kd \end{pmatrix}  \right) 
		.\] 
		Außerdem gilt
		\begin{align*}
			&D_1\left( \begin{pmatrix} a+a' & b \\ c + c' & d \end{pmatrix}  \right)\\
			=&(a+a')b+(c+c')d\\
			=&(ab+cd)+(a'b+c'd)\\
			=&D_1\left( \begin{pmatrix} a & b \\ c & d \end{pmatrix}  \right) +D_1\left( \begin{pmatrix} a' & b \\ c' & d \end{pmatrix}  \right) 
		\end{align*}
		und
		\begin{align*}
			&D_1\left( \begin{pmatrix} a & b+b' \\ c & d+d' \end{pmatrix}  \right)\\
		=&a(b+b')+c(d+d')\\
		=&(ab+cd)+(ab'+cd')\\
		=&D_1\left( \begin{pmatrix} a & b \\ c & d \end{pmatrix} \right) +D_1\left( \begin{pmatrix} a & b' \\ c & d' \end{pmatrix}  \right)
		\end{align*}
		also $D_1$ ist linear. $D_1$ ist auch schiefsymmetrisch, da
		\[
			D\left( \begin{pmatrix} b & a \\ d & c \end{pmatrix} \right) =ab-cd
		.\]
		Aber $ab-cd=-(ab-cd)$, weil es nur $2$ Elemente in $\Z / 2\Z$ gibt, und f\"{u}r alle Elemente $x\in \Z / 2\Z$ gilt $x+x=0$. Dann ist $D_1$ schiefsymmetrisch. $D_1$ ist jedoch nicht alternierend. 
		\[
			D_1\left( \begin{pmatrix} 1 & 1 \\ 0 & 0 \end{pmatrix}  \right) =1\cdot 1-0\cdot 0 =1\neq 0
		.\] 
	\end{parts}
\end{proof}
