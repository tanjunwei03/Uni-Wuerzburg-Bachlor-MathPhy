\documentclass[prb,12pt]{revtex4-2}

\usepackage{amsmath, amssymb,physics,amsfonts,amsthm}
\usepackage{enumitem}
\usepackage{cancel}
\usepackage{booktabs}
\usepackage{tikz}
\usepackage{hyperref}
\usepackage{enumitem}
\usepackage{transparent}
\usepackage{float}
\usepackage{multirow}
\newtheorem{Theorem}{Theorem}
\newtheorem{Proposition}{Theorem}
\newtheorem{Lemma}[Theorem]{Lemma}
\newtheorem{Corollary}[Theorem]{Corollary}
\newtheorem{Example}[Theorem]{Example}
\newtheorem{Remark}[Theorem]{Remark}
\theoremstyle{definition}
\newtheorem{Problem}{Problem}
\theoremstyle{definition}
\newtheorem{Definition}[Theorem]{Definition}
\newenvironment{parts}{\begin{enumerate}[label=(\alph*)]}{\end{enumerate}}
%tikz
\usetikzlibrary{patterns}
% definitions of number sets
\newcommand{\N}{\mathbb{N}}
\newcommand{\R}{\mathbb{R}}
\newcommand{\Z}{\mathbb{Z}}
\newcommand{\Q}{\mathbb{Q}}
\newcommand{\C}{\mathbb{C}}
\begin{document}
	\title{Lineare Algebra 1 Hausaufgabenblatt Nr. 1}
	\author{Jun Wei Tan}
	\email{jun-wei.tan@stud-mail.uni-wuerzburg.de}
	\affiliation{Julius-Maximilians-Universit\"{a}t W\"{u}rzburg}
	\date{\today}
	\maketitle
\begin{Problem}
	Gegeben sei die Relation $~\subseteq (\R^2 \ \{0\}) \times  (\R^2 \ \{O\})$ mit $x ~ y$ genau dann, wenn es eine Gerade $L \subseteq \R2$ gibt, die $0$, $x$ und $y$ enthält.

	\begin{parts}
		\item Bestimmen Sie alle $y \in \R^2 \backslash \{(0, 0)\}$ mit $(0, 1) ~ y$ bzw. $(1, 0) ~ y$ und skizzieren Sie die beiden Mengen in einem geeigneten Koordinatensystem.
		\item Begründen Sie, dass ~ eine Äquivalenzrelation ist.
		\item Bleibt $~$ auch dann eine Äquivalenzrelation, wenn man sie als Relation in $\R^2$ betrachtet?
	\end{parts}
\end{Problem}
\end{document}
