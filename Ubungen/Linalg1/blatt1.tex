\begin{Definition}\label{def:line1}
	Sind $a_1,a_2,b\in \R$  mit $(a_1, a_2) \neq (0, 0)$, so bezeichnet man die Menge $g := \{(x_1, x_2) \in \R^2 | a_1x_1 + a_2x_2 = b\}$ als Gerade.
\end{Definition}

\begin{Theorem}
	Zu jeder Geraden gibt es $c_1, c_2, d_1, d_2 \in \R$, sodass die Gerade in der Form
\[\{(c_1, c_2) + t(d_1, d_2) : t \in \R\}\]
geschrieben werden kann. Weiterhin ist obige Menge im Fall $(d_1, d_2) \neq  (0, 0)$ immer eine Gerade
\end{Theorem}

\begin{Remark}
	Der Parameterform f\"{u}r Geraden und Ebenen ist in der Vorlesung bewiesen.
\end{Remark}
\begin{Problem}\label{pr:linalg1-3}
	Beweisen Sie folgende Aussage:
Gegeben seien zwei Punkte $p, q \in \R^2$ mit $p \neq  q$. Dann gibt es genau eine Gerade $g \subseteq \R^2$ mit $p \in g$ und $q \in g$. Diese ist gegeben durch $g_{p,q} = \{x \in \R^2 |x_1 (q_2 - p_2 ) - x_2(q_1 - p_1 ) = p_1q_2-  p_2q_1\}$.
\end{Problem}

\begin{proof}
	Wir nutzen Def.~\ref{def:line1}. Weil $p$ und $q$ in der Gerade sind, können wir zwei Gleichungen schreiben\ldots
	\begin{align*}
		a_1p_1+a_2p_2&=b\\
		a_1q_1+a_2q_2&=b
	\end{align*}
	Dann gilt
	\begin{align*}
		a_1p_1+a_2p_2=& a_1q_1+a_2q_2\\
		a_1(p_1-q_1)=& a_2(q_2-p_2)
	\end{align*}
	Daraus folgt die L\"{o}sungsmenge
	\begin{align*}
		a_1=&t\\
		a_2=&t\frac{p_1-q_1}{q_2-p_2}\\
		b=&p_1t+p_2\frac{p_1-q_1}{q_2-p_2}t
	\end{align*}
	Es ist klar, dass die gegebene Gerade eine Lösung zu die Gleichung ist, mit $t=q_2-p_2$. Was passiert mit andere $t$? Sei $t=q_2-p_2$ und $t'\in \R$. Vergleich dann die Gleichungen
	\begin{align*}
		x_1t+x_2t\frac{p_1-q_1}{q_2-p_2}=&p_1t+p_2\frac{p_1-q_1}{q_2-p_2}t\\
		x_1t'+x_2t'\frac{p_1-q_1}{q_2-p_2}=&p_1t'+p_2\frac{p_1-q_1}{q_2-p_2}t'
	\end{align*}
	Es ist klar, dass die zweite Gleichung nur die erste Gleichung durch $t' / t$ multipliziert ist. Deshalb habe die zwei Gleichungen die gleiche Lösungsmengen, dann sind die Gerade, die durch die Gleichungen definiert werden, auch gleich.

Wenn $q_1=q_2$ d\"{u}rfen wir die L\"{o}sungemenge nicht so schreiben. Aber wir k\"{o}nnen den Beweis wiederholen, aber mit $a_2$ als das freie Parameter. Es darf nicht, dass $\left( q_1-p_1,q_2-p_2 \right) =(0,0)$, weil $\va q\neq \va 0$
\end{proof}
\begin{Problem}
	In Beispiel 1.2.8 wurde der Schnitt von zwei Ebenen bestimmt. Er hatte eine ganz bestimmte Form, die wir für den Kontext dieser Aufgabe als Gerade bezeichnen wollen, formal:
	
	Ist $(v_1 , v_2, v_3) \in \R^3 \backslash \{(0, 0, 0)\}$ und $(p_1 , p_2 , p_3 ) \in \R^3$ beliebig, dann ist die Menge
	\[\{(p_1 + t \cdot v_1, p_2 + t \cdot v_2 , p_3 + t \cdot v_3 )|t \in \R\}\]
	eine Gerade.
	\begin{parts}
		\item  Finden Sie zwei Ebenen, deren Schnitt die Gerade $g = \{(1 + 3t, 2 + t, 3 + 2t)|t \in \R\}$ ist. Erläutern Sie, wie Sie die Ebenen bestimmt haben und beweisen Sie anschließend, dass Ihr Ergebnis korrekt ist.
		\item Ist der Schnitt von zwei Ebenen immer eine Gerade? Wenn ja, begründen Sie das, wenn nein, geben Sie ein Gegenbeispiel an.
		\item Zeigen Sie: Für den Schnitt einer Geraden $g$ mit einer Ebene $E$ gilt genau einer der folgenden drei Fälle:
		\begin{itemize}
			\item $g\cap E=\varnothing$ 
			\item $|g\cap E|=1$ 
			\item $g\cap E=g$
		\end{itemize}
		Geben Sie für jeden der Fälle auch ein Geraden-Ebenen-Paar an, dessen Schnitt genau die angegebene Form hat.
	\end{parts}
\end{Problem}
\begin{proof}
	\begin{parts}
		\item  Wir suchen zwei Ebenen, also 6 Vektoren $\va p_1, \va u_1, \va u_2, \va p_2, \va v_1, \va v_2\in \R^3$, die zwei Ebenen durch
		
		\begin{align*}
			E_1=&\left\{ \va p_1+t_1\va u_1+t_2\va u_2|t_1,t_2\in \R \right\}\\
			E_2=&\left\{ \va p_2+t_1'\va v_1+t_2'\va v_2|t_1',t_2'\in \R \right\} 
		\end{align*}
		definieren. Einfachste wäre, wenn $p_1=p_2\in g$. Sei dann $p_1=p_2=(1,2,3)^T$. Wenn $\va u_1=\va v_1=(3,1,2)^T$, ist es auch klar, dass der Schnitt  $g$ entschließt ($t_2=t_2'=0$). Dann mussen wir $\va u_2, \va v_2$ finden, f\"{u}r die gelten, 
		\[(t_,t_2')\neq (0,0)\implies t_1\va u_1+t_2\va u_2\neq t_1' \underbrace{\va u_1}_{\va u_1=\va v_1}+t_2'\va v_2\forall t_1,t_1'\in \R,\]
		also
		\[
		\xi_1 \va u_1\neq t_2'\va v_2-t_2 \va u_2 \qquad (t_2,t_2')\neq (0,0), \forall \xi_1\in \R
		.\] 

		Das bedeutet
		\begin{align*}
			\xi_1=0&: \va v_2\neq k\va u_2 \qquad \forall k\in \R \\
			\xi_1\neq 0&: \va u_1\not\in \text{span}(\va v_2,\va u_2)
		\end{align*}
	
		\begin{Remark}
			Wir k\"{o}nnen uns einfach f\"{u}r solchen $\va v_2, \va u_2$ entscheiden. Wir brauchen nur
			\[
\left<\va u_2, \va v_2 \right> = \left<\va u_1, \va u_2 \right>	=\left<\va u_1, \va v_2 \right> =0		
			.\] 
			Aber weil das innere Produkt nicht in der Vorlesung nicht diskutiert worden ist, mussen wir es nicht systematisch finden.
		\end{Remark}

		\begin{Remark}
			Eigentlich braucht man keine spezielle Grunde, um $\va u_2$ und $\va v_2$ zu finden. Wenn man irgindeine normalisierte Vektoren aus einer Gleichverteilung auf $\R^3$ nimmt, ist die Wahrscheinlichkeit, dass die eine L\"{o}sung sind, $1$.
		\end{Remark}
		Daher entscheide ich mich ganz zufällig f\"{u}r zwei Vektoren\ldots
		\begin{align*}
			\va v_2=&(1,0,0)^T\\
			\va u_2=&(0,1,0)^T
		\end{align*}
			Der Schnitt von der Ebenen kann berechnet werden\ldots
		\[
			\cancel{\va p}+t_1\va u_1+t_2\va u_2=\cancel{\va p}+t_1'\va v_1+t_2'\va v_2
		,\]
		\[
		\xi_1\va u_1+t_2\va u_2=t_2'\va v_2
		.\] 
		Also
		\begin{align*}
			\xi_1 \begin{pmatrix} 3 \\ 1 \\ 2 \end{pmatrix} +t_2 \begin{pmatrix} 0 \\ 1 \\ 0 \end{pmatrix} =&t_2'\begin{pmatrix} 1 \\ 0 \\ 0 \end{pmatrix}, 
		\end{align*}
		oder
		\[
			\begin{pmatrix} 3 & 0 & -1 \\ 1 & 1 & 0 \\ 2 & 0 & 0 \end{pmatrix} \begin{pmatrix} \xi_1\\ t_2\\ t_2' \end{pmatrix} =\begin{pmatrix} 0 \\ 0 \\ 0 \end{pmatrix} 
		.\]
		\begin{Remark}
			Hier ist es noch einmal klar, dass die einzige Lösung $\xi_1=t_2=t_2'=0$ ist, weil $\det(\dots)\neq 0$. Aber wir mussen noch eine langere Beweis schreiben\ldots
		\end{Remark}
\begin{gather*}
	\left( \begin{matrix} 3 & 0 & -1 \\ 1 & 1 & 0 \\ 2 & 0 & 0\end{matrix}\right|\left.\begin{matrix}0 \\ 0 \\ 0\end{matrix}  \right)\to \left(\begin{matrix}6 & 0 & -2\\0 & 3 & 3\\0 & 0 & 3\end{matrix}\right|\left.\begin{matrix}0 \\ 0 \\ 0\end{matrix}\right)\to \left(\begin{matrix} 1 & 0 & 0 \\ 0 & 1 & 0 \\ 0 & 0 & 1\end{matrix}\right|\left. \begin{matrix}0 \\ 0 \\ 0\end{matrix}\right)
\end{gather*}
also die einzige Lösung ist $\xi_1=t_2=t_2'=0 \implies t_2=t_2'=0,t_1=t_2\implies E_1\cap E_2=g$ 

		\item Nein.
		\[
		E_1=\begin{pmatrix} 0 \\ 0 \\ 0 \end{pmatrix} +u_1 \begin{pmatrix} 1 \\ 0 \\ 0 \end{pmatrix} +u_2 \begin{pmatrix} 0 \\ 1 \\ 0 \end{pmatrix} \qquad u_1,u_2\in \R
		,\] 
		\[
		E_2=\begin{pmatrix} 0 \\ 0 \\ 1 \end{pmatrix} +u_1\begin{pmatrix} 1 \\ 0 \\ 0 \end{pmatrix}+u_2\begin{pmatrix} 0 \\ 1 \\ 0 \end{pmatrix}, \qquad u_1,u_2\in \R
		.\]
		Dann ist $E_1\cap E_2=\varnothing$
	\item 
		\begin{Theorem}
			Sei $\va a, \va b\in \R^n, n\in \N$. Dann gibt es genau eine Gerade $g$, wof\"{u}r gilt $\va a \in g, \va b \in g$. Es kann als
			\[
			\va a+t(\va b-\va a), t\in \R
			\] 
			geschrieben werden.
		\end{Theorem}
		\begin{proof}
			Es ist klar, dass
			\begin{gather*}
				\va a\in g\qquad(t=0)\\
				\va b\in g\qquad(t=1)
			\end{gather*}
			Sei dann eine andere Gerade $g'$, wof\"{u}r gilt $\va a\in g'$ und $\va b\in g'$. $g'$ kann als
			\[
			\va u+t\va v, t\in \R
			\]
			geschrieben werden, wobei $\va u, \va v\in\R^n$. Es existiert $t_1,t_2\in\R$, sodass
			\begin{gather*}
				\va u+t_1\va v=\va a\\
				\va u+t_2\va v=\va b
			\end{gather*}
			Es gilt dann
			\begin{gather*}
				\va u=\va a-t_1\va v\\
				\va a-t_1\va v+t_2\va v=\va b\\
				\va v = \frac{1}{t_2-t_1}(\va b - \va a)\qquad t_1\neq t_2\text{ weil }\va a \neq \va b
			\end{gather*}
			Es gilt dann f\"{u}r $g'$: 
			\begin{align*}
				g'=& \left\{ \va u+t\va v|t\in \R \right\} \\
				=& \left\{ \va a-\frac{t_1}{t_2-t_1}(\va b-\va a)+\frac{t}{t_2-t_1}(\va b-\va a)|t\in \R \right\}\\
				=& \left\{\va a+\left( \frac{t}{t_2-t_1}-\frac{t_1}{t_2-t_1} \right) \left( \va b -\va a \right) | t\in \R \right\}
			\end{align*}
			Wenn man $t'=\frac{t}{t_2-t_1}-\frac{t_1}{t_2-t_1}$ definiert, ist es dann klar, dass $g'=g$
		\end{proof}

			Es ist klar, dass maximal eines der F\"{a}lle gelten kann. Wir nehmen an, dass die erste zwei F\"{a}lle nicht gelten. Dann gilt
		\[
		|g\cap E|\ge 2
		.\] 
		Es gibt dann mindestens zwei Punkte in $g\cap E$. Es ist auch klar, dass die Verbindungsgerade zwische die beide Punkte $g$ ist (Pr.~\ref{pr:linalg1-3})


		\begin{Theorem}
			Sei $\va v_1, \va v_2\in E$. Dann ist die Verbindungsgerade zwischen $\va v_1$ und $\va v_2$ auch in $E$.
		\end{Theorem}
		\begin{proof}
			Sei
			\[
			E=\left\{ \va p_1+t_1\va u+t_2\va v|t_1,t_2\in \R \right\} 
			.\] 
			Es wird angenommen, dass $a_1,a_2,b_1,b_2$ existiert, sodass
			\begin{align*}
				\va v_1=& \va p+a_1\va u+a_2\va v\\
				\va v_2=&\va p+b_1\va u+b_2\va v
			\end{align*}
			Dann ist
			\[
			\va v_2-\va v_1=(b_1-a_1)\va u+(b_2-a_2)\va v,\]
			also
			\begin{align*}
				\va v_1+t(\va v_2-\va v_1)=& \va p+a_1\va u+a_2\va v+t\left[ (b_1-a_1)\va u+(b_2-a_2)\va v \right]\\
				=& \va p+\left[ a_1+t(b_1-a_1) \right] \va u+\left[ a_2+t(b_2-a_2) \right] \va v\in E
			\end{align*}
		\end{proof}

		Deshalb ist $g\subseteq g\cap E$. Weil $g\cap E\subseteq g$, ist $g=g\cap E$
	\end{parts}
\end{proof}

