\begin{Problem}\label{pr:linalg1blatt8-1}
Entscheiden Sie, welche der folgenden Abbildungen linear sind.	
\begin{parts}
	\item $f:\R^2\to \R~(x,y)\to x\cdot y$ 
	\item $g:\R^2\to \R~(x,y)\to x+y$ 
	\item $h:\Q[t]\to \Q[t]~p(t)\to p(t^2)$ 
	\item $k:\Q\to \Q$ mit $k(t)=t+2$ 
	\item $l:\C\to \C$ mit $l(z)=\overline{z}$ mit $\C$ als $\R$-Vektorraum
	\item $l$, aber mit $\C$ als $\C$-Vektorraum
\end{parts}
\end{Problem}
\begin{proof}
	\begin{parts}
	\item Nein. $f((1,1))=1\cdot 1 = 1$, aber $f(2(1,1))=f((2,2))=2\cdot 2=4\neq 2(1)$.
	\item Ja. Sei $(x_1,x_2),(y_1,y_2)\in \R^2$. Es gilt
		\begin{align*}
			f((x_1,x_2)+(y_1,y_2))=&f((x_1+y_1,x_2+y_2))\\
			=&(x_1+y_1)+(x_2+y_2)\\
			=&(x_1+x_2)+(y_1+y_2)\\
			=&f((x_1,x_2))+f((y_1,y_2))
		\end{align*}
		Sei außerdem $\lambda \in \R$. Es gilt
		\begin{align*}
			f(\lambda(x_1,x_2))=&f((\lambda x_1, \lambda x_2))\\
			=& \lambda x_1+\lambda x_2\\
			=& \lambda(x_1+x_2)\\
			=&\lambda f((x_1,x_2))
		\end{align*}
	\item Ja. Sei $p,q\in Q[t]$, $p=p_0+p_1t+p_2t^2+\dots + p_n t^n$ und $q=q_0+q_1t+q_2t^2+\dots+q_nt^n$. Es gilt
		\begin{align*}
			h(p(t))=&p_0+p_1t^2+p_2t^4+\dots+p_nt^{2n}\\
			h(q(t))=&q_0+q_1t^2+q_2t^4+\dots+q_nt^{2n}\\
			h(p(t))+h(q(t))=&(p_0+q_0))+(p_1+q_1)t^2+\dots+(p_n+q_n)t^{2n}\\
			=&h(p+q)
		\end{align*}
		Sei außerdem $\lambda\in \Q$. Es gilt
		\begin{align*}
			h(\lambda p(t))=&\lambda p_0+\lambda p_1t^2+\lambda p_2t^4+\dots+\lambda p_n t^{2n}\\
			=& \lambda\left( p_0+p_1t^2+\dots+p_nt^{2n} \right) \\
			=&\lambda h(p(t))
		\end{align*}
	\item Nein. Es gilt $k(2)=4$, aber $k(2\cdot 2)=k(4)=6\neq 2 k(2)=8$.
	\item Ja. Sei $z_1,z_2\in \C$, Es gilt $\overline{z_1+z_2}=\overline{z}_1+\overline{z}_2$. 

		Sei außerdem $\lambda\in \R$. Es gilt dann
		\[
		\overline{\lambda z_1}=\overline{\lambda}\overline{z}_1=\lambda \overline{z}_1
		.\] 
	\item Nein. Die erste Eigenschaft bleibt wie in (e), aber die zweite nicht. Sei $\lambda\in \C$. Es gilt
		\[
		\overline{\lambda z_1}=\overline{\lambda}\overline{z}_1\neq \lambda \overline{z_1}
	\]
	solange $\lambda \not\in \R$. Sei z.B. $\lambda=i,~z_1=i$. Dann gilt $\lambda z_1=-1$ und $\overline{\lambda z_1}=-1$. Das ist aber ungleich $\lambda \overline{z}_1=i(\overline{i})=i(-i)=1$.\qedhere
	\end{parts}
\end{proof}

\begin{Problem}
Entscheiden Sie, welche der folgenden linearen Abbildungen injektiv, surjektiv bzw. bijektiv sind.
\begin{parts}
\item $\R^3\to \R^2,x\to Ax$ mit
	\[
		A=\begin{pmatrix} 3 & 4 & 5\\0 & 42 & 0 \end{pmatrix} 
	.\] 
\item $\R^3\to \R^3,x\to Ax$ mit
	\[
		A=\begin{pmatrix} 3 & 4 & 5 \\ 0 & 42 & 0 \\ 4 & 3 & 2 \end{pmatrix} 
	.\] 
\item $\Q[t]\to \Q[t],p(t)\to p'(t)$
\item $\C^2\to \C^2$ mit $(z,w)\to (z+w,z-\overline{w})$, wobei wir $\C^2$ als $\R$-Vektorraum auffassen.
\item $\text{End}_{\R}(\C)\to\text{End}_{\R}$ mit $f\to \text{Re}(f|_\R)+\text{Im}(f|_{\R})$, wobei $f|_{\R}:\R\to \R$ mit $f|_{\R}(x):=f(x)$ f\"{u}r $x\in \R$ und $\Re$ bzw. $\Im$ den Real bzw. Imaginärteil bezeichnen.
\end{parts}
\end{Problem}
\begin{proof}
	\begin{parts}
	\item Nicht injektiv, weil die Spalten nicht linear unabhängig sind. Insbesondere gilt
		\[
		A\begin{pmatrix} 5 \\ 0 \\ 0 \end{pmatrix} =A\begin{pmatrix} 0 \\ 0 \\ 3 \end{pmatrix} =\begin{pmatrix} 15 \\ 0 \end{pmatrix} 
		.\] 
		Es ist surjektiv, weil die erste zwei Spalten eine Basis sind.
	\item
		\begin{gather*}
		\left(
\begin{array}{ccc}
 3 & 4 & 5 \\
 0 & 42 & 0 \\
 4 & 3 & 2 \\
\end{array}
\right) \xrightarrow{R_1\times \frac{1}{3}} \left(
\begin{array}{ccc}
 1 & \frac{4}{3} & \frac{5}{3} \\
 0 & 42 & 0 \\
 4 & 3 & 2 \\
\end{array}
\right) \xrightarrow{R_3-4R_1} \\\left(
\begin{array}{ccc}
 1 & \frac{4}{3} & \frac{5}{3} \\
 0 & 42 & 0 \\
 0 & -\frac{7}{3} & -\frac{14}{3} \\
\end{array}
\right) \xrightarrow{R_3\times 18} \left(
\begin{array}{ccc}
 1 & \frac{4}{3} & \frac{5}{3} \\
 0 & 42 & 0 \\
 0 & -42 & -84 \\
\end{array}
\right) \xrightarrow{R_3+R_2} \left(
\begin{array}{ccc}
 1 & \frac{4}{3} & \frac{5}{3} \\
 0 & 42 & 0 \\
 0 & 0 & -84 \\
\end{array}
\right)	
		\end{gather*}
		also es ist injektiv und surjektiv, daher bijektiv.
	\item Nicht injektiv. Sei $p=x+1$ und $q=x+2$. Dann ist $p'=q'=1$, aber $p\neq q$. 

		Es ist aber surjektiv. Sei $\Q[t]\ni p=a_0+a_1t+a_2t^2+\dots + a_nt^n$. Dann ist $q=a_0t+\frac{a_1}{2}t^2+\frac{a_2}{3}t^3+\dots + \frac{a_n}{n+1}t^{n+1}$ ein Polynom, dessen Bild $p$ ist.
	\item Es ist nicht injektiv. Sei $(z_1,w_1),(z_2,w_2)\in \C^2$, $z_1=0,z_2=i,w_1=2+3i,w_2=2+2i$. Es gilt dann
		\begin{align*}
			(z_1+w_1,z_1-\overline{w}_1)=&(2+3i,-2+3i)\\
			(z_2+w_2,z_2-\overline{w}_2)=&(2+3i,-2+3i)
		\end{align*}
		Es ist auch nicht surjektiv. Es gilt
		\[
		\Im(z_1+w_1)=\Im(z_1)+\Im(w_1)
	\]
	und
	\[
	\Im(z_1-\overline{w}_1)=\Im(z_1)-\Im(\overline{w}_1)=\Im(z_1)+\Im(z_2)
	.\] 
	Dann gilt f\"{u}r alle $(z,w)$ im Bild, dass $\Im(z)=\Im(z)$. Da es gibt Punkte in $\C^2$, die das nicht erfüllen, ist die Abbildung nicht surjektiv.
\item Es ist nicht injektiv. Sei $h:\R\to \R, h(x)=x$. Dadurch definieren wir zwei Abbildungen $f,g:\R\to \C$:
	\begin{align*}
		f(x)=&h(x)\\
		g(x)=&i h(x)
	\end{align*}
	Dann gilt $f\neq g$. Aber
	\begin{align*}
		\Re(f)=&\Im(g)=h\\
		\Im(f)=&\Re(g)=0
	\end{align*}
	also die zwei Funktionen werden auf die gleiche Funktion abgebildet.

	Es ist aber surjektiv. F\"{u}r jede $f:\R\to \R$ definieren wir $g:\R\to \C$ mit $g=f$. Dann wird $g$ auf $f$ abgebildet.\qedhere
	\end{parts}
\end{proof}
\begin{Problem}
Geben Sie je eine lineare Abbildung mit den folgenden Eigenschaften an. Sie müssen Ihre Aussagen ausnahmsweise nicht beweisen. 	
\begin{parts}
\item $L_1:\R^2\to \R$ mit $L(x)=x$ nur f\"{u}r $x=(0,0)$.
\item $L_2:\R^3\to \R^2$, sodass $L_2((1,1,1))=L_2((1,1,0))$.
\item $L_3:\Q[t]\to \Q[t]$, sodass $\text{deg}(L_3(p(t)))\ge 3\text{deg}(p(t))$ f\"{u}r alle $p\in \Q[t]$.
\item $L_3:V \to V$, die injektiv, aber nicht surjektiv ist, für einen $\Q$-Vektorraum Ihrer Wahl. 
\item $L_5:(\Z / 3\Z)^2\to (\Z / 3\Z)^2$, sodass es genau drei verschiedene Elemente $x,y,z\in (\Z / 3\Z)^2$ mit $L_5(x)=L_5(y)=L_5(z)=(1,0)$ gibt. 
\end{parts}
\end{Problem}
\begin{proof}
	\begin{parts}
	\item $L_1((x,y))=(2x,2y)$ ist linear, aber $L(x)=x$ nur f\"{u}r $x=(0,0)$.
	\item Projektor: $L_2((x,y,z))=(x,y)$.
	\item $p(t)\to p(t^3)$ (wie in \ref{pr:linalg1blatt8-1})
	\item F\"{u}r $V=\Q[t]$: $L_5:p(t)\to p(t)t$.\qedhere
	\end{parts}
\end{proof}

\begin{Problem}
Die folgenden linearen Abbildungen können jeweils auch in der Form $x\to Ax$ mit einer Matrix $A$ geschrieben werden. Bestimmen Sie für jede der Abbildungen eine geeignete Matrix. 	
\begin{parts}
\item $f:\R^3\to \R^2$ mit
	\[
	\begin{pmatrix} a \\ b \\ c \end{pmatrix} \to \begin{pmatrix} a +b\\b-c \end{pmatrix} 
	.\] 
\item $g:\R^2\to \R^3$ mit
	\[
	\begin{pmatrix} x \\ y \end{pmatrix} \to \begin{pmatrix} x+y\\x-y\\x+y \end{pmatrix} 
	.\] 
\item $f\circ g$.
\item $g\circ f$.
\end{parts}
\end{Problem}

\begin{proof}
	\begin{parts}
	\item 
		\[
			\begin{pmatrix} 1 & 1 & 0 \\0 & 1 & -1 \end{pmatrix} 
		.\] 
		Wir verifizieren es:
		\[
			\begin{pmatrix} 1 & 1 & 0 \\ 0 & 1 & -1 \end{pmatrix} \begin{pmatrix} a \\ b \\ c \end{pmatrix} =\begin{pmatrix} a+b\\b-c \end{pmatrix} 
		.\] 
	\item 
		\[
			\begin{pmatrix} 1 & 1 \\ 1 & -1 \\ 1 & 1 \end{pmatrix} 
		.\] 
		Noch einmal können wir direkt verifizieren:
		\[
			\begin{pmatrix} 1 & 1 \\ 1 & -1 \\ 1 & 1 \end{pmatrix} \begin{pmatrix} x \\ y \end{pmatrix} =\begin{pmatrix} x + y \\ x - y \\ x + y \end{pmatrix} 
		.\] 
	\item Die Matrixdarstellung ist nur das Produkt:
\[
	\begin{pmatrix} 1 & 1 \\ 1 & -1 \\ 1 & 1 \end{pmatrix} \begin{pmatrix} 1 & 1 & 0 \\ 0 & 1 & -1 \end{pmatrix}=\begin{pmatrix} 1 & 2 & -1 \\ 1 & 0 & 1 \\ 1 & 2 & -1 \end{pmatrix}  
.\] 
\item Noch einmal:
	\[
		\begin{pmatrix} 1 & 1 & 0 \\ 0 & 1 & -1 \end{pmatrix} \begin{pmatrix} 1 & 1 \\ 1 & -1 \\ 1 & 1 \end{pmatrix} =\begin{pmatrix} 2 & 0 \\ 0 & -2 \end{pmatrix} 
	.\qedhere\] 
	\end{parts}
\end{proof}

\begin{Problem}
	Wir betrachten die Abbildung $S_n:\Q[t]_{\le n}\to \Q[t]_{\le n}$ mit $p(t)\to p'(t)+\tilde{p}(0)t^n$.
	\begin{parts}
	\item Beweisen Sie: $S_n$ ist f\"{u}r jedes $n\in \N$ linear.
	\item Untersuchen Sie $S_n$ auf Injektivität, Surjektivität und Bijektivität.
	\item Beweisen Sie: $S_n^k(t^k)=k!$ und $S_n^{n-k}(t^n)=n! / k!t^k$ f\"{u}r $k=0,\dots,n$.
	\item Folgern Sie: $S_n^{n+1}(p(t))=n!p(t)$ f\"{u}r alle $n\in \N$, $p(t)\in \Q[t]_{\le n}$.
	\end{parts}
\end{Problem}
\begin{proof}
	\begin{parts}
	\item Sei $q,p\in \Q[t]_{\le n}$. Es gilt
		\begin{align*}
			S_n(q+p)=&(q+p)'(t)+\widetilde{q+p}(0)t^n\\
			=&q'(t)+p'(t)+\tilde{q}(0)t^n+\tilde{p}(0)t^n\\
			=&(q'(t)+\tilde{q}(0)t^n)+(p'(t)+\tilde{p}(0)t^n)\\
			=&S_n(q)+S_n(p).
		\end{align*}
		Sei außerdem $\lambda\in \Q$. Es gilt
		\begin{align*}
			S_n(\lambda q)=&(\lambda q)'(t)+\widetilde{\lambda q}(0)t^n\\
			=&\lambda q'(t)+\lambda\tilde{q}(0)t^n\\
			=&\lambda\left( q'(t)+\tilde{q}(0)t^n \right) \\
			=&\lambda S_n(q)
		\end{align*}
	\item Wir schreiben die Wirkung von $S_n$ auf einem Polynom:
		\[
		S_n:(a_0,a_1,a_2,\dots, a_n)\to (a_1,2a_2,3a_3,\dots, na_n, a_0)
		.\] 
		Daraus folgt die Injektivität und Surjektivität: Sei $p=(a_0,a_1,\dots, a_n)$ und $q=(b_0,b_1,\dots, b_n)$. Wann ist $S_n(p)=q$? Es gilt genau dann, wenn
		\[
			(a_1,2a_2,3a_3,\dots, na_n, a_0)=(b_0,b_1,b_2,\dots, b_n)
		.\] 
		Dann ist es klar: $a_1=b_0$, $2a_2=b_1$, $\dots$. Weil alle Koeffizienten noch rational sind, gilt: 
	\end{parts}
\end{proof}
