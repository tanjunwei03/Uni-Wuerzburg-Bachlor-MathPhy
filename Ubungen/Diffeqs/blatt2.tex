\begin{Problem}
Bestimmen Sie die Lösungen der Anfangswertprobleme
\begin{parts}
	\item $\dot{x}=(5t+5x)^2,\qquad x(0)=0$ und
	\item $\dot{x}=\frac{t^2-x^2}{-5tx},\qquad x(1)=1$.
\end{parts}
\end{Problem}
\begin{proof}
	\begin{parts}
	\item Die Abbildung $T:\R^2\to \R^2,~(t,x)\mapsto (t, x+t)=:(t, u)$ ist ein Diffeomorphismus. Damit ist die Ableitung
	\[
		\dot{x}=\dot{u}-1
	\]
	und die Gleichung ist
	\[
		\dot{u}-1=25u^2
	.\]
	Diese Gleichung ist separabel und besitzt L\"{o}sung
	\[
	\int_0^u \frac{1}{1+25s^2}\dd{s}=\int_0^t \dd{r}
	,\]
	also
	\[
		\frac{1}{5}\tan^{-1}(5u)=t
	.\] 
	Setzt man $x=u-t$ wieder ein, so erhält man
	\[
	x=\frac{1}{5}\tan (5t)-t
	.\] 
\item Man verifiziere einfach, dass die Gleichung homogen ist. Daher verwenden wir wie im Skript die Substitution $u=x / t$ bzw. der Diffeomorphismus
	\[
	T:(t,x)\mapsto (t, x / t)
	.\] 
Wie im Skript ist die transformierte DGL
\[
	\dot{u}=\frac{\frac{1-u^2}{-5u}-u}{t}
\] 
was nach Vereinfachung
\[
	\dot{u}=\frac{1}{t}\frac{1-u^2+5u^2}{-5u}=-\frac{1}{t}\frac{1+4u^2}{5u}
\] 
ist. Die Gleichung ist jetzt separabel mit L\"{o}sung gegeben durch
\[
	\int_1^u \frac{5s}{1+4s^2}\dd{s}=\int_1^t -\frac{1}{r}\dd{r}
,\] 
oder
\[
\frac{5}{8}\ln \left[ \frac{1}{5}\left( 1+4u^2 \right)  \right]=-\ln t 
.\] 
Daraus ergibt sich
\[
	\frac{1}{5}(1+4u^2)=t^{-8 / 5}
\] 
und
\[
	x=t\sqrt{\frac{1}{4}\left( 5t^{-8 / 5}-1 \right) }=\frac{t}{2}\sqrt{5t^{-8 / 5}-1}  
.\qedhere\] 
	\end{parts}
\end{proof}
\begin{Problem}
	Zeigen Sie, dass die Differentialgleichung
	\begin{parts}
	\item $\cos(t)+\sin(x)+(t\cos(x)+x)\dot{x}=0\qquad x(0)=0$ exakt ist und bestimmen Sie die Stammfunktion $F_0(t,x)$.
	\item $\frac{1}{4}t^4 \dot{x}+3x+t^2\dot{x}+t^3x=-3t\dot{x}-5\dot{x}-2tx,\qquad x(0)=1$ exakt ist, bestimmen Sie Stammfunktion $F_0(t, x)$ und bestimmen Sie eine Lösung.
	\end{parts}
\end{Problem}
\begin{proof}
	\begin{parts}
	\item Die Koeffizientenfunktionen sind auf ganz $\R^2$ definiert, was ein Rechteck ist. Daher betrachten wir wie im Skript
		\[
			\pdv{(\cos t+\sin x)}{x}=\cos t=\pdv{(t\cos x + x)}{t}
		.\] 
		Die DGL ist also exakt. Eine mgliche Stammfunktion ist
		\[
		F_0(t, x)=t\sin x+\frac{x^2}{2}+\sin t
		.\] 
	\item Die Gleichung umgeformt ist
		\[
			\underbrace{\left(3x+t^3x+2tx \right)}_{M(t, x)}+\underbrace{\left( t^2+3t+5+\frac{t^4}{4} \right)}_{N(t,x)}\dot{x}=0
		.\] 
Die Koeffizientenfunktionen sind wieder auf $\R^2$ definiert. Daher betrachten wir
\[
	\pdv{M}{x}=3+t^3+2t=\pdv{N}{t}
.\] 
Die DGL ist also exakt. Eine Stammfunktion ist
\[
F_0(t, x)=3xt+\frac{1}{4}t^4 x+t^2 x+5x
.\] 
Dann setzen wir $F_0(t_0, x_0)$ ein und erhalten
\[
F_0(0,1)=5
.\] 
Die Lösung ist gegeben durch
\[
F_0(t,x)=x\left( 3t+\frac{t^4}{4}+t^2+5 \right) =F_0(0,1)=5
,\] 
und die $x(t)$ ist
\[
x(t)=\frac{5}{3t+\frac{t^4}{4}+t^2+5}
.\qedhere\] 
	\end{parts}
\end{proof}


\begin{Problem}
	Gegeben sei eine Differentialgleichung für $(t,x)\in U\subseteq \R \times \R$ mit der Form
	\[
		M(t, x)+N(t, x)\dot{x}=0
	.\] 
	die der Exaktheitsbedingung $\pdv{M}{x}=\pdv{N}{t}$ \emph{nicht} genügt.
	\begin{parts}
	\item Seien $N,M$ stetig differenzierbare Funktionen auf $\tilde{D}\to \R$, wobei $M(t, x)\neq 0$ f\"{u}r alle $(t,x)\in \tilde{D}$ f\"{u}r ein offenes Rechteck $\tilde{D}\subseteq U$. Zeigen Sie: H\"{a}ngt $\beta(t, x):=\frac{1}{M}\left( \pdv{M}{x}-\pdv{N}{t} \right) $ allein von $x$ ab, so ist $\mu(x):=\exp\left( -\int_{x_0}^x \beta(x)\dd{s} \right) $ f\"{u}r $(t_0, x_0)\in \tilde{D}$ ein integrierende Faktor von der Gleichung.
	\item \"{U}berpr\"{u}fen Sie die Differentialgleichung
		\[
			-2tx+(3t^2-x^2)\dot{x}=0,\qquad x(1)=1
		\]
		auf Exaktheit und bestimmen Sie dann eine Stammfunktion $F_0(t,x)$ im Falle der Exaktheit. Falls sie nicht exakt ist, finden Sie einen integrierenden Faktor und bestimmen Sie dann eine Stammfunktion $F_0(t,x)$.
	\end{parts}
\end{Problem}
\begin{proof}
	\begin{parts}
	\item Da $\tilde{D}$ ein Rechteck ist, ist die DGL exakt genau dann, wenn
	\[
		\pdv{M(t, x)\mu(x)}{x}=\pdv{N(t, x)\mu(x)}{t} 
	.\] 
	Dann berechnen wir die Ableitungen
	\begin{align*}
		\pdv{M(t, x)\mu(x)}{x}&= M(t, x)\pdv{\mu(x)}{x}+\mu(x)\pdv{M(t,x)}{x}\\
				      &=M(t, x)\mu(x)(-\beta(x))+\mu(x)\pdv{M(t,x)}{x}\\
		\pdv{(N(t,x)\mu(x))}{t}&=\mu(x)\pdv{N(t,x)}{t}
	\end{align*}
	Die beide sind genau dann gleich, wenn
	\[
		\pdv{M(t, x)}{x}-\pdv{N(t,x)}{t}=\beta(x)M(t,x)
	,\] 
	was per Definition von $\beta(x)$ erf\"{u}llt ist.
\item Da die Koeffizientenfunktionen auf $\R^2$ definiert sind, ist die DGL genau dann exakt, falls
\end{parts}
\end{proof}
\begin{Problem}
	\begin{parts}
	\item Bestimmen Sie die L\"{o}sung des Anfangswertproblems
		\[
			\dot{x}-\frac{3t^2+4}{t^3+4t}x=t\qquad x(2)=0
		.\] 
	\item F\"{u}r welche Anfangswerte $x(1)=x_0,~x_0\in \R$, hat die Differentialgleichung
		\[
			t\dot{x}=x+2t^3
		\]
		eine L\"{o}sung? Bestimmen Sie alle m\"{o}glichen L\"{o}sungen des Anfangswertproblems.
	\end{parts}
\end{Problem}
\begin{proof}
	\begin{parts}
	\item Wir verwenden Variation der Konstante. Zunächst lösen wir die homogene DGL
		\[
			\dot{x}-\frac{3t^2+4}{t^3+4t}x=0
		.\] 
		Die Koeffizientfunktion hat Stammfunktion
		\[
			\dv{t} \ln |t^3 + 4t|=\frac{3t^2+4}{t^3+4t}
		\]
		und die Lösung ist damit
		\[
			x(t) = C|t^3 + 4t|
		.\] 
		Da $[t^3+4t]_{t=2}>0$, können wir eine offene Umgebung finden, sodas $t^3+4t$ in dieser Umgebung positiv ist. Daher schreiben wir einfach
		\[
		x(t)=C(t^3+4t)
		.\] 
		Für die Methode ersetzen wir die Konstante $C$ durch eine Funktion $C(t)$. Mit diesem Ansatz ergibt sich eine Gleichung für $C(t)$: 
		\[
		C'(t)(t^3+4t)=t
		.\] 
		Die allgemeine Lösung ergibt sich einfach durch Integration:
		\[
			C(t)=\frac{1}{2}\tan^{-1}\frac{t}{2}+A
		\] 
		und die Lösung für $x(t)$ ist
		\[
			x(t)=\left( \frac{1}{2}\tan^{-1}\frac{t}{2}+A \right)(t^3+4t)
		.\] 
		Nun wird $t=2$ eingesetzt, um $A$ zu bestimmen. Die Lösung ist
		\[
		A=-\frac{\pi}{8}
	\]
	und
	\[
		x(t)=\frac{1}{2}\left( \tan^{-1}\frac{t}{2}-\frac{\pi}{4} \right)(t^3+4t)
	.\] 
\item Wieder durch Variationen der Konstanten: Zunächst lösen wir die homogene DGL
	\[
		t\dot{x}=x
	\]
	und erhalten als allgemeine Lösung
	\[
	x=Ct,~C\in \R
	.\] 
	Dann setzen wir als Ansatz $x=c(t)t$ ein und erhalten eine Gleichung für $c(t)$:
	\[
t^2 c'(t)=2t^3
	\] 
	mit Lösung
	\[
	c(t) = t^2+A
	.\] 
	Die allgemeine Lösung für $x(t)$ ist also
	\[
	x(t) = \left(t^2+A \right)t
	.\] 
	Die Gleichung $x(1)=x_0$ ist
	\[
	x_0=1+A
,\]
was immer eine Lösung für $A$ besitzt. Daher gibt es eine Lösung für alle $x_0\in \R$.\qedhere
	\end{parts}
\end{proof}
