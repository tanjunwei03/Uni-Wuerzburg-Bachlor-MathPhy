\begin{Problem}
	Sei $A:\R\to \R^{2\times 2}$. Gegeben ist die Differentialgleichung $\dot{x} = A(t)x$ durch
	\begin{align*}
		\dot{x}_1&= (3t-1)x_1-(1-t)x_2,\\
		\dot{x}_2&= -(t+2)x_1 + (t-2)x_2.
	\end{align*}
	\begin{parts}
		\item  Zeigen Sie, dass die Wronski-Determinante $w$ durch $w = c\cdot e^{2t^2-3t}, c \in \R$ gegeben ist.
		\item Eine Lösung der Differentialgleichung ist durch $\varphi_1 : \R \to \R^2 , \varphi_1 (t) =
		\begin{pmatrix} 1 \\ -1 \end{pmatrix}e^{t^2}$ gegeben. Bestimmen Sie mit Hilfe der Wronski-Determinante $w$ eine weitere von $\varphi_1$ linear unabhängige Lösung der Differentialgleichung.
		
		(Hinweis: Setzen Sie dazu mit Begründung bei der Wronski-Determinante aus Teil a) für c einen festen Wert ein und benutzen Sie als Ansatz für die zweite Lösung $\varphi_2 (t) = (u(t), v(t))^T$.)
	\end{parts}
\end{Problem}
\begin{proof}
	\begin{parts}
		\item Das Gleichungsssystem können wir umschreiben als
		\[\dv{t}\begin{pmatrix} x_1 \\ x_2\end{pmatrix}= \underbrace{\begin{pmatrix}
		3t-1 & t-1 \\
		-(t+2) & t-2 
		\end{pmatrix}}_{A(t)}
		\begin{pmatrix}
			x_1\\
			x_2
		\end{pmatrix}\]
		mit Spur $\Tr(A)(t)=4t-3$. Die Wronski-Determinante ist damit bestimmt durch die Gleichung
		\[\dot{w}=(4t-3)w\]
		Die DGL hat L\"{o}sung
		\[w = e^{\int 4t-3 \dd{t}}=c\cdot e^{2t^2 - 3t}.\]
		\item Die Wronski-Matrix ist gegeben durch
		\[W = \begin{pmatrix}
			e^{t^2} & u(t)\\
			-e^{t^2} & v(t)
		\end{pmatrix}\]
	mit Determinante
	\[\det(W)= [u(t)+v(t)]e^{t^2}=c \cdot e^{2t^2 - 3t}.\]
	Wir wählen $c=1$ und damit
	\[u(t)+v(t) = e^{t^2 - 3t}.\]
	Eine L\"{o}sung ist
	\[u(t)=v(t) = \frac 12 e^{t^2-3t}.\qedhere\]
	\end{parts}
\end{proof}

\begin{Problem}
	Gegeben sei die Differentialgleichung $\dot{x} = Ax$ mit 
	$
	A = \begin{pmatrix}
		1 & -1 \\
		1 & 1
	\end{pmatrix}.
	$	
	\begin{parts}
		\item Bestimmen Sie die reellen Eigenschwingungen und mit diesen die allgemeine Lösung der Differentialgleichung und eine Fundamentalmatrix.
		\item Bestimmen Sie mit den reellen Eigenschwingungen die Lösung der Differentialgleichung zum Anfangswert 
		$
		x(1) = \begin{pmatrix}
			2 \\
			1
		\end{pmatrix}.
		$
	\end{parts}
\end{Problem}
\begin{proof}
	\begin{parts}
		\item $A$ hat charakteristisches Polynom $\lambda^2-2\lambda+2$ und damit Eigenwerte $1\pm i$. Da die Eigenwerte unterschiedlich sind, ist $A$ diagonalisierbar. Die Eigenvektoren sind $(i, 1)^T$ und $(-i, 1)^T$ und 
	\end{parts}
\end{proof}
\begin{Problem}
	\begin{parts}
		\item Berechnen Sie die Fundamentalmatrizen und Lösungen von
		\begin{enumerate}
			\item[i)] $\dot{x} = Ax, \, x(0) = \begin{pmatrix} 0 \\ 2 \end{pmatrix}$ mit 
			$
			A = \begin{pmatrix}
				3 & -1 \\
				-1 & 3
			\end{pmatrix}
			$
			und
			\item[ii)] $\dot{x} = Bx, \, x(0) = \begin{pmatrix} 1 \\ 1 \\ 1 \end{pmatrix}$ mit 
			$
			B = \begin{pmatrix}
				0 & -1 & 1 \\
				-3 & -2 & 3 \\
				-2 & -2 & 3
			\end{pmatrix}.
			$
			\item[iii)] $\dot{x} = Cx, \, x(0) = \begin{pmatrix} 1 \\ 3 \\ -2 \end{pmatrix}$ mit 
			$
			C = \begin{pmatrix}
				0 & 0 & -1 \\
				0 & -1 & 0 \\
				1 & 0 & 2
			\end{pmatrix}.
			$
		\end{enumerate}
		
		\item Wir erweitern a.iii) zu $\dot{x} = Cx + c(t)$ mit 
		$
		c(t) = \begin{pmatrix}
			-t \, e^{-t} \\
			e^{-t} \\
			1 + t
		\end{pmatrix}.
		$
		Lösen Sie dieses Anfangswertproblem.
	\end{parts}
\end{Problem}
\begin{proof}
	\begin{parts}
	\item 
		\begin{enumerate}[label=\roman*)]
			\item $A$ ist orthogonal diagonalisierbar 
		\end{enumerate}
	\end{parts}
\end{proof}
\begin{Problem}
	Beurteilen Sie, ob die folgenden 4 Behauptungen wahr oder falsch sind.
	
	Sie müssen bei dieser Aufgabe keine Begründungen angeben.
	
	Für jede richtig beantwortete Frage gibt es einen Punkt.
	
	Für jede falsch beantwortete Frage wird ein Punkt abgezogen.
	
	Für jede nicht beantwortete Frage gibt es keine Punkte.
	
	Die gesamte Aufgabe wird mit mindestens 0 Punkten bewertet (sie können also nicht z. B. -1 Punkte bekommen). Insgesamt können bis zu 3 Punkte erreicht werden.
	
	\begin{tabularx}{\textwidth}{X|p{2cm}|p{2cm}|}
		& Wahr & Falsch \\\hline
		Die Differentialgleichung $\dot{x}(t) = \frac{\arctan(t)x^3}{1 + x^2} + e^{-t^2} \sin(x) + 1, \, x(t_0) = x_0$ hat für alle $(t_0, x(t_0)) \in \mathbb{R}^2$ eine eindeutige Lösung. & & \\\hline
	Die Funktionen $\varphi_1, \ldots, \varphi_n, \varphi_i(t) = e^{\mu_i t}, \, i = 1, \ldots, n$, genau dann $\mathbb{R}$-linear unabhängig sind, wenn $\mu_1, \ldots, \mu_n \in \mathbb{R}$ paarweise verschieden sind. & & \\\hline
		Es sei $\dot{x} = f(t, x), \, x(t_0) = x_0, \, f : Z_{a,b} \to \mathbb{R}$. Ist $f$ in $x$ lipschitz-stetig, so ist $f$ auch in $x$ stetig, womit es nach dem Satz von Picard-Lindelöf eine eindeutige Lösung gibt. & & \\\hline
	\end{tabularx}
\end{Problem}
\begin{proof}
		\begin{tabularx}{\textwidth}{X|p{2cm}|p{2cm}|}
		& Wahr & Falsch \\\hline
		Die Differentialgleichung $\dot{x}(t) = \frac{\arctan(t)x^3}{1 + x^2} + e^{-t^2} \sin(x) + 1, \, x(t_0) = x_0$ hat für alle $(t_0, x(t_0)) \in \mathbb{R}^2$ eine eindeutige Lösung. & & \\\hline
		Die Funktionen $\varphi_1, \ldots, \varphi_n, \varphi_i(t) = e^{\mu_i t}, \, i = 1, \ldots, n$, genau dann $\mathbb{R}$-linear unabhängig sind, wenn $\mu_1, \ldots, \mu_n \in \mathbb{R}$ paarweise verschieden sind. & & $\times$\\\hline
		Es sei $\dot{x} = f(t, x), \, x(t_0) = x_0, \, f : Z_{a,b} \to \mathbb{R}$. Ist $f$ in $x$ lipschitz-stetig, so ist $f$ auch in $x$ stetig, womit es nach dem Satz von Picard-Lindelöf eine eindeutige Lösung gibt. & & $\times$ \\\hline
	\end{tabularx}
\end{proof}