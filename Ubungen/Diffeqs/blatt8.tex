\begin{Problem}
	Bestimmen Sie die allgemeine Lösung der Differentialgleichungen
	\begin{parts}
		\item $\ddot{x} - 4\dot{x} - 5x = 8e^t$.
		\item $\ddot{x}(t) + x(t) = 4t \sin(t) - 2 \sin(t)$.
	\end{parts}
\end{Problem}
\begin{proof}
	\begin{parts}
	\item Die charakteristische Gleichung ist
\[
\lambda^2 - 4\lambda - 5 = 0
\]
mit L\"{o}sungen
\[
\lambda = -1, 5
.\] 
Zwei lineare unabhängige Lösungen der homogenen DGL sind $e^{-t}$ und $e^{5t}$. F\"{u}r eine partikulare L\"{o}sung verwenden wir als Ansatz $x = Ae^{t}$. Eingesetzt ist
\[
A e^t -4 Ae^t - 5 Ae^t =-8 A e^t = 8e^t	 
.\]
und damit $A=-1$. Die allgemeine L\"{o}sung ist
\[
	\varphi(t) = c_1 e^{-t} + c_2 e^{5t} - 8 e^t, c_1,c_2\in \R
.\] 
\item Die charaktische Gleichung der homogenen DGL ist $x^2 + 1 = 0$ und hat L\"{o}sungen $x=\pm 1$. Die L\"{o}sung der homogenen DGL ist damit
	\[
	x(t) = c_1 \cos t + c_2 \sin t, c_1,c_2\in \R
	.\] 
	Zur Bestimmung einer L\"{o}sung der inhomogenen DGL verwenden wir den Ansatz
	\[
	x(t) = A t^2 \cos t+Bt \cos t
	\]
	mit
	\begin{align*}
		\dot{x}=&2A t \cos t - A t^2 \sin t + B \cos t - Bt\sin t\\
		\ddot{x}=&2A \cos t - 2 At\sin t - 2At\sin t - At^2\cos t\\
			 &-B\sin t - B\sin t -Bt\cos t\\
		=&2A\cos t-4At\sin t - At^2\cos t - 2B\sin t - Bt\cos t
	\end{align*}
	und
	\[
		\ddot{x}+x =2A\cos t - 4At\sin t - 2 B\sin t 
	.\] 
	\end{parts}
\end{proof}
\begin{Problem}
	Bestimmen Sie mit Begründung eine lineare Differentialgleichung zweiter Ordnung mit konstanten Koeffizienten, die folgende Lösungen besitzt:
	\begin{align*}
	\varphi_1 &: \mathbb{R} \to \mathbb{R}, \quad t \mapsto 2e^{3t} + \sin(3t),\\
	\varphi_2 &: \mathbb{R} \to \mathbb{R}, \quad t \mapsto 3e^{-2t} + \sin(3t),\\
	\varphi_3 &: \mathbb{R} \to \mathbb{R}, \quad t \mapsto e^{-2t} + 5e^{3t} + \sin(3t).
	\end{align*}	
\end{Problem}
\begin{proof}
	Wenn die DGL homogen wäre, bräuchten wir die folgenden Nullstellen der charakteristischen Polynom: $-2, 5, \pm 3i$. Da 4 L\"{o}sungen zu viel sind, muss die DGL inhomogen sein.

	Die drei L\"{o}sungen haben $\sin 3t$ gemeinsam. Dies muss daher der inhomogenen Teil sein. Wir suchen also eine DGL, deren homogenen Teil $e^{-2t}$ und $e^{3t}$ als linear unabhängige Lösungen hat und $\sin 3t$ als partikuläre Lösung hat. Ein charakteristisches Polynom ist
	\[
		(x+2)(x-3) = x^2 - x - 6
	\]
	und damit ist ein möglicher homogener Teil
	\[
		\ddot{x}-\dot{x}-6x=0
	.\] 
	Zur Bestimmung der inhomogenen Teil setzen wir einfach $\sin 3t$ ein:
	\[
	-3 \sin 3t - 3 \cos 3t - 6\sin 3t = -9\sin 3t - 3 \cos 3t
	.\] 
	Die DGL ist damit
	\[
		\ddot{x}-\dot{x}-6x = -9\sin 3t - 3 \cos 3t
	.\qedhere\] 
\end{proof}
\begin{Problem}
	Sei $D = \mathbb{R} \times \mathbb{R}^n$. Wir betrachten das Anfangswertproblem
	\begin{equation}\label{eq:diffeqblatt8-1}
		\dot{x} = f(t, x), \quad x(t_0) = x_0, \tag{1}
	\end{equation}
	wobei $f : D \to \mathbb{R}^n$ stetig auf $D$ und lokal Lipschitz-stetig in $x$ ist. Weiterhin sei $C \geq 0$, sodass
	\[
	\langle f(t, x), x \rangle \leq C|x|_2^2
	\]
	für alle $(t, x) \in D$. $\langle \cdot, \cdot \rangle$ bezeichne hier das Standardskalarprodukt und $|\cdot|_2$ die euklidische Norm im $\mathbb{R}^n$.
	
	Zeigen Sie: Für das maximale Existenzintervall $I = (t^-, t^+)$ der Lösung $\varphi : I \to \mathbb{R}$ von \eqref{eq:diffeqblatt8-1}, gilt $t^+ = +\infty$.
	
	\textbf{Bemerkung:} Eine Möglichkeit diese Aufgabe zu lösen, verwendet folgende Hinweise:
	\begin{enumerate}
		\item Verwenden Sie als Ansatz $y(t) = |x(t)|_2^2$.
		\item In dieser Aufgabe können Sie die Separation der Variablen ohne Beweis auch auf Differentialungleichungen der Form $y' \leq f(y)$ anwenden.
	\end{enumerate}
	\end{Problem}
	
	\newpage
	\section*{Präzensblatt}
	\begin{Problem}
		Bestimmen Sie die allgemeine reelle Lösung der Differentialgleichung
		\[
		\ddot{x} + 2x = 2\cos(t).
		\]
		\end{Problem}
	\begin{proof}
		\[
		x = A \cos(\sqrt{2} t) + B\sin (\sqrt{2} t)+2 \cos t
		.\qedhere\] 
	\end{proof}	
		\begin{Problem}
			Seien $\varphi_1, \varphi_2, \varphi_3 : \mathbb{R} \to \mathbb{R}$ gegeben durch
			\[
			\varphi_1(t) = 1, \quad \varphi_2(t) = t, \quad \varphi_3(t) = t^2
			\]
			für alle $t \in \mathbb{R}$. Es ist bekannt, dass $\varphi_1, \varphi_2, \varphi_3$ Lösungen einer linearen inhomogenen Differentialgleichung zweiter Ordnung sind.
			
			Geben Sie die Menge aller Lösungen dieser Differentialgleichung an. Die Differentialgleichung selbst ist dabei nicht zu bestimmen.  
			\textit{(Hinweis: Beachten Sie, dass die Lösungen einer linearen inhomogenen Differentialgleichung einen affinen Unterraum aufspannen, siehe Satz 7.3.)}
		\end{Problem}
	\begin{proof}
		\[A+ B t + t^2\]
	\end{proof}	
		\begin{Problem}
			Bei zeitunabhängigen linearen Differentialgleichungen $\dot{x} = Ax$ können wir mit Hilfe der Matrix-Exponentialfunktion $\exp(At)$ eine Fundamentalmatrix angeben. Man könnte daher versuchen zu beweisen, dass bei zeitabhängigen linearen Differentialgleichungen $\dot{x} = A(t)x$ die Matrix-Exponentialfunktion
			\[
			\Phi(t) := \exp\left( \int_{t_0}^t A(s) \, ds \right)
			\]
			eine Fundamentalmatrix ist.
			
			Erklären Sie, an welcher Stelle der Beweis schief gehen würde und belegen Sie dies mit einem Gegenbeispiel.
		\end{Problem}
		\begin{proof}
			\begin{align*}
				\Phi(t) &= \sum_{k=0}^\infty \frac 1{k!}\left( \int_{t_0}^t A(s)\dd{s} \right)^k\\
				\Phi'(t)&=\sum_{k=1}^\infty \frac{k}{k!}\left( \int_{t_0}^t A(s)\dd{s} \right)^{k-1}A(t)\\
					&= \left[ \sum_{k=1}^\infty \frac{1}{(k-1)!}\left( \int_{t_0}^t A(s)\dd{s} \right)^{k-1} \right]A(t)\\
					&= \exp\left( \int_{t_0}^t A(s)\dd{s} \right) A(t)
			\end{align*}
		\end{proof}
