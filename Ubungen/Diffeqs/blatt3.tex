\begin{Problem}
	\begin{parts}
		\item Eine Riccati-Differentialgleichung hat die Form
			\begin{equation}\label{eq:diffeqblatt3-1}
	\dot{x}=a(t)+b(t)x+c(t)x^2
,\end{equation}
wobei $a,b,c:D\to \R,~D\subset \R$ offen, stetige Funktionen sind. Zeigen Sie:

Kennt man eine L\"{o}sung $\varphi(t)$ von \eqref{eq:diffeqblatt3-1}, so kann man \eqref{eq:diffeqblatt3-1} mit der Transformation $y=x-\varphi(t)$ in eine Bernoullische Differentialgleichung umwandeln.
\item Kreuzen Sie bei der folgenden Tabelle an, welche Differentialgleichungen linear, nicht linear, homogen, inhomogen, bernoullisch oder riccatisch sind.

	\begin{tabular}{c|c|c|c|c|c|c}
		DGL & linear & nichtlinear & homogen & inhomogen & bernoullisch & riccatisch \\\hline
		$\sin(4t)x+3\dot{x}+\sqrt{t} =0$ & & & & & & \\\hline
		$t\dot{x}-2x+2tx^2=0$ & & & & & & \\\hline
		$\dot{x}-2e^{2t}x+3x^2=1+e^{2t}$ & & & & & & \\\hline
		$\dot{x}+3x=0$ & & & & & & \\\hline
		$\dot{x}+2x-tx^4=0$ & & & & & & 
	\end{tabular}
\end{parts}
\end{Problem}
\begin{proof}
\begin{parts}
\item Es gilt
	\begin{align*}
		x&=y+\varphi(t)\\
		\dot{x}&=\dot{y}+\dot{\varphi}(t).
	\end{align*}
	Daher ist nach Einsetzen
	\begin{align*}
		\dot{y}+\varphi'(t)&=a(t)+b(t)(y+\varphi(t))+c(t)(y^2+2y\varphi(t)+\varphi(t)^2) \\
		\dot{y}+\textcolor{blue}{\varphi'(t)}&=\textcolor{blue}{a(t)}+b(t)y+\textcolor{blue}{b(t)\varphi(t)}+c(t)y^2+2y\varphi(t)+\textcolor{blue}{c(t)\varphi(t)^2}\\
		\dot{y}&= [b(t)+2\varphi(t)]y + c(t)y^2
	\end{align*}
	wobei wir die blauen Terme kürzen dürfen, da $\varphi$ bekanntermaßen eine Lösung der DGL ist. Die Gleichung am Ende ist offensichtlich eine bernoullische DGL.
\item 

	\begin{tabular}{c|c|c|c|c|c|c}
	DGL & linear & nichtlinear & homogen & inhomogen & bernoullisch & riccatisch \\\hline
	$\sin(4t)x+3\dot{x}+\sqrt{t} =0$ & $\times$ & & & $\times $ & & $\times$ \\\hline
	$t\dot{x}-2x+2tx^2=0$ & & $\times$ & & $\times$ & $\times$ & $\times$ \\\hline
	$\dot{x}-2e^{2t}x+3x^2=1+e^{2t}$ & & $\times$ & & $\times$ & & $\times$ \\\hline
	$\dot{x}+3x=0$ & $\times$ & & $\times$ & & $\times$ & $\times$ \\\hline
	$\dot{x}+2x-tx^4=0$ & & $\times$ & & $\times$ & $\times$ & 
\end{tabular}
\end{parts}	
\end{proof}
\begin{Problem}
	In der Literatur findet man den \emph{Potenzreihenansatz} zur L\"{o}sung von Differentialgleichungen und Anfangswertproblemen. Um die Vorgehensweise dieses Ansatzes zu verstehen, betrachten wir die Differentialgleichung des Federpendels
	\begin{equation}\label{eq:diffeqblatt3-2}
		m\ddot{x}+Dx=0,
	\end{equation}
	wobei $m>0$ (Masse) und $D>0$ (Federkonstante) gilt.
	\begin{parts}
	\item Ermitteln Sie mit dem Ansatz
		\[
			\varphi(t)=\sum_{n=0}^{\infty} a_n t^n,~a_n\in \R~\text{f\"{u}r alle }n\in \N_0
		\] 
		durch gliedweises Differenzieren, Einsetzen in \eqref{eq:diffeqblatt3-2} und Koeffizientenvergleich eine Rekursionsgleichung f\"{u}r die Koeffizienten $a_n$.
	\item Leiten Sie mit Hilfe der Rekursionsgleichung aus Teil (a)
		\[
			\varphi(t)=a_0 \sum_{k=0}^{\infty} \frac{(-1)^k}{(2k)!}\left(\frac{D}{m}\right)^k t^{2k}+a_1 \sum_{k=0}^{\infty} \frac{(-1)^k}{(2k+1)!}\left( \frac{D}{m} \right)^k t^{2k+1}
		\]
		her.
	\item Zeigen Sie, dass
\[
\varphi(t)=a_0 \cos\omega t + \frac{a_1}{\omega}\sin \omega t,~\omega:=\sqrt{\frac{D}{M}} 
\]
gilt.
\item L\"{o}sen Sie mit Hilfe des Potenzreihenansatzes das Anfangswertproblem
	\[
		\dot{x}+x=t,\qquad x(0)=x_0
	.\] 
	\emph{(Hinweis: Falls Sie Probleme haben die entstehende Potenzreihe zu erkennen, dann lösen Sie für sich die Differentialgleichung mit dem Ansatz der Variation der Konstanten und vergleichen Sie Ihre Lösung mit der Potenzreihe.)} 
	\end{parts}
\end{Problem}
\begin{proof}
	\begin{parts}
		\item Da Potenzreihen gleichmäßig konvergieren, ist
		\begin{align*}
		\varphi''(t)&=\sum_{n=0}^\infty n(n-1)a_n t^{n-2}\\
		&=\sum_{n=-2}^\infty (n+2)(n+1)a_{n+2}t^n\\
		&=\sum_{n=0}^\infty (n+2)(n+1)a_{n+2}t^n
	\end{align*}
	Setzt man diesen Ansatz in \eqref{eq:diffeqblatt3-2} ein, so erhält man
	\[\sum_{n=0}^\infty\left[m(n+2)(n+1)a_{n+2}+Da_n\right]t^n=0.\]
	Da dies f\"{u}r alle $t$ verschwinden muss, muss die Koeffizienten verschwinden, also
	\[
		m(n+2)(n+1)a_{n+2}+Da_n=0
	.\] 
\item Aus der Rekursionsgleichung bekommt man zwei Folgen $(a_n)_{n\text{ gerade}}$ und $(a_n)_{n\text{ ungerade}}$. Die L\"{o}sungen sind:
	\begin{align*}
		a_{2k+1}&=\frac{(-1)^k}{(2k+1)!}\left( \frac{D}{m} \right)^k a_1,\\
		a_{2k} &= \frac{(-1)^k}{(2k)!}\left( \frac{D}{m} \right)^k a_0.
	\end{align*}
	Da eine Potenzreihe innerhalb der Konvergenzkreisscheibe absolut konvergiert, können wir die Reihe umordnen bzw. in dieser zwei Reihen zerlegen.
	\begin{align*}
		\varphi(t) &= \sum_{k\text{ gerade}}a_k t^k + \sum_{k\text{ ungerade}} a_kt^k\\
			   &=\sum_{k=0}^\infty a_{2k} t^{2k}+\sum_{k=0}^\infty a_{2k+1}t^{2k+1}\\
			   &=a_0\sum_{k=0}^{\infty}\frac{(-1)^k}{(2k)!} \left( \frac{D}{m} \right)^k t^{2k} +a_1\sum_{k=0}^\infty \frac{(-1)^k}{(2k+1)!}\left( \frac{D}{m} \right)^{k}t^{2k+1}
	\end{align*}
\item Wir können die Gleichung umschreiben als
	\begin{align*}
		\varphi(t)&= a_0\sum_{k=0}^\infty \frac{(-1)^k}{(2k)!}\omega^{2k} t^{2k}+\frac{a_1}{\omega}\sum_{k=0}^\infty \frac{(-1)^k}{(2k+1)!}\omega^{2k+1}t^{2k+1}\\
			  &= a_0 \cos\omega t + \frac{a_1}{\omega}\sin \omega t.	
	\end{align*}
\item Wir verwenden wieder den Potenzreiheansatz:
	\[
	\varphi(t)=\sum_{n=0}^{\infty} a_n t^n
\]
und
\begin{align*}
	\varphi'(t)&=\sum_{n=0}^\infty n a_n t^{n-1}\\
		   &=\sum_{n=0}^\infty (n+1)a_{n+1}t^n
\end{align*}
Eingesetz ist
\begin{align*}
	\sum_{n=0}^\infty (n+1)a_{n+1}t^n + \sum_{n=0}^\infty a_n t_n&=t\\
	\sum_{n=0}^\infty \left[ (n+1)a_{n+1}+a_n \right]t^n = t
\end{align*}
Die Rekursionsgleichung ist also
\[
	(n+1)a_{n+1}+a_n = \begin{cases}
		1 & n = 1\\
		0 & \text{sonst}
	\end{cases}
.\]
Sei $a_0\in \R$ fest. Dann ist $a_1=-a_0$. Danach gilt
\[
2a_2+a_1 = 1
,\]
also
\[
a_2 = \frac{1-a_1}{2}=\frac{1+a_0}{2}
.\]
Umgekehrt ist $a_0 = 2a_2-1$. F\"{u}r die weitere Koeffizienten ist
\[
a_n = (-1)^n\frac{2}{n!}a_2,\qquad n \ge 2
.\] 
Insgesamt ist die Potenzreihe
\begin{align*}
	\varphi(t)&=(2a_2-1)+(1 - 2a_2)t + \sum_{k=2}^\infty (-1)^k \frac{2}{k!}a_2 t^k\\
		  &=(2a_2-1)+(1-2a_2)t+2a_2(-1+e^{-t}+t)\\
		  &=-1+t+2a_2 e^{-t}
\end{align*}
Nun setzen wir die Initialbedingungen ein:
\[
\varphi(0)=x_0=-1 + 2a_2
\]
mit L\"{o}sung
\[
2a_2 = x_0+1
.\] 
Damit ist die L\"{o}sung der DGL
\[
	\varphi(t) = -1 + t + (x_0+1)e^{-t}
.\qedhere\] 
	\end{parts}
\end{proof}
\begin{Problem}
	Sei $a\in \R,~a>0$. Zeigen Sie, dass das Anfangswertproblem
	\[
		\dot{x}=|x|^a,\qquad x(0)=0
	\]
	genau im Fall $a\ge 1$ eine eindeutige L\"{o}sung f\"{u}r $t\ge 0$ besitzt.
\end{Problem}
\begin{proof}
	Offensichtlich ist $x(t)=0$ eine L\"{o}sung f\"{u}r alle $a>0$. Jetzt betrachten wir Lipschitz-Stetigkeit: Da $|x|^a = x^a$ f\"{u}r alle $x\ge 0$, was stetig differenzierbar in $[0,\infty)$ f\"{u}r alle $a\ge 1$ ist, ist $|x|^a$ lokal lipschitz stetig und die L\"{o}sung ist somit eindeutig. 

	Jetzt betrachten wir den Fall $a<1$. Sei $t_0>0,~\varphi:[0,\infty)\to \R$ und
	\[
	\varphi(t) = \begin{cases}
		0 & t \le t_0,\\
		[(1-a)(t-t_0)]^{\frac{1}{1-a}} & \text{sonst.}
	\end{cases}
	.\] 
	Es ist klar, dass die DGL in sowohl $[0,t_0)$ als auch in $(t_0,\infty)$ erf\"{u}llt ist. Dann bleibt es nur die Ableitung in $t_0$ zu betrachten. Es gilt
	\begin{align*}
		\lim_{t \searrow t_0} \frac{\varphi(t)-\varphi(t_0)}{t-t_0}&=\lim_{t \searrow t_0} \frac{1}{t-t_0}(1-a)^{\frac{1}{1-a}}(t-t_0)^{\frac{1}{1-a}}\\
		&=\lim_{t \searrow t_0} (1-a)^{\frac{1}{1-a}}(t-t_0)^{\frac{a}{1-a}}\\	&= 0
	\end{align*}
	da
	\[
	\frac{a}{1-a}=\frac{1}{\frac{1}{a}-1}>0
	.\] 
	Es ist klar, dass $\lim_{t \nearrow t_0} \frac{\varphi(t)-\varphi(t_0)}{t-t_0}=0$. $\varphi(t)$ ist damit in $t_0$ differenzierbar mit Ableitung $0$. Die DGL ist also erf\"{u}llt. Da $\varphi(t)$ eine L\"{o}sung f\"{u}r alle $x_0>0$ ist, ist die L\"{o}sung nicht eindeutig.
\end{proof}
