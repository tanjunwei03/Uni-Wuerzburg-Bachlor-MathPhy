\begin{Problem}
	Gegeben sei das Anfangswertproblem
	\begin{equation}\label{eq:diffeqblatt6-1}
	\dot{x}=x(x-2)e^{\cos x},\qquad x(t_0)=x_0.
\end{equation}
	\begin{parts}
		\item Bestimmen Sie alle $x_0$, f\"{u}r die \eqref{eq:diffeqblatt6-1} konstante L\"{o}sungen hat.
		
		Nun sei die Anfangwertbedingung gegeben durch $x(0)=1$.
		\item Zeigen Sie, dass das Anfangswertproblem eine eindeutige L\"{o}sung $\varphi(t)$ besitzt.
		\item Zeigen Sie, dass die L\"{o}sung aus Teilaufgabe (b) beschr\"{a}nkt, streng monoton fallend ist und auf ganz $\R$ existiert.
		\item Zeigen Sie, dass die Grenzwerte $\lim_{t\to \pm \infty} \varphi(t)$ existieren und bestimmen Sie diese Grenzwerte.
	\end{parts}
\end{Problem}
\begin{proof}
	\begin{parts}
		\item Bei einer konstanten L\"{o}sung muss $\dot{x}=0$ f\"{u}r alle Zeitpunkte gelten, insbesondere in $t_0$. Da $\exp(x)\neq 0\forall x\in \R$ ist, muss $x(x-2)=0$, also $x_0=0$ oder $x_0=2$.
		\item Die Funktion $f(t, x)=x(x-2)e^{\cos x}$ ist stetig in $t$ (da sie nicht von $t$ abhängt) und lokal Lipschitz stetig in $x$, da sie nach $x$ stetig differenzierbar ist.
		\item Die L\"{o}sung darf nur zwischen $0$ und $2$ sein. 
		
		Angenommen es g\"{a}be $\tilde{t}$ mit $\varphi(t)=2$. Dann ist $\varphi$ eine L\"{o}sung des Anfangswertproblems \eqref{eq:diffeqblatt6-1}, $x(\tilde{t})=2$. Die konstante L\"{o}sung ist aber auch eine L\"{o}sung, was ein Widerspruch zur lokalen Eindeutigkeit ist. Analog ist $\varphi(t)\neq 0$. Aufgrund Stetigkeit muss dann $0<\varphi<2$.
		
		Die L\"{o}sung ist streng monoton fallend, wachsend oder konstant nach Präzensaufgabe 5.2. Da $\dot{x}(0)<0$, ist sie streng monoton fallend.
	\end{parts}
\end{proof}
\begin{Problem}
	Sei $A:\R\to \R^{2\times 2}$ gegeben durch
	\[A(t)=\begin{pmatrix}
		t & 0 \\
		1 & t
	\end{pmatrix}.\]
	Finden Sie die \"{U}bergangsmatrix und eine Fundamentalmatrix des Differentialgleichungssystems
	\[\dot{x}=Ax\]
\end{Problem}
\begin{proof}
	Sei $x=(x_1,x_2)^T$. Damit ist die DGL
	\begin{align*}
		\dot{x}_1 &= t x_1\\
		\dot{x}_2 &= x_1 + t x_2
	\end{align*}
	Die erste DGL kann man direkt l\"{o}sen mit TDV:
	\[x_1 = Ae^{\frac{t^2}{2}},~A\in \R\]
	Eingesetzt ist die zweite DGL eine DGL nur in $x_2$. Wir k\"{o}nnen die L\"{o}sung durch Variation der Konstanten bestimmen:
	\[x_2(t)=At e^{\frac{t^2}{2}}+B e^{\frac{t^2}{2}}.\]
	Zwei linear unabhängige Lösungen sind also
	\[x=\begin{pmatrix}
		e^{\frac{t^2}{2}} \\ t e^{\frac{t^2}{2}}
	\end{pmatrix},\qquad x=\begin{pmatrix}
	0 \\ e^{\frac{t^2}{2}}
\end{pmatrix}\]
und eine Fundamentalmatrix ist
\[\Phi(t) = \begin{pmatrix}
	e^{\frac{t^2}{2}} & 0\\ t e^{\frac{t^2}{2}} &  e^{\frac{t^2}{2}}
\end{pmatrix}.\]
Die Übergangsmatrix ist
\[\Psi(t,t_0) = \Phi(t)\Phi(t_0)^{-1}=e^{\frac 12 (t^2-t_0^2)}\begin{pmatrix}
	1 & 0 \\
	t-t_0 & 1
\end{pmatrix}.\qedhere\]
\end{proof}
\begin{Problem}
	Sei $I$ ein offenes Intervall und $A: I\to \R^{n\times n}$ stetig. Sei $\Phi: I\to \R^{n\times n}$ eine Fundamentalmatrix von 
	\begin{equation}\label{eq:diffeqblatt6-2}
		\dot{x}=A(t)x
	\end{equation}
	\begin{parts}
		\item Bestimmen Sie eine stetige Abbildung $B : I \to \R^{n\times n}$ , so dass $\Psi : I \to \R^{n\times n}$ mit $\Psi(t) := (\Phi^T)^{-1}(t)$ eine Fundamentalmatrix von $\dot{x} = B(t)x$ definiert.
			
			(Hinweis: Als Fundamentalmatrix muss $\Psi(t)$ die Differentialgleichung $\dot{\Psi}(t) = B(t)\Psi(t)$ erfüllen.)
			\item Sei $C \in \R^{n\times n}$ mit $\det(C) \neq 0$ und sei $\Psi : I \to \R^{n\times n}$ definiert durch $\Psi(t) := C\Phi(t)$. Zeigen Sie, dass $\Psi(t)$ genau dann eine Fundamentalmatrix von \eqref{eq:diffeqblatt6-2} ist, wenn
			\[CA(t)=A(t)C\]
			f\"{u}r alle $t\in I$ gilt.
	\end{parts}
\end{Problem}
\begin{proof}
	\begin{parts}
		\item Die Fundamentalmatrizen $\Phi$ und $\Psi$ erf\"{u}llen die folgenden Differentialgleichungen:
		\begin{align*}
			\dot{\Phi}(t)&=A(t)\Phi(t)\\
			\dot{\Psi}(t)&=B(t)\Psi(t)
		\end{align*}
	Wir wissen, dass $\Psi(t)= (\Phi^T)^{-1}(t)$ ist. Dafür müssen wir zunächst ein Lemma beweisen:
	\begin{Lemma}
		Sei $A(t)$ eine von Zeit abhängige Matrix. Es gilt
		\[\dv{A^{-1}}{t}=-A^{-1}\dv{A}{t}A\]
	\end{Lemma}
\begin{proof}
	Es gilt $AA^{-1}=0$ und damit
	\begin{align*}
		0&=\dv{A}{t}A^{-1} + A\dv{A^{-1}}{t}\\
		\dv{A^{-1}}{t}&=-A^{-1}\dv{A}{t}A^{-1}.\qedhere
	\end{align*}
\end{proof}
also
\begin{align*}
\dot{\Psi}&= -(\Phi^T)^{-1}{\dv{\Phi^T}{t}}(\Phi^T)^{-1}=B(t)(\Phi^T)^{-1}\\
-(\Phi^T)^{-1}\dv{\Phi^T}{t}&= B(t)\\
B(t)&= -(\Phi^T)^{-1}\left(A(t) \Phi(t)\right)^T\\
&= -(\Phi^T)^{-1} \Phi^T A(t)^T\\
&= A(t)^T
\end{align*}
\item $\Psi$ ist eine Fundamentalmatrix, wenn $\dot{\Psi}(t)=A(t)\Psi(t)$, oder
\begin{align*}
	C\dot{\Phi}&= AC\Phi\\
	CA\Phi&=AC\Phi\\
	0&= (CA-AC)\Phi
\end{align*}
da $\Phi$ invertierbar ist, ist dies genau dann erfüllt, wenn $CA-AC=0$ f\"{u}r alle $t\in I$.\qedhere
	\end{parts}
\end{proof}
\begin{Problem}
	Beurteilen Sie, ob die folgenden 4 Behauptungen wahr oder falsch sind.
	
	Sie müssen bei dieser Aufgabe keine Begründungen angeben.
	
	Für jede richtig beantwortete Frage gibt es einen Punkt.
	
	Für jede falsch beantwortete Frage wird ein Punkt abgezogen.
	
	Für jede nicht beantwortete Frage gibt es keine Punkte.
	
	Die gesamte Aufgabe wird mit mindestens 0 Punkten bewertet (sie können also nicht z. B. -1 Punkte bekommen). Insgesamt können bis zu 4 Punkte erreicht werden.
	
		\begin{tabularx}{\textwidth}{X|p{2cm}|p{2cm}|}
	& Wahr & Falsch \\\hline
	Sei $f : \R \times \R^n\to  \R^n$ lipschitz-stetig und $\dot{x} = f (t, x)$ eine Differentialgleichung.
	Weiter sei $\varphi_\text{max}$ eine maximale Lösung mit maximalem Existenzintervall	$I_\text{max} = (-\infty, t^+ ), t^+ < \infty$. Dann ist $\varphi_\text{max}$ beschränkt. & & \\\hline
	Sei $\ddot{x} = \frac{\dot{x}-1}{x}$ eine skalare Differentialgleichung, $x \in \R \setminus \{0\}$. Dann ist $\varphi(t) = t \ln(t), t > 1$, eine Lösung dieser Differentialgleichung. & & \\\hline
	Sei $D \subseteq \R \times \R^n , f : D \to \R^n$ stetig und $\varphi : I \to \R^n$ eine Lösung von $\dot{x} = f (t, x)$. Ist $I = (t_0, t_1)$ und hat $\varphi$ in $t_1$ einen linksseitigen Grenzwert $x_1 = \lim_{t\to t_1} \varphi(t)$ mit $(t_1 , x_1 ) \in D$, so ist $\varphi$ nach rechts fortsetzbar. & & \\\hline
	Jede Fundamentalmatrix $\Phi(t)$ ist auch eine Wronski-Matrix $W(t)$. & & \\\hline
	\end{tabularx}
\end{Problem}
\begin{proof}\noindent
	
		\begin{tabularx}{\textwidth}{X|p{2cm}|p{2cm}|}
	& Wahr & Falsch \\\hline
	Sei $f : \R \times \R^n\to  \R^n$ lipschitz-stetig und $\dot{x} = f (t, x)$ eine Differentialgleichung.
	Weiter sei $\varphi_\text{max}$ eine maximale Lösung mit maximalem Existenzintervall	$I_\text{max} = (-\infty, t^+ ), t^+ < \infty$. Dann ist $\varphi_\text{max}$ beschränkt. & & $\times$ \\\hline
	Sei $\ddot{x} = \frac{\dot{x}-1}{x}$ eine skalare Differentialgleichung, $x \in \R \setminus \{0\}$. Dann ist $\varphi(t) = t \ln(t), t > 1$, eine Lösung dieser Differentialgleichung. & $\times$ & \\\hline
	Sei $D \subseteq \R \times \R^n , f : D \to \R^n$ stetig und $\varphi : I \to \R^n$ eine Lösung von $\dot{x} = f (t, x)$. Ist $I = (t_0, t_1)$ und hat $\varphi$ in $t_1$ einen linksseitigen Grenzwert $x_1 = \lim_{t\to t_1} \varphi(t)$ mit $(t_1 , x_1 ) \in D$, so ist $\varphi$ nach rechts fortsetzbar. & $\times$ & \\\hline
	Jede Fundamentalmatrix $\Phi(t)$ ist auch eine Wronski-Matrix $W(t)$. & $\times$ & \\\hline
\end{tabularx}
\end{proof}