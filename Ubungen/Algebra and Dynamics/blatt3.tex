\begin{Problem}[Norm]
	Which of the maps $\mathcal{M}_2 \to [0,\infty)$
	\begin{align}
		M &\mapsto |\det M| \tag{1a}\\
		M &\mapsto |\operatorname{tr} M| \tag{1b}\\
		M &\mapsto \sup_{ij} |M_{ij}| \tag{1c}\\
		M &\mapsto \sup_{\substack{v\in\mathbb{C}^2, \\ \|v\|=1}} \|Mv\| \quad \text{(with } \|v\| = \sqrt{|v_1|^2 + |v_2|^2}) \tag{1d}
	\end{align}
	define a norm on the algebra $\mathcal{M}_2$ of complex $2\times 2$ matrices? Which turn $\mathcal{M}_2$ into a $C^*$-algebra?
\end{Problem}
\begin{proof}
	The first is not homogeneous, because multiplying $M$ by a constant multiplies $\det M$ by the squared constant.
	
	The second is not positive definite: $\Tr(\text{diag}(1,-1))=0$, but $\text{diag}(1,-1)\neq 0$.
	
	The third is not a norm:
	\begin{enumerate}
		\item It is homogeneous.
		\item It is positive definite
		\item It satisfies the triangle inequality.
		\item It is not submultiplicative: Consider
			\[
				A = \begin{pmatrix} 1 & 1 \\ 1 & 1 \end{pmatrix} 
			.\] 
			Then
			\[
				A^2 = \begin{pmatrix} 2 & 2 \\ 2 & 2 \end{pmatrix} 
			,\]
			thus
			\[
			2=\|A^2\|\not\leq \|A\|\|A\|=1
			.\] 
	\end{enumerate}
\item This is the functional norm.\qedhere
\end{proof}
\begin{Problem}[Spectrum and Resolvent]
	Use the Pauli matrices
	\[
	\sigma_1 = \begin{pmatrix} 0 & 1 \\ 1 & 0 \end{pmatrix}, \quad
	\sigma_2 = \begin{pmatrix} 0 & -i \\ i & 0 \end{pmatrix}, \quad
	\sigma_3 = \begin{pmatrix} 1 & 0 \\ 0 & -1 \end{pmatrix}
	\tag{2}
	\]
	to parametrize a general complex $2\times 2$ matrix $M \in \mathcal{M}_2$ by four complex numbers $(a_0, \vec{a})$:
	\[
	M(a_0, \vec{a}) = a_0 1 + \vec{a}\cdot \vec{\sigma} 
	= a_0 1 + a_1\sigma_1 + a_2\sigma_2 + a_3\sigma_3
	= \begin{pmatrix}
		a_0 + a_3 & a_1 - i a_2 \\
		a_1 + i a_2 & a_0 - a_3
	\end{pmatrix}.
	\tag{3}
	\]
	
	\begin{enumerate}
		\item For which $(a_0, \vec{a}) \in \mathbb{C}^4$ is $M(a_0, \vec{a})$
		\begin{enumerate}[label=(\alph*)]
			\item normal,
			\item isometric,
			\item unitary,
			\item self-adjoint,
			\item positive?
		\end{enumerate}
		
		\item Determine the resolvent set $r_{\mathcal{M}_2}(M(a_0, \vec{a}))$ and spectrum $\sigma_{\mathcal{M}_2}(M(a_0, \vec{a}))$ for all $(a_0, \vec{a})$. Handle exceptional cases.
		
		\item Test the general results for the spectrum of normal, isometric, unitary, self-adjoint, and positive matrices.
		
		\item Compute the resolvent
		\[
		R^{a_0,\vec{a}} : r_{\mathcal{M}_2}(M(a_0, \vec{a})) \to \mathcal{M}_2, \quad 
		z \mapsto (z1 - M(a_0, \vec{a}))^{-1}
		\tag{4}
		\]
		as a $2\times 2$ matrix (i.e., perform the matrix inversion explicitly!).
		
		\item \textbf{NB:} In the lecture on \textit{Tuesday, November 4, 2025}, it will be shown that $P^{M(a_0,\vec{a})}_\mathcal{C}$ is indeed a projection, if $\mathcal{C}$ encircles a part of the spectrum $\sigma(M(a_0,\vec{a}))$. You can wait until then to complete the exercise, take a peek at the script, or just do the integral choosing typical examples for $\mathcal{C}$ based on your earlier results for $\sigma(M(a_0,\vec{a}))$.
		
		Compute the projections
		\[
		P^{M(a_0,\vec{a})}_\mathcal{C} = \int_{\mathcal{C}} \frac{dz}{2\pi i} \, R^{a_0,\vec{a}}(z)
		\tag{5}
		\]
		for ``interesting'' $\mathcal{C}$ explicitly. Are there qualitatively different cases to consider?
	\end{enumerate}
\end{Problem}
\begin{proof}
	\begin{enumerate}
		\item 
			\begin{parts}
			\item We compute $A A^\dagger - A^\dagger A$ explicitly: We have
				\begin{align*}
					A A^\dagger &= \left( \sum_{i=0}^3 a_i \sigma_i \right)\left( \sum_{j=0}^3 a_j^* \sigma_j \right)   \\
						    &= \sum_{i,j=0}^3 a_i a_j^* \sigma_i \sigma_j \\
						    &= |a_0|^2 + \sum_{i=1}^{3} a_i a_0^* \sigma_i + \sum_{i=1}^{3} a_0 a_i^* \sigma_i + \sum_{i,j=1}^{3} a_i a_j^* \sigma_i \sigma_j \\
						    &= |a_0|^2 + \sum_{i=1}^3 (a_i a_0^* + a_i^* a_0)\sigma_i + \sum_{i,j=1}^{3}a_i a_j^* \left(\delta_{ij}I + i\sum_k\epsilon_{ijk}\sigma_k\right) \\
						    &= \sum_i |a_i|^2 + \sum_{i=1}^{3} (a_i a_0^* + a_i^* a_0)\sigma_i + i \sum_{i,j,k=1}^{3} \epsilon_{ijk}a_i a_j^* \sigma_k 
				\end{align*}
				By the symmetry, we get that
				\[
					A^\dagger A = \sum_i |a_i|^2 + \sum_{i=1}^{3} (a_i a_0^* + a_i^* a_0)\sigma_i + i\sum_{i,j,k=1}^{3} \epsilon_{ijk}a_i^* a_j\sigma_k
				.\] 
				Thus, we subtract them and demand that their difference vanishes, which leads to
				\[
					0 = \sum_{i,j,k=1}^3 [\epsilon_{ijk}a_i a_j^*-\epsilon_{ijk}a_i^* a_j]\sigma_k \\
				.\]
				Using the orthogonality of the $\sigma_k$ s, this is equivalent to the expression that
				\[
					\vec{a}\times \vec{a}^* = 0
				.\] 
			\item As in the last part, we have
				\[
					A^\dagger A = \sum_i |a_i|^2 + \sum_{k=1}^3 \left[ (a_ka_0^* + a_k^* a_0)+i\sum_{i,j=1}^3 \epsilon_{ijk}a_i^* a_j \right]\sigma_k  
				.\]
				Since this must be the identity, we must have $\sum_{i=0}^3 |a_i|^2$ and
				\[
					a_k a_0^* + a_k^* a_0 + i(\vec{a}^* \times \vec{a})_k=0
				.\] 
			\item It is unitary if it is isometric and normal. By substituting, we get
				\begin{align*}
					\vec{a}^* \times \vec{a} &= 0 \\
					\Re(a_k a_0^*) &= 0
				\end{align*}
			\item It is self adjoint if all $a$s are real.
			\item It is positive if
				\[
					a_0 \ge \sqrt{a_1^2 + a_2^2 + a_3^2} 
				.\] 
			\end{parts}
		\item The eigenvalues are
			\[
			a_0\pm \sqrt{a_1^2 + a_2^2 + a_3^2} 
			.\] 
		\item As example unitary matrix, we choose
			\[
			M = `\frac{1}{9\sqrt{2} } \left(
\begin{array}{cc}
 2+3 i & 10+7 i \\
 -10+7 i & 2-3 i \\
\end{array}
\right)
			\] 
			which corresponds to vector
			\[
				\begin{pmatrix} a_0 \\ a_1 \\ a_2 \\ a_3 \end{pmatrix} = \begin{pmatrix}  \frac{\sqrt{2}}{9} \\\frac{7 i}{9 \sqrt{2}} \\\frac{5 i \sqrt{2}}{9} \\\frac{i}{3 \sqrt{2}} \end{pmatrix}
			.\] 
			We can see clearly that it satisfies the equations for a unitary matrix. 
		\item 
		\[
			\frac{1}{z - M(a_0, \vec{a})} =\left(
\begin{array}{cc}
 -\frac{-a_0+a_3+z}{2 a_0 z-a_0^2+a_1^2+a_2^2+a_3^2-z^2} & -\frac{a_1-i a_2}{2 a_0 z-a_0^2+a_1^2+a_2^2+a_3^2-z^2} \\
 -\frac{a_1+i a_2}{2 a_0 z-a_0^2+a_1^2+a_2^2+a_3^2-z^2} & -\frac{-a_0-a_3+z}{2 a_0 z-a_0^2+a_1^2+a_2^2+a_3^2-z^2} \\
\end{array}
\right) 
		.\] 
	\item We expect qualitatively that there should be 4 classes of contours $\mathcal{C}$: Two, that enclose either point of the spectrum, one that encloses both and one that encloses neither. 

		We write the integral explicitly:
		\[
			\mathcal{P}_\mathcal{C}^{M(a_0, \vec{a})} = -\int_\mathcal{C}\frac{\dd{z}}{2\pi i} \frac{1}{2a_0z - a_0^2 + |\vec{a}|^2 - z^2} \begin{pmatrix} z - a_0 + a_3 & a_1 - i a_2 \\ a_1 + i a_2 & z - a_0 - a_3 \end{pmatrix} 
		.\] 
		Factorising the denominator yields
		\[
			z^2 - 2a_0z + a_0^2 - |\vec{a}|^2 = 
		.\] 
	\end{enumerate}
\end{proof}
