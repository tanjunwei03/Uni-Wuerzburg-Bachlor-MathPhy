\begin{Problem}[CCRs vs. Boundedness]
Consider two bounded operators $A$ and $B$ on a Hilbert space $\mathcal{H}$, i.e.
\begin{subequations}
	\begin{align}
		\exists a \in \mathbb{R} : \forall \psi \in \mathcal{H} : \|A \psi\| \leq a \|\psi\|, \label{1a} \\
		\exists b \in \mathbb{R} : \forall \psi \in \mathcal{H} : \|B \psi\| \leq b \|\psi\|. \label{1b}
	\end{align}
\end{subequations}

Show that the canonical commutation relations
\begin{equation}
	[A,B] = AB - BA = i
\end{equation}
are inconsistent with the assumption of boundedness for the operators $A$ and $B$.

\noindent \textbf{NB:} It is not necessary to find an original proof. It suffices to find, understand and present a proof from the literature.
\end{Problem}

\begin{proof}
	The proof comes from rudin: Let $A$ be a normed algebra, and $x,y\in A$. We assume that
	\[xy - yx = 1\]
	The first step is to prove that $xy^n - y^n x = ny^{n-1}$ for all $n\in \N$. This is true for $n=1$. Then, by induction, if
	\[
	xy^n - y^n x = ny^{n-1}
	,\] 
	it follows that
	\begin{align*}
		xy^{n+1} - y^{n+1}x &= (xy^n - y^n x)y + y^n (xy - yx)\\
		&= ny^n + y^n\\
		&= (n+1)y^n
	\end{align*}
	Additionally, we note that $y^n\neq 0$ for all $n$. Otherwise, we would choose the minimum $n$, and get
	\[
	0 = xy^n - y^n x = n y^{n-1}
	,\]
	a contradiction to the minimality of $n$. Now, we have
	\[
	n \|y^{n-1}\|= \|xy^n - y^n x\|\le 2 \|x\|\|y\|\|y\|^{n-1}
	\] 
	and
	\[
	2\|x\|\|y\|\ge n
	,\]
	a contradiction.
\end{proof}
\begin{Problem}[Classical Dynamics on the 2-Torus]

Consider a classical dynamical system with the 2-Torus $T^2 = S^1 \times S^1$ as phase space $\Gamma$ (this is not a cotangent bundle, but it has the technical advantage of being compact).

Using standard coordinates $(\theta_1, \theta_2) \in [0, 2\pi)^2$, a consistent Poisson bracket is given by
\begin{equation}
	\{f,g\} = \frac{\partial f}{\partial \theta_1} \frac{\partial g}{\partial \theta_2} - \frac{\partial f}{\partial \theta_2} \frac{\partial g}{\partial \theta_1}.
\end{equation}

Assume that the Hamiltonian is
\begin{align}
	H : \Gamma &\to \mathbb{R} \nonumber \\
	(\theta_1, \theta_2) &\mapsto H(\theta_1, \theta_2) = c\cos \theta_1.
\end{align}

In order to be well defined globally, the Hamiltonian must be periodic in $\theta_1$ and $\theta_2$. This is the simplest choice.

\begin{enumerate}
	\item Derive the equations of motion.
	\item Determine the flow $\Phi$ of a phase space point $(\theta_1, \theta_2) \in \Gamma$.
	\item Determine the time evolution of the state $\omega$, where
	\begin{equation}
		\omega(f) = \int_{\Gamma} \dd[2]{\theta}f(\theta) \, \omega(\theta)
	\end{equation}
	with
	\begin{align}
		\omega : \Gamma &\to \mathbb{R}\nonumber \\
		(\theta_1, \theta_2) &\mapsto \omega(\theta_1, \theta_2) = \frac{1}{\pi^2} \sin^2 \theta_1 \sin^2 \theta_2.
	\end{align}
\end{enumerate}
\end{Problem}
\begin{revision}
	\section*{Hamiltonian Dynamics}
The Hamiltonian is a function on the phase space
\[H = H(q_1, \dots, q_n, p_1, \dots, p_n).\]
The flow solves the canonical equations of motion
\begin{align*}
\dv{q_i}{t} &= \pdv{H}{p_i}\\
\dv{p_i}{t} &= -\pdv{H}{q_i}
\end{align*}
We can also define the Poisson bracket
\[\{f,g\} = \sum_i \left(\pdv{f}{q_i}\pdv{g}{p_i} - \pdv{f}{p_i}\pdv{g}{q_i}\right)\]
It is immediately clear that the Poisson bracket is antisymmetric; additionally, we have the canonical commutation relations
\[\{q,p\}= 1, \{q,q\}=\{p,p\}=0\]
Clearly, because $\pdv{q}{p} = \pdv{p}{q} = 0$, we can rewrite the canonical commutation relations as
\begin{align*}
	\dv{q_i}{t} &= \{q_i, H\}\\
	\dv{p_i}{t} &= \{p_i, H\}
\end{align*}
\section*{States as Functionals}
\end{revision}
\begin{proof}
	\begin{enumerate}
		\item The equations of motion are
		\[\dv{\theta_1}{t} = \pdv{H}{\theta_2}=0\]
		and 
		\[\dv{\theta_2}{t} = -\pdv{H}{\theta_1}=c\sin\theta_1\]
		\item Clearly, the first equation tells us that $\theta_1$ is a constant; since the right hand side of equation 2 is now a constant, $\theta_2$ varies linearly with time.
		\[\Phi^t \begin{pmatrix}
			\theta_1 \\ \theta_2
		\end{pmatrix} = \begin{pmatrix}
		\theta_1 \\ \theta_2 + ct\cos\theta_1
	\end{pmatrix}\]
\item The flow is defined by
\[\omega_t(f) = \omega(\Phi_t(f))=\omega(f\circ \Phi_t).\]
On the right hand side, we desire
\begin{align*}
\omega_t(f) &= \frac{1}{\pi^2}\int_\Gamma \dd[2]{\theta}f(\theta_1, \theta_2 + ct\cos\theta_1)\omega(\theta_1,\theta_2) \\
&= \frac{1}{\pi^2}\int_\Gamma \dd[2]{\theta}f(\theta_1, \theta_2)\omega(\theta_1,\theta_2-ct\cos\theta_1).\qedhere
\end{align*}
	\end{enumerate}
\end{proof}
\begin{Problem}[Classical Dynamics on the 2-Sphere]

Consider a classical dynamical system with the 2-Sphere $S^2$ as phase space $\Gamma$ (this is again not a cotangent bundle, but the technical advantage of being compact and is highly symmetric).

Using standard spherical coordinates $(\theta, \phi) \in [0, \pi] \times [0, 2\pi)$, a consistent Poisson bracket is given by
\begin{equation}
	\{f,g\} = \frac{1}{\sin \theta} \left(\frac{\partial f}{\partial \theta} \frac{\partial g}{\partial \phi}- \frac{\partial f}{\partial \phi} \frac{\partial g}{\partial \theta} \right).
\end{equation}

Assume that the Hamiltonian is
\begin{align}
	H : \Gamma &\to \mathbb{R} \nonumber \\
	(\phi, \theta) &\mapsto H(\phi, \theta) = c\cos \theta.
\end{align}

In order to be well defined globally, the Hamiltonian must be periodic in $\theta$ and $\phi$. This is one of the simplest choices.

\begin{enumerate}
	\item Show that the Poisson bracket satisfies all requirements.
	\item Determine the flow $\Phi$ of a phase space point $(\theta, \phi) \in \Gamma$.
	\item Determine the time evolution of the state $\omega$, where
	\begin{equation}
		\omega(f) = \int_{\Gamma}\sin\theta \dd{\theta}\dd{\phi} f(\theta, \phi) \, \omega(\theta, \phi)
	\end{equation}
	with
	\begin{align}
		\omega : \Gamma &\to \mathbb{R}\nonumber\\
		(\theta, \phi) &\mapsto \omega(\theta, \phi) = \frac{2}{\pi^2} \sin \theta \cos^2 \phi.
	\end{align}
\end{enumerate}
\end{Problem}
\begin{proof}
	\begin{enumerate}
		\item It is clearly antisymmetric
		\item We have the equations of motion
		\begin{align*}
			\dv{\theta}{t} &= \{\theta, H\} = 0\\
			\dv{\phi}{t} &= \{\phi, H\} = -\frac{1}{\sin\theta} (-c\sin\theta) = c
		\end{align*}
	and solutions
	\begin{align*}
		\theta(t) &= \theta(0)\\
		\phi(t) &= \phi(0) + ct
	\end{align*}
\item Again
\begin{align*}
	\omega_t(f) &= \omega(f\circ \Phi_t)\\
	&= \frac2{\pi^2} \int_\Gamma \sin\theta \dd{\theta}\dd{\phi} f(\theta, \phi+ct) \, \omega(\theta, \phi)\\
	&=\frac2{\pi^2} \int_\Gamma \sin\theta \dd{\theta}\dd{\phi} f(\theta, \phi) \, \omega(\theta, \phi-ct)
\end{align*}
	\end{enumerate}
\end{proof}
