\documentclass[prb,12pt]{revtex4-2}

\usepackage{amsmath, amssymb,physics,amsfonts,amsthm}
\usepackage[most]{tcolorbox}
\usepackage{enumitem}
\usepackage{cancel}
\usepackage{booktabs}
\usepackage{polynom}
\usepackage{tabularx}
\usepackage{tikz}
\usepackage{hyperref}
\usepackage{enumitem}
\usepackage{ulem}
\usepackage{transparent}
\usepackage{caption}
\usepackage{float}
\usepackage{multirow}
\newtheorem{Theorem}{Theorem}
\newtheorem{Proposition}{Theorem}
\newtheorem{Lemma}[Theorem]{Lemma}
\newtheorem{Corollary}[Theorem]{Corollary}
\newtheorem{Example}[Theorem]{Example}
\newtheorem{Remark}[Theorem]{Remark}
\theoremstyle{definition}
\newtheorem{Problem}{Problem}
\theoremstyle{definition}
\newtheorem{Definition}[Theorem]{Definition}
\newenvironment{parts}{\begin{enumerate}[label=(\alph*)]}{\end{enumerate}}
%tikz
\usetikzlibrary{patterns}
\usetikzlibrary{matrix}
%tcolorbox
\tcbset{breakable=true,toprule at break = 0mm,bottomrule at break = 0mm}
% definitions of number sets
\newcommand{\N}{\mathbb{N}}
\newcommand{\R}{\mathbb{R}}
\newcommand{\Z}{\mathbb{Z}}
\newcommand{\Q}{\mathbb{Q}}
\newcommand{\C}{\mathbb{C}}
\allowdisplaybreaks
\setlength{\parindent}{0cm}
\captionsetup[table]{name=Tabelle}
\setlist[enumerate]{parsep=0pt}

\begin{document}
\title{Geometric Analysis Exam Presentation Outline}
	\author{Jun Wei Tan}
	\email{jun-wei.tan@stud-mail.uni-wuerzburg.de}
	\affiliation{Julius-Maximilians-Universit\"{a}t W\"{u}rzburg}
	\date{\today}
	\maketitle
\section{Introduction}
\begin{enumerate}
	\item \textbf{Define} Lie groups
	\item \textbf{State} example of $\text{GL}(n,\R)$ and \textbf{prove} that it is a Lie group.
	\item \textbf{State} that Lie groups provide a way to move between elements (group multiplication)
	\item \textbf{Prove} left translation is diffeo
	\item \textbf{Prove} Lie group homos have constant rank
	\item \textbf{Prove} Identity component is only connected open subgroup, all connected components are diffeo to identity component
	\item \textbf{Draw} picture corresponding to previous proof
	\item \textbf{State} that embedded subgroups are open and closed. 
\end{enumerate}
\section{Group actions}
\begin{enumerate}[resume]
	\item \textbf{Define} what it means for a smooth function to intertwine actions
	\item \textbf{Prove:} If group action on $M$, $N$ is transitive and $F$ intertwines actions, $F$ has constant rank.
	\item \textbf{Prove} orbit map $G\to M$ (fixed $p$) is constant rank.
\end{enumerate}
\section{Lie Algebras}
\begin{enumerate}[resume]
	\item State commutator properties
		\item \textbf{Define} a lie algebra
	\item \textbf{Define} left invariant vector fields
	\item \textbf{Prove} that left invariant vector fields are closed under the commutator
	\item \textbf{Prove} that dim(Lie($G$)) = dim($G$) by showing that the evaluation map is an isomorphism.
	\item \textbf{Deduce} as a corollary that all left invariant vector fields on a lie group are smooth.
\end{enumerate}
\subsection{Matrix Lie Group \& Algebra}
\begin{enumerate}[resume]
	\item \textbf{State} that GL($n, \R$) is an open subset of $\mathfrak{gl}(n, \R)$.
	\item \textbf{Prove} that $\text{GL}(n, \R)\cong T_{I_n}\text{GL}(n, \R)\cong \mathfrak{gl}(n, \R)$
\end{enumerate}
\subsection{Lie Algebra Homomorphisms}

\end{document}
