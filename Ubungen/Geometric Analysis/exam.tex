\documentclass[prb,12pt]{revtex4-2}

\usepackage{amsmath, amssymb,physics,amsfonts,amsthm}
\usepackage[most]{tcolorbox}
\usepackage{enumitem}
\usepackage{cancel}
\usepackage{booktabs}
\usepackage{tikz-cd}
\usepackage{polynom}
\usepackage{tabularx}
\usepackage{tikz}
\usepackage{hyperref}
\usepackage{enumitem}
\usepackage{ulem}
\usepackage{transparent}
\usepackage{caption}
\usepackage{float}
\usepackage{multirow}
\newtheorem{Theorem}{Theorem}
\newtheorem{Proposition}{Theorem}
\newtheorem{Lemma}[Theorem]{Lemma}
\newtheorem{Corollary}[Theorem]{Corollary}
\newtheorem{Example}[Theorem]{Example}
\newtheorem{Remark}[Theorem]{Remark}
\theoremstyle{definition}
\newtheorem{Problem}{Problem}
\theoremstyle{definition}
\newtheorem{Definition}[Theorem]{Definition}
\newenvironment{parts}{\begin{enumerate}[label=(\alph*)]}{\end{enumerate}}
%tikz
\usetikzlibrary{patterns}
\usetikzlibrary{matrix}
%tcolorbox
\tcbset{breakable=true,toprule at break = 0mm,bottomrule at break = 0mm}
% definitions of number sets
\newcommand{\N}{\mathbb{N}}
\newcommand{\R}{\mathbb{R}}
\newcommand{\Z}{\mathbb{Z}}
\newcommand{\Q}{\mathbb{Q}}
\newcommand{\C}{\mathbb{C}}
\allowdisplaybreaks
\setlength{\parindent}{0cm}
\captionsetup[table]{name=Tabelle}
\setlist[enumerate]{parsep=0pt}

\begin{document}
\title{Geometric Analysis Exam Presentation Outline}
	\author{Jun Wei Tan}
	\email{jun-wei.tan@stud-mail.uni-wuerzburg.de}
	\affiliation{Julius-Maximilians-Universit\"{a}t W\"{u}rzburg}
	\date{\today}
	\maketitle
\section{Introduction}
\begin{enumerate}
	\item \textbf{Define} Lie groups
	\item \textbf{State} example of $\text{GL}(n,\R)$ and \textbf{prove} that it is a Lie group.
	\item \textbf{State} that Lie groups provide a way to move between elements (group multiplication)
	\item \textbf{Prove} left translation is diffeo
	\begin{enumerate}
		\item Invertible ($L_{g^{-1}}$)
		\item Smooth by definition
	\end{enumerate}
	\item \textbf{Prove} Lie group homos have constant rank
		\begin{enumerate}
		\item Compare to rank at $e$
		\item Consider $F(L_{g_0}(g))=L_{F(g_0)}(F(g))$
		\item Take differential at $g=e$
	\end{enumerate}
	\item \textbf{Prove} that open subgroups are closed.
		\begin{enumerate}
		\item Consider cosets
	\end{enumerate}
	\item \textbf{Prove} identity component is only connected open subgroup, all connected components are diffeo to identity component
	\begin{enumerate}
		\item Connected subsets generate connected subgroups
		\item Consider elements that can be expressed as a product of $k$ elements of the set.
		\item Because they share 1 element, the union is connected.
		\item Consider subgroup generated by identity component.
		\item Use previous result (open subgroups are closed) to prove uniqueness
	\end{enumerate}
	\item \textbf{Draw} picture corresponding to previous proof
\end{enumerate}
\section{Group actions}
\begin{enumerate}[resume]
	\item \textbf{Define} a smooth group action of $G$ on $M$ as assigning a smooth map $G\times M \to M$.
	\item \textbf{Prove} that smooth actions are diffeos (ext. smooth inv) 
	\item \textbf{Define} what it means for a smooth function to intertwine actions. If $G$ is a lie group acting on manifolds $M$ and $N$ with actions $\theta$ and $\varphi$ respectively, then $F:M\to N$ intertwines actions if the following diagram commutes for all $g$:
	\begin{center}
		\begin{tikzcd}
			M\arrow[r, "F"]\arrow[d, "\theta_g"] & N\arrow[d,"\varphi_g"] \\ M\arrow[r, "F"] & N
			\end{tikzcd}
	\end{center}
	\item \textbf{Prove:} If group action on $M$, $N$ is transitive on $M$ and $F$ intertwines actions, $F$ has constant rank.
		\begin{center}
		\begin{tikzcd}
			T_pM\arrow[r, "\dd{F}_p"]\arrow[d, "\dd{(\theta_g)_p}"] & T_{F(p)}N\arrow[d,"\dd{(\varphi_g)}_{F(p)}"] \\ T_qM\arrow[r, "\dd{F}_q"] & T_{F(q)}N
		\end{tikzcd}
	\end{center}
	\item \textbf{Prove} orbit map $G\to M$ (fixed $p$) is constant rank.
		\begin{enumerate}
		\item Orbit map is equivariant wrt the action
	\end{enumerate}
\end{enumerate}
\section{Lie Algebras}
\begin{enumerate}[resume]
	\item \textbf{State} commutator properties
	\begin{enumerate}[label=(\alph*)]
		\item Bilinearity
		\item Anticommutativity
		\item Jacobi Identity
		\[[X,[Y,Z]]+[Y,[Z,X]]+[Z,[X,Y]]=0\]
	\end{enumerate}
		\item \textbf{Define} a lie algebra
	\item \textbf{Define} left invariant vector fields
	
	Invariant under all left translations, so
	\[\dd{(L_g)_{g'}}(X_{g'})=X_{gg'}\]
	forall $g,g'\in G$
	\item \textbf{Prove} that left invariant vector fields are closed under the commutator
	\begin{enumerate}[label=(\alph*)]
	\item $Y$ is $F$-related to $X$ iff for every $f:N\to \R$
	\[X(f\circ F)=(Yf)\circ F\]
	Proof:
	\begin{gather*}
		X(f\circ F)(p)=X_p(f\circ F)=\dd{F}_p(X_p)(f)\\
		(Yf)\circ F(p)=(Yf)(F(p))=Y_{F(p)}f
	\end{gather*}
	\item Show that $\dd{F}[X,Y] = [\dd{F}X,\dd{F}Y]$.
	
	Do this by showing
	\[XY(f\circ F) = (\dd{F}(X)\dd{F}(Y)f)\circ F\]
	and then showing that
	\item Apply this result to a left translation and use the invariance under left translator.
\end{enumerate}
\item \textbf{Define} Lie algebra as the vector space of left invariant vector fields under the commutator
	\item \textbf{Prove} that dim(Lie($G$)) = dim($G$) by showing that the evaluation map at the identity is an isomorphism.
	\begin{enumerate}[label=(\alph*)]
		\item Injectivity: Assume that the vector field at the origin is 0. Translate it to show that the entire vector field is 0.
		\item Surjectivity: We let $v\in T_eG$ be arbitrary, and define
		\[X_g = \dd{(L_g)_e}(v)\]
		Then we show that this is smooth by considering a smooth curve $\gamma$ through the origin defining the vector $v$.
		\[\left.\dv{t}\right|_{t=0}(f\circ L_g\circ \gamma)(t)\]
	\end{enumerate}
	\item \textbf{Deduce} as a corollary that all left invariant vector fields on a lie group are smooth.
\end{enumerate}
\subsection{Matrix Lie Group \& Algebra}
\begin{enumerate}[resume]
	\item \textbf{State} that GL($n, \R$) is an open subset of $\mathfrak{gl}(n, \R)$.
	\item \textbf{Prove} that $\text{GL}(n, \R)\cong T_{I_n}\text{GL}(n, \R)\cong \mathfrak{gl}(n, \R)$
\end{enumerate}

\end{document}
