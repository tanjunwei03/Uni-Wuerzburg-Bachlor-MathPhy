\begin{Problem}
	Es sei $n\in \N_0$. Bestimmen Sie alle ganzen Funktionen, die für alle $r>0$ folgende Eigenschaft erfüllen:
	\[
		\frac{1}{2\pi}\int_0^{2\pi}|f(re^{it})|\dd{t}\le r^n
	.\] 
\end{Problem}

\begin{Problem}
	Es sei $r>0$ und $f\in \mathcal{H}(K_r(0)) $. Ferner sei für $z\in \C$ die (formale) Potenzreihe
	\[
		F(z):=\sum_{k=0}^{\infty} \frac{f^{(k)}(0)}{(k!)^2}z^k
	\]
	gegeben. Zeigen Sie, dass $F:\C\to \C$ eine ganze Funktion definiert und dass für alle $z\in \C$ und alle $0\le R < r$ folgende Ungleichung gilt
	\[
		F(z)\le \|f\|_{\partial K_r(0)}\exp\left( \frac{|z|}{R} \right) 
	.\] 
\end{Problem}
\begin{proof}
	Wir zeigen, dass die Potenzreihe konvergiert. Sei $z\in \C$ beliebig. Es gibt $z_0\in K_r(0)$ mit $z=Rz_0$ und $\R\ni R > 0$. Mit dem Würzelkriterium 
\end{proof}
\begin{Problem}
	Es sei $f:\C\to \C$ eine ganze, nullstellenfreie Funktion mit der Eigenschaft
	\[
	|f(2z)|\le 2|f(z)|,\qquad z\in \C
	.\] 
	Zeigen Sie, dass $f$ konstant ist.
\end{Problem}
\begin{proof}
	Sei
	\[
		M=\sup_{z\in \overline{K_r(0)}}|f(z)|
	.\] 
	Das ist endlich, weil $|f|$ auf $\overline{K_r(0)}$ stetig ist.
	\begin{Theorem}
		Sei $|z|<2^n$. Dann gilt $|f(z)|\le M 2^n$. 
	\end{Theorem}
	\begin{proof}
	Das beweisen wir durch Induktion. Für $n=0$ ist das die Definition von $f$. Jetzt nehmen wir an, dass die Aussage für beliebiges $n\in \N$ gilt. Wir betrachten $K_{2^{n+1}}(0)$. Für $z\in K_{2^{n+1}}(0)$ gilt $z / 2\in K_{2^n}(0)$ und daher per Induktionsvoraussetzung $|f(z / 2)|\le M 2^n$.

	Aus der in der Aufgabe gegebenen Voraussetzung gilt dann
	\[
		|f(z)| \le M 2^{n+1}
	.\qedhere\] 
\end{proof}
Sei $z\in \C$ beliebig und $n\in \N_0$, sodass $2^{n+1}>|z|\ge 2^n$. Daraus folgt
\begin{align*}
	|f(z)|<& M 2^{n+1}\\
	<& 2M|z|
\end{align*}
also wir dürfen Satz 7.12 anwenden. $f(z)$ ist daher ein Polynom vom Grad $\le 1$, $f(z)=az+b,~a,b\in \C$. Wir wissen aber, dass ein Polynom vom Grad 1 eine Nullstelle besitzt (entweder mit dem Fundamentalsatz der Algebra oder wir können die Gleichung direkt lösen und erhalten $z_0= - b / a$.)

Nach Aufgabenstellung ist $f$ jedoch nullstellenfrei, also $a=0$ und $f$ ist konstant.
\end{proof}
\begin{Problem}
	Seien $U\subseteq \C$ offen, $z_0\in U$ und $f\in \mathcal{H}(U\setminus \{z_0\} )$. Zeigen Sie, dass jede der Voraussetzungen hinreichend für die Existenz einer holomorphen Fortsetzung $\tilde{f}:U\to \C$ von $f$ auf $U$ ist.
	\begin{parts}
		\item $f(U\setminus \{z_0\} )\subseteq \mathbb{H}^+=\{z\in \C|\Re(z)>0\} $.
		\item Es existieren $C>0$ und $\alpha>-1$ derart, dass
			\[
			|f(z)|\le C|z-z_0|^\alpha
		\]
		für alle $z\in U\setminus \{z_0\} $ ist.
	\end{parts}
\end{Problem}
