\begin{Problem}
	Seien $\mathcal{C}$, $\mathcal{C}'$ und $\mathcal{C}_0$ Halbkreise mit Durchmessern $AC,~AB$ bzw. $BC$, sodass $A,~B$ und $C$ auf einer Gerade liegen. Wir betrachten erner Kreise $\mathcal{C_n}$ f\"{u}r $n\in \N$ tangential zu den Halbkreisen $\mathcal{C}$ und $\mathcal{C}'$, sodass ferner $\mathcal{C}_n$ tangential zu $\mathcal{C}_{n-1}$ in einem Punkt $P_n$ ist. Zeigen Sie, dass es eine Kreislinie gibt, die alle Berührpunkte $P_0,P_1,\dots$ enthält.

	{
	\tikzset{circle through 3 points/.style n args={3}{%
insert path={let \p1=($(#1)-(#2)$),\p2=($(#1)!0.5!(#2)$),
    \p3=($(#1)-(#3)$),\p4=($(#1)!0.5!(#3)$),\p5=(#1),\n1={(-(\x2*\x3) + \x3*\x4 + \y3*(-\y2 +
    \y4))/(\x3*\y1 - \x1*\y3)},\n2={veclen(\x5-\x2-\n1*\y1,\y5-\y2+\n1*\x1)} in
    ({\x2+\n1*\y1},{\y2-\n1*\x1}) circle (\n2)}
}}	
	\begin{center}
		\begin{tikzpicture}[scale=1.5]
			\draw[thick] (0,0) arc(180:0:2);
			\draw[thick] (0,0) arc(180:0:2.5);
			\draw[thick] (4,0) arc(180:0:0.5);
			\foreach \x in {1,2,..., 20}{
				\coordinate (A) at ({0.225/((0.225)^2 + ((\x*2-1)*0.025)^2)}, {(\x*2-1)*0.025/((0.225)^2 + ((\x*2-1)*0.025)^2)});
				\coordinate (B) at ({0.2/((0.2)^2 + (\x*0.05)^2)}, {\x*0.05/((0.2)^2 + (\x*0.05)^2)});
				\coordinate (C) at ({0.25/((0.25)^2 + (\x*0.05)^2)},{\x*0.05/((0.25)^2 + (\x*0.05)^2)});
				\draw[thick,circle through 3 points={A}{B}{C}];
			}
		\draw[thick] (0,0) node[anchor=north] {$A$} -- (5,0) node[anchor=north] {$C$};
		\draw (4,0) node[anchor=north] {$B$};
		\end{tikzpicture}
	\end{center}
}
\end{Problem}
\begin{proof}
	Wir führen ein Möbiustransformation durch. OBdA nehmen wir an, dass $A$ das Ursprung ist. 

	Danach betrachten wir die Inversion $z\mapsto 1 / z$. Die reelle Achse wird offensichtlich auf die reelle Achse abgebildet. Insbesondere bleiben $B$ und $C$ auf der reellen Achse. $A$ wird auf $\infty$ abgebildet. 
	
	\textbf{Schritt 1:} Die große Halbkreisen werden auf vertikale Geraden abgebildet.
	
	Wir schreiben die Koordinaten von $B$ bzw. $C$ als $x,~x\in \R$. Das höchste Punkt hat Koordinaten $x/2 + ix/2$. Das wird auf
	\[\frac{1}{\frac x2 + i \frac x2}=\frac{1-i}{x}=\frac{1}{x}-\frac{i}{x}.\]
	Das heißt, dass die Halbkreisen auf vertikalen Geraden bei x-Koordinaten $1/(2R_1)$ und $1/(2R_2)$ abgebildet werden.
\end{proof}
\begin{Problem}
	Es sei $z_0\in \C,~(a_k)\subseteq \C$ und $K_R(z_0)$, $0<R<\infty$, die Konvergenzkreisscheibe der Potenzreihe
	\[
		f(z)=\sum_{k=0}^\infty a_k(z-z_0)^k,~z\in K_R(z_0)
	.\] 
	Zeigen Sie:
	\begin{parts}
		\item F\"{u}r jedes $r\in [0,R)$ gilt $\sum_{k=0}^\infty |a_k|^2 r^{2k}=\frac{1}{2\pi}\int_0^{2\pi} |f(z_0+re^{it})|^2\dd{t}$.
		\item Falls $|f(z)|\le M$ f\"{u}r alle $z\in K_R(z_0)$, so gilt $|a_k|\le M \frac{1}{R^k}$ f\"{u}r alle $k\in \N_0$.
	\end{parts}
\end{Problem}

\begin{proof}
	\begin{parts}
	\item Es gilt
		\begin{align*}
			f(z_0+re^{it})=& \sum_{k=0}^\infty a_kr^k e^{kit}\\
			\overline{f}(z_0+re^{it})=&\sum_{k=0}^\infty \overline{a}_k r^k e^{-kit}\\
			|f(z_0+re^{it})|^2=&\left( \sum_{k=0}^\infty a_k r^k e^{kit}\right)\left( \sum_{k=0}^\infty \overline{a}_k r^k e^{-kit} \right)\\
			=&\sum_{k=0}^\infty\sum_{l=0}^k\left(a_l r^l e^{lit} \overline{a}_{k-l} r^{k-l}e^{-(k-l)it} \right)\\
			=&\sum_{k=0}^\infty \left[ r^k\sum_{l=0}^k a_l \overline{a}_{k-l}e^{(2l-k)it} \right] 
		\end{align*}
		Die Reihe konvergiert gleichmäßig, also wir dürfen die Summe und das Integral vertauschen
		\begin{align*}
			&\int_0^{2\pi} |f(z_0+re^{it})|^2\dd{t}\\
			=&\int_{0}^{2\pi} \sum_{k=0}^\infty \left[ r^k \sum_{l=0}^k a_l \overline{a}_{k-l} e^{(2l-k)it} \right] \dd{t}\\
			=&\sum_{k=0}^\infty r^k \sum_{l=0}^k a_l \overline{a}_{k-l} \int_0^{2\pi} e^{(2l-k)it}\dd{t}
		\end{align*}
		Wir wissen aber, dass das Integral nur ungleich Null ist genau dann, wenn $2l-k=0$. Weil $2l$ gerade ist, muss $k$ auch gerade sein. Daher substituieren wir $k:=2p,~p\in \N_0$. Der Ausdruck ist
		\begin{align*}
			&\int_0^{2\pi} |f(z_0+re^{it})|^2\dd{t}\\
			=& \sum_{p=0}^\infty r^{2p} \sum_{l=0}^{2p} a_l \overline{a}_{2p-l} \int_0^{2\pi} e^{(2l-2p)it}\dd{t}\\
			=&\sum_{p=0}^\infty r^{2p}|a_p|^2 (2\pi) &\text{Nur }l=p\text{ Fall bleibt}\\
\sum_{k=0}^\infty r^{2k} |a_k|^2=& 			\frac{1}{2\pi}\int_0^{2\pi} |f(z_0+re^{it})|\dd{t}
		\end{align*}
	\item Es gilt
		\begin{align*}
			\sum_{k=0}^\infty |a_k|^2 r^{2k}=& \frac{1}{2\pi}\int_0^{2\pi} |f(z_0+re^{it})|^2\dd{t}\\
			\le& \frac{1}{2\pi}\int_0^{2\pi} M^2\dd{t}\\
			=& M^2
		\end{align*}
		andererseits gilt, weil alle Terme in der Summe positiv sind,
		\[
		\sum_{k=0}^\infty |a_k|^2r^{2k} \ge |a_p|^2 r^{2p}
		\] 
		f\"{u}r alle $p\in \N_0$. Daraus folgt
		\[
		|a_k| \le \frac{M}{r^k}
		\]
		f\"{u}r alle $r\in [0,R)$. Dann muss $|a_k|\le M / R^k$ sein.\qedhere
	\end{parts}
\end{proof}
\begin{Problem}\label{pr:complexanal4-3}
	Seien $z_0\in \C,~r>0$ und $k\in \Z\setminus \{1\} $. Zeigen Sie, dass f\"{u}r alle $z\in K_r(z_0)$ folgende Identität gilt:
	\[
		\int_{\partial K_r(z_0)}\frac{1}{(w-z)^k}\dd{w}=0
	.\] 
	Warum schließen wir $k=1$ aus?

	\emph{Hinweis: Betrachten Sie zunächst den Spezialfall} $z=z_0$ \emph{und versuchen Sie anschließend den allgemeinen Fall auf diesen zurückzuführen.}
\end{Problem}
\begin{proof}
	Spezialfall $z=z_0$: Wir parametrisieren den Weg durch $\gamma(t)= z_0+ re^{it},~t\in [0,2\pi]$ und $\gamma'(t) = i re^{it}$. 
\begin{align*}
	\int_{\partial K_r(z_0)} \frac{1}{(w-z_0)^k}\dd{w}=& \int_0^{2\pi} \frac{\gamma'(t)}{(\gamma(t)-z_0)^k}\\
	=&\int_0^{2\pi} \frac{i re^{it}}{(z_0+re^{it}-z_0)^k}\dd{t}\\
	=&\int_0^{2\pi} i r^{1-k} e^{(1-k)it}\dd{t}\\
	=&i r^{1-k} \left[ \frac{1}{i(1-k)}e^{(1-k)it} \right]_0^{2\pi} & k\neq 0\\
	=& \frac{r^{1-k}}{1-k}\left[ e^{2\pi(1-k)i} - e^{0} \right] \\
	=& \frac{r^{1-k}}{1-k}\left[ 1- 1\right] =0
\end{align*}
Jetzt im Allgemein:
\begin{align*}
	\int_{\partial K_r(z_0)}\frac{1}{(w-z)^k}\dd{w}=&\int_{\partial K_r(z_0)}\frac{1}{(w-z_0+z_0-z)^k}\dd{w}\\
	=&\int_{\partial K_r(z_0)}\frac{1}{(w-z_0)^k}\frac{1}{\left( 1+\left( \frac{z_0-z}{w-z_0} \right) \right)^k}\dd{w}
\end{align*}
Da $|w-z_0|=r$ und $|z_0-z|<r$, ist
\[
 \left|\frac{z_0-z}{w-z_0} \right| <1
.\]
Daher dürfen wir den Ausdruck in einer Potenzreihe entwickeln.
\begin{align*}
&	\int_{\partial K_r(z_0)} \frac{1}{(w-z)^k}\dd{w}
\\=&\int_{\partial K_r(z_0)}\frac{1}{(w-z_0)^k}\sum_{n=0}^\infty \binom{-k}{n}\left( \frac{z_0-z}{w-z_0} \right)^n\dd{w}\\
=&\int_{\partial K_r(z_0)}\sum_{n=0}^\infty \binom{-k}{n} \frac{(z_0-z)^n}{(w-z_0)^{k+n}}\dd{w}\\
=&\sum_{n=0}^\infty \int_{\partial K_r(z_0)} \frac{(z_0-z)^n}{(w-z_0)^{k+n}}\dd{w}\\
=& \sum_{n=0}^\infty 0\\
=&0
\end{align*}
wobei wir die Summe und das Integral tauschen dürfen, weil die Reihe gleichmäßig konvergiert.
\end{proof}

\begin{Problem}
	Sei $f:\mathbb{D}\to \C$ holomorph und $g(z)=zf(z)$.
	\begin{parts}
	\item Sei $K\subset \mathbb{D}$ kompakt. Beweisen Sie, dass die Funktionreihe $\sum_{n=1}^\infty g(z^n)$ gleichmäßig auf $K$ konvergiert.
	\item Zeigen Sie, dass die Funktionenreihe $\sum_{n=1}^\infty g(z^n)$ nicht notwendigerweise gleichmäßig auf der ganzen Einheitskreisscheibe $\mathbb{D}$ konvergiert.
	\end{parts}
\end{Problem}
