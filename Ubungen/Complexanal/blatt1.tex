\begin{Problem}
	Es seien $\mathbb{D}=K_1(0)=\{z\in \C:|z|<1\} $ die offene Einheitskreisscheibe in $\C$ und $a\in \mathbb{D}$.
	\begin{parts}
	\item Zeigen Sie
		\[
		|a-z|<|1-\overline{a}z| \iff |z| < 1
		\]
		und
		\[
		|a-z|=|1-\overline{a}z|\iff |z| = 1
		.\] 
		{\footnotesize\emph{Hinweis: }$|\cdot|^2$}
	\item Es sei die folgende bijektive (holomorphe) Funktion definiert:
		\[
		T_a:\mathbb{D}\to \mathbb{D},~T_a(z):=\frac{a-z}{1-\overline{a}z}
		.\] 
		Bestimmen Sie die Umkehrabbildung von $T_a$.
	\end{parts}
\end{Problem}
\begin{proof}
	\begin{parts}
	\item 
		\begin{align*}
			|a-z|^2=&(a-z)\overline{(a-z)}\\
			=&(a-z)(\overline{a}-\overline{z})\\
			=&|a|^2-a\overline{z}-\overline{a}z+|z|^2\\
			|1-\overline{a}z|^2=&(1-\overline{a}z)(1-\overline{a}z)^*\\
			=&(1-\overline{a}z)(1-a\overline{z})\\
			=&1-\overline{a}z-a\overline{z}+|a|^2|z|^2.
		\end{align*}
		Damit ist
		\begin{align*}
			&|a-z|<|1-\overline{a}z|\\
			\iff&|a-z|^2<|1-\overline{a}z|^2\\
			\iff&|a|^2-\overline{a}z-a\overline{z}+|z|^2\\
			&<1-\overline{a}z-a\overline{z}+|a|^2|z|^2,
		\end{align*}
		was genau dann erf\"{u}llt ist, wenn
		\begin{align*}
			|a|^2+|z|^2<&1+|a|^2|z|^2\\
			1>&|a|^2+|z|^2-|a|^2|z|^2\\
			=&|a|^2+|z|^2(1-|a|^2)
		\end{align*}
		Offensichtlich ist die rechte Seite der Gleichung eine monoton steigende Funktion von $|z|^2$. Da $a\in \mathbb{D}$, ist $0\le |a|^2<1$. Wenn wir $a$ als fest betrachten, und nach der passenden $|z|$ suchen, brauchen wir also nur einen Wert von $|z|$. Denn
		\[
		|a|^2+(1)(1-|a|^2)=1
		,\]
		ist 1 der ``kritische Wert'', also 
		\begin{align*}
			&1>|a|^2+|z|^2(1-|a|^2)\\
			\iff&1>|z|^2\\
			\iff& 1>|z|
		\end{align*}
		Der zweite Teil folgt aus den gleichen Ausdrucke.
	\item 
\begin{align*}
			T_a(x)=&\frac{a-z}{1-\overline{a}z}\\
			=& a \frac{1-z / a}{1-\overline{a}z}\\
			=&a \frac{1-\frac{z}{a}+\overline{a}z-\overline{a}z}{1-\overline{a}z}\\
			=&a\left[ 1+ \frac{\overline{a}z-\frac{z}{a}}{1-\overline{a}z} \right]\\
			=&a\left[ 1+\frac{\overline{a}-\frac{1}{a}}{\frac{1}{z}-\overline{a}} \right] 
		\end{align*}
		Also
		\begin{align*}
			\frac{\overline{a}-\frac{1}{a}}{\frac{1}{z}-\overline{a}}=& \frac{T_a(z)}{a}-1\\
			\frac{1}{z}-\overline{a}=&\frac{\overline{a}-\frac{1}{a}}{\frac{T_a(z)}{a}-1}\\
			\frac{1}{z}=&\frac{1}{\frac{T_a(z)}{a}-1}\left[ \overline{a}-\frac{1}{a}+\frac{\overline{a}}{a}T_a(z)-\overline{a} \right] \\
			=&\frac{1}{a}\frac{1}{\frac{T_a(z)}{a}-1}\left[ \overline{a}T_a(z)-1 \right] 
		\end{align*}
		Damit ist
		\[
		T_a^{-1}(z)=\frac{z-a}{\overline{a}z-1}
		.\qedhere\] 
	\end{parts}
\end{proof}
\begin{Problem}
	\begin{parts}
	\item Sei $z=x+iy$ mit $x,y\in \R$ und $y\neq 0$. Zeigen Sie, dass
		\[
		|\sin(z)|\ge \frac{1}{2}(e^{|y|}-e^{-|y|})
		\]
		ist.

		{\footnotesize \emph{Hinweis: Verwenden Sie die Darstellung }$\sin(z)=(e^{iz}-e^{-iz}) / (2i)$ \emph{vgl. Beispiel 1.8}}
	\item Gegeben sei die Funktionfolge $\{f_n\}_{n=1}^\infty$ mit
		\[
		f_n(z):=\frac{\sin(nz)}{n},\qquad z\in \C.\] 
		Geben Sie die Menge $M$ aller Punkte $z\in \C$ an, f\"{u}r die $\{f_n(z)\}_{n=1}^\infty$ konvergiert, und bestimmen Sie die Grenzfunktion
		\[
		f(z):=\lim_{n \to \infty} f_n(z),\qquad z\in M
		.\] 
	\item Konvergiert die Funktionfolge $\{f_n\} $ gleichm\"{a}ßig auf $M$?
	\end{parts}
\end{Problem}

\begin{proof}
	\begin{parts}
	\item 
		\begin{align*}
			|\sin(z)|=& \left| \frac{e^{iz}-e^{-iz}}{2i} \right| \\
			=&\frac{1}{2}\left| e^{iz}-e^{-iz} \right| \\
			\ge&\frac{1}{2}\left[ |e^{iz}|-|e^{-iz}| \right] \addtocounter{equation}{1}\tag{\theequation}\label{eq:companal1-1} 
		\end{align*}
		Ähnlich ist
		\begin{equation}
			\sin(z)\ge \frac{1}{2}\left[ |e^{-iz}|-|e^{iz}| \right] \label{eq:companal1-2}
		.\end{equation}
		Nebenrechnung:
		\begin{align*}
			|e^{\pm iz}|=&|e^{\pm i(x+iy)}|\\
			=&|e^{\pm ix}e^{\mp y}|\\
			=&|e^{\pm ix}| |e^{\mp y}|\\
			=&|e^{\mp y}|\\
			=&e^{\mp y}
		\end{align*}
		Einsetzen in entweder Gl.~\eqref{eq:companal1-1} oder GL.~\eqref{eq:companal1-2} liefert
		\begin{equation}
			|\sin(z)| \ge \frac{1}{2}\left( e^{|y|}-e^{-|y|} \right)\label{eq:companal1-3} 
		.\end{equation}
	\item Wir betrachten deren Betrag
		\begin{align*}
			|f_n(z)|=&\left| \frac{\sin(nz)}{n} \right| \\
			\ge& \frac{1}{2n}[e^{|n y|}-e^{-|ny|}]\\
			\ge& \frac{1}{2n}e^{n|y|}
		\end{align*}
Wenn $|y|\neq 0$, konvergiert die Folge nicht, da
\[
\lim_{x \to \infty} \frac{e^x}{x}=\lim_{x \to \infty} \frac{e^x}{1}=\infty
\]
nach L'Hopital. 

Damit konvergiert $\{f_n\} $ nicht, wenn $z \not\in \R$ gilt .

Wenn $z\in \R$ ist, konvergiert $\{f_n\} $, da
\[
|f_n(z)|= \left| \frac{\sin (nz)}{n} \right| \le \left| \frac{1}{n} \right| 
,\]
was gegen $0$ geht.

Insgesamt ist $M=\R$ und $f(z)=0$. 
\item Ja. Da $|f_n(z)|\le |1 / n|$, ist auch $\sup_{z\in \R}|f_n(z)|\le |1 / n|$. Damit konvergiert $\{f_n\} $ im Supremumsnorm. $\{f_n\} $ konvergiert also gleichmäßig.\qedhere 
	\end{parts}
\end{proof}
