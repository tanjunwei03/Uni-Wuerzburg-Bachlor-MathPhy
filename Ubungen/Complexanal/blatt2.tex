\begin{Problem}
	Beweisen oder widerlegen Sie
	\begin{parts}
		\item Die Funktion
			\[
			f: \{z\in \C:|\Im(z)|<1\} \to \C,\qquad f(z)=\frac{1}{1+z^2}
		\]
		ist beschr\"{a}nkt.
	\item Es sei $U\subseteq \C$ eine offene Menge und $f\in H(U)$ nicht konstant. Dann ist die Funktion $z\to f(\overline{z})$ holomorph auf $U^*:=\{z\in C|\overline{z}\in U\} $.
	\item Seien $f,g:K_1(0)\to \C$ stetige Funktionen. Sei außerdem die Funktion
		 \[
		h:K_1(0)\to \C,\qquad h(z)=f(z)\cdot g(z)
	\]
	holomorph. Dann ist auch $f$ oder $g$ holomorph auf $K_1(0)$.
\item Es sei $G$ ein Gebiet in $\C$ und $f\in H(G)$ mit $\Re(f(z))=1$ f\"{u}r alle $z\in G$. Dann ist $f$ konstant.
	\end{parts}
\end{Problem}

\begin{proof}
	\begin{parts}
	\item Falsch. Man betrachte einfach $x\to i$. Für alle $\R\ni N > 0$ gibt es $z\in \{iy, y\in (0,1)\} $ sodass $\frac{1}{1+z^2}=\frac{1}{1-y^2}>y$ ist. Damit ist $f$ nicht beschränkt. 
	ist $f$ nicht beschränkt.
\item Falsch. Beweis wie: $z\mapsto \overline{z}$ ist nicht differenzierbar.
\[
\dv{z}(f(\overline{z}))=\lim_{y \to y_0} \frac{f(x_0-iy)-f(x_0-i y_0)}{iy_0}=i \pdv{f}{y}=-f'(\overline{z_0})
,\]
jedoch
\[
\dv{z}(f(\overline{z}))=\lim_{x \to x_0} \frac{f(x-i y_0) - f(x_0 - i y_0)}{x_0}= f'(\overline{z_0})
.\] 
Weil $f$ nicht konstant ist, gibt es mindestens ein Punkt $z_0$, sodass $f'(\overline{z_0})\neq 0$. Daher kann $f'(\overline{z_0})$ nicht immer gleich $-f'(\overline{z_0})$ sein, also $f(\overline{z})$ ist nicht differenzierbar.
\item Falsch. Sei
\[
f(z)=\begin{cases}
	1 & z\in \Q^2\\
	0 & \text{sonst.}
\end{cases}
\]
und
\[
g=\begin{cases}
	0 & z\notin \Q^2\\
	1 & \text{sonst.}
\end{cases}
\]
im Sinne, dass wir $\C$ mit $\R^2$ sowie $\Q^2$ mit einer Teilmenge von $\R^2$ idenfizieren. $f$ und $g$ sind dann in keinem Punkt differenzierbar, weil die in keinem Punkt stetig sind. Deren Produkt ist aber $f\cdot g = 0$, eine konstante Funktion, also differenzierbar.
\item Wahr. Schreibe $f=u+iv$ mit $u=\Re(f)$ und $v=\Im(f)$. Es gilt $u=1$. Es gilt auch die Cauchy-Riemann-Gleichungen
	\[
		0=\pdv{u}{x}=\pdv{v}{y}\qquad 0=\pdv{u}{y}=-\pdv{v}{x}
	.\] 
	Da $f$ holomorph ist, verwenden wir die Wirtinger-Ableitung
	\[
		f'=\pdv{f}{z}=\frac{1}{2}\left( \pdv{f}{x}-i\pdv{f}{y} \right) =0
	.\] 
	Dann ist $f$ konstant.\qedhere
	\end{parts}
\end{proof}

\begin{Problem}
	\begin{parts}
	\item Es seien $U,V$ offene Menge in $\C$ sowie $f:U\to V$ eine stetige und $g:V\to \C$ eine holomorphe Funktion. Ferner sei $g'(w)\neq 0$ f\"{u}r alle $w\in V$ und es gelte $g(f(z))=z$ f\"{u}r alle $z\in U$. Zeigen Sie, dass $f$ holomorph auf $U$ ist und $f'(z)=1 / (g'(f(z))$ f\"{u}r alle $z\in U$.
	\item Es sei $U\subseteq \C$ offen und $f:U\to \C$ stetig und nullstellenfrei. Zeigen Sie, dass aus $f^2\in H(U)$ bereits $f\in H(U)$ folgt.
	\end{parts}
\end{Problem}
\begin{proof}
	\begin{parts}
	\item $g(f(z))$ ist differenzierbar. Insbesondere
		\begin{align*}
			1=&\lim_{z \to z_0} \frac{g(f(z))-g(f(z_0))}{z-z_0}\\
			=&\lim_{z \to z_0} \frac{g(f(z))-g(f(z_0))}{f(z)-f(z_0)}\frac{f(z)-f(z_0)}{z-z_0}
		\end{align*}
		Der Grenzwert
		\[
		\lim_{z \to z_0} \frac{g(f(z))-g(f(z_0))}{z-z_0}
	\]
	existiert, da $f$ stetig ist, und sogar
	\[
		\lim_{z \to z_0} \frac{g(f(z))-g(f(z_0))}{f(z)-f(z_0)}=\lim_{z \to f(z_0)} \frac{g(z)-g(f(z_0))}{z-f(z_0)}=g'(f(z_0))
	.\] 
	D.h. der andere Grenzwert 
\[
\lim_{z \to z_0} \frac{f(z)-f(z_0)}{z-z_0}
.\] 
	existiert auch ($f$ ist holomorph), und
	\[
	1=g'(f(z_0)) \lim_{z \to z_0} \frac{f(z)-f(z_0)}{z-z_0}=g'(f(z_0)) f'(z_0)
	.\]
	Die Behauptung folgt.
\item Es gilt
	\begin{align*}
		\lim_{z \to z_0} \frac{f(z)^2-f(z_0)^2}{z-z_0}=& \lim_{z \to z_0} \frac{(f(z)-f(z_0))(f(z)+f(z_0))}{z-z_0}\\
		=& \lim_{z \to z_0} \left[ \frac{f(z)-f(z_0)}{z-z_0} \right]\left[ f(z)+f(z_0) \right] 
	\end{align*}
	Der Grenzwert
	\[
	\lim_{z \to z_0} [f(z)+f(z_0)]
	\]
	existiert, und zwar konvergiert der Limes gegen $2f(z_0)\neq 0$, weil $f$ stetig ist, und auch weil $f(z_0)\neq 0$ ist ($f$ ist nullstellenfrei). Das heißt, der Grenzwert
	\[
	\lim_{z \to z_0} \frac{f(z)-f(z_0)}{z-z_0}=\lim_{z \to z_0} \frac{f(z)^2-f(z_0)^2}{z-z_0}\frac{1}{f(z)+f(z_0)}
	\]
	existiert auch.\qedhere
	\end{parts}
\end{proof}
\begin{Problem}
	Es sei $z=x+iy$ mit $x,y\in \R$ und
	\[
	f:\C\to \C,\qquad z\mapsto \begin{cases}
		\frac{x^3y(y-ix)}{x^6+y^2} & z\neq 0\\
		0 & z = 0
	\end{cases}
	.\] 
	Ferner sei $z_0=0$. Beweisen Sie die folgenden Aussagen:
	\begin{parts}
	\item Die Funktion $f$ ist in $z_0$ partiell differenzierbar.
	\item Die Funktion erf\"{u}llt in $z_0$ die Cauchy-Riemannsche Differentialgleichung.
	\item Es sei $t\in \R$ fixiert. Dann besitzt $f$ einen radialen Grenzwert in $z_0$, also es existiert folgender Grenzwert
		\[
			\lim_{r \to 0,r>0} \frac{f(e^{it}r)-f(0)}{e^{it}r}
		.\] 
	\item Die Funktion $f$ ist in $z_0$ \emph{nicht} complex differenzierbar. Begr\"{u}nden Sie außerdem, warum dies nicht im Widerspruch zu Korollar 2.10 steht.
	\end{parts}
\end{Problem}
\begin{proof}
	\begin{parts}
	\item Wir schreiben wie üblich $f=u+iv$ und idenfizieren
		\[
		u=\frac{x^3 y^2}{x^6 + y^2}\qquad v = -\frac{x^4 y}{x^6+ y^2}
		\] 
		f\"{u}r $(x,y)\neq (0,0)$. Damit sind die partielle Ableitungen
		\begin{align*}
			\left.\pdv{u}{x}\right|_{(0,0)}=&\lim_{x \to 0} \frac{u(x,0)}{x}\\
			=&\lim_{x \to 0} 0 =0\\
			\left. \pdv{u}{y}\right|_{(0,0)}=&\lim_{y \to 0} \frac{u(0,y)}{y}\\
			=&\lim_{y \to 0} 0=0\\
			\left. \pdv{v}{x}\right|_{(0,0)}=&\lim_{x \to 0} \frac{v(x,0)}{x}\\
			=&\lim_{x \to 0} 0=0\\
			\left.\pdv{v}{y}\right|_{(0,0)}=&\lim_{y \to 0} \frac{v(0,y)}{y}\\
			=&\lim_{y \to 0} 0=0
		\end{align*}
		Da alle $4$ Ableitungen existieren, ist $f$ partiell differenzierbar.
	\item Die Cauchy-Riemann Gleichungen sind erfüllt (man setze die Ausdrucke aus (a) ein) 
	\item 
		\[
			f(e^{it} r)=\frac{(r\cos t)^3(r\sin t)(r\sin t - i r\cos t)}{r^6 \cos^6 t + r^2\sin^2 t}
		.\] 
		Da $f(0)=0$, ist
		\begin{align*}
			\lim_{r \to 0} \left|\frac{f(e^{it}r)-f(0)}{e^{it}r}\right|=&\lim_{r \to 0} \left|\frac{f(e^{it}r)}{e^{it}r}\right|\\
			=&\lim_{r \to 0} \left|e^{-it} \frac{r^2 \cos^3 t \sin t(\sin t - i \cos t)}{r^4 \cos^6 t + \sin^2 t}\right|\\
			=&0
		\end{align*}
	\item Wir betrachten das Weg parametrisiert durch $y=t^3$, $x=t$. Damit ist
		\begin{align*}
			\lim_{t \to 0} \frac{f(z(t))}{z(t)}=& \lim_{t \to 0} \frac{t^3(t^3)(t^3-it)}{t^6+t^6} \frac{1}{t+it^3}\\
			=&\lim_{t \to 0} \frac{t^2-i}{2(1+i t^2)}\\
			=& -\frac{i}{2}
		\end{align*}
		Aber wenn die Funktion holomorph wäre, wäre der Grenzwert gleich $\pdv{f}{x}=0$ sein. Das dies offensichtlich nicht der Fall ist, ist die Funktion nicht holomorph.\qedhere
	\end{parts}
\end{proof}
