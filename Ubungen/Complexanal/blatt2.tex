\begin{Problem}
	Beweisen oder widerlegen Sie
	\begin{parts}
		\item Die Funktion
			\[
			f: \{z\in \C:|\Im(z)|<1\} \to \C,\qquad f(z)=\frac{1}{1+z^2}
		\]
		ist beschr\"{a}nkt.
	\item Es sei $U\subseteq \C$ eine offene Menge und $f\in H(U)$ nicht konstant. Dann ist die Funktion $z\to f(\overline{z})$ holomorph auf $U^*:=\{z\in C|\overline{z}\in U\} $.
	\item Seien $f,g:K_1(0)\to \C$ stetige Funktionen. Sei außerdem die Funktion
		 \[
		h:K_1(0)\to \C,\qquad h(z)=f(z)\cdot g(z)
	\]
	holomorph. Dann ist auch $f$ oder $g$ holomorph auf $K_1(0)$.
\item Es sei $G$ ein Gebiet in $\C$ und $f\in H(G)$ mit $\Re(f(z))=1$ f\"{u}r alle $z\in G$. Dann ist $f$ konstant.
	\end{parts}
\end{Problem}

\begin{proof}
	\begin{parts}
	\item Falsch. Man betrachte einfach $x\to -1$ (sogar eingeschränkt auf der reellen Achse). Da
		\[
		\lim_{z \to -1} \frac{1}{1+z^2}=\infty
	,\]
	ist $f$ nicht beschränkt.
\item Falsch. Beweis wie: $z\mapsto \overline{z}$ ist nicht differenzierbar.
\[
	\lim_{y \to y_0} \frac{f(x_0-iy)-f(x_0-i y_0)}{iy_0}=i \pdv{f}{y}=-f'(z_0)
\]
also $f(\overline{z})$ ist nicht differenzierbar.
\item Falsch. Sei $f=0$ und $g$ irgendeine nicht holomorphe Funktion, z.B. die Konjugationsabbildung. Dann ist $f(z)\cdot g(z)=0$ und somit konstant und auch holomorph. 
\item Wahr. Schreibe $f=u+iv$ mit $u=\Re(f)$ und $v=\Im(f)$. Es gilt $u=1$. Es gilt auch die Cauchy-Riemann-Gleichungen
	\[
		0=\pdv{u}{x}=\pdv{v}{y}\qquad 0=\pdv{u}{y}=-\pdv{v}{x}
	.\] 
	Da $f$ holomorph ist, verwenden wir die Wirtinger-Ableitung
	\[
		f'=\pdv{f}{z}=\frac{1}{2}\left( \pdv{f}{x}-y\pdv{f}{y} \right) =0
	.\] 
	Dann ist $f$ konstant.\qedhere
	\end{parts}
\end{proof}

\begin{Problem}
	\begin{parts}
	\item Es seien $U,V$ offene Menge in $\C$ sowie $f:U\to V$ eine stetige und $g:V\to \C$ eine holomorphe Funktion. Ferner sei $g'(w)\neq 0$ f\"{u}r alle $w\in V$ und es gelte $g(f(z))=z$ f\"{u}r alle $z\in U$. Zeigen Sie, dass $f$ holomorph auf $U$ ist und $f'(z)=1 / (g'(f(z))$ f\"{u}r alle $z\in U$.
	\item Es sei $U\subseteq \C$ offen und $f:U\to \C$ stetig und nullstellenfrei. Zeigen Sie, dass aus $f^2\in H(U)$ bereits $f\in H(U)$ folgt.
	\end{parts}
\end{Problem}
\begin{proof}
	\begin{parts}
	\item $g(f(z))$ ist differenzierbar. Insbesondere
		\begin{align*}
			1=&\lim_{z \to z_0} \frac{g(f(z))-g(f(z_0))}{z-z_0}\\
			=&\lim_{z \to z_0} \frac{g(f(z))-g(f(z_0))}{f(z)-f(z_0)}\frac{f(z)-f(z_0)}{z-z_0}
		\end{align*}
		Der Grenzwert
		\[
		\lim_{z \to z_0} \frac{g(f(z))-g(f(z_0))}{z-z_0}
	\]
	existiert, da $f$ stetig ist, und sogar
	\[
		\lim_{z \to z_0} \frac{g(f(z))-g(f(z_0))}{f(z)-f(z_0)}=\lim_{z \to f(z_0)} \frac{g(z)-g(f(z_0))}{z-f(z_0)}=g'(f(z_0))
	.\] 
	D.h. der andere Grenzwert 
\[
\lim_{z \to z_0} \frac{f(z)-f(z_0)}{z-z_0}
.\] 
	existiert auch ($f$ ist holomorph), und
	\[
	1=g'(f(z_0)) \lim_{z \to z_0} \frac{f(z)-f(z_0)}{z-z_0}=g'(f(z_0)) f'(z_0)
	.\]
	Die Behauptung folgt.
\item 
	\end{parts}
\end{proof}
\begin{Problem}
	Es sei $z=x+iy$ mit $x,y\in \R$ und
	\[
	f:\C\to \C,\qquad z\mapsto \begin{cases}
		\frac{x^3y(y-ix)}{x^6+y^2} & z\neq 0\\
		0 & z = 0
	\end{cases}
	.\] 
	Ferner sei $z_0=0$. Beweisen Sie die folgenden Aussagen:
	\begin{parts}
	\item Die Funktion $f$ ist in $z_0$ partiell differenzierbar.
	\item Die Funktion erf\"{u}llt in $z_0$ die Cauchy-Riemannsche Differentialgleichung.
	\item Es sei $t\in \R$ fixiert. Dann besitzt $f$ einen radialen Grenzwert in $z_0$, also es existiert folgender Grenzwert
		\[
			\lim_{r \to 0,r>0} \frac{f(e^{it}r)-f(0)}{e^{it}r}
		.\] 
	\item Die Funktion $f$ ist in $z_0$ \emph{nicht} complex differenzierbar. Begr\"{u}nden Sie außerdem, warum dies nicht im Widerspruch zu Korollar 2.10 steht.
	\end{parts}
\end{Problem}
\begin{proof}
	\begin{parts}
	\item Als Produkt bzw. Quotient differenzierbare Funktonen (von $x$ bzw. $y$) ist $f$ partiell differenzierbar. (Sogar überall, nicht nur in $z_0$)
	\item 
		\[
		f=\frac{x^3 y^2}{x^6 + y^2}-i \frac{x^4 y}{x^6 + y^2}
		.\] 
		Wir schreiben wie üblich $f=u+iv$. Die Ableitungen sind
		\begin{align*}
			\pdv{f}{x}=&\underbrace{-\frac{3 x^8 y^2}{\left(x^6+y^2\right)^2}+\frac{3 x^2 y^4}{\left(x^6+y^2\right)^2}}_{\partial u / \partial x}+i \underbrace{\left(\frac{2 x^9 y}{\left(x^6+y^2\right)^2}-\frac{4 x^3 y^3}{\left(x^6+y^2\right)^2}\right)}_{\partial v/\partial x}\\
			\pdv{f}{y} =& \underbrace{\frac{2 x^9 y}{\left(x^6+y^2\right)^2}}_{\partial u / \partial y}+i\underbrace{\left(\frac{x^4 y^2}{\left(x^6+y^2\right)^2}-\frac{x^{10}}{\left(x^6+y^2\right)^2}\right)}_{\partial v/\partial y}
		\end{align*}
		Offensichtlich ist $\pdv{f}{x}(0)=0$ und $\pdv{f}{y}(0)=0$. Damit ist $\pdv{f}{x}(0)=-i\pdv{f}{y}(0)$ und die Cauchy-Riemann-Gleichungen sind erfüllt.
	\item 
		\[
			f(e^{it} r)=\frac{(r\cos t)^3(r\sin t)(r\sin t - i r\cos t)}{r^6 \cos^6 t + r^2\sin^2 t}
		.\] 
		Da $f(0)=0$, ist
		\begin{align*}
			\lim_{r \to 0} \left|\frac{f(e^{it}r)-f(0)}{e^{it}r}\right|=&\lim_{r \to 0} \left|\frac{f(e^{it}r)}{e^{it}r}\right|\\
			=&\lim_{r \to 0} \right|e^{-it} \frac{r^2 \cos^3 t \sin t(\sin t - i \cos t)}{r^4 \cos^6 t + \sin^2 t}\right|\\
			\le &e^{-it} \frac{r^2 \cos^3 t \sin t(\sin t - i \cos t)}{\sin^2 t}\\
		\end{align*}
	\end{parts}
\end{proof}
