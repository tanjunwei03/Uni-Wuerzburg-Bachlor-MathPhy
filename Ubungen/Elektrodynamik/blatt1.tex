\begin{Problem}
	\begin{parts}
		\item Geben Sie die Definitionen von Gradient, Rotation und Divergenz an.
		\item Wir schreiben die Komponenten des dreidimensionalen Vektorprodukts als
			\[
				(\vec{a}\times \vec{b})_i=\sum_{j,k=1}^3 \epsilon_{ijk}a_jb_k,\] 
				wobei $\epsilon_{ijk}$ der total antisymmetrische Tensor f\"{u}r $\R^3$ ist, mit $\epsilon_{ijk}=1$. Zeigen Sie, dass gilt:
				\begin{align*}
					\sum_{i=1}^3 \epsilon_{ijk}\epsilon_{ilm}=&\delta_{jl}\delta_{km}-\delta_{jm}\delta_{kl}\\
					\frac{1}{2}\sum_{i,j=1}^3 \epsilon_{ijk}\epsilon_{ijl}=&\delta_{kl},
				\end{align*}
				mit $\delta$ dem Kronecker-$\delta$.
			\item Zeigen Sie mit den Formeln aus (b) die folgenden Identitäten für beliebige Vektorfelder $\vec{a},~\vec{b},~\vec{c},~\vec{d}$:
				\begin{align*}
					\vec{a}\cdot(\vec{b}\times\vec{c})=&\vec{b}\cdot (\vec{c}\times \vec{a})=\vec{c}\cdot (\vec{a}\times \vec{b})\\
					\vec{a}\times(\vec{b}\times\vec{c})=&(\vec{a}\cdot\vec{c})\vec{b}-(\vec{a}\cdot\vec{b})\vec{c},\\
					(\vec{a}\times\vec{b})\cdot(\vec{c}\times\vec{d})=&(\vec{a}\cdot\vec{c})(\vec{b}\cdot\vec{d})-(\vec{a}\cdot\vec{d})(\vec{b}\cdot\vec{c})
				\end{align*}
			\item Zeigen Sie damit, dass f\"{u}r beliebige skalare Funktionen $F(\vec{x})$ und Vektorfelder $\vec{A}(\vec{x})$ gilt:
				\begin{align*}
					\curl{\grad{F}}=&0\\
					\div{(\curl{A})}=&0\\
					\curl{\curl{A}}=&\grad{(\div{A})}-\laplacian{A}\\
					\div{(F\vec{A})}=&(\grad{F})\cdot\vec{A}+F\div{\vec{A}},
				\end{align*}
				mit $\Delta$ dem Laplace-Operator.
	\end{parts}
\end{Problem} 
\begin{proof}
	\begin{parts}
	\item 
		\begin{align*}
			\text{grad }F=\sum_{i=1}^3 \pdv{F}{x_i}\hat{x}_i\\
			\text{div }\vec{F}=\sum_{i=1}^3 \pdv{F_i}{x_i}\\
			\text{curl }\vec{F}=&\curl{F}
		\end{align*}
	\item Bemerkung: $j,k,l$ und $ m$ sind freie Indizes. $i$ wird summiert.

	Spezialfall $j=k$ oder $l=m$: Die Summe ist 0

	Voraussetzung, um eine nicht Null Summe zu erhalten: $i$ muss nicht gleich $j,k,l$ oder $m$ sein. Da es nur insgesamt 3 Möglichkeiten für die Indizes gibt, muss für die Mengen gelten:
	 \[
	\{j,k\} =\{l,m\} 
	.\] 
	Wenn die vorherige Gleichung gilt, ist die Summe $\neq 0$ genau dann, wenn
	\[
	i\in \{1,2,3\} \setminus \{j,k\} 
	.\] 
	In diesem Fall ist die Summe eindeutig bestimmt. 

	Der eindeutige Ausdruck mit den gleichen Eigenschaften ist
	\[
		\delta_{jl}\delta_{km}-\delta_{jm}\delta_{kl}
	.\] 
	Für den zweiten Teil betrachten wir wieder die Fälle:
	
	Wenn $k\neq l$, gälte einer der Fälle
	\begin{enumerate}[label=(\roman*)]
		\item $i=k$, oder
		\item $i=k$, oder $j=k$, oder
		\item $i=l$, oder $j=l$
	\end{enumerate}
	Damit sind alle Terme in der Summe und auch die ganze Summe $0$.

	Im Fall $k=l$: Sei $\{1,2,3\} = \{a,b,k\} =\{a,b,l\} $. Dann gibt es nur nicht Null Terme: $\epsilon_{abk}\epsilon_{abl}=(\epsilon_{abl})^2$ und $(\epsilon_{bal})^2$.

	Unabhängig vom Signum der Permutation ist dessen Quadrat 1. Die Summe ist damit $2$ und der Ausdruck $1$.
\item Doppel auftretende Indizes werden immer implizit summiert.
	\begin{enumerate}[label=(\roman*)]
		\item 
			\begin{align*}
				\va a\cdot (\va b\times \va c)=& a_i(\epsilon_{ijk}b_jc_k)\\
				=&c_j\epsilon_{kij}a_ib_j\\
				=&\va c\cdot (\va c\times \va b)\\
				=&b_j\epsilon_{jki}c_ka_i\\
				=&\va b\cdot(\va c\times \va a)
			\end{align*}
		\item 
			\begin{align*}
				[\va a\times(\va b\times \va c)]_i =&\epsilon_{ijk}a_j(\va b\times \va c)_k\\
				=&\epsilon_{ijk}a_j(\epsilon_{klm}b_lc_m)\\
				=&\epsilon_{kij}\epsilon_{klm}a_jb_lc_m\\
				=&(\delta_{il}\delta_{jm}-\delta_{im}\delta_{jl})a_jb_lc_m\\
				=&b_ia_mc_m-c_ia_lb_l\\
				=&b_i(\va a\cdot\va c)-c_i(\va a\cdot\va b)\\
				\va a\times(\va b\times\va c)=&\va b(\va a\cdot\va c)-\va c(\va a \cdot\va b)
			\end{align*}
		\item 
			\begin{align*}
				(\va a\times \va b)\cdot (\va c\times \va d)=&(\va a\times \va b)_i(\va c\times \va d)_i\\
				=&(\epsilon_{ijk}a_jb_k)(\epsilon_{ilm}c_ld_m)\\
				=&(\delta_{jl}\delta_{km}-\delta_{jm}\delta_{kl})a_jb_kc_ld_m\\
				=&a_lc_lb_md_m-a_md_mb_lc_l\\
				=&(\va a\cdot\va c)(\va b\cdot \va d)-(\va a\cdot\va d)(\va b\cdot\va c)
			\end{align*}
	\end{enumerate}
\item Doppel auftretende Indizes werden immer implizit summiert.
	\begin{enumerate}[label=(\roman*)]
		\item 
			\begin{align*}
				[\curl{\grad{F}}]_i=&\epsilon_{ijk}\partial_j \partial_k F\\
				=&-\epsilon_{ikj}\delta_j\delta_k F & \text{Indizes in }\epsilon\text{ vertauscht}\\
				=&-\epsilon_{ikj}\partial_k\partial_j F &\text{Satz von Schwarz}\\
				=&-\epsilon_{ijk}\partial_j\partial_k F & \text{Umnummerierung }j\leftrightarrow k\\
				=&-[\curl{\grad{F}}]_i\\
				[\curl{\grad{F}}]_i=&0&\forall i\\
				\curl{\grad{F}}=&\va 0
			\end{align*}
		\item 
			\begin{align*}
				\div{\curl{A}}=&\partial_i(\curl{A})_i\\
				=&\partial_i(\epsilon_{ijk}\partial_jA_k)\\
				=&\epsilon_{ijk}\partial_i\partial_j A_k\\
				=&0
			\end{align*}
			mit dem gleichen Argument, aber $k$ ist kein freier Index.
		\item 
			\begin{align*}
				[\curl{\curl{\va A}}]_i=& \epsilon_{ijk}\partial_j (\curl{\va A})_k\\
				=&\epsilon_{ijk}\partial_j\epsilon_{klm}\partial_l A_m\\
				=&\epsilon_{kij}\epsilon_{klm}\partial_j\partial_l A_m\\
				=&(\delta_{il}\delta_{jm}-\delta_{im}\delta_{jl})\partial_j\partial_l A_m\\
				=&\partial_m\partial_i A_m - \partial_l\partial_l A_i\\
				=& \partial_i\partial_m A_m - \partial_l \partial_l A_i\\
				\curl{\curl{A}}=&\grad{(\div{\va A})}-\laplacian{\va A}
			\end{align*}
		\item 
			\begin{align*}
				\div{F\va A}=&\partial_i(F A_i)\\
				=&A_i\partial_i F+F\partial_i A_i\\
				=&\va A\cdot\grad{F}+F\div{A}
			\end{align*}
	\end{enumerate}
	\end{parts}
\end{proof}
\begin{Problem}
\begin{parts}
\item Berechnen Sie das elektrische Feld $\vec E(\vec x)$ einer Punktladung $q$, die sich am Ort $\vec{x} = 0$ befindet.		
\item Berechnen Sie das elektrische Feld $\vec E(\vec x)$ einer homogen geladenen Kugel mit Radius $R$ und Gesamtladung $Q$, für Orte sowohl innerhalb als auch außerhalb der Kugel. 
\end{parts}
\end{Problem}

\begin{proof}
\begin{parts}
\item Wir betrachten $B_r(0)$. Aus Symmetrie ist $\vec E=E\vu{r}$. Nach dem Satz von Gauss gilt
\begin{align*}
	\frac{q}{\epsilon_0}=&\frac{1}{\epsilon_0}\iiint_{B_r(0)}\rho\dd{V}\\
	=& \iint_{\partial B_r(0)} \va{E}\cdot \dd{\va S}\\
	=& \iint_{\partial B_r(0)} E\dd{S}\\
	=& 4\pi r^2 E
\end{align*}
also
\[
\va{E}(r,\theta,\varphi)=\frac{q}{4\pi \epsilon_0 r^2}\vu{r}
.\] 
\item Es gilt
	\[
	\rho=\frac{Q}{\frac{4}{3}\pi R^3}
	.\]
	Ähnlich betrachten wir $B_r(0)$. Es gilt
	\begin{align*}
		\frac{1}{\epsilon_0}\iiint \rho\dd{V}=&\begin{cases}
			\frac{4}{3}\pi r^3 \rho & r < R\\
			Q & \text{sonst}.
		\end{cases}\\
		=&\begin{cases}
			\frac{r^3}{R^3}Q & r < R\\
			Q & \text{sonst.}
		\end{cases}\\
		=& \iint_{\partial B_r(0)} \va E\cdot \dd{\va S}\\
		=& \iint_{\partial B_r(0)}E\dd{S}\\ 
		=&4\pi r^2 E
	\end{align*}
	also
	\[
	\va E(r,\theta,\phi)=\frac{1}{4\pi r^2\epsilon_0}\begin{cases}
		\frac{r^3}{R^3}Q & r < R\\
		Q & \text{sonst.}
	\end{cases}
	.\qedhere\] 
\end{parts}
\end{proof}
