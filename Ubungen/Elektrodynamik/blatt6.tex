\begin{Problem}
	\textbf{Vektorpotential für ein homogenes Magnetfeld} Wir betrachten ein konstantes, homogenes Magnetfeld $\vec{B}$, das entlang der $z$-Richtung orientiert ist, d.h. $\vec{B} = (0, 0, B)$.
	\begin{parts}
	\item  Zeigen Sie, dass $\vec{A} = (0, Bx, 0)$ ein Vektorpotential für $\vec{B}$ ist.
	\item Finden Sie für $\vec{B}$ ein Vektorpotential $\vec{A}'$, das in $x$-Richtung orientiert ist. Geben Sie eine Eichtransformation an, die $\vec{A}'$ in $\vec{A}$ überführt.
	\item  Zeigen Sie, dass auch $\vec{A}^{\prime\prime} =\frac{1}{2}\vec{B}\times \vec{x}$ ein Vektorpotential für $\vec{B}$ ist. Geben Sie auch hier eine Eichtransformation an, die $\vec{A}^{\prime\prime}$ in $\vec{A}$ überführt.	
	\end{parts}
\end{Problem}

\begin{proof}
\begin{parts}
\item 
	\begin{align*}
		\curl{\vec{A}}=&\begin{vmatrix}
			\vu{e}_x & \vu{e}_y & \vu{e}_z\\
			\pdv{x} & \pdv{y} & \pdv{z}\\
			0 & Bx & 0
		\end{vmatrix}\\
		=&B\vu{e}_z
	\end{align*}
\item Durch eine direkte Rechnung
	\[
	\vec{A}'= -By\vu{e}_y
	.\] 
	Die gewünschte Eichtransformation ist definiert durch eine Funktion $f=-Bxy$, und die Eichtransformation
	\[
	\vec{A}'\mapsto \vec{A}' + \grad{f}
	.\] 
\item Die Rechnung ergibt das Potential
	\[
	\vec{A}^{\prime\prime}=\frac{1}{2}\begin{pmatrix} -By \\ Bx\\0 \end{pmatrix} 
	.\] 
	Aus $\vec{A}^{\prime\prime}=\frac{1}{2}(\vec{A}+\vec{A}')$ sieht man einfach, dass das ein Vektorpotential ist.
	\begin{align*}
		\curl{\vec{A}^{\prime\prime}}=&\frac{1}{2}\curl{\left( \vec{A}+\vec{A}' \right)}\\
		=&\frac{1}{2}\curl{\vec{A}}+\frac{1}{2}\curl{\vec{A}'}\\
		=&\frac{1}{2}\vec{B}+\frac{1}{2}\vec{B}\\
		=&\vec{B}
	\end{align*}
	Die Eichtransformation ist definiert durch $f=-Bxy / 2$.\qedhere 
\end{parts}	
\end{proof}
