\begin{Problem}
Man betrachte den Raum zwischen zwei unendlich großen, geerdeten Leiterplatten, die parallel zueinander an den Punkten $x = 0$ und $x = d$ aufgestellt sind. Eine weitere Platte mit Flächenladungsdichte $\sigma$ befindet sich bei $x = a$ mit $0 < a < d$.
\begin{parts}
\item Zeigen Sie, dass die Herleitung des Potentials $\varphi(x)$ f\"{u}r $0\le x \le d$ \"{a}quivalent ist zur Berechnung einer eindimensionalen Green’schen Funktion und somit zur Lösung der
\[
\Delta_x G(x,a)=-\delta(x-a),~\Delta_x=\pdv[2]{x}
,\]
mit den Dirichlet-Randbedingungen $G(0,a)=G(d,a)=0$.

Um die Lösung der folgenden Teilaufgaben zu vereinfachen, bietet es sich an, das Koordinatensystem so zu verschieben, dass die geladene Platte sich bei $x = 0$ und die geerdeten Leiterplatten sich bei $x = -a$ bzw. $x = d - a$ befinden.
\item Teilen Sie den Raum in zwei ladungsfreie Regionen $-a < x < 0$ und $0 < x < d - a$ auf, und lösen Sie dort die zwei separaten, homogenen Laplace-Gleichungen für das Potential. Integrieren Sie dann die Poisson-Gleichung von $x = -\epsilon$ bis $x = +\epsilon$, und betrachten Sie den Grenzwert $\epsilon\to 0$, um die Bedingungen zu finden, welche die zwei Lösungen für das Potential in $x = 0$ verbinden. Bestimme schließlich das Potential für den gesamten Bereich $-a < x < d - a$.
\item Bestätigen Sie das obige Resultat, indem Sie die Differenzialgleichung für das Potential unter Berücksichtigung der Randbedingungen bei $x = -a$ und $x = d - a$ direkt integrieren. 
\end{parts}
\end{Problem}

\begin{proof}
	\begin{parts}
	\item Die Ladungsverteilung ist 0 außer wenn $x=a$, also die Ladungsverteilung ist proportional zu $\delta(x-a)$. Die Definition der Greensche Funktion ist also proportional zu die Poisson-Gleichung.
	
		Die Greensche Funktion verschwindet genau dann wenn die Potential verschwindet, also bei $x=0$ und $x=d$.
	\item Die Lösungen in einer ladungsfreien Region ist
		\[
		\dv[2]{V}{x}=0\implies V = a x + b
		.\] 
		Der Gradient ist das elektrische Feld, also $\frac{\sigma}{2\epsilon_0}$. Die Lösungen sind also
		\begin{align*}
			-a<x<0:&~V = \frac{\sigma}{2\epsilon_0}(x+a)\\
			0<x<d-a:&V= \frac{\sigma}{2\epsilon_0}(d-a-x)
		\end{align*}
		Integriert zwischen $x=-\epsilon$ und $x=+\epsilon$ ergibt
\begin{align*}
	\dv[2]{V}{x}=& \delta(x)\\
	\int_{-\epsilon}^\epsilon \dv[2]{V}{x}\dd{x}=& \int_{-\epsilon}^{\epsilon} \delta(x)\dd{x}\\
	V'(\epsilon)-V'(-\epsilon) =& 1
\end{align*}
	\end{parts}
	Das Potential muss stetig sein, also das Potential ist
\end{proof}
