\begin{Problem}
Man betrachte den Raum zwischen zwei unendlich großen, geerdeten Leiterplatten, die parallel zueinander an den Punkten $x = 0$ und $x = d$ aufgestellt sind. Eine weitere Platte mit Flächenladungsdichte $\sigma$ befindet sich bei $x = a$ mit $0 < a < d$.
\begin{parts}
\item Zeigen Sie, dass die Herleitung des Potentials $\varphi(x)$ f\"{u}r $0\le x \le d$ \"{a}quivalent ist zur Berechnung einer eindimensionalen Green’schen Funktion und somit zur Lösung der
\[
\Delta_x G(x,a)=-\delta(x-a),~\Delta_x=\pdv[2]{x}
,\]
mit den Dirichlet-Randbedingungen $G(0,a)=G(d,a)=0$.

Um die Lösung der folgenden Teilaufgaben zu vereinfachen, bietet es sich an, das Koordinatensystem so zu verschieben, dass die geladene Platte sich bei $x = 0$ und die geerdeten Leiterplatten sich bei $x = -a$ bzw. $x = d - a$ befinden.
\item Teilen Sie den Raum in zwei ladungsfreie Regionen $-a < x < 0$ und $0 < x < d - a$ auf, und lösen Sie dort die zwei separaten, homogenen Laplace-Gleichungen für das Potential. Integrieren Sie dann die Poisson-Gleichung von $x = -\epsilon$ bis $x = +\epsilon$, und betrachten Sie den Grenzwert $\epsilon\to 0$, um die Bedingungen zu finden, welche die zwei Lösungen für das Potential in $x = 0$ verbinden. Bestimme schließlich das Potential für den gesamten Bereich $-a < x < d - a$.
\item Bestätigen Sie das obige Resultat, indem Sie die Differenzialgleichung für das Potential unter Berücksichtigung der Randbedingungen bei $x = -a$ und $x = d - a$ direkt integrieren. 
\end{parts}
\end{Problem}

\begin{proof}
	\begin{parts}
	\item Die Ladungsverteilung ist 0 außer wenn $x=a$, also die Ladungsverteilung ist proportional zu $\delta(x-a)$. Die Definition der Greensche Funktion ist also proportional zu die Poisson-Gleichung.
	
		Die Greensche Funktion verschwindet genau dann wenn die Potential verschwindet, also bei $x=0$ und $x=d$.
	\item Die Lösungen in einer ladungsfreien Region ist
		\[
		\dv[2]{V}{x}=0\implies V = a x + b
		.\] 
		Der Gradient ist das elektrische Feld, also $\frac{\sigma}{2\epsilon_0}$. Die Lösungen sind also
		\begin{align}
			-a<x<0:&~V = k_1(x+a)\label{eq:electrodynamics3-1}\\
			0<x<d-a:&V= k_2(d-a-x)\label{eq:electrodynamics3-2}
		\end{align}
		Integriert zwischen $x=-\epsilon$ und $x=+\epsilon$ ergibt
\begin{align*}
	\dv[2]{V}{x}=& \delta(x)\\
	\int_{-\epsilon}^x \dv[2]{V}{x}\dd{x}=& -\int_{-\epsilon}^{x} \delta(x)\dd{x}\\
	=& \begin{cases}
		0 & x < 0\\ 
		-1 & x \ge 0
	\end{cases}\\
		V'(x)=&V'(-\epsilon) - \begin{cases}
		0 & x < 0\\
		1 & x \ge 0
	\end{cases}\\
			V(\epsilon) =& V(-\epsilon) + \int_{-\epsilon}^\epsilon V'(x)\dd{x}\\
			=& V(-\epsilon) + 2\epsilon V'(-\epsilon) - \epsilon
\end{align*}
Eingesetzt in die vorherige Lösungen \eqref{eq:electrodynamics3-1} und \eqref{eq:electrodynamics3-2} liefert
\begin{align*}
	V'(-\epsilon) =& k_1\\
	V(-\epsilon) =& k_1 (a-\epsilon)\\
	V(\epsilon) =& k_1(a-\epsilon)+2\epsilon k_1 - \epsilon\\
	=&k_2(d-a-\epsilon) 
\end{align*}
Wir lösen die Gleichung für $k_2$ und erhalten
\[
k_2= \frac{k_1(a-\epsilon) + 2\epsilon k_1-\epsilon}{d-a-\epsilon}
.\] 
Die Lösungen sind also (außerhalb $(-\epsilon,\epsilon)$)
\[
V(x)=\begin{cases}
	k_1(x + a) & -a < x < -\epsilon,\\
	\frac{k_1(a-\epsilon)+2\epsilon k_1 - \epsilon}{d-a-\epsilon}(d-a-x) & \epsilon < x < d-a.
\end{cases}
\]
Wenn wir $\epsilon\to 0$ nehmen, ist
\[
V(x) = \begin{cases}
	k_1(x+a) & -a < x < 0, \\
	\frac{k_1}{d-a}(d-a-x) & 0 < x < d-a.
\end{cases}
\] 
\item Die Differentialgleichung ist
	\[
		\dv[2]{V}{x}=-\delta(x)
	.\] 
Mit Randbedingungen $V(-a)=V(d-a)=0$.
\begin{align*}
	\int_{-a}^x \dv[2]{V}{x}\dd{x}=&-\int_{-a}^x \delta(x)\dd{x}\\
	V'(x)-V'(-a)=&\begin{cases}
		-1 & x > 0\\
		0 & x\le 0
	\end{cases}\\
		V'(x)=&V'(-a)-\begin{cases}
			1 & x > 0\\
			0 & x \le 0
		\end{cases}\\
			V(x) - \cancelto{0}{V(-a)} =& \int_{-a}^x V'(x) \dd{x}\\
			=& V'(-a)(x+a) - x\theta(x)\\
			:=& k_1(x+a)-x\theta(x).
\end{align*}
Wir w
	\end{parts}
\end{proof}
