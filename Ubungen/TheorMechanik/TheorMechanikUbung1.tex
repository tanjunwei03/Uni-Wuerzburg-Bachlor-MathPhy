\documentclass[prb,12pt]{revtex4-2}

\usepackage{amsmath, amssymb,physics,amsfonts,amsthm}
\usepackage{enumitem}
\usepackage{cancel}
\usepackage{booktabs}
\usepackage{tikz}
\usepackage{hyperref}
\usepackage{enumitem}
\usepackage{transparent}
\usepackage{float}
\usepackage{multirow}
\newtheorem{Theorem}{Theorem}
\newtheorem{Proposition}{Theorem}
\newtheorem{Lemma}[Theorem]{Lemma}
\newtheorem{Corollary}[Theorem]{Corollary}
\newtheorem{Example}[Theorem]{Example}
\newtheorem{Remark}[Theorem]{Remark}
\theoremstyle{definition}
\newtheorem{Problem}{Problem}
\theoremstyle{definition}
\newtheorem{Definition}[Theorem]{Definition}
\newenvironment{parts}{\begin{enumerate}[label=(\alph*)]}{\end{enumerate}}
%tikz
\usetikzlibrary{patterns}
% definitions of number sets
\newcommand{\N}{\mathbb{N}}
\newcommand{\R}{\mathbb{R}}
\newcommand{\Z}{\mathbb{Z}}
\newcommand{\Q}{\mathbb{Q}}
\newcommand{\C}{\mathbb{C}}
\begin{document}
	\title{Theoretische Mechanik Hausaufgabenblatt Nr. 1}
	\author{Jun Wei Tan}
	\email{jun-wei.tan@stud-mail.uni-wuerzburg.de}
	\affiliation{Julius-Maximilians-Universit\"{a}t W\"{u}rzburg}
	\date{\today}
	\maketitle
\begin{Problem}
	Betrachten Sie den harmonischen Oszillator in einer Dimension, d. h. das Anfangswertproblem
	\begin{align*}
		m\dv[2]{x}{t}(t)=&F(x(t))=-kx(t)\\
		x(t_0)=&x_0\in \R\\
		\dv{x}{t}(t_0)=&v_0\in \R
	\end{align*}
\begin{enumerate}
	\item Zeigen Sie, daß wenn eine komplexwertige Funktion $z : I \to \C$ mit $t_0 \in I \subseteq \R$ die Differentialgleichung (1a) löst, ihr Realteil $x(t) = \Re z(t)$ zur Lösung des reellen Anfangswertproblems (1) benutzt werden kann.
	\item Was ist die allgemeinste Form der rechten Seite der Differentialgleichung (1a), für die der Realteil einer komplexen Lösung selbst eine Lösung ist? Geben Sie Gegenbeispiele an.
	\item Machen Sie den üblichen Exponentialansatz für lineare Differentialgleichungen mit konstanten Koeffizienten\ldots 
\end{enumerate}
\end{Problem}
\begin{proof}
	\begin{enumerate}
		\item Sei $x(t)=x_r(t)+i x_i(t), x_r, x_i: I\to\R$.

			Dann gilt

			 \[
				 m\left( \dv[2]{x_r}{t}+i\dv[2]{x_i}{t} \right) =-k(x_r+ix_i)
			.\] 
			Weil das eine Gleichung von zwei komplexe Zahlen ist, gilt auch
			\[
				m\dv[2]{x_r}{t}=-kx_r
			.\] 
		\item Das passt f\"{u}r alle reelle lineare Kombinationen der Ableitungen von $x(t)$.

			\[
				\sum_{i=0}^n a_i\dv[i]{x}{t}=0, \qquad a_i\in \R
			.\] 
			\paragraph{Gegenbeispiele}
			\begin{enumerate}[label=(\roman*)]
				\item Irgendeine $a_i\not\in \R$

\[
	m\dv[2]{x}{t}=-ikx(t), \qquad k\in \R
.\] 
Hier ist es klar, dass \emph{keine} Abbildung $x:\R\to \R$ eine L\"{o}sung sein kann, weil die linke Seite reelle wird, aber die rechte Seite nicht reelle wird.

Daraus folgt: Das Realteil der L\"{o}sung ist kein L\"{o}sung.

\item Nichtlineare Gleichung, z.B.

\[
	m\dv[2]{x}{t}=-k\left( \dv{x}{t} \right) ^2
.\] 
			\end{enumerate}

		\item 
			\begin{gather*}
				x(t)=\alpha e^{\lambda t}\\
				\ddot{x}(t)=\lambda^2 \alpha e^{\lambda t}
			\end{gather*}
			Dann
			\begin{gather*}
				m\cancel\alpha\lambda^2 \cancel{e^{\lambda t}}=-k\cancel\alpha \cancel{e^{\lambda t}}\\
				\lambda^2=-\frac{k}{m}\\
				\lambda=\pm i\sqrt{\frac{k}{m}}=\pm i\omega \qquad \omega :=\sqrt{\frac{k}{m}}    
			\end{gather*}
			Daraus folgt, f\"{u}r $z_1(t)$:
\begin{align*}
	z_1(0)=&\alpha_{1,+}+\alpha_{1,-}=x_0\\
	z_1'(0)=&-i\omega \alpha_{1,+}+i\omega \alpha_{1,-}=v_0\\
		&-\alpha_{1,+}+\alpha_{1,-}=-\frac{iv_0}{\omega}\\
	2\alpha_{1,-}=&x_0-\frac{iv_0}{\omega}\\
	2\alpha_{1,+}=&x_0+\frac{iv_0}{\omega}\\
	z_1(t)=& \frac{1}{2}\left[ \left( x_0+\frac{iv_0}{\omega} \right) e^{-i\omega t}+\left( x_0-\frac{iv_0}{\omega} \right) e^{i\omega t} \right] 
\end{align*}
Daraus folgt die andere Formen der L\"{o}sungen:
\begin{enumerate}[label=(\roman*)]
	\item $x_2(t)$ 
		\begin{align*}
			&\frac{1}{2}\left[ \left( x_0+\frac{iv_0}{\omega} \right)e^{-i\omega t}+\left( x_0-\frac{iv_0}{\omega} \right) e^{i\omega t}  \right] \\
			&= \Re\left[ \left( x_0+\frac{iv_0}{\omega} \right) e^{-i\omega t} \right] \\
			&=\Re\left[ \left( x_0+\frac{iv_0}{\omega} \right) \left( \cos(\omega t)-i\sin(\omega t) \right)  \right] \\
			&=\Re\left[ x_0\cos\omega t+\frac{v_0}{\omega}\sin\omega t+i(\dots) \right] \\
			&= x_0\cos\omega t+\frac{v_0}{\omega}\sin\omega t
		\end{align*}
	\item $x_3(t)$ 
		(R-Formula)
		\begin{gather*}
			x_0\cos\omega t+\frac{v_0}{\omega}\sin\omega t=\alpha_3 \sin(\omega t+\delta_3)\\
			\alpha_3=\sqrt{x_0^2+\left( \frac{v_0}{\omega} \right) ^2} \\
			\delta_3=\arctan \frac{v_0}{x_0\omega}
		\end{gather*}
	\item $x_4(t)$ 
		\[
		\sin\left( x+\frac{\pi}{2} \right) =\cos x
		.\] 
		Daraus folgt:
		\[
		\alpha_4=\alpha_3 \qquad \delta_4=\delta_3+\frac{\pi}{2}
		.\] 
\end{enumerate}
	\end{enumerate}
\end{proof}

\begin{Problem}
	\ldots	
\end{Problem}

\begin{proof}
	\begin{enumerate}
	\item 	
	\begin{gather*}
		x(t)=\alpha e^{\lambda t}\\
		\dot{x}(t)=\alpha\lambda e^{\lambda t}\\
		\ddot{x}(t)=\alpha\lambda^2e^{\lambda t}
	\end{gather*}
Daraus folgt
\begin{align*}
	m\lambda^2\cancel\alpha\cancel{e^{\lambda t}}=&-k\cancel{\alpha e^{\lambda t}}-2m\gamma\lambda\cancel{\alpha e^{\lambda t}}\\
	0=&m\lambda^2+2m\gamma\lambda+k\\
	\lambda=&-\gamma\pm \sqrt{\gamma^2-\frac{k}{m}}
\end{align*}
Falls $\gamma^2 \neq \frac{k}{m}$ :
\[
	x(t)=e^{-\gamma t}\left[ Ae^{\sqrt{\gamma^2-\frac{k}{m}} t}+Be^{-\sqrt{\gamma^2-\frac{k}{m}} t} \right] 
,\] 
\begin{align*}
	x'(t)=&-\gamma e^{-\gamma t}\left[ Ae^{\sqrt{\gamma^2-\frac{k}{m}} t}+Be^{-\sqrt{\gamma^2-\frac{k}{m}} t} \right] \\
	      &+e^{-\gamma t}\left[ A\sqrt{\gamma^2-\frac{k}{m}} e^{\sqrt{\gamma^2-\frac{k}{m}} t}-B\sqrt{\gamma^2-\frac{k}{m}} e^{-\sqrt{\gamma^2-\frac{k}{m}} t} \right] 
\end{align*}
und
\begin{gather*}
	x(0)=A+B=x_0\\
	x'(0)=\sqrt{\gamma^2-\frac{k}{m}} \left( A-B \right)=v_0\\
	2A=x_0+\frac{v_0}{\sqrt{\gamma^2-\frac{k}{m}} }\\
	2B=x_0-\frac{v_0}{\sqrt{\gamma^2-\frac{k}{m}} }
\end{gather*}
Es ist zu beachten, dass es möglich ist, dass $\gamma^2<\frac{k}{m}$. In diesem Fall ist $\sqrt{\gamma^2-\frac{k}{m}} =i\sqrt{\frac{k}{m}-\gamma^2} $, aber der Form der L\"{o}sung bleibt.

F\"{u}r $\gamma^2=\frac{k}{m}$ ist die L\"{o}sung
\[
	x(t)=Ae^{-\gamma t}+Bte^{-\gamma t}
.\] 
Es gilt
\[
	x'(t)=-\gamma Ae^{-\gamma t}+Be^{-\gamma t}-Bt\gamma e^{-\gamma t}
.\]
Dann
\begin{align*}
	x(0)=& A=x_0\\
	x'(0)=& -\gamma A+B=v_0\\
	B=& v_0+\gamma x_0\\
	x(t)=& x_0e^{-\gamma t}+(v_0+\gamma x_0)te^{-\gamma t}
\end{align*}
\item Wir suchen eine Partikularl\"{o}sung f\"{u}r die Gleichung
	\[
		m\dv[2]{x}{t}+2m\gamma\dv{x}{t}+kx=F_0e^{-i\omega_0t}
	\]
	mit dem Form
	\[
		x(t)=Ae^{-i\omega_0t}
	.\] 
	Es gilt
	\begin{align*}
		x'(t)=&-i\omega_0 Ae^{-i\omega_0 t} \\
		x''(t)=&-\omega_0^2 Ae^{-i\omega_0 t}
	\end{align*}
	Dann ist
	\[
		-\omega_0^2 Ame^{-i\omega_0t}-2m\gamma i\omega_0 Ae^{-i\omega_0 t}+Ake^{-i\omega_0 t}=F_0e^{-i\omega_0t}
	,\]
	\[
	A=\frac{F_0}{-m\omega_0^2-2m\gamma i\omega_0+k}
	.\] 
\item 
\end{enumerate}
\end{proof}
\end{document}
