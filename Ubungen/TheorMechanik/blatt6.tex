\begin{Problem}
	Betrachten Sie das Potential des sphärischen harmonischen Oszillators mit einer kleinen Störung $\delta V$:
	\[
	V(\va x)=\frac{1}{2}m\omega^2|\va x|^2+\delta V(\va x)
	.\] 
	\begin{enumerate}
		\item Es sei
			 \[
			\delta V(\va x)=\lambda_1x_3
			.\] 
Formulieren und lösen Sie die entkoppelten Bewegungsgleichungen für die kartesischen Koordinaten. Was ist die geometrische Interpretation der Lösung? Kann man dies schon aus dem Potentials (1) ablesen?
\item Es sei nun
	\[
	\delta V(\va x)=\lambda_2 x_3^2
	.\] 
Formulieren und lösen Sie auch hier die die entkoppelten Bewegungs- gleichungen für die kartesischen Koordinaten. Was erwarten Sie anschaulich für die Trajektorien im Vergleich zu den elliptischen Bahnen des ungestörten Problems?
	\end{enumerate}
\end{Problem}

\begin{proof}
	\begin{enumerate}
		\item $L=\frac{1}{2}m|\dot{\va x}|^2-\frac{1}{2}m\omega^2|\va x|^2+\delta V(\va x)$. Es folgt
\[
	m\ddot{\va x}=m\omega^2\va x+\lambda_1\vu{e}_3
.\] 
	\end{enumerate}
\end{proof}
