\documentclass[prb,12pt]{revtex4-2}

\usepackage{amsmath, amssymb,physics,amsfonts,amsthm}
\usepackage{enumitem}
\usepackage{cancel}
\usepackage{booktabs}
\usepackage{tikz}
\usepackage{hyperref}
\usepackage{enumitem}
\usepackage{transparent}
\usepackage{float}
\usepackage{xtab}
\usepackage{multirow}
\newtheorem{Theorem}{Theorem}
\newtheorem{Proposition}{Theorem}
\newtheorem{Lemma}[Theorem]{Lemma}
\newtheorem{Corollary}[Theorem]{Corollary}
\newtheorem{Example}[Theorem]{Example}
\newtheorem{Remark}[Theorem]{Remark}
\theoremstyle{definition}
\newtheorem{Problem}{Problem}
\theoremstyle{definition}
\newtheorem{Definition}[Theorem]{Definition}
\newenvironment{parts}{\begin{enumerate}[label=(\alph*)]}{\end{enumerate}}
%tikz
\usetikzlibrary{patterns}
% definitions of number sets
\newcommand{\N}{\mathbb{N}}
\newcommand{\R}{\mathbb{R}}
\newcommand{\Z}{\mathbb{Z}}
\newcommand{\Q}{\mathbb{Q}}
\newcommand{\C}{\mathbb{C}}
\renewcommand*{\proofname}{Solution}
\begin{document}
	\title{Klassische Physik 1 Hausaufgabenblatt Nr. 1}
	\author{Jun Wei Tan}
	\email{jun-wei.tan@stud-mail.uni-wuerzburg.de}
	\affiliation{Julius-Maximilians-Universit\"{a}t W\"{u}rzburg}
	\date{\today}
	\maketitle

\begin{Problem}
	\begin{parts}
		\item Man berechne den Abstand der Punkte $P (8| - 2| - 4)$ und $Q(5|4| - 6)$
		\item Berechnen Sie den Fl\"{a}cheninhalt des durch die Punkte $P_1(2|3|1)$, $P_2(0|2|3)$ und $P_3(1|2|2)$ gegebenen Dreiecks.
		\item Zeigen Sie, dass gilt $(|\va a|=a$ usw.): $\qquad \left| \va a \cdot \va b \right| \le a \cdot b$.
	\end{parts}
\end{Problem}
\begin{proof}
	\begin{parts}
		\item
	\begin{align*}
		d(P, Q)=&\sqrt{(8-5)^2+(-2-4)^2+(-4+6)^2}\\
		=& \sqrt{3^2+6^2+2^2}\\
		=&\sqrt{49} =7
	\end{align*}
\item
	{\allowdisplaybreaks
	\begin{align*}
		P_2-P_1=&\begin{pmatrix} -2 \\ -1 \\ 2 \end{pmatrix} \\
		P_3-P_1=& \begin{pmatrix} -1 \\ -1 \\ 1 \end{pmatrix}\\
		(P_2-P_1)\times (P_3-P_1)=&\begin{bmatrix} \vu i & \vu j & \vu k \\ -2 & -1 & 2 \\ -1 & -1 & 1 \end{bmatrix} \\
		=& \begin{pmatrix} 1 \\ 0 \\ 1 \end{pmatrix}\\
		|(P_2-P_1)\times (P_3-P_1)|=& \sqrt{1^2+0^2+1^2} =\sqrt{2}\\
		A=& \frac{\sqrt{2} }{2}
\end{align*}}
\item $|\va a \cdot \va b|=|ab\cos\theta|=ab|\cos\theta| \le ab$. \qedhere
	\end{parts}
\end{proof}

\begin{Problem}
	Auf der A3 zwischen dem Kreuz Biebelried und der Abfahrt Randersacker ist ein 4 km langer Stau vor einer Baustelle.
	\begin{parts}
	\item Wie viele Autos stehen ungef\"{a}hr in diesem Stau?	
	\item Wie viele Personen befinden sich in diesem Stau?
	\end{parts}
\end{Problem}
\begin{proof}
	\begin{parts}
	\item Ein Auto ist ungef\"{a}hr $2$ m lang, und es gibt daher $\frac{4\text{ km}}{2\text{ m}}=2000$ Autos in ein $4\text{ km}$ lange Gerade. Aber es gibt ungef\"{a}hr 6 Spuren, und die Anzahl von Autos ist ungef\"{a}hr $2000\times 6 = 12000$ Autos im Stau.
	\item Es gibt wahrscheinlich durchschnittlich $2$ Personen pro Auto, also $24000$ Personen im Stau. \qedhere
	\end{parts}
\end{proof}
\begin{Problem}
	Eine Familie macht einen Sonntagsspaziergang. Auf dem R\"{u}ckweg beschließt das Kind 2,0 km entfernt vom Haus mit dem Fahrrad vorauszufahren. Die Eltern spazieren gem\"{u}tlich mit einer konstanten Geschwindigkeit von 3,0 km/h in Richtung Haus. Das Kind f\"{a}hrt mit einer konstanten Geschwindigkeit von 8,0 km/h nach Hause. Dort angekommen dreht es um und f\"{a}hrt zur\"{u}ck zu seinen Eltern. Wenn es auf diese trifft, dreht es wieder um und f\"{a}hrt wieder zum Haus zur¨uck usw. bis die Eltern schließlich auch zu Hause ankommen.
	\begin{parts}
	\item In welcher Entfernung zum Haus treffen die Eltern das Kind das 1. Mal, 2. Mal, 3. Mal nach dessen Losfahren wieder? (Skizze!)
	\item Welche Gesamtstrecke hat das Kind am Schluss (wenn alle zu Hause angekommen sind) zur\"{u}ckgelegt?
	\end{parts}
\end{Problem}
\begin{proof}
Sei
\begin{align*}
	d=&2\text{ km}\\
	v_1=&3\text{ kmh}^{-1}\\
	v_2=&8\text{ kmh}^{_1}
\end{align*}
	\begin{parts}
	\item 		Sei $t_n$ die Zeit, die die Eltern und das Kind sich zum $n$. Mal treffen. Die Eltern sind $x_n=d-v_1t_n$ vom Haus entfernt. Wann treffen sie sich zum nächsten Mal ($t_{n+1}$)? Es gilt:
	\begin{align*}
		x_{n+1}=& d-v_1t_{n+1}\\
		x_{n+1}+x_n=& v_2(t_{n+1}-t_n)\\
		\underbrace{d-v_1t_{n+1}}_{x_{n+1}}+\underbrace{d-v_1t_n}_{x_n}=&v_2(t_{n+1}-t_n)\\
		2d-v_1(t_{n+1}+t_n)=& v_2(t_{n+1}-t_n)\\
		(v_2+v_1)t_{n+1}+(v_1-v_2)t_n=&2d\\
		t_{n+1}=&\frac{2d}{v_1+v_2}+\frac{v_2-v_1}{v_1+v_2}t_n
	\end{align*}
	Sei dann $t_0=0,a=\frac{v_2-v_1}{v_1+v_2},b=\frac{2d}{v_1+v_2}$ und die L\"{o}sung ist:
	\[
	t_n=\frac{(1-a^n)b}{1-a}+C, \qquad C\in \R
	.\] 
	Weil $t_0=0$, ist $C=0$. Numerisch ist $a=5 / 11\approx 0.455$ und $b\approx 0.634\text{ h}$. Daraus folgt die erste $t_n$ und auch $x_n=d-v_1t_n$:

	\begin{center}
		\begin{tabular}{p{0.1\textwidth}p{0.2\textwidth}p{0.2\textwidth}}
			\toprule
			$n$ & $t_n\text{ (h)}$ & $x_n\text{ (km)}$\\\midrule
0 & 0 & 2.00 \\\midrule
1 & 0.364 & 0.909 \\\midrule
2 & 0.529 & 0.413 \\\midrule
3 & 0.604 & 0.188 \\\midrule
4 & 0.638 & 0.0854 \\\midrule
5 & 0.654 & 0.0388 \\\midrule
6 & 0.661 & 0.0176 \\\bottomrule
		\end{tabular}
	\end{center}
\item $d=\frac{2\text{ km}}{3\text{ kmh}^{-1}}8\text{ kmh}^{-1}\approx 5.33\text{ km}$
	\end{parts}
\end{proof}
\end{document}
