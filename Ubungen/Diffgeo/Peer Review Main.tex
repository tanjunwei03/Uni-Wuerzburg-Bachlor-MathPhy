\documentclass[prb,12pt]{revtex4-2}
%fonts
% Palatino for main text and math
\usepackage[osf,sc]{mathpazo}
% Helvetica for sans serif
% (scaled to match size of Palatino)
\usepackage[scaled=0.90]{helvet}

% Bera Mono for monospaced
% (scaled to match size of Palatino)
\usepackage[scaled=0.85]{beramono}
\usepackage{amsmath, amssymb,physics,amsfonts,amsthm}
\usepackage{enumitem}
\usepackage{cancel}
\usepackage[most]{tcolorbox}
\usepackage{booktabs}
\usepackage{tikz}
\usepackage{hyperref}
\usepackage{enumitem}
\usepackage{float}
\usepackage{hwemoji}
\usepackage{multirow}
\newtheorem{Theorem}{Theorem}
\newtheorem{Proposition}{Proposition}
\newtheorem{Lemma}[Theorem]{Lemma}
\newtheorem{Corollary}[Theorem]{Corollary}
\newtheorem{Example}[Theorem]{Example}
\newtheorem{Remark}[Theorem]{Remark}
\theoremstyle{definition}
\newtheorem{Problem}{Aufgabe}
\theoremstyle{definition}
\newtheorem{Definition}[Theorem]{Definition}
\newenvironment{parts}{\begin{enumerate}[label=(\alph*)]}{\end{enumerate}}
%tikz
\usetikzlibrary{patterns}
\usepackage{tikz-cd}
% definitions of number sets
\newcommand{\N}{\mathbb{N}}
\newcommand{\R}{\mathbb{R}}
\newcommand{\Z}{\mathbb{Z}}
\newcommand{\Q}{\mathbb{Q}}
\newcommand{\C}{\mathbb{C}}
\allowdisplaybreaks

\renewcommand\qedsymbol{🎉}
\newcommand\contradiction{💀}
\renewcommand\proofname{Beweis}
\begin{document}
	\title{Einf\"{u}hrung in die Differentialgeometrie Hausaufgaben Blatt Nr. 5}
	\author{Max Mustermann}
	\affiliation{Julius-Maximilians-Universit\"{a}t W\"{u}rzburg}
	\date{\today}
	\maketitle

%\begin{Problem}
Seien $U,~V,~W$ endlich-dimensionale Vektorräume über $\R$ und sei $B : U \times V \to W$ eine $\R$-bilineare Abbildung. Ferner sei $I \subseteq R$ ein nicht-triviales Interval und seien $f : I \to U$ und $g : I \to V$ stetig differenzierbare Abbildungen. Betrachten Sie die Abbildung $B(f, g): I \to W$, die folgendermaßen definiert ist:	
\[
B(f,g)(t):=B(f(t),g(t)),~t\in I
.\] 
Zeigen Sie: Die Abbildung $B(f, g) : I \to W$ ist ebenfalls stetig differenzierbar, und es gilt die \emph{Produktregel}
\[
B(f,g)'(t)=B(f'(t),g(t))+B(f(t),g'(t)),~\forall t\in I
.\] 
\end{Problem}
\begin{proof}
Per Definition gilt
\begin{align*}
	&\lim_{\delta t \to 0} \frac{B(f,g)(t+\delta t)-B(f,g)(t)}{\delta t}\\
	=&\lim_{\delta t \to 0} \frac{B(f(t+\delta t), g(t+\delta t))-B(f(t),g(t))}{\delta t}\\
	=&\lim_{\delta t \to 0}\frac{1}{\delta t}[B(f(t+\delta t),g(t+\delta t))+B(f(t+\delta t), g(t))\\
	&-B(f(t+\delta t), g(t)) - B(f(t),g(t))] \\
	=&\lim_{\delta t \to 0} \frac{1}{\delta t}[B(f(t+\delta t), g(t+\delta t)-g(t))\\
	&+B(f(t+\delta t)-f(t), g(t))]\\
	=&\lim_{\delta t \to 0} \left[ B\left( f(t+\delta t), \frac{g(t+\delta t)-g(t)}{\delta t} \right)\right.\\
	&+\left. B\left( \frac{f(t+\delta t)-f(t)}{\delta t}, g(t) \right) \right]\\
	=& B(f'(t),g(t))+B(f(t),g'(t)).\qedhere
\end{align*}
\end{proof}

%\begin{Problem}
	\begin{parts}
	\item Sei $-\infty\le a < b\le \infty$ und seien $r:(\alpha,\beta)\to [0,\infty)$ und $\varphi: (\alpha,\beta)\to \R$ glatte Funktionen. Leiten Sie eine Formel für die Bogenlänge der in Polarkoordinaten parametrisierten Kurve $\gamma:(\alpha,\beta)\to \R^2$,
	\[
	\gamma(t):=(r(t)\cos\varphi(t), r(t)\sin\varphi(t)),\qquad t\in (\alpha,\beta)
	\]
	auf dem Intervall $[a,b]$ her, wobei $\alpha<a<b<\beta$.
\item Berechnen Sie die Bogenlänge der archimedischen Spirale $\gamma:(0,\infty)\to \R^2$ mit
	\[
	\gamma(t):=(t\cos t, t\sin t),\qquad t\in (0,\infty)
	\]
	auf dem Intervall $[\pi,2\pi]$. Skizzieren Sie die Kurve $\gamma$ (oder zeichnen Sie mit Geogebra).
\end{parts}
\end{Problem}
\begin{proof}
	\begin{parts}
	\item 
		\begin{align*}
			\gamma'(t)=& \begin{pmatrix} r'(t)\cos\varphi(t)-r(t)\varphi'(t)\sin\varphi(t) \\ r'(t)\sin\varphi(t) + r(t)\varphi'(t)\cos\varphi(t) \end{pmatrix}\\
				\|\gamma'(t)\|=&\sqrt{r'(t)^2+r(t)^2\varphi'(t)^2}\\
				l=&\int_a^b \sqrt{r'(t)^2+r(t)^2\varphi'(t)^2} \dd{t}
		\end{align*}
	\item Wir idenfizieren $r(t)=t~\varphi(t)=t$. Damit ist
		\begin{align*}
			l=&\int_\pi^{2\pi} \sqrt{1^2+t^2} \dd{t}\\
			t=&\sinh u~\dd{t}=\cosh u\dd{u}\\
			l=&\int_{\sinh^{-1}\pi}^{\sinh^{-1}2\pi}\cosh^2 u \dd{u}\\
			=&\int_{\sinh^{-1}\pi}^{\sinh^{-1}2\pi}\frac{1}{4}\left( e^u + e^{-u} \right)^2 \dd{u}\\
			=&\frac{1}{4}\left[ \frac{1}{2}e^{2u}-\frac{1}{2}e^{-2u}+2u \right]_{\sinh^{-1}\pi}^{\sinh^{-1}2\pi}\\
			=&\frac{1}{4}\left[\sinh(2u)+2u \right]_{\sinh^{-1}\pi}^{\sinh^{-1}2\pi}\\
			=&\frac{1}{2}\left[ t(1+t^2)+\ln\left( t+\sqrt{1+t^2}  \right)  \right]_{t=\pi}^{t=2\pi}.\qedhere
		\end{align*}
	\end{parts}
\end{proof}

\begin{Problem}
	\begin{parts}
	\item Sei $r > 0$ und $I \subseteq \R$ ein nicht-leeres offenes Intervall. Bezeichne $rI := \{rs : s \in I\}$. Seien $\alpha : I \to \R^2$, $\beta : rI \to \R^2$ glatte und reguläre, nach der Bogenlänge parametrisierte Kurven. Wir nehmen an, es gelte $\kappa_\beta(rs) = \frac{1}{r}\kappa_\alpha(s)$, für alle $s \in I$. Zeigen Sie, dass $\alpha$ und $\beta$ im folgenden Sinne ähnlich sind: Es gibt eine orientierungserhaltende Isometrie $F : \R^2 \to \R^2$ mit $r\alpha(s) = F(\beta(rs))$, für alle $s \in I$. (In dem Fall heißt $r$ manchmal Streckfaktor.)
	\item Für $a > 0$ bezeichne $\gamma_a : \R \to \R^2$ die nach Bogenlänge parametrisierte \emph{Klothoide} mit Krümmungsfunktion $\kappa_a(s) := \frac{s}{a^2},~s \in \R$. Seien $a, b > 0$ beliebig. Zeigen Sie, dass $\gamma_a$ und $\gamma_b$ im Sinne von (a) ähnlich sind, und bestimmen Sie den zugehörigen Streckfaktor $r$. 
\end{parts}
\end{Problem}
\begin{proof}
	\begin{parts}
	\item Wir bezeichnen $I=[a,b]$. Zunächst wählen wir $v\in \R^2$, sodass $t_v(\beta(ra))=r\alpha(a)$. Per Voraussetzung gilt
	\begin{align*}
		\kappa_\beta(rs)=&\beta''(rs)\cdot J\beta'(rs)\\
		\kappa_\alpha(s)=&\alpha''(s)\cdot J\alpha'(s)\\
		\|\beta''(rs)\|=&\frac{1}{r}\|\alpha''(s)\|
	\end{align*}
	Das heißt: Es gibt eine Matrix $U\in O(3,\R)$ sodass
	\[
	U\beta''(rs)=\frac{1}{r}\alpha''(s)
	.\] 
	Jetzt integrieren wir die Gleichung bzgl $s$:
	\begin{align*}
		\int_a^x U\beta''(rs)\dd{s}=& \int_{ra}^{xa} \frac{1}{r}\beta''(rs)\dd{(rs)}\\
		=&\frac{1}{r}(U\beta'(rx) - U\beta'(ra))\\
		=&\frac{1}{r}\int_a^x \alpha''(s)\dd{s}\\
		=& \frac{1}{r}(\alpha'(x)-\alpha'(a))
	\end{align*}
	Diese Gleichung integrieren wir noch einmal
	\begin{align*}
		\int_a^x U(\beta'(rs)-\beta'(ra))\dd{s}=&\frac{1}{r}\int_{ra}^{rx} (\beta'(rs)-\beta'(ra))\dd{(rs)}\\
		=&\frac{1}{r}U[\beta(rx)-\beta(ra)-\beta'(ra)(rx-ra)]\\
		=&\frac{1}{r}\int_a^x (\alpha'(s)-\alpha'(a))\dd{s}\\
		=&\alpha(x)-\alpha(a)-\alpha'(a)(x-a)
	\end{align*}
	also
	\[
	U\beta(rs)-U\beta(ra)-rU\beta'(ra)(s-a)=r\alpha(s)-r\alpha(a)-r\alpha'(a)(s-a)
	\] 
	oder
	\[
	r\alpha(s)=U\beta(rs) + \left[ r\alpha(a)-U\beta(ra) \right]+r(\alpha'(a)-\beta'(ra))(s-a)
	.\] 
\item Es gilt
	\begin{align*}
		\kappa_a(s)=&\frac{s}{a^2}\\
		\kappa_b(s)=&\frac{s}{b^2}
	\end{align*}
	daraus folgt
	\[
	\kappa_b(s)=\frac{a^2}{b^2}\kappa_a(s)
	,\]
	also $r=a^2 / b^2$.\qedhere
\end{parts}
\end{proof}

\end{document}
