\documentclass[prb,12pt]{revtex4-2}
%fonts
% Palatino for main text and math
\usepackage[osf,sc]{mathpazo}

% Helvetica for sans serif
% (scaled to match size of Palatino)
\usepackage[scaled=0.90]{helvet}

% Bera Mono for monospaced
% (scaled to match size of Palatino)
\usepackage[scaled=0.85]{beramono}
\usepackage{amsmath, amssymb,physics,amsfonts,amsthm}
\usepackage{enumitem}
\usepackage{cancel}
\usepackage[most]{tcolorbox}
\usepackage{booktabs}
\usepackage{tikz}
\usepackage{hyperref}
\usepackage{enumitem}
\usepackage{float}
\usepackage{hwemoji}
\usepackage{multirow}
\newtheorem{Theorem}{Theorem}
\newtheorem{Proposition}{Proposition}
\newtheorem{Lemma}[Theorem]{Lemma}
\newtheorem{Corollary}[Theorem]{Corollary}
\newtheorem{Example}[Theorem]{Example}
\newtheorem{Remark}[Theorem]{Remark}
\theoremstyle{definition}
\newtheorem{Problem}{Aufgabe}
\theoremstyle{definition}
\newtheorem{Definition}[Theorem]{Definition}
\newenvironment{parts}{\begin{enumerate}[label=(\alph*)]}{\end{enumerate}}
%tikz
\usetikzlibrary{patterns}
\usepackage{tikz-cd}
% definitions of number sets
\newcommand{\N}{\mathbb{N}}
\newcommand{\R}{\mathbb{R}}
\newcommand{\Z}{\mathbb{Z}}
\newcommand{\Q}{\mathbb{Q}}
\newcommand{\C}{\mathbb{C}}
\allowdisplaybreaks

\renewcommand\qedsymbol{🎉}
\newcommand\contradiction{💀}
\renewcommand\proofname{Beweis}
\begin{document}
	\title{Einf\"{u}rung in die Differentialgeometrie Hausaufgaben Blatt Nr. 1}
	\author{Max Mustermann}
	\affiliation{Julius-Maximilians-Universit\"{a}t W\"{u}rzburg}
	\date{\today}
	\maketitle

\begin{Problem}
Seien $U,~V,~W$ endlich-dimensionale Vektorräume über $\R$ und sei $B : U \times V \to W$ eine $\R$-bilineare Abbildung. Ferner sei $I \subseteq R$ ein nicht-triviales Interval und seien $f : I \to U$ und $g : I \to V$ stetig differenzierbare Abbildungen. Betrachten Sie die Abbildung $B(f, g): I \to W$, die folgendermaßen definiert ist:	
\[
B(f,g)(t):=B(f(t),g(t)),~t\in I
.\] 
Zeigen Sie: Die Abbildung $B(f, g) : I \to W$ ist ebenfalls stetig differenzierbar, und es gilt die \emph{Produktregel}
\[
B(f,g)'(t)=B(f'(t),g(t))+B(f(t),g'(t)),~\forall t\in I
.\] 
\end{Problem}
\begin{proof}
Per Definition gilt
\begin{align*}
	&\lim_{\delta t \to 0} \frac{B(f,g)(t+\delta t)-B(f,g)(t)}{\delta t}\\
	=&\lim_{\delta t \to 0} \frac{B(f(t+\delta t), g(t+\delta t))-B(f(t),g(t))}{\delta t}\\
	=&\lim_{\delta t \to 0}\frac{1}{\delta t}[B(f(t+\delta t),g(t+\delta t))+B(f(t+\delta t), g(t))\\
	&-B(f(t+\delta t), g(t)) - B(f(t),g(t))] \\
	=&\lim_{\delta t \to 0} \frac{1}{\delta t}[B(f(t+\delta t), g(t+\delta t)-g(t))\\
	&+B(f(t+\delta t)-f(t), g(t))]\\
	=&\lim_{\delta t \to 0} \left[ B\left( f(t+\delta t), \frac{g(t+\delta t)-g(t)}{\delta t} \right)\right.\\
	&+\left. B\left( \frac{f(t+\delta t)-f(t)}{\delta t}, g(t) \right) \right]\\
	=& B(f'(t),g(t))+B(f(t),g'(t)).\qedhere
\end{align*}
\end{proof}


\end{document}
