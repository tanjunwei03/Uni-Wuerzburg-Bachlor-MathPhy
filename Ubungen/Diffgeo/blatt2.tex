\begin{Problem}
In dieser Aufgabe befassen wir uns mit der Matrixdarstellung von Bilinearformen. (Im Anhang zum Vorlesungsskript findet man einen kurzen Überblick über einige Grundlagen zu Matrizen.)	

Seien $V,W$ endlich-dimensionale Vektorr\"{a}ume \"{u}ber einem K\"{o}rper $K$ mit $\text{dim}_K V=m>0$ und $\text{dim}_K W=n>0$. Ferner sei $E:V\times W\to K$ eine Bilinearform.
\begin{parts}
\item Sei $\mathcal{B}$ bzw. $\mathcal{C}$ eine (geordnete) Basis f\"{u}r $V$ bzw. $W$. Zeigen Sie, dass es eine eindeutige $(m\times n)-$Matrix $A$ mit Eintr\"{a}gen aus $K$ gibt, sodass
	\[
	E(v,w)=([v]_\mathcal{B}^t)A[w]_{\mathcal{C}}
	,\qquad\text{f\"{u}r alle }v\in V\text{ und }w\in W.\] 
	Geben Sie explizit die Eintr\"{a}ge von $A$ in Abh\"{a}ngigkeit von $E$ sowie $\mathcal{B}$ und $\mathcal{C}$. Wir bezeichnen $[E]_{\mathcal{B},\mathcal{C}}:=A$.
\item Seien $\mathcal{B}_1,~\mathcal{B}_2$ Basen f\"{u}r $V$ und seien $\mathcal{C}_1,~\mathcal{C}_2$ Basen f\"{u}r $W$. Zeigen Sie, dass
	\[
	[E]_{\mathcal{B}_1,\mathcal{C}_1}=(
	.\] 
\end{parts}
\end{Problem}
