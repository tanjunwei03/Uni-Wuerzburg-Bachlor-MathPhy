\begin{Problem}
Auf einem Kreis mit Radius 4 rollt innen ein Kreis mit Radius 1 ab. Die Kurve, die dabei ein fest gewählter Punkt auf dem kleineren Kreis beschreibt, heißt Astroide (dt. ``sternähnliche Kurve''). 	
\begin{parts}
	\item Bestimmen Sie eine Parametrisierung der Astroide und fertigen Sie eine Skizze der Kurve an (oder visualisieren Sie sie auf Geogebra). An welchen Stellen ist Ihre Parametrisierung singulär?
	\item Seien $a,b$ mit $a<b$ beliebige aus dem Definitionsbereich Ihrer Parametrisierung. Leiten Sie eine Formel für die Bogenlänge Ihrer parametrisierten Kurve auf dem Intervall $[a, b]$ her.  
\end{parts}
\end{Problem}
\begin{proof}
	\begin{parts}
	\item Der Schwerpunkt der Masse bewegt sich mit Winkelgeschwindigkeit $\omega$. Der Kreis dreht deswegen mit Winkelgeschwindigkeit $3\omega$. 

		Ein Punkt hat die Parametrisierung
		 \[
		\va r(t)=\left( 3\cos \omega t + \cos\left(3\omega t+\delta\right),3\sin\omega t + \sin\left( 3\omega t +\delta\right)   \right) 
		.\] 
		OBdA ist $\omega=1$ und in diesem Fall ist $t\in [0,2\pi)$. 
		\[
		\va r(t)=\left( 3\cos t + \cos(3t+\delta), 3\sin t-\sin(3t+\delta)  \right) 
		.\] 
		\begin{center}
			\begin{tikzpicture}
				\begin{axis}[trig format plots=rad]
					\addplot[domain=0:2*pi,samples=200] ({3*cos(x)+cos(3*x)}, {3*sin(x)-sin(3*x)});
				\end{axis}
			\end{tikzpicture}
		\end{center}
	\item \ldots
	\end{parts}
\end{proof}
\begin{Problem}
	Betrachten Sie die \emph{Traktrix} (dt. ``Ziehkurve'') $\alpha:(0,\pi)\to \R^2$,
	\[
	\gamma(t):=\left( \cos t+\log\left( \tan\left( \frac{t}{2} \right)  \right), \sin t \right),\qquad t\in (0,2\pi)
	.\] 
	\begin{parts}
	\item Skizzieren Sie die gegebene parametrisierte Kurve (oder visualisieren Sie sie auf Geogebra). 
	\item Zeigen Sie, dass jede Tangente der Traktrix die $x$-Achse schneidet, und dass die Länge der Strecke der Tangente zwischen dem Berührungspunkt mit der Traktrix und dem Schnittpunkt mit der $x$-Achse für alle Tangenten der Traktrix gleich ist. 
	\end{parts}
\end{Problem}
