\documentclass[prb,12pt]{revtex4-2}

\usepackage{amsmath, amssymb,physics,amsfonts,amsthm}
\usepackage{enumitem}
\usepackage[most]{tcolorbox}
\usepackage{cancel}
\usepackage{booktabs}
\usepackage{tikz}
\usepackage{hyperref}
\usepackage{enumitem}
\usepackage{transparent}
\usepackage{float}
\usepackage{multirow}
\usepackage{subcaption}
\newtheorem{Theorem}{Theorem}
\newtheorem{Proposition}{Theorem}
\newtheorem{Lemma}[Theorem]{Lemma}
\newtheorem{Corollary}[Theorem]{Corollary}
\newtheorem{Example}[Theorem]{Example}
\newtheorem{Remark}[Theorem]{Remark}
\theoremstyle{definition}
\newtheorem{Problem}{Problem}
\theoremstyle{definition}
\newtheorem{Definition}[Theorem]{Definition}
\newenvironment{parts}{\begin{enumerate}[label=(\alph*)]}{\end{enumerate}}
%tikz	
\tcbset{breakable=true,toprule at break = 0mm,bottomrule at break = 0mm}
\usetikzlibrary{patterns}
\usetikzlibrary{matrix}
\usepackage{pgfplots}
\pgfplotsset{compat=1.18}
% definitions of number sets
\newcommand{\N}{\mathbb{N}}
\newcommand{\R}{\mathbb{R}}
\newcommand{\Z}{\mathbb{Z}}
\newcommand{\Q}{\mathbb{Q}}
\newcommand{\C}{\mathbb{C}}
\allowdisplaybreaks
\begin{document}
\title{Funktionalanalysis Hausaufgaben Blatt 2}
	\author{Jun Wei Tan}
	\email{jun-wei.tan@stud-mail.uni-wuerzburg.de}
	\affiliation{Julius-Maximilians-Universit\"{a}t W\"{u}rzburg}
	\date{\today}
	\maketitle
\begin{Problem}
	Es seien die Punkte $x_0, x_1, \dots, x_n$ mit $x_i \in \R$ gegeben. Wir definieren den Operator
	\[
		\Phi:\R_{\le n}[x]\to \R^{n+1}, p\to y, \text{ mit }p(x_i)=y_i, i=0,\dots,n
	\] 
	wobei wir mit $\R_{\le n}[x]$ den Raum der Polynome mit reellen Koeffizienten vom Grad höchsten $n$ bezeichnen und $p(x)$ die Auswertung des Polynoms $p$ im Punkt $x$ beschreibt.
	\begin{parts}
		\item  Zeigen Sie: Sind die Punkte $x_i$ paarweise verschieden, so ist die Abbildung $\Phi$ wohldefiniert und isomorph. (Eine Konsequenz hieraus ist die eindeutige Lösbarkeit der Polynominterpolation.)
		\item Was passiert, wenn Sie nicht fordern, dass die $x_i$ paarweise verschieden sind? Kann $\Phi$ im Allgemeinen noch injektiv (surjektiv) sein?
	\end{parts}
\end{Problem}
\begin{proof}
	\begin{parts}
\item Injektiv: Nehme an, dass es zwei unterschiedliche Polynome $p_1$, $p_2$ gibt, mit $p_1(x_i)=p_2(x_i)\forall i=0,\dots,n$. Dann ist $p(x):=p_1(x)-p_2(x)$ auch ein Polynom, mit $p(x_i):=0\forall i\in \{0,\dots,n\}$. Weil  $ \deg(p)\le n$ist, folgt daraus, dass $\forall x,p(x)=0, p_1(x)=p_2(x)$. Das ist ein Widerspruch.

	Surjektive: Sei $(y_0,\dots,y_n)\in \R^{n+1}$. Dann ist
	\[
	p(x)=(x-y_0)(x-y_1)\dots(x-y_n)
	\]
	auch ein Polynom mit $\Phi(p)=(y_0,\dots,y_n)$.

	Linearität: Sei $p_1(x),p_2(x)\in \R_{\le n}[x], a\in \R$. Sei auch $p(x)=p_1(x)+p_2(x)$. Es gilt dann
	\[
	p(x_i)=p_1(x_i)+p_2(x_i),i=0,\dots,n\] und daher
	\[
	\Phi(p)=\Phi(p_1+p_2)=\Phi(p_1)+\Phi(p_2)
	.\] 
	Es gilt auch, f\"{u}r $p(x):=ap_1(x)$, dass
	\[
	p(x_i)=ap_1(x_i), i=0,\dots,n
	,\]
	und daher
	\[
	\Phi(p)=\Phi(ap_1)=a\Phi(p_1)
	.\] 
\item Nein. Sei, zum Beispiel, $n=1$, $x_0=x_1=0$. Dann gilt
	\begin{align*}
		\Phi(x)=&(0,0)^T\\
		\Phi(x^2)=&(0,0)^T
	\end{align*}
	Aber die zwei Polynome sind ungleich.
	\end{parts}
\end{proof}

\begin{Problem}
	\begin{parts}	
	\item Es sei eine Matrix $A \in \mathbb{K}^{n\times n}$ gegeben. Wir bilden die erweiterte Matrix
	\[
		B=(A|1_n)\]
		mit $1_n$ die Einheitsmatrix in $\R^n$. Zeigen Sie: $A$ ist genau dann invertierbar, wenn $A$ durch elementare Zeilenumformung in die Einheitsmatrix überführt werden kann. Verfizieren Sie weiterhin: Werden die dafür benötigten Zeilenumformungen auf ganz $B$ angewendet, so ergibt sich im hinteren Teil, wo zu Beginn die Einheitsmatrix stand, genau $A^{-1}$.
	\item Es sei nun
		\[
			A=\begin{pmatrix} 1 & 0 & 0 & 1\\0 & -1 & 2 & 0 \\ 0 & 0 & 0 & -2\\3 & 0 & 1 & 2 \end{pmatrix} 
		.\] 
		Bestimmen Sie $A^{-1}$.
	\end{parts}
\end{Problem}
\begin{proof}
	\begin{parts}
	\item Definiert $(x,y),x\in \mathbb{K}^n,y\in\mathbb{K}^m$ durch $\mathbb{K}^{n+m}\ni(x,y)=(x_1,\dots,x_n,y_1,\dots,y_n)$. Eine solche erweiterte Matrix bedeutet eine Gleichungssystem durch
		\[
		B(x, -y)=Ax-1_ny=0
		,\]
		wobei $x,y\in \mathbb{K}^n$. F\"{u}r jeder $x\in\mathbb{K}^n$ gibt es $y\in \mathbb{K}^n,$ so dass $B(x,-y)=0$. Nehme an, dass wir durch elementare Zeilenumformung
		\[
		B=(A|1_n)\to (1_n, A'):=B'
		\]
		kann. Die Gleichungssystem ist dann $x=A'y$. Dadurch können wir f\"{u}r jeder  $y\in\mathbb{K}^n$ eine $A'y=x\in\mathbb{K}^n$ rechnen, f\"{u}r die gilt, dass $Ax=y$. Das heißt, dass $A'=A^{-1}$. 
	\item
		{\allowdisplaybreaks
		\begin{gather*}
			\left(
\begin{array}{cccc|cccc}
 1 & 0 & 0 & 1 & 1 & 0 & 0 & 0 \\
 0 & -1 & 2 & 0 & 0 & 1 & 0 & 0 \\
 0 & 0 & 0 & -2 & 0 & 0 & 1 & 0 \\
 3 & 0 & 1 & 2 & 0 & 0 & 0 & 1 \\
\end{array}
\right) \xrightarrow{R_4-3R_1} \left(
\begin{array}{cccc|cccc}
 1 & 0 & 0 & 1 & 1 & 0 & 0 & 0 \\
 0 & -1 & 2 & 0 & 0 & 1 & 0 & 0 \\
 0 & 0 & 0 & -2 & 0 & 0 & 1 & 0 \\
 0 & 0 & 1 & -1 & -3 & 0 & 0 & 1 \\
\end{array}
\right) \xrightarrow{R_2\times -1}\\ \left(
\begin{array}{cccc|cccc}
 1 & 0 & 0 & 1 & 1 & 0 & 0 & 0 \\
 0 & 1 & -2 & 0 & 0 & -1 & 0 & 0 \\
 0 & 0 & 0 & -2 & 0 & 0 & 1 & 0 \\
 0 & 0 & 1 & -1 & -3 & 0 & 0 & 1 \\
\end{array}
\right) \xrightarrow{R_3\leftrightarrow R_4} \left(
\begin{array}{cccc|cccc}
 1 & 0 & 0 & 1 & 1 & 0 & 0 & 0 \\
 0 & 1 & -2 & 0 & 0 & -1 & 0 & 0 \\
 0 & 0 & 1 & -1 & -3 & 0 & 0 & 1 \\
 0 & 0 & 0 & -2 & 0 & 0 & 1 & 0 \\
\end{array}
\right) \xrightarrow{R_2+2R_3}\\ \left(
\begin{array}{cccc|cccc}
 1 & 0 & 0 & 1 & 1 & 0 & 0 & 0 \\
 0 & 1 & 0 & -2 & -6 & -1 & 0 & 2 \\
 0 & 0 & 1 & -1 & -3 & 0 & 0 & 1 \\
 0 & 0 & 0 & -2 & 0 & 0 & 1 & 0 \\
\end{array}
\right) \xrightarrow{R_2-R_4} \left(
\begin{array}{cccc|cccc}
 1 & 0 & 0 & 1 & 1 & 0 & 0 & 0 \\
 0 & 1 & 0 & 0 & -6 & -1 & -1 & 2 \\
 0 & 0 & 1 & -1 & -3 & 0 & 0 & 1 \\
 0 & 0 & 0 & -2 & 0 & 0 & 1 & 0 \\
\end{array}
\right) \xrightarrow{R_4\times -\frac{1}{2}}\\ \left(
\begin{array}{cccc|cccc}
 1 & 0 & 0 & 1 & 1 & 0 & 0 & 0 \\
 0 & 1 & 0 & 0 & -6 & -1 & -1 & 2 \\
 0 & 0 & 1 & -1 & -3 & 0 & 0 & 1 \\
 0 & 0 & 0 & 1 & 0 & 0 & -\frac{1}{2} & 0 \\
\end{array}
\right) \xrightarrow{R_1-R_4} \left(
\begin{array}{cccc|cccc}
 1 & 0 & 0 & 0 & 1 & 0 & \frac{1}{2} & 0 \\
 0 & 1 & 0 & 0 & -6 & -1 & -1 & 2 \\
 0 & 0 & 1 & -1 & -3 & 0 & 0 & 1 \\
 0 & 0 & 0 & 1 & 0 & 0 & -\frac{1}{2} & 0 \\
\end{array}
\right) \xrightarrow{R_3+R_4} \\\left(
\begin{array}{cccc|cccc}
 1 & 0 & 0 & 0 & 1 & 0 & \frac{1}{2} & 0 \\
 0 & 1 & 0 & 0 & -6 & -1 & -1 & 2 \\
 0 & 0 & 1 & 0 & -3 & 0 & -\frac{1}{2} & 1 \\
 0 & 0 & 0 & 1 & 0 & 0 & -\frac{1}{2} & 0 \\
\end{array}
\right)
		\end{gather*}
	}
	\end{parts}
\end{proof}
\begin{Problem}
	Es seien die Vektorräume $V, W$ über $\mathbb{K}$ gegeben mit $\dim(V) = n$ und $\dim(W ) = m$. Wir betrachten eine lineare Abbildung
	\[
	T:V\to W, v\to T(v)\] 
Seien $B_V$ und $B_W$ Basen von $V$, bzw. $W$. Wir nehmen an $T$ ist nicht die konstante Nullabbildung. Beweisen Sie:
\begin{parts}
\item Der Kern von $_{B_W}[T]_{B_V}$ ist entweder trivial (d.h. nur die 0) oder hängt nur von der Wahl von $B_V$ ab, aber nicht von $B_W$.
\item Das Bild von $_{B_W}[T]_{B_V}$ ist entweder der ganze $\mathbb{K}^m$ oder hängt nur von der Wahl von $B_W$ ab, aber nicht von $B_v$. 
\item Der Rang von $_{B_W}[T]_{B_V}$ ist unabh\"{a}ngig von $B_w$ und $B_V$.
aber nicht von $B_W$.
\end{parts}
\end{Problem}
\begin{proof}
Nach Korollar 5.43 gilt, f\"{u}r $A,A' \subseteq V$ und $B,B'\subseteq W$ Basen der Vektorräume $V$ und $W$ über $\mathbb{K}$, und $\Phi\in \text{Hom}(V,W)$.
 \[
	 _{B'}\left[ \Phi \right]_{A'}={}_{B'}[\text{id}_W]_B\cdot{}_B[\Phi]_A\cdot{}_A[\text{id}_V]_{A'}
.\] 
\begin{Lemma}
	Jeder Basiswechsel f\"{u}r sowohl $B_V$ als auch $B_W$ kann als zwei Basiswechseln interpretiert werden, wobei eine Basiswechsel nur $B_V$ verändert, und die andere nur $B_W$.
\end{Lemma}
\begin{proof}
	\[
		_{B'}\left[ \Phi \right]_{A'}={}_{B'}[\text{id}_W]_B\cdot{}_B[\Phi]_A\cdot{}_A[\text{id}_V]_{A'}={}_{B'}[\text{id}_W]_B\left( {}_B[\text{id}_W]_B\cdot {}_B[\Phi]_A\cdot {}_A[\text{id}]_{A'} \right) {}_A[\text{id}_V]_A
	.\]
	(In den Klammern gibt es zuerst ein Basiswechsel in $V$, dann ein Basiswechsel in $W$ ). Ein ähnliche Argument zeigt, dass wir zuerst ein Basiswechseln in $W$ betrachten kann.
\end{proof}
\begin{Corollary}
	In jedem Teilaufgabe muss man nur das Fall betrachten, in dem entweder $B_V$ oder $B_W$ sich verändert. 
\end{Corollary}
	\begin{parts}
	\item 
	\end{parts}
\end{proof}

\end{document}
