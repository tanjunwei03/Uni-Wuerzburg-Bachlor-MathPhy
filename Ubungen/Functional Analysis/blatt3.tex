\begin{Problem}
	Let $V$ be a $\mathbb{K}$-valued vector space. A seminorm on $V$ is a homogeneous map $p:V\to [0,\infty)$ satisfying the triangle inequality, i.e.
	\[
	p(v+w)\le p(v)+p(w)
\]
and
\[
p(\lambda v)=|\lambda|p(v)
\]
for any two vectors $v,w\in V$ and every scalar $\lambda\in\mathbb{K}$.
\begin{parts}
	\item Show that the kernel of a seminorm $p$ is a subspace of $V$.

		Given a seminorm $p$, we say that two vectors $v,w\in V$ are equivalent if there is a vector $u\in \text{ker }p$ such that $w=v+u$. Make yourself clear that this yields an equivalence relation on $V$.
	\item Show that the quotient space $V / \text{ker }p:=V / \sim$ carries a canonical linear structure.
	\item Show that the map
		\[
			\overline{p}:V / \text{ker }p\ni [v]\mapsto p(v)
		\]
		yields a well-defined norm on the quotient space.
\end{parts}
\end{Problem}
\begin{proof}
	\begin{parts}
	\item Clear by definition
	\item Also clear
	\end{parts}
\end{proof}
\begin{Problem}\label{pr:funcanalblatt3-2}
	Let $M$ be a topological space. Show that the space $(\mathcal{C}_b(M), \|\cdot\|_\infty)$ of continuous and bounded $\mathbb{K}$-valued functions endowed with the supremum norm is complete.
\end{Problem}
\begin{proof}
	Suppose we have a Cauchy sequence in the supremum norm. Clearly, this implies pointwise convergence (since the real numbers are complete).

	It remains to show that the new function is continuous and bounded. Clearly, if $\|f-f_n\|\le \epsilon$, then $\|f\|\le \|f_n\|+\epsilon$, which means it is bounded.

	Continuity follows from uniform convergence.\qedhere
\end{proof}
\begin{Problem}
	In this exercise we weaken the conditions of Homework~\ref{pr:funcanalblatt3-2} by considering functions that are only essentially bounded. The goal is to find a suitable seminorm on this function space such that the corresponding quotient becomes a Banach space. But first, we shall settle the term ``essentially bounded''. To this end, we need the following definitions:

	Let $X$ be a set and $\mathfrak{a}\in 2^X$. We call $\mathfrak{a}$ a $\sigma$-algebra if
	\begin{itemize}
		\item $\varnothing\in\mathfrak{n}$,
		\item $X \setminus A\in \mathfrak{a}$ for every $A\in \mathfrak{a}$ and
		\item $\bigcup_{n\in \N} A_n\in \mathfrak{a}$ for every sequence $(A_n)_{n\in \N}\subset \mathfrak{a}$
	\end{itemize}
	The pair $(X, \mathfrak{a})$ is called a measurable space. One can check that for every $A\in \mathfrak{a}$ one obtains a new $\sigma$-algebra $a|_{X\setminus A}\subseteq 2^{X \setminus A}$, where $B\in \mathfrak{a}_{X \setminus A}$iff there is some $C\in \mathfrak{a}$ such that $B=C \setminus A$.

	A function $f: (X, \mathfrak{a})\to \mathbb{K}$ if $f^{-1}(B_r(z))\subseteq \mathfrak{a}$ for every $z\in \mathbb{K}$ and $r>0$. We denote the set of measurable $\mathbb{K}$-valued functions by $\mathcal{M}(X, a)$. Clearly, the restriction of a measurable function $f\in \mathcal{M}(X, a)$ to $X \setminus A$ yields a measurable function $f|_{X \setminus A}\in \mathcal{M}(X \setminus A, \mathfrak{a}|_{X \setminus A})$.

	Finally, a subset $\mathfrak{n}\subseteq \mathfrak{a}$ is called a $\sigma$-ideal if
	\begin{itemize}
		\item $\varnothing \in \mathfrak{n}$,
		\item $\bigcup_{n\in \N} A_n\in \mathfrak{a}$ for every sequence $(A_n)_{n\in \N}\subset \mathfrak{n}$ and
		\item for all $A\in \mathfrak{n}$ and $B\in\mathfrak{a}$ one has the implication $B\subseteq A\implies B \in \mathfrak{n}$.
	\end{itemize}
	\begin{parts}
	\item For $f\in \mathcal{M}(X, \mathfrak{a})$ we define the essential range
		\[
			\text{ess range}(f):=\{z\in \mathbb{K}:f^{-1}(B_r(z))\not\in \mathfrak{n}\text{ for all }r>0\} 
		\]
		and the essential supremum
		\[
			\text{ess sup}(f):=\sup \{|z|:z\in \text{ess range}(f)\} 
		.\] 
		Show that $\text{ess range}(f)\subseteq \mathbb{K}$ is closed and $f^{-1}(\mathbb{K}\setminus\text{ess range}(f))\in \mathfrak{n}$.
	\item Show that two functions $f,g\in \mathcal{M}(X, \mathfrak{a})$ have the same essential range if the essential range of $f-g$ contains only $0$.
	\item The set of essentially bounded functions on $X$ is defined as
		\[
			\mathcal{L}^\infty (X, \mathfrak{a}, \mathfrak{n}):=\{f\in \mathcal{M}(X, a):\|f\|_\text{ess sup}:=\text{ess sup}(f)<\infty\} 
		.\] 
		Show that $\|\cdot\|_\text{ess sup}$ defines a seminorm on $\mathcal{L}^\infty(X, \mathfrak{a}, \mathfrak{n})$ and compute its kernel. Moreover, show that the essential supremum of $f\in \mathcal{L}^\infty(X, \mathfrak{a}, \mathfrak{n})$ is given by
		\[
			\text{ess sup}(f)=C_f:=\inf \{C>0:|f|^{-1}([C,\infty))\in \mathfrak{n}\}  
		.\] 
		Hint: You can use that $\mathcal{M}(X, \mathfrak{a})$ and $\mathcal{L}^\infty(X, \mathfrak{a}, \mathfrak{n})$ are $\mathbb{K}$-vector spaces without proof.
	\item Show that $L^\infty(X, \mathfrak{a}, \mathfrak{n}):= \mathcal{L}^\infty (X, \mathfrak{a}, \mathfrak{n}) / \text{ker}\|\cdot\|_{ess sup}$ is a Banach space, i.e. a complete normed space.

		Hint: Consider the sequence $(f_n)_n$ on a suitable subset of $X$ and copy your proof of Homework~\ref{pr:funcanalblatt3-2}. You can use that a pointwise limit of a sequence of measurable functions is again measurable without proof.
	\end{parts}
\end{Problem}
\begin{proof}
	\begin{parts}
	\item We consider the complement of the essential range. Suppose $z$ is not in the essential range. Then $f^{-1}(B_r(z))\not\in \mathfrak{n}$ for some $r$. Clearly, this is the open set that we seek that does not intersect the essential range. Thus, the essential range is closed.

		We prove the second part by contradiction. Suppose that the inverse image were not in $\mathfrak{n}$. Then
		\[
			f^{-1}(\mathbb{K} \setminus \text{ess range}(f))=\mathcal{M}\setminus f^{-1}(\text{ess range}(f))
		.\] 
		We seek to show that this is not in $\mathfrak{n}$. Suppose it were. Then 
	\end{parts}
\end{proof}
