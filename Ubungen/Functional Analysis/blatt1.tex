\begin{Problem}\label{pr:funcanalblatt1-1}
	Let $(M, d)$ be a metric space. Consider a sequence $(a_n)_{n\in \N}\subset \text{Map}(N, M)$ of Cauchy sequences in $M$ i.e. $a_n=(a_{mn})_{m\in \N}\subset M$ for every $n\in \N$.
	\begin{parts}
		\item Show that the sequence $(d_k^{(mn)})_{k\in N}\subset \R$ defined by.
			\[
				d_k^{(mn)}:=d(a_{nk}, a_{m k })
			\]
			is convergent.
			In the following, we assume that for every $\epsilon>0$ there is a natural number $N\in \N$ such that $\lim_{k \to \infty} d_k^{(nm)}<\epsilon$ for every  $n,m\ge N$.
		\item For a strictly monotonously increasing sequence $(m_k)_{k\in \N} \subset \N$, we define the diagonal sequence $ (D_k)_{k\in \N}$ as follows
			\[
				D_k:=a_{k m_k}
			.\] 
			Show that there exists a diagonal Cauchy sequence $(D_k)_k$ such that $\lim_{k \to \infty} d(a_{nk}, D_k)$ converges to zero in the limit $n\to \infty$. Moreover, show that every other diagonal Cauchy sequence $(D_k')_k$ with the same property satisfies $\lim_{k \to \infty} d(D_k, D_k')=0$.
		\item Assume now that $M$ is complete. Show that $(D_k)_k$ converges and compute its limit.
	\end{parts}
\end{Problem}
\begin{proof}
	\begin{parts}
		\item We show that the sequence is Cauchy. Choose $N\in \N$ such that for all $k_1,k_2\ge N$, we have $d(a_{nk_1}, a_{nk_2})<\epsilon$ and $d(a_{m k_1}, a_{m k_2})<\epsilon$. Then we apply the triangle inequality
			\begin{align*}
				d(a_{nk_1}, a_{m, k_1})&\le d(a_{nk_1}, a_{n k_2})+d(a_{n k_2}, a_{m k_2})+d(a_{m k_2}, a_{m k_1})\\
				d(a_{nk_1}, a_{m, k_1})-d(a_{n k_2}, a_{m k_2})&\le d(a_{nk_1}, a_{n k_2})+d(a_{m k_2}, a_{m k_1})
			\end{align*}
			Thus the sequence is Cauchy. Since $\R$ is complete, it is convergent.
		\item We construct this sequence as follows: Let us fix $k\ge 0$ and choose 
	\end{parts}
\end{proof}

\begin{Problem}
	Let $(M, d)$ be a metric space. We write $\tilde{M}$ for the set of Cauchy sequences in $M$. 
	\begin{enumerate}
		\item We say that two Cauchy sequences $(a_n)_n, (b_n)_n\in \tilde{M}$ are equivalent if
			\[
			\lim_{n \to \infty} d(a_n, b_n)=0
			\] 
			and write $(a_n)_n\sim (b_n)_n$. Show that this defines an equivalence relation on $\tilde{M}$
		\item Show that there exists a metric $\hat{d}$ on the quotient space $\hat{M}:=\tilde{M} / \sim$ such that $(\hat{M}, \hat{d})$ is a completion of $(M, d)$.
		\item Let $(M', D')$ be another completion of $(M, d)$. Show that $M'$ is isometrically isomorphic to $\hat{M}$, i.e. there exists a bijective isometry $\phi: \hat{M}\to M'$.
		\item Now, assume $(M', d')$ to be another complete metric space and let $\Phi:M\to M'$ be a uniformly continuous map. Show that there is a unique continuous map $\phi:\hat{M}\to M'$ such that
			\[
			\Phi=\phi\circ \iota
			.\] 
			Conclude that $\phi$ is even uniformly continuous.
	\end{enumerate}
\end{Problem}
\begin{proof}
	\begin{parts}
	\item Clear.
	\item We define the metric between two Cauchy sequences $(a_n)_n$ and $(b_n)_n$ by $\hat{d}(a, b)=\lim_{n \to \infty} d(a_n, b_n)$ (cf Pr.~\ref{pr:funcanalblatt1-1}).

		Note that this is actually not a metric on $\tilde{M}$, since it is not positive. Two applications of the triangle inequality show that this is well defined on the quotient space. 

		We map each element of $M$ to the (equivalence class containing the) constant sequence at that element. Note that these sequences are convergent, and this map is an isometry. 

		It remains to show that (the image of) $M$ is dense, and that $\hat{M}$ is complete. 

		Let us consider a Cauchy sequence $(a_n)_n$ and an $\epsilon>0$. Since the sequence is Cauchy, we have $N\in \N$ such that for all $n,m\ge N$, $d(a_n, a_m)<\epsilon$. In particular, $d(a_N, a_m)<\epsilon$. Thus $\hat{d}(a, a_N)<\epsilon$ (Note: $a_N$ is the constant Cauchy sequence at $a_N$). Thus, the open ball with radius $\epsilon$ intersects $M$. $M$ is therefore dense.

		Now we show completeness. Consider a Cauchy sequence of elements of $\hat{M}$. Since this is Cauchy, it satisfies the conditions of Pr.~\ref{pr:funcanalblatt1-1}(b), and converges to the diagonal sequence mentioned there.

		Comment: Pr~\ref{pr:funcanalblatt1-1}(c) shows that this construction would yield $M$ if $M$ were complete, because the diagonal sequence would converge to a limit.
	\item Consider a map from $\hat{M}$ to $M'$ that sends all elements of $M$ to elements of $M$. Clearly, when we restrict the map to these elements, it is an isometry. 

		The question is: Can we now define 
	\end{parts}
\end{proof}

\begin{Problem}
	Let $(M, \mathcal{M})$ be a topological space and $A,B\subseteq M$ be subsets. Prove the following identities.
	\begin{parts}
	\item 
		\[
		(A\cup B)^\text{cl}=A^\text{cl}\cup B^\text{cl}
		\]
		and 
		\[
		(A\cup B)^{\circ}\supseteq A\circ \cup B^\circ
		.\] 
	\item 
		\[
		(A\cap B)^\text{cl}\subseteq A^\text{cl}\cap B^\text{cl}
		\]
		and
		\[
		(A\cap B)^\circ = A^\circ \cap B^\circ
		.\] 
	\item 
		\[
		(M\setminus A)^\text{cl}=M\setminus A^{\circ}
		\]
		and
		\[
		(M\setminus A)^\circ = M \setminus A^\text{cl}
		.\] 
	\end{parts}
	For the identities with inequalities, give examples where one has strict subsets.
\end{Problem}
\begin{proof}
	\begin{parts}
	\item The reverse inclusion is obvious, because if all neighbourhoods of a point $p$ intersect $A$ or $B$, then they intersect $A\cup B$. Conversely, suppose a point is neither in the closure of $A$ nor in the closure of $B$. Then it has a neighbourhood that does not intersect $A$, and a neighbourhood that does not intersect $B$. The intersection of these two neighbourhoods does not intersect $A\cup B$.

		For the second inclusion, we simply note that $A^\circ$ and $B^\circ$ are open; therefore, their union is an open set contained in $A\cup B$.

		An example where equality fails is the Cantor set and its complement.
	\item The closures of $A$ and $B$ are closed sets containing $A\cap B $; therefore, their intersection is closed and contains $(A\cap B)^\text{cl}$.

		Equality fails

		Suppose $x$ is in the interior of $A$ and of $B$. Then it has a neighbourhood entierly contained in $A$ and $B$, and thus in $A\cap B$. The same thing holds in reverse.
	\item 
	\end{parts}
\end{proof}
