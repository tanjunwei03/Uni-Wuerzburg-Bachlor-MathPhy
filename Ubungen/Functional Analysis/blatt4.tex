\begin{Problem}
	For $p\in [1,\infty]$ we define the set
	\[
	\ell^p:=\begin{cases}
		\{x:=(x_n)_{n\in N}\subset \mathbb{K}|\|x\|_p:=(\sum_{n=1}^\infty |x_n|^p)^{\frac{1}{p}}<\infty\} & p<\infty\\
		\{x:=(x_n)_{n\in \N}\subset \mathbb{K}|\|x\|_\infty := \sup_{n\in \N}|x_n|<\infty\} & p=\infty.
	\end{cases}
	\] 
Show that the usual operations on sequences induce a vector space structure on $\ell^p$. Moreover, show that $\ell^p$ is a subspace of $\ell^r$ for $p\le r$.
\end{Problem}
\begin{proof}
	Clearly, multiplying a vector by a constant multiplies its norm by a constant in both cases. 

	We show the inclusion as follows: Since the series converges, the terms (all positive) must converge to $0$. Thus we can choose $N$ such that for $|x_n|<1$ for all $n\ge N$. For $|x|<1$, we have $|x|^p\ge |x|^r$. This shows that the vector is also in $\ell^r$.
\end{proof}
\begin{Problem}
	In this exercise, we consider the spaces $\ell^p$ for $p\in (1,\infty)$. Note that for every such $p$ there exists a conjugate number $q\in (1,\infty)$ which satisfies $\frac{1}{p}+\frac{1}{q}=1$.
	\begin{parts}
	\item Show that the product of two non-negative real numbers $a,b\in [0,\infty)$ satisfies Young's inequality, that is
		\[
		ab\le \frac{a^p}{p}+\frac{b^q}{q}
		.\] 
		\emph{Hint: Use the AM-GM inequality.}
	\item Prove that H\"{o}lder's inequality
		\[
		\|xy\|_1:=\sum_{n=1}^{\infty} |x_ny_n|\le \|x\|_p \|y\|_q
		\]
		holds true for any two sequences $x\in \ell^p$ and $y\in \ell^q$.
	\item Show Minkowsky's inequality, that is
		\[
		\|x+y\|_p\le \|x\|_p + \|y\|_p
		\]
		for $x,y\in \ell^p$.
	\item Let $\lambda:=(\lambda_n)_{n\in \N}\subset [0,1]$ be a sequence in $\ell^1$ with $\|\lambda\|_1=1$. Show that Jensen's inequality
		\[
		f\left( \sum_{n=1}^\infty \lambda_n x_n \right)\le \sum_{n=1}^{\infty} \lambda_n f(x_n)
		\]
		holds true for every convex function $f\in \mathcal{C}(I)$ on an open interval $I\subseteq \R$ and every sequence $(x_n)_{n\in \N}\subset I$ such that $\sum_{n=1}^{\infty} \lambda_n x_n$ and $\sum_{n=1}^{\infty} \lambda_n f(x_n)$ converge and $\sum_{n=1}^{\infty} \lambda_n x_n\in I$. Conclude that $\|x\|_r\le \|x\|_p$ for every $x\in \ell^p$ and $p\le r$.
	\end{parts}
\end{Problem}
\begin{proof}
	\begin{parts}
		\item The weighted AM-GM inequality yields
		\[
		\frac{q a^p + p b^q}{pq}\ge \sqrt[pq]{a^{pq} b^{pq}} =ab
		.\]
		\item Suppose either norm is 0. Then that sequence must be 0 everywhere, and thus the inequality is fulfilled. 
		
		Now suppose either $p$ or $q$ is infinite --- without loss of generality, we assume $p$ is. Then the inequality reduces to
		\[\sum_{n=1}^\infty |x_n y_n| \le \left(\sup_{n\in \N} x_n\right) \|y\|_1\]
		which is obviously true, as we can see by replacing $x_n$ with its supremum. 
		
		Hence, we assume that both $p$ and $q$ are finite, and that neither norm is $0$. We can thus divide each sequence by their norm, and assume WLOG that $\|x\|_p=1=\|y\|_q$.  Now, we apply Young's inequality
		\begin{align*}
			\|xy\|_1 &= \sum_{n=1}^\infty |x_n y_n|\\
			&\le\sum_{n=1}^{\infty} \left[ \frac{|x_n|^p}{p}+\frac{|y_n|^q}{q} \right] \\
			&=\frac{1}{p}\|x\|_p^p+\frac{1}{q}\|y \|_q^q\\
			&=\frac{1}{p}+\frac{1}{q}=1
		\end{align*}
	as desired. 
	\item 
	\begin{align*}
		\|x+y\|_p &= \left[\sum_{n=1}^\infty |x_n+y_n|^p\right]^{1/p}\\
		&=\left[\sum_{n=1}^\infty |x_n+y_n||x_n+y_n|^{p-1}\right]^{1/p}\\
		&\le\left[\sum_{n=1}^\infty \left(|x_n+y_n|+|x_n+y_n|^{p-1}\right)\right]^{1/p}
	\end{align*}
	\end{parts}
\end{proof}
\begin{Problem}
	Let $p\in [1,\infty)$. Consider the sequences $(e_n:=(\delta_{nm})_{m\in \N})_{n\in \N}\subset \ell^p$. Show that for every sequence $x=(x_n)_{n\in \N}\in \ell^p$ the series $\sum_{n\in \N} x_n e_n$ converges unconditionally towards $x$ with respect to $\|x\|_p$. Does it converge absolutely? Moreover, sh ow that a sequence $x=(x_n)_{n\in \N}$ lies in $\ell^p$ if the series $\sum_{n\in \N}x_n e_n$ converges unconditionally with respect to $\|\cdot\|_p$.

	\emph{Hint: Having Minkowski's inequality, you can use that } $(\ell^p, \|\cdot\|_p)$ \emph{is a normed space without proof}
\end{Problem}
\begin{Problem}
	In the upcoming exercise sheets, we will prove the Stone-Weierstraß theorem in several steps. Here, we do some necessary preparation we will need for the actual proof.

	By recursion, define the polynomials
	\[
	p_0(x)=0,\qquad\text{and}\qquad p_{n+1}(x)=p_n(x)+\frac{1}{2}(x-p_n^2(x))
	.\] 
	\begin{parts}
	\item Show $p_n(0)=0$ and the estimates
		\[
		p_n(x)\ge 0,\qquad \text{and}\qquad 0\le \sqrt{x} -p_n(x)\le \frac{2\sqrt{x} }{2+n\sqrt{x} }
		\]
		for $x\in [0,1]$.

		\emph{Hint: First show the coarser estimates} $0\le p_n(x)\le 1$ \emph{ for }$x\in [0,1]$ \emph{by induction. Use this in a second induction to improve the estimates.}
	\item Conclude that $(p_n)_{n\in \N}$ converges uniformly to the square root function on the interval $[0,1]$.
	\item Let $\alpha>0$. Construct a sequence of polynomials that converges uniformly to the square root function on $[0,\alpha]$.
	\item Let $\alpha>0$. Construct a sequence of polynomials that converges uniformly to the absolute value function on $[-\alpha, \alpha]$.
	\end{parts}
\end{Problem}
