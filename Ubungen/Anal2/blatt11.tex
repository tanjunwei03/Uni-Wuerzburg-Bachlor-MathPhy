\begin{Problem}
	Der Laplace-Operator $\Delta$ ist f\"{u}r $f\in C^1(\R^n)$ definiert durch
	\[
		\Delta f=\sum_{i=1}^n \partial_i^2 f
	.\] 
	Allgemeiner ist ein (homogener) Differentialoperator $P$ zweiter Ordnung f\"{u}r $f\in C^2(\R^n)$ definiert durch
	\[
		Pf=\sum_{i,j=1}^n a_{ij}\partial_i\partial_j f
	.\] 
	Zeigen Sie, dass die einzigen rotationsinvarianten Differentialoperatoren, d.h. solche, welche
	\[
		P(f(Qx))=(Pf)(Qx)
	.\] 
	f\"{u}r alle $x\in \R^n$ und alle orthogonalen Matrizen $Q\in \R^{n\times n}$ erfüllen, Vielfache des Laplace-Operators darstellen.
\end{Problem}
\begin{proof}
	Sei $Q$ orthogonal und beliebig mit Elemente $Q_{ij},i,j\in \{1,\dots, n\} $. Sei auch $x\in \R^n$ beliebig. Die Voraussetzung ist dann
	Es gilt (Kettensregel)
	\[
		\partial_i f(Qx)=\sum_{j=1}^n(\partial_j f)(Qx)Q_{ji}
	.\]
	Dann ist 
	\begin{align*}
		\sum_{i,j=1}^n a_{ij}\partial_i \partial_j f(Qx)=&	\sum_{i,j=1}^n a_{ij}\partial_i \left( \sum_{k=1}^n (\partial_k f)(Q_{kj}) \right) \\
		=&\sum_{i,j=1}^n \sum_{l=1}^n\sum_{k=1}^n a_{ij}(\partial_l \partial_k f)(Qx)Q_{kj}Q_{li}\\
		=&\sum_{k,l=1}^n (\partial_l\partial_k f)\sum_{i,j=1}^n Q_{li}a_{ij}Q_{kj}\\
		=&\sum_{k,l=1}^n (\partial_l\partial_k f)\sum_{i,j=1}^n Q_{li}a_{ij}(Q^T)_{jk}\\
		\overset{?}{=}&\sum_{i,j=1}^n a_{ij}(\partial_i\partial_j f)(Qx)
	\end{align*}
	Sei $B:=QAQ^T$. Offensichtlich muss dann, f\"{u}r alle orthogonale Matrizen $Q$, $QAQ^T=A$ gelten.
\end{proof}

\begin{Problem}
	Beweisen Sie: Ist $f\in \mathcal{C}([a,b],\R)$, so existiert f\"{u}r jedes $\epsilon>0$ ein Polynom $p:[a,b]\to \R$ mit
	\[
	\|f-p\|_\infty\le\epsilon
	.\] 
	Die Menge der Polynome auf dem Intervall $[a,b]$ ist also dicht im Raum der stetigen Funktionen bzgl. der Supremumsnorm.

	Gehen Sie wie folgt vor:
	\begin{parts}
		\item (Hutfunktionen) Es sei
			\[
				h_{a,b}(x)=\max\left\{ 0,1-\frac{|x-a|}{b} \right\} ,x\in \R
			\]
			f\"{u}r $a\in \R,b>0$. Begründen Sie, dass auf jedem kompakten Intervall $I$ f\"{u}r jedes $\epsilon$ ein Polynom $p$ existiert mit $\|h_{a,b}-p\|_{\infty,I}\le\epsilon$.
		\item (Lineare Interpolante) Zu einer gegebenen Partition von $[a,b]$ mit St\"{u}tzstellen $x_0=a<x_1<\dots<x_N=b$ definieren wir die lineare Interpolante von $f$ durch
			\[
				H(x)=\sum_{i=0}^N h_{x_i, \Delta x_i}(x)f(x_i)
			,\]
			wobei $\Delta x_i=x_i-x_{i-1}$ f\"{u}r $i=1,\dots, n$ und $\Delta x_0=x_1-a$. Bestimmen Sie zu gegebenem $\epsilon>0$ eine Partition von $[a,b]$, sodass $\|H-f\|_\infty\le\epsilon$ gilt.
		\item Beweisen Sie den Approximationssatz von Weierstraß. 
	\end{parts}
\end{Problem}
\begin{proof}
	\begin{parts}
	\item Wir brauchen hier die Aufgaben
		\begin{tcolorbox}
			Zeigen Sie, dass die Funktion
	\[f:[-1,1]\to\R\qquad f(x)=\text{max}\left\{ x,0 \right\} \]
	gleichmäßig durch Polynome approximiert werden kann.	
		\end{tcolorbox}
	\end{parts}
\end{proof}
\begin{Problem}
	Zeigen Sie, dass die Funktion
	\[
	f:\R^2\to\R,~f(x,y)=\begin{cases}
		\frac{xy(x^2-y^2)}{x^2+y^2} & (x,y)\neq (0,0)\\
		0 & (x,y)=(0,0)
	\end{cases}
	\] 
	in $(0,0)$ zweimal partiell differenzierbar ist und dass
	\[
	\frac{\partial^2 f}{\partial x\partial y}(0,0)\neq \frac{\partial^2 f}{\partial y\partial x}(0,0)
\]
gilt. Ist $f$ in $\R^2$ stetig differenzierbar?
\end{Problem}

\begin{Problem}
	Betrachten Sie die Funktion $f:\R^2\to\R$ mit
	\[
	f(x,y)=(y-x^2)(y-2x^2)
	.\] 
	Zeigen Sie:
	\begin{parts}
	\item F\"{u}r jede Richtung $v\in \R^2\backslash \{0\} $ nimmt $f_v(t)=f(tv),~t\in \R$ ein striktes lokales Minimum in $t=0$ an.
	\item Die Funktion $f$ besitzt in $(0,0)$ kein lokales Minimum
	\end{parts}
\end{Problem}
\begin{proof}
\begin{parts}
\item Sei $v=(a,b)^T$. Es gilt 
	\begin{align*}
		f_v(t)=& f(tv)\\
		=&(bt-a^2t^2)(bt-2a^2t^2)\\
		=&b^2t^2+2a^4t^4-2ba^2t^3-ba^2t^3\\
		=&b^2t^2+2a^4t^4-3ba^2t^3\\
		=&t^2(2a^4t^2-3ba^2t+b^2)
	\end{align*}
	Die Ableitungen sind\ldots
\item Offensichtlich ist $f(0,0)=0$. Sei $ x$ fest. Wir wählen $x^2<y<2x^2$. Damit ist
	\[
		f(x,y)=\underbrace{(y-x^2)}_{>0}\underbrace{(y-2x^2)}_{<0}<0
	,\]
	also $f(x,y)<0$. Da
	\begin{align*}
		\|(x,y)\|=& \sqrt{x^2+y^2} \\
		\le& \sqrt{x^2+(2x^2)^2} \\
		=& \sqrt{x^2+4x^4} \\
		=& |x|\sqrt{1+4x^2} 
	\end{align*}
	was wir beliebig klein wählen kann, gibt es f\"{u}r jedes $\epsilon>0$ ein Punkt $p\in B_\epsilon((0,0))$ so dass $f(p)<f((0,0))$, also $f$ besitzt kein lokales Minimum in $(0,0)$.
\end{parts}
\end{proof}
