\begin{Problem}
	In dieser Aufgabe beweisen wir, dass die Verknüpfung zweier Riemann-integrierbarer Funktionen i.A. nicht Riemann-integrierbar ist. Dazu gehen wir wie folgt vor:
	\begin{enumerate}[label=(\alph*)]
		\item Es sei $q:\N \to \Q\cap [0,1]$ eine Abzählung von $\Q\cap [0,1]$, d.h. eine bijektive Abbildung von $\N$ nach $\Q\cap [0,1]$. Weiterhin sei
			 \[
			f(x)=\begin{cases}
				0 & x\in [0,1]\backslash \Q, \\
				\frac{1}{n} & x=q_n.
			\end{cases}
			.\] 
			Zeigen Sie, dass $f$ Riemann-integrierbar ist.
		\item Weiterhin sei
			\[
			g(x)=\begin{cases}
				0 & x\in [0,1]\backslash \left\{ \frac{1}{n}|n\in \N \right\},\\
				1 & x=\frac{1}{n}\text{ f\"{u}r ein }n\in \N.
			\end{cases}
			\] 
			Zeigen Sie, dass $g$ Riemann-integrierbar ist, die Verknüpfung $g\circ f$ mit der Funktion $f$ jedoch nicht. 
	\end{enumerate}
\end{Problem}
\begin{proof}
	\begin{parts}
	\item Wir definieren rekursiv eine Menge
	\end{parts}
\end{proof}
\begin{Problem}
	Es sei $f:[a,b]\to \R$ Riemann-integrierbar auf dem echten Intervall $[a,b]$ mit
	\[
		\int_a^b f(x)\dd{x}>0
	.\] 
	Zeigen SIe, dass es ein echtes Intervall $J\subset [a,b]$ gibt, auf dem $f$ strikt positiv ist, d.h. mit $f(x)>0$ f\"{u}r alle $x\in J$.

	{\footnotesize Hinweis: Eine Möglichkeit ist, die Charakterisierung der Darboux-Integrierbarkeit zu benutzen und Untersummen zu betrachten.}
\end{Problem}
\begin{proof}
	Wir beweisen es per Widerspruch. Nehme an, dass in jedem Interval es mindestens ein Punkt $x_0$ gibt, f\"{u}r die $f(x_0)\le 0$. Insbesondere gilt das f\"{u}r alle abgeschlossen Intervalle $[c,d]\subseteq [a,b]$.

	Sei jetzt $\mathcal{J}$ eine beliebige Zerlegung von $[a,b]$, $\mathcal{J}=\left\{ t_0,t_1,\dots,t_N \right\} $, mit die übliche Voraussetzung $t_0< t_1 < t_2<\dots< t_N$. Es gilt
	\begin{align*}
		\mathcal{U}_{\mathcal{J}}=&\sum_{i=1}^{N} \text{inf}\left( f|_{[t_{i-1},t_i]} \right) (t_i-t_{i-1})\\
		\le& \sum_{i=1}^{N} (0)(t_i-t_{i-1})\\
		=& 0
	\end{align*}
	Weil $\mathcal{J}$ beliebig war, gilt das f\"{u}r alle Zerlegungen, und
	\[
		\mkern3mu\underline{\vphantom{\intop}\mkern7mu}\mkern-10mu\int_a^b f(x)\dd{x}\le 0
	,\]
	also
	\[
		\int_a^b f(x)\dd{x}\le 0
	,\] 
	ein Widerspruch.
\end{proof}
\begin{Problem}
	Beweisen oder widerlegen Sie die folgenden Aussagen:
	\begin{parts}
	\item Ist $f:[a,b]\to \R$ eine Funktion und $|f|$ integrierbar auf $[a,b]$, so ist es auch $f$.
	\item Ist $f:[a,b]\to \R$ integrierbar und $f(x)\ge \delta$ f\"{u}r alle $x\in [a,b]$ und ein $\delta>0$, so ist auch $\frac{1}{f}$ \"{u}ber $[a,b]$ integrierbar.
	\item Sind $f,g:[a,b]\to \R$ integrierbar, so gilt
		\[
			\int_a^b(f\cdot g)(x)\dd{x}=\int_a^b f(x)\dd{x}\cdot \int_a^b g(x)\dd{x}
		.\] 
	\end{parts}
\end{Problem}
\begin{proof}
	\begin{parts}
	\item Falsch. Sei $f:[0,1]\to \R$
		\[
		f(x)=\begin{cases}
			1 & x\in \Q\\
			-1 & x\not\in \Q
		\end{cases}
		.\] 
	\item Wahr.
	\item Falsch. Sei $f$ und $g$ Treppefunktionen, $f,g:[0,1]\to \R$:
		\begin{align*}
			f(x)=&\begin{cases}
				1 & 0 \le x \le 0.5\\
				0 & \text{sonst}.
			\end{cases}\\
				g(x)=&\begin{cases}
					1 & 0.5 < x \le 1\\
					0 & \text{sonst}.
				\end{cases}
		\end{align*}
		Es gilt $(f\cdot g)(x)=0$, und daher $\int_0^1 (f\cdot g)(x)\dd{x}=0$.
	\end{parts}
\end{proof}
\begin{Problem}
	\textbf{(Wanderdüne)}	Man gebe eine Folge von nicht-negativen Funktionen $f_n:[0,1]\to \R$ an, sodass
	\begin{itemize}
		\item $\lim_{n \to \infty} \int_0^1 f_n(x)\dd{x}=0$,
		\item $f_n\not\to0$ f\"{u}r jedes $x\in [0,1]$.
	\end{itemize}
\end{Problem}
\begin{proof}
	Sei
	\[
		g_{a,b}(x)=\begin{cases}
			\sin\left( \pi \frac{x-a}{b-a} \right) & x\in [a,b]\cap [0,1]\\
			0 & \text{sonst}.
		\end{cases}
	\]
	Es gilt
	\begin{align*}
		\int_0^1 g_{a,b}(x)\dd{x}\le& \int_a^b g_{a,b}(x)\dd{x}\\
		=&\int_a^b \sin\left( \pi \frac{x-a}{b-a} \right) \dd{x}\\
		=&\frac{2(b-a)}{\pi}
	\end{align*}
\end{proof}
