\begin{Problem}
\begin{parts}
	\item Benutzen Sie Proposition 5.6.9, um zu zeigen, dass
		\[
		g(x)=\sin(x)\cosh(x), \qquad x\in \R
		\] 
durch die zugehörige Taylorreihe im Punkt $x_0 = 0$ mit Konvergenzradius $R=+\infty$ dargestellt wird.
\item Zeigen Sie, dass $f:\R\to \R$ mit
	\[
	f(x)=\begin{cases}
		\exp\left( -\frac{1}{x^2} \right) & \qquad x \neq 0\\
		0 & \text{sonst}
	\end{cases}
	\]
	\emph{nicht} durch ihre Taylorreihe um x = 0 dargestellt wird. Warum ist dies kein Widerspruch zu Proposition 5.6.9?
\end{parts}	
\end{Problem}

\begin{Problem}
	Es sei $f : \R_0^+ \to \R$ definiert durch $f (x) = \sqrt[3]{x}$. Geben Sie das Taylorpolynom $P_2$ von f mit Entwicklungspunkt $x_0 = 1$ an und schätzen Sie den maximalen Fehler von $|f(x) - P_2(x)|$ auf dem Intervall $\left[ \frac{1}{2},\frac{3}{2} \right] $ ab.
\end{Problem}
\begin{proof}
	Es gilt $f(x)=x^{1 / 3}$, und daher
	\[
		f^{(n)}(x)=\left[\prod_{i=0}^{n-1} \left( \frac{1}{3}-i \right) \right]x^{\frac{1}{3}-n} 
	,\] 
	also
	\[
		f^{(n)}(1)=\prod_{i=1}^{n-1} \left( \frac{1}{3}-i \right)  
	.\] 
	Es gilt daher
	\[
	P_2(x)=1+\frac{1}{3}(x-1)-\frac{1}{9}(x-1)^2
	.\] 
\end{proof}
\begin{Problem}
	Bestimmen Sie die Taylorpolynome vom Grad 30 der folgenden Funktionen in $x_0$.
	\begin{parts}
	\item $f(x)=x^3-3x^2+3x+2$ im Punkt $x_0=2$.
	\item $g(x)=\sin^2\left( \pi x \right) $ in $x_0=3$.
	\item $h(x)=\sin^{-1}(x)$ in $x_0=0$.
	\end{parts}
\end{Problem}
\begin{proof}
	\begin{parts}
	\item 
		{\allowdisplaybreaks
		\begin{align*}
			f(2)=&2^3-3(2)^2+3(2)+2=4\\
			f'(x)=&3x^2-6x+3\\
			f'(2)=&3\\
			f''(x)=&6x-6\\
			f''(2)=&6\\
			f'''(x)=&6=f(2)\\
			f''''(x)=&0
		\end{align*}
	}
		Das Taylorpolynom ist dann
		\[
		4+3(x-2)+3(x-2)^2+(x-2)^3
		.\] 
	\item {\allowdisplaybreaks
		\begin{align*}
			g(x)=&\sin^2(\pi x)=\frac{1}{2}\left( 1-\cos(2\pi x) \right) \\
			g(3)=&0\\
			g'(x)=&\pi \sin(2\pi x)\\
			g^{(n)}(x)=&\left( -1 \right)^{\left\lfloor (n-1) / 2 \right\rfloor}\frac{(2\pi)^{n}}{2}\begin{cases}
				\sin(2\pi x) & n\text{ ungerade}\\
				\cos(2\pi x) & n\text{ gerade}
			\end{cases} & n\geq 1\\
				g^{(n)}(3)=& \left( -1 \right) ^{\left\lfloor (n - 1) / 2 \right\rfloor}\frac{(2\pi)^n}{2}\begin{cases}
					0 & n\text{ ungerade}\\
					1 & n\text{ gerade}	
				\end{cases} & n \ge 1
		\end{align*}
	}
		Das Taylorpolynom vom Grad 30 ist
		\[
			\sum_{n=1}^{15}\left[ \left( -1 \right) ^{\left\lfloor (2n -1) / 2 \right\rfloor}\frac{(2\pi)^{2n}}{2(2n)!}(x-3)^{2n}\right]
		.\] 
	\item
		{\allowdisplaybreaks
		\begin{align*}
			h(x)=& \sin^{-1}x\\
			h(0)=&0\\
			h'(x)=&\frac{1}{\sqrt{1-x^2} }=(1-x^2)^{-1 / 2}\\
			h'(0)=&1\\
			h''(x)=&-\frac{1}{2}\left( 1-x^2 \right)^{-3 / 2}(-2x)\\
			=& x(1-x^2)^{-3 / 2}\\
			h''(0) =& 0\\
			h'''(x)=&(1-x^2)^{-3 / 2}-\frac{3x}{2}\left( 1-x^2 \right) ^{- 5 / 2}(-2x)\\
=&(1-x^2)^{-3 / 2}-3x^2\left( 1-x^2 \right) ^{- 5 / 2}\\
h'''(0)=&1
		\end{align*}
	}
	\end{parts}
\end{proof}
\begin{Problem}
	Bestimmen Sie die Ober- und Untersummen von $\exp : [0, 1] \to\R$ für die markierten Zerlegungen $(J_n, \Xi_n)$ mit der Auswahl $\Xi_n = \left\{ 0,\frac{1}{n},\frac{1}{n},\dots,\frac{n-1}{n},1 \right\} $ für $n \in \N$. Zeigen Sie anschließend, dass die zugehörigen Ober- und Untersummen gegen denselben Wert konvergieren.
\end{Problem}

\begin{proof}
	\begin{parts}
	\item 
		\begin{Lemma}
			\[\lim_{n \to \infty} n\left( 1-e^{- 1 / n} \right) =1.\]
		\end{Lemma}
		\begin{proof}
			\begin{align*}
				\lim_{n \to \infty} n\left( 1-e^{-1 / n} \right) =&\lim_{n \to \infty} \frac{1 - e^{-1 / n}}{1 / n}\\
				=&\lim_{x \to 0^+} \frac{1-e^{-x}}{x} & x= 1 / n\\
				=&\lim_{x \to 0^+} \frac{e^{-x}}{1}&\text{L'Hopital}\\
				=&1\qedhere
			\end{align*}
		\end{proof}
\begin{align*}
	\mathfrak{O}_{\Xi_n}(f)=&\sum_{k=0}^{n-1} \left( \frac{1}{n}\exp\left( \frac{k+1}{n} \right)  \right)\\ 
	=&\frac{1}{n}\sum_{k=0}^{n-1} \exp\left( \frac{k+1}{n} \right)\\
	=& \frac{1}{n}\frac{(e-1)e^{1 / n}}{e^{1 / n}-1}\\
	=&\frac{1}{n}\frac{e-1}{1-e^{-1 / n}} 
\end{align*}
Es folgt daraus
\[
	\lim_{n \to \infty} \mathfrak{D}_{\Xi_n}(f)=\lim_{n \to \infty} \frac{e-1}{n\left( 1-e^{-1 / n} \right) }=e-1
.\] 
\item 
	\begin{align*}
		\mathfrak{U}_{\Xi_n}(f)=&\sum_{k=0}^{n-1} \left( \frac{1}{n}\exp\left( \frac{k}{n} \right)  \right) \\
		=&\frac{1}{n}\sum_{k=0}^{n-1} \exp\left( \frac{k}{n} \right)\\
		=&\frac{1}{n}\frac{e-1}{e^{1 / n}-1}\\
		=&\frac{1}{n}\frac{e-1}{e^{1 / n}\left( 1-e^{- 1 / n} \right) }
	\end{align*}
	Daraus folgt:
	\[
		\lim_{n \to \infty} \mathfrak{U}_{\Xi_n}(f)=\lim_{n \to \infty} \frac{e-1}{n(e^{1 / n}\left( 1-e^{-1 / n} \right) }=e-1
	.\qedhere\] 
	\end{parts}
\end{proof}
