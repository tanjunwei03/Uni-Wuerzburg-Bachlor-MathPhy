\documentclass[prb,12pt]{revtex4-2}

\usepackage{amsmath, amssymb,physics,amsfonts,amsthm}
\usepackage{enumitem}
\usepackage{cancel}
\usepackage{booktabs}
\usepackage{tikz}
\usepackage{hyperref}
\usepackage{enumitem}
\usepackage{transparent}
\usepackage{float}
\usepackage{multirow}
\newtheorem{Theorem}{Theorem}
\newtheorem{Proposition}{Theorem}
\newtheorem{Lemma}[Theorem]{Lemma}
\newtheorem{Corollary}[Theorem]{Corollary}
\newtheorem{Example}[Theorem]{Example}
\newtheorem{Remark}[Theorem]{Remark}
\theoremstyle{definition}
\newtheorem{Problem}{Problem}
\theoremstyle{definition}
\newtheorem{Definition}[Theorem]{Definition}
\newenvironment{parts}{\begin{enumerate}[label=(\alph*)]}{\end{enumerate}}
%tikz
\usetikzlibrary{patterns}
% definitions of number sets
\newcommand{\N}{\mathbb{N}}
\newcommand{\R}{\mathbb{R}}
\newcommand{\Z}{\mathbb{Z}}
\newcommand{\Q}{\mathbb{Q}}
\newcommand{\C}{\mathbb{C}}
\begin{document}
	\title{Analysis 2 Hausaufgabenblatt Nr. 1}
	\author{Jun Wei Tan}
	\email{jun-wei.tan@stud-mail.uni-wuerzburg.de}
	\affiliation{Julius-Maximilians-Universit\"{a}t W\"{u}rzburg}
	\author{Lukas Then}
	\affiliation{Julius-Maximilians-Universit\"{a}t W\"{u}rzburg}
	\date{\today}
	\maketitle
\begin{Problem}
Berechnen Sie Ableitungen der folgenden Funktionen:
\begin{parts}
	\item $f(x)=\frac{\arctan\sin x^2}{e^{1-x}}\text{ f\"{u}r }x\in \R$ 
	\item $g(x)=x^{(x^x)}\text{ f\"{u}r }x>0$
\end{parts}
\end{Problem}

	\begin{parts}
		\item 
			\[
				f(x)=e^{x-1}\arctan\sin x^2
			.\] 
			\begin{align*}
				f'(x)=& e^{x-1}\dv{x}\arctan\sin x^2+\left( \arctan\sin x^2 \right) \dv{x}e^{x-1}\\
			=&\frac{e^{x-1}}{1+\sin^2 x^2}\dv{x}\sin x^2+\left( \arctan\sin x^2 \right)e^{x-1}\dv{x}(x-1)\\
			=&\frac{e^{x-1}}{1+\sin^2 x^2}(\cos x^2)(2x)+(\arctan\sin x^2)e^{x-1}\\
			=& \frac{2x\cos x^2 e^{x-1}}{1+\sin^2 x^2}+(\arctan\sin x^2)e^{x-1}
			\end{align*}
		\item 
			\begin{align*}
				g(x)=& x^{(x^x)}\\
				\ln g(x)=& x^x\ln x
			\end{align*}

			\begin{Lemma}
				\begin{align*}
					h(x)&:=x^x\\
					h'(x)&=x^x(1+\ln x)
				\end{align*}
			\end{Lemma}
			\begin{proof}
				\[
				\ln h(x)=x\ln x
				.\] 
				\begin{align*}
					\dv{x}|\ln h(x)=&\dv{x}\left( x\ln x \right) \\
					\frac{h'(x)}{h(x)}=&\ln x+1\\
					h'(x)=&h(x)\left( 1+\ln x \right) \\
					=&x^x\left( 1+\ln x \right) 
				\end{align*}
			\end{proof}
			Dann gilt
	\begin{align*}
		\dv{x}\ln g(x)=&\dv{x}\left( x^x \ln x \right) \\
		\frac{g'(x)}{g(x)}=&\frac{1}{x}x^x+(\ln x)\dv{x}\left( x^x \right) \\
		=&\frac{x^x}{x}+(\ln x)x^x (1+\ln x)\\
		g'(x)=&g(x)x^x\left[ \frac{1}{x}+\ln x+\ln^2x \right] \\
		=& x^{x^x+x}\left[ \frac{1}{x}+\ln x+\ln^2x \right] \\
		=& x^{x^x+x-1}\left[ 1+x\ln x+x\ln^2 x \right] 
	\end{align*}
		\end{parts}

\begin{Problem}
	Untersuchen Sie, für welche Argumente des Definitionsbereiches die folgenden Funktionen differenzierbar sind:
	\begin{parts}
	\item $f(x)=|x|, x\in \R$ 
	\item $g(x)= \begin{cases}
			x^2 & x\in \Q \\
			0 & x\in \R \backslash \Q
	\end{cases}$ 
\item $h(z)=\overline{z}, z\in \C$
	\end{parts}
\end{Problem}

\begin{parts}
\item F\"{u}r $x_0\neq 0$ gibt es eine Umgebung auf $x_0$, worin $|x|=x$ oder $|x|=-x$. Dann ist die Ableitung von $|x|$ gleich mit die Ableitung von entweder $x$ oder $-x$, also $f'(x_0)$ existiert f\"{u}r $x_0\neq 0$.

	F\"{u}r $x_0=0$ gilt $|0|=0$, und auch
	\begin{align*}
		\lim_{x \to 0^{+}} \frac{f(x)-f(0)}{x-0}&=\lim_{x \to 0^{+}} \frac{x}{x}=1\\
		\lim_{x \to 0^{-}} \frac{f(x)-f(0)}{x-0}&=\lim_{x \to 0^{-}} \frac{-x}{x}=-1
	\end{align*}
	Weil die beide ungleich sind, existiert die Grenzwert und daher auch die Ableitung nicht.

\item Sei $x_0\neq 0$ und $y_0=x_0^2$. Dann f\"{u}r $0<\epsilon<y_0$ existiert keine $\delta>0$, sodass $|x-x_0|<\delta\implies |g(x)-g(x_0)|<\epsilon$.

	Beweis: Es gibt zwei F\"{a}lle:
	\begin{enumerate}[label=(\roman*)]
		\item $x_0\in \Q$. Dann in jeder offenen Ball $(x_0-\delta, x_0+\delta)$ gibt es ein Zahl $x \in \R \backslash \Q$, also $|g(x)-g(x_0)|=g(x_0)=y_0>\epsilon$

		\item Sei $x_0\in \R \backslash \Q$. Dann in jeder offenen Ball $(x_0-\delta, x_0+\delta)$ gibt es ein Zahl $x\in \Q$, also $|g(x)-g(x_0)|=g(x)>y_0>\epsilon$
	\end{enumerate}
	Sei $x_0=0$. Dann gilt $g(x_0)=0$, und auch:
	\begin{enumerate}[label=(\roman*)]
		\item $x\in \Q$, also
			\begin{align*}
				\frac{g(x)-g(0)}{x-0}=& \frac{x^2}{x}\\
				=& x
			\end{align*}

		\item oder $x \not\in \Q$, also
			\[
			g(x)-g(0)=0-0=0\implies \frac{g(x)-g(0)}{x-0}=0
			.\] 
	\end{enumerate}
	Deshalb ist
	\[
	g'(0)=\lim_{x \to 0} \frac{g(x)-g(0)}{x-0}=0
	.\]
\item Zu berechnen:
	\[
		\lim_{z \to z_0} \frac{f(z)-f(z_0)}{z-z_0}=\lim_{z \to z_0} \frac{\overline{z}-\overline{z_0}}{z-z_0}=\lim_{z \to z_0} \frac{\overline{z-z_0}}{z-z_0}
	.\] 

	Sei $z=z_0+x, x\in \R$. Dann, falls die Grenzwert existiert, ist es gleich
	\begin{align*}
		\lim_{z \to z_0} \frac{\overline{z-z_0}}{z-z_0}=& \lim_{x \to 0} \frac{\overline{z_0+x-z_0}}{z_0+x-z_0}\\
		=& \lim_{x \to 0} \frac{\overline{x}}{x}\\
		=&\lim_{x \to 0} \frac{x}{x}\\
		=& 1
	\end{align*}

	Sei jetzt $z=z_0+ix, x\in \R$. Falls die Grenzwert existiert ist es gleich
	\begin{align*}
		\lim_{z \to z_0} \frac{\overline{z-z_0}}{z-z_0}=&\lim_{x \to 0} \frac{\overline{z_0+ix-z_0}}{z_0+ix-z_0}\\
		=& \lim_{x \to 0} \frac{\overline{ix}}{ix}\\
		=& \lim_{x \to 0} \frac{-ix}{ix} \\
		=& -1
	\end{align*}

	Weil das Grenzwert, wenn durch zwei Richtungen berechnet wurde, ungleich ist, existiert das Grenzwert nicht (f\"{u}r alle $z\in \C$)
\end{parts}

\begin{Problem}
	Man zeige, dass die Gleichung
	\[
	x=\cos\left( \frac{\pi x}{2} \right) \] 
	auf $[0,1]$ genau eine L\"{o}sung besitzt.
\end{Problem}

Sei $f(x)=x-\cos\left( \frac{\pi x}{2} \right) $. Dann ist die Gleichung gleich $f(x)=0$. $f(x)$ ist auf $[0,1]$ stetig, und auf $(0,1)$ differenzierbar.
\begin{align*}
	f(0)=& 0-1=-1\\
	f(1)=& 1-0=1
\end{align*}
Wegen des Zwischenwertsatzes gibt es mindestens eine L\"{o}sung zu der Gleichung $f(x)=0$. Dann
\[
	f'(x)=1+\frac{\pi}{2}\sin\left( \frac{\pi x}{2} \right) >0\text{ f\"{u}r }x\in [0,1]
.\] 
$f$ is dann monoton wachsend, und es gibt maximal eine L\"{o}sung zu $f(x)=0$.

Deswegen besitzt die Gleichung genau eine L\"{o}sung.

\begin{Problem}
	Bestimmen Sie die folgenden Grenzwerte:
	\begin{parts}
	\item $\lim_{k \to \infty} k\ln \frac{k-1}{k}$ 
	\item  $\lim_{x \to \infty} \frac{x^{\ln x}}{e^x}$
	\end{parts}
\end{Problem}

\begin{parts}
\item \[
k\ln \frac{k-1}{k}=\frac{\ln (k-1)-\ln k}{1 / k}
.\] 
Weil $\ln x$ und $1 / x$ auf $x\in (0,\infty)$ differenzierbar sind, kann man den Satz von L'Hopital verwenden:
\begin{align*}
	\dv{k}\left[ \ln(k-1)-\ln k \right] =& \frac{1}{k-1}-\frac{1}{k}=\frac{1}{k(k-1)}\\
	\dv{k}\frac{1}{k}=&-\frac{1}{k^2}
\end{align*}
Dann gilt
\begin{align*}
	\lim_{k \to \infty}  \frac{\frac{1}{k(k-1)}}{-\frac{1}{k^2}}=&\lim_{k \to \infty} \left( -\frac{k}{k-1} \right)\\
	=& \lim_{k \to \infty} \left( -\frac{1}{1-\frac{1}{k}} \right) \\
	=&-1
\end{align*}
Weil das Grenzwert auf $\R\cup \left\{ \pm\infty \right\} $ existiert, ist
\[
\lim_{k \to \infty} k\ln\left( \frac{k-1}{k} \right) =-1
.\]
\item \[
		\frac{x^{\ln x}}{e^x}=\frac{\left( e^{\ln x} \right)^{\ln x}}{e^x}=\frac{e^{\ln^2 x}}{e^x}=e^{\ln^2 x-x}=e^{x\left( \frac{\ln^2 x}{x}-1 \right) }
.\]
\begin{Lemma}
	\[
	\lim_{x \to \infty} \frac{\ln^p x}{x^q}=0,\qquad p,q>0
	.\] 
\end{Lemma}
\begin{proof}
	\begin{align*}
		\lim_{x \to \infty} \frac{\ln^p x}{x^q}=&\left( \lim_{x \to \infty} \frac{\ln x}{x^{p / q}} \right)^q\\
		=&\left( \lim_{x \to \infty} \frac{1}{x(x^{p / q})} \right)^q & \text{L'Hopital}\\
		=& 0^q=0
	\end{align*}
\end{proof}
\begin{Corollary}
	\[
	\lim_{x \to \infty} \left[ x\left( \frac{\ln^2x}{x}-1 \right)  \right] =-\infty
	.\] 
\end{Corollary}
Deswegen ist
\[
	\lim_{x \to \infty} \frac{x^{\ln x}}{e^x}=\lim_{x \to \infty} e^{x\left( \frac{\ln^2 x}{x}-1 \right) }=0
.\] 
\end{parts}
\begin{Problem}
	Überprüfen Sie die Funktion $f : [-1, +\infty) \to \R$ definiert durch
	\[
	f(x)=\begin{cases}
		x^2+1 & -1\le x < 1\\
		\frac{8}{\pi}\arctan \frac{1}{x} & x \ge 1
	\end{cases}
	\]
auf lokale und globale Extrema.
\end{Problem}

Es gilt
\[
f'(x)=\begin{cases}
	2x & -1< x < 1 \\
	\frac{8}{\pi}\frac{1}{1+\frac{1}{x^2}}\left( -\frac{1}{x^2} \right) & x>1
\end{cases}
.\] 

Es ist klar, dass $x=0$ eine L\"{o}sung zu $f'(x)=0$ ist. Weil $f''(0)=2>0$, ist es ein lokales Minimum. Es gibt auch $a,b\in \R, a<1<b$, wof\"{u}r gilt
\begin{align*}
	& f'(x)>0 & x\in (a,1)\\
	& f'(x)<0 & x\in (1,b)
\end{align*}
Falls $f(1)\ge \lim_{x \to 1^{-}} f(x)$, ist $f(1)$ ein lokales Maximum (sogar wenn $f$ nicht auf $1$ stetig ist). Weil
\[
	\lim_{x \to 1^{-}} f(x)=\lim_{x \to 1^{-}} \left( x^2+1 \right) =2\] 
	und
	\[
	f(1)=\frac{8}{\pi}\arctan 1=\frac{8}{\pi}\frac{\pi}{4}=2\] 
	ist $f(1)$ ein lokales Maximum. Weil $f(x)<2$ f\"{u}r $x>1$ kann kein Punkt $x>1$ ein globales Maximum sein. Es gilt auch, dass
	\[
	f(-1)=(-1)^2+1=2
	.\] 
	Außer $x\in \left\{ -1,0,1 \right\} $ gibt es keine M\"{o}glichkeiten f\"{u}r ein globales Maximum. Daher sind die globale Maxima auf $x\in \left\{ -1,1 \right\} $ 

	F\"{u}r $x\in [1,1)$ gilt $f(x)\ge 1$. Dennoch ist
	\[
	\lim_{x \to \infty} \frac{8}{\pi}\arctan\left( \frac{1}{x} \right) =0
	.\] 
	Deswegen gibt es \emph{keine} globales Maximum auf $\R$. Wenn man $f(\infty)$ definiert durch $f(\infty)=\lim_{x \to \infty} f(x)$, ist $f(\infty)$ das globale Maximum.
\end{document}
