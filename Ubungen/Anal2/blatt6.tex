\begin{Problem}
	Im Folgenden ist $a\in \R$ und $b\in \R \cup \left\{ +\infty \right\} $. Zeigen oder widerlegen Sie:
	\begin{parts}
		\item Sind $f,g:[a,b)\to \R$ uneigentlich integrierbar, so auch $f+g$.
		\item Sind $f,g:[a,b)\to \R$ uneigentlich integrierbar, so auch $f\cdot g$.
		\item Sind $f:[a,b)\to \R$ uneigentlich integrierbar und $g:\R \to \R$ stetig, so auch $g\circ f$.
		\item Es sei $f:(0,1]\to \R$ stetig, nicht-negativ und uneigentlich integrierbar. Dann konvergiert
			\[
				\int_0^1 \sqrt{f(x)} \dd{x}
			.\] 
	\end{parts}
\end{Problem}
\begin{proof}
	\begin{parts}
	\item Sei $a<c<b$. Wir wissen, dass $f+g$ auf $[a,c]$ integrierbar f\"{u}r alle solchen $c $ ist und außerdem
		\[
			\int_a^c (f+g)(x)\dd{x}=\int_a^c f(x)\dd{x}+\int_a^c f(x)\dd{x}
		.\] 
		Es gilt dann
		\begin{align*}
			\lim_{c \to b} \int_a^b (f+g)(x)\dd{x}=&\lim_{c \to b} \int_a^c f(x)\dd{x}+\lim_{c \to \infty} \int_a^c g(x)\dd{x}\\
			=&\int_a^b f(x)\dd{x}+\int_a^b g(x)\dd{x}
		\end{align*}
		also $f+g$ ist integrierbar.
	\end{parts}
\end{proof}
\begin{Problem}
Untersuchen Sie die folgenden uneigentlichen Riemannintegrale auf Konvergenz bzw. absolute Konvergenz:
\begin{parts}
\item $\int_{-\infty}^\infty e^{-x^2}\dd{x}$,
\item $\int_{-\infty}^\infty \frac{\sin x}{x}\dd{x}$,
\item $\int_{-\infty}^\infty \sin\left( x^2 \right) \dd{x}$.
\end{parts}
\end{Problem}
\begin{proof}
	\begin{parts}
	\item Wir betrachten zuerst $\int_1^\infty e^{-x^2}\dd{x}$. Weil $e^{-x^2}$ monoton fallend ist, konvergiert das Integral genau dann, wenn
		\[
			\sum_{k=1}^\infty e^{-k^2}
		\]
		konvergiert. Es gilt $e^{-k^2}\ge e^{-k}$ f\"{u}r $k\in \N$ und
		\[
			\sum_{k=1}^\infty e^{-k}
		\]
		konvergiert, weil es eine geometrische Reihe ist. 

		Nachdem wir gezeigt haben, dass $\int_{-\infty}^{1}f(x)\dd{x}$ existiert, existiert auch das Integral $\int_{-\infty}^\infty e^{-x^2}\dd{x}$. Es gilt, f\"{u}r $c\le -1$, dass
		\[
			\int_{c}^1 f(x)\dd{x}=\int_{c}^{-1} f(x)\dd{x}+\int_{-1}^1 f(x)\dd{x}
		\]
		und
		\[
			\lim_{c \to -\infty} \int_c^1 f(x)\dd{x}=\lim_{c \to -\infty} \int_{c}^{-1}f(x)\dd{x}+\int_{-1}^1f(x)\dd{x}
		.\] 
		Weil $\exp(-x^2)$ auf $\R$ stetig ist, ist $\exp(-x^2)$ auf $[-1,1]$ integrierbar, also das Grenzwert $\int_{-\infty}^1 e^{-x^2}\dd{x}$ existiert genau dann, wenn $\int_c^{-1}e^{-x^2}\dd{x}$. Aber wir wissen, weil $\exp(-(-x)^2)=\exp(-x^2)$, dass
		\[
			\int_{c}^{-1}e^{-x^2}\dd{x}=\int_1^c e^{-x^2}\dd{x}
		,\] 
		also
		\[
			\lim_{c \to -\infty} \int_{c}^{-1}e^{-x^2}\dd{x}=\lim_{c \to \infty} \int_1^c e^{-x^2}\dd{x}
		.\] 
		Weil das Grenzwert auf der rechten Seite existiert existiert das auf der linken Seite, also $\int_{-\infty}^\infty e^{-x^2}\dd{x}$ existiert.

		Weil $e^{-x^2}>0$ f\"{u}r alle $x\in \R$, konvergiert das Integral genau dann, wenn es absolut konvergiert, also es konvergiert absolut.
	\item Es steht schon im Skript, dass
		\[
			\int_0^\infty \frac{\sin x}{x}\dd{x}
		\]
		existiert. Aus
		\[
			\frac{\sin(-x)}{-x}=\frac{\sin x}{x}
		.\] 
		existiert auch $\int_{-\infty}^0 \frac{\sin x}{x}$, daher auch
		\[
			\int_{-\infty}^\infty \frac{\sin x}{x}\dd{x}
		.\] 
		Wir untersuchen es also f\"{u}r absolut Konvergenz. Es gilt
		\begin{align*}
			\int_0^{2\pi N}\left|\frac{\sin x}{x}\right|\dd{x}=&\sum_{j=0}^{N-1}\int_{2\pi j}^{2\pi(j+1)}\left|\frac{\sin x}{x}\right|\dd{x}\\
			\ge&\sum_{j=0}^{N-1}\frac{1}{2\pi (j+1)}\int_{2\pi j}^{2\pi(j+1)}|\sin x|\dd{x}\\
			=&\sum_{j=1}^{N-1}\frac{2}{\pi(j+1)}
		\end{align*}
		Dann nehmen wir die Limes $N\to \infty$. Weil die Summe divergent ist, ist das Integral auch divergent.
	\item Wie vorher ist $\sin((-x)^2)=\sin x$, also wir müssen nur zeigen, dass $\int_0^\infty \sin(x^2)$ konvergiert bzw. absolut konvergiert.

		Sei $u=x^2,~\dd{u}=2x\dd{x}=2\sqrt{u} \dd{x}$ und $\R\ni a,b, 0<a<b$. Es folgt:
		\[
			\int_a^b \sin x^2\dd{x}=\int_{a^2}^{b^2} \frac{\sin u}{2\sqrt{u} }\dd{u} 
		.\] 
		Dann machen wir partielle Integration
		\begin{align*}
			\int_{a^2}^{b^2}\frac{\sin u}{\sqrt{u} }\dd{u}=&\left.\frac{-\cos u}{\sqrt{u} }\right|_{a^2}^{b^2}-\int_{a^2}^{b^2}\frac{\cos u}{2u^{\frac{3}{2}}}\dd{u}\\
				\left| \int_{a^2}^{b^2}\frac{\sin u}{\sqrt{u} }\dd{u} \right|\le&\left| \left.\frac{-\cos u}{\sqrt{u} }\right|_{a^2}^{b^2} \right|+\int_{a^2}^{b^2} \frac{1}{2u^{3 / 2}}\dd{u}\\
					\ge& \frac{1}{b}+\frac{1}{a}-\left. \frac{1}{\sqrt{u} }\right|_{a^2}^{b^2}\\
						=& \frac{2}{a},
		\end{align*}
		was unabhängig von der oberen Grenze ist. Also $\int_a^\infty x^2 \sin(x^2)	$ konvergiert f\"{u}r alle $a$. Da $\sin(x^2)$ überall in $\R$ definiert und stetig ist, ist auch
		\[
			\int_0^\infty \sin(x^2)\dd{x}
		\]
		konvergent. Daraus folgt, dass
		\[
			\int_{-\infty}^\infty\sin(x^2)\dd{x}
		\]
		konvergent ist. Wir untersuchen es dann f\"{u}r absolut Konvergenz. Es gilt
		\begin{align*}
			\int_0^{2\pi N}\left| \frac{\sin x}{2\sqrt{x} } \right| \dd{x}=&\sum_{j=1}^{N-1}\int_{2\pi j}^{2\pi (j+1)}\frac{\sin x}{2\sqrt{x} }\dd{x}\\
			\ge&\sum_{j=0}^{N-1}\frac{1}{2\sqrt{2\pi j} }\int_{2\pi j}^{2\pi (j+1)}|\sin x|\dd{x}\\
			=&\sum_{j=0}^{N-1}\frac{2}{\sqrt{2\pi} \sqrt{j} }
		\end{align*}
	\end{parts}
	Weil die Summe divergent ist, divergiert auch das Integral.
\end{proof}
\begin{Problem}
	Beweisen Sie
	\[
	\lim_{n \to \infty} \left( \frac{1}{n+1}+\frac{1}{n+2}+\dots+\frac{1}{2n} \right)=\ln 2 
	.\] 
\end{Problem}
\begin{proof}
	Es gilt, f\"{u}r alle $n\in \N$,
\begin{align*}
	\ln 2=&\ln 2n-\ln n\\
	=&\int_1^{2n} \frac{1}{x}\dd{x}-\int_1^n \frac{1}{x}\dd{x}\\
	=& \int_n^{2n}\frac{1}{x}\dd{x}
\end{align*}
also wir brauchen
\[
	\lim_{n \to \infty} \int_n^{2n}\frac{1}{x}\dd{x}=\lim_{n \to \infty} \sum_{j=n}^{2n}\frac{1}{j}
.\] 
Es gilt
\[
	\sum_{j=n}^{2n}\frac{1}{j}\le \int_n^{2n}\frac{1}{x}\dd{x}\le \sum_{j=n}^{2n} \frac{1}{j+1}
\]
weil $\frac{1}{x}$ auf $(0,\infty)$ monoton fallend ist. Außerdem ist
\begin{align*}
	\lim_{n \to \infty} \left[ \sum_{j=n}^{2n}\frac{1}{j}-\sum_{j=n}^{2n}\frac{1}{j+1} \right] =& \lim_{n \to \infty} \left[ \frac{1}{n}-\frac{1}{2n+1} \right] \\
	=& 0
\end{align*}
also wenn wir die Limes gegen $\infty$ nehmen, konvergiert die beide Summen nach dem gleichen Wert, also das Integral. 
\[
	\lim_{n \to \infty} \sum_{j=n}^{2n}\frac{1}{j}=\lim_{n \to \infty} \int_n^{2n}\frac{1}{x}\dd{x}
.\] 
Die Behauptung folgt.
\end{proof}
\begin{Problem}
	Wir zeigen die Irrationalität von $\pi$ durch einen Widerspruch. Angenommen, es gälte $\pi=\frac{a}{b}$ f\"{u}r $a,b\in \N$. Außerdem seien $f,F:[0,\pi]\to \R$ definiert durch
	\begin{align*}
		f(x)=&x^n\frac{(a-bx)^n}{n!},\\
		F(x)=&f(x)-f^{(2)}(x)+f^{(4)}(x)-\dots+(-1)^nf^{(2n)}(x).
	\end{align*}
	f\"{u}r $n\in \N$.
	\begin{parts}
	\item Zeigen Sie, dass $f^{(k)}(0),f^{(k)}(\pi)\in \N$ f\"{u}r alle $k\in \N$ gilt.
	\item Zeigen Sie
		\[
		f(x)\sin x=\left( F'(x)\sin x-F(x)\cos x \right)'
		.\] 
		Folgern Sie anschließend, dass $F(\pi)+F(0)$ ebenfalls eine natürliche Zahl ist, was allerdings nicht mit den Eigenschaften von $x\to f(x)\sin x$ f\"{u}r hinreichend großes $n\in \N$ vereinbar ist. 
	\end{parts}
\end{Problem}
\begin{proof}
	\begin{parts}
	\item Wir leiten es ab. Sei $g(x)=x^n,~h(x)=(a-bx)^n$. Es gilt
		\[
			f^{(p)}(x)=\frac{1}{n!}\sum_{k=0}^n \binom{p}{k}g^{(k)}(x)h^{(p-k)}(x)
		.\] 
		Es gilt außerdem
		\[
			g^{(k)}(x)=n(n-1)\dots(n-k+1)x^{n-k}
		,\]
		also $g^{(k)}(0)\neq 0$ genau dann, wenn $k=n$ und in diesem Fall ist $g^{(n)}(0)=n!$. Ähnlich ist $h^{(k)}(x)=n(n-1)\dots(n-k+1)(-b)^k(a-bx)^{n-k}$, also $h^{(k)}(0)\in \N$. Dann im Summe müssen wir nurdem Fall $k=n$ betrachten, also
		\[
			f^{(p)}(0)=\binom{p}{n}h^{(p-k)}(0)\in \N
		.\] 
		Ähnlich bei $\pi$ gilt $g^{(k)}(\pi)=n(n-1)\dots (n-k+1)\pi^{n-k}$ und
		\begin{align*}
			g^{(k)}(\pi)=&n(n-1)\dots(n-k+1)(-b)^k(a-b\pi)^{n-k}
=&n(n-1)\dots(n-k+1)(-b)^k(a-b(a / b))^{n-k}\\
=&\begin{cases}
	0 & k \neq n\\
	n!(-b)^n & k = n
\end{cases}
		\end{align*}
		also wir betrachten im Summe nur den Fall $p-k=n$, also
		\[
		f^{(p)}(x)=\frac{1}{n!}\binom{p}{p-n}g^{(p-n)}(\pi)h^{(n)}(\pi)=\binom{p}{p-n}g^{(p-n)}(\pi)(n!)(-b)^n\in \N
		.\] 
	\item Es gilt
		\begin{align*}
			\left( F'(x)\sin x-F(x)\cos x \right)'=& F''(x)\sin x+F'(x)\cos x-F'(x)\cos x+F(x)\sin x\\
			=& \left[ F''(x)+F(x) \right] \sin x\\
			=&\sin(x)[\cancel{f^{(2)}(x)}-\cancel{f^{(4)}(x)}+\cancel{f^{(6)}(x)}-\dots+(-1)^nf^{(2n+2)}(x)\\
			 &+f(x)-\cancel{f^{(2)}(x)}+\cancel{f^{(4)}(x)}-\dots+\cancel{(-1)^n f^{(2n)}(x)}]\\
			=&f(x)\sin x+(-1)^n f^{(2n+2)}(x).
		\end{align*}
		Es bleibt zu zeigen: $f^{(2n+2)}(x)=0$. Wir wissen: $g^{(n+1)}(x)=h^{(n+1)}(x)=0$. Daraus folgt:
	\end{parts}
\end{proof}
