\begin{Problem}
	\textbf{(Stetigkeit, partielle und totale Differenzierbarkeit)}	Sind die Funktionen mit den Funktionswerten
	\begin{parts}
		\item $f(x,y)=(x^2+y^2)^{1 / 4}$,
		\item $f(x,y)=\begin{cases}
				(x^2+y^2)\sin\left( \frac{1}{\sqrt{x^2+y^2} } \right) & \text{f\"{u}r }(x,y)\neq (0,0),\\
				0 & \text{sonst,}
		\end{cases}$ 
	\item $f(x,y)=\begin{cases}
			\frac{y(x^2+y^2)^{3 / 2}}{(x^2+y^2)^2+y^2} & \text{f\"{u}r }(x,y)\neq (0,0),\\
			0 & \text{sonst}
	\end{cases}$
	\end{parts}
	stetig, partiell oder total differenzierbar in $(0,0)$?
\end{Problem}
\begin{proof}
	\begin{parts}
	\item Die Funktion ist stetig. Sei $\epsilon>0$ gegeben. Sei $\delta=\epsilon^2$. Dann f\"{u}r alle $r\in \R^2$, so dass $\|r-0\|=\|r\|<\delta$ gilt $f(x,y)=(\|r\|^2)^{1 / 4}=\|r\|^{1 / 2}<\epsilon$.

		Die Funktion ist nicht partiell differenzierbar. F\"{u}r die Gerade $x=0$ gilt $f(0,y)=(y^2)^{1 / 4}=\sqrt{|y|} $. Aber $g(y)=\sqrt{|y|} $ ist nicht bei $0$ differenzierbar. Ähnlich ist sie auch nicht durch $x$ partiell differenzierbar.

		Weil die Funktion nicht partiell differenzierbar ist, ist sie auch nicht total differenzierbar.
	\item Die Funktion ist stetig. Es gilt
		\[
		\left|(x^2+y^2)\sin\left( \frac{1}{\sqrt{x^2+y^2} } \right)  \right| \le (x^2+y^2)
	.\]
	Da $x^2+y^2\to 0$ wenn $(x,y)\to (0,0)$, gilt es auch f\"{u} $f(x,y)$ und $f(x,y)$ ist in $(0,0)$ stetig.

	Sie ist nicht partiell differenzierbar. z.B. F\"{u}r die Gerade $y=0$ ist $f(x,0)=x^2 \sin(1 / |x|)$, was nicht differenzierbar bei $0$ ist. Ähnlich existiert auch $\pdv{f}{y}(0,0)$ nicht. 

	Weil $f$ nicht partiell differenzierbar ist, ist $f$ nicht total differenzierbar.
\item Die Funktion ist stetig. Es gilt
	\begin{align*}
		f(x,y)=& \frac{y(x^2+y^2)^{3 / 2}}{(x^2+y^2)^2+y^2}\\
		=&\frac{y}{\sqrt{x^2+y^2} +\frac{y}{(x^2+y^2)^{3 / 2}}}
	\end{align*}
	Da $\frac{y}{(x^2+y^2)^{3 / 2} }\to \infty$ wenn $(x,y)\to (0,0)$, geht die Funktion $f(x,y)\to 0$. Dann ist $\lim_{(x,y)\to (0,0)}f(x,y)=0$, also die Funktion ist stetig.

	Sie ist auch partiell differenzierbar. Da $f(x,0)=0$, ist $\pdv{f}{x}(0,0)=0$. Es gilt
	\[
		f(0,y)=\frac{y(y^2)^{3 / 2}}{y^4+y^2}
	.\] 
	Daraus folgt:
	\begin{align*}
		\lim_{y \to 0} \frac{f(0,y)-f(0,0)}{y}=&\lim_{y \to 0}  \frac{(y^2)^{3 / 2}}{y^4+y^2}\\
		=&\lim_{y \to 0} \frac{|y|^{3}}{y^4+y^2}\\
		=&\lim_{y \to 0} \frac{|y|}{y^2+1} 
	\end{align*}
	Da $\frac{1}{y^2+1}$ stetig ist und $\frac{1}{y^2+1}\to 0$ wenn $y\to 0$, existiert der Grenzwert. Der Grenzwert ist auch $0$. 

	Sie ist aber nicht total differenzierbar. Falls eine Ableitung existiere, kann die Ableitung durch die partielle Ableitungen dargestellt werden. Diese Darstellung liefert eine totale Ableitung von $0$. Es würde also gelten:
	\[
		\lim_{(x,y) \to (0,0)} \frac{f(x,y)-f(0,0)}{\|(x,y)\|}=\lim_{(x,y) \to (0,0)} \frac{f(x,y)}{\|(x,y)\|}= 0
	.\]
	Es gilt aber
	\begin{align*}
		\lim_{(x,y) \to (0,0)} \frac{f(x,y)}{\|(x,y)\|}=& \lim_{(x,y) \to (0,0)}  \frac{f(x,y)}{\sqrt{x^2+y^2} }\\
		=&\lim_{(x,y) \to (0,0)} \frac{y(x^2+y^2)}{(x^2+y^2)^2+y^2}
	\end{align*}
	was nicht existiert.\qedhere
	\end{parts}
\end{proof}
\begin{Problem}
	\textbf{(Tangenten von Kurven)} Für eine stetig differenzierbare Kurve $\gamma:[a,b]\to \R$ heißt $t\in [a,b]$ ein regulärer Punkt, falls $\gamma'(t)\neq 0$. Andernfalls nennen wir $t$ ein singulären Punkt.

Bestimmen Sie die Menge der regulären/singulären Punkte folgender Kurven:	
\begin{parts}
\item $\gamma_1:\R\to \R^2$ mit $\gamma_1(t)=(t^2,t^3)^T$,
\item $\gamma_2:[0,2\pi]\to \R^3$ mit $\gamma_2(t)=(\cos^3(t),\sin^3(t))^T$,
\item $\gamma_3:[0,2\pi]\to \R^3$ mit $\gamma_3(t)=(t\cos t,t\sin t, t)^T$.
\end{parts}
\end{Problem}
\begin{proof}
	\begin{parts}
	\item $\gamma_1(t)=(2t,3t^2)^T$.

		Singulären Punkte: $\{0\} $.
		
		Regularären Punkte: $\R\backslash \{0\} $.
	\item $\gamma_2'(t)=(3\cos^2(t)(-\sin t), 2\sin^2(t)\cos t)$, also

		Singulären Punkte: $S=\{0,\pi / 2, \pi, 3\pi / 2, 2\pi\} $

		Regulären Punkte: $[0,2\pi] \backslash S$.
	\item  $\gamma_3'(t)=(\cos t-t\sin t,\sin t-t\cos t,1)^T$

		Die Ableitung ist nie der Nullvektor, also

		Singulären Punkte: $\varnothing$ 

		Regulären Punkte: $[0,2\pi]$.\qedhere
	\end{parts}
\end{proof}
\begin{Problem}
	\textbf{(Rechnen mit der Kettenregel)} Der reelwertigen Funktionen $f(u_1,\dots, u_n)$ und $u_1(x_1,\dots, x_m),\dots, u_n(x_1,\dots, x_m)$ seien auf den offenen Mengen $U\subset \R^n$ bzw. $G\subset \R^m$ erklärt, und die Funktion
	\[
	\varphi(x_1,\dots, x_m):=f(u_1(x_1,\dots, x_m),\dots, u_n(x_1,\dots, x_m))
\]
existiere auf $G$.

Im Folgenden ist jeweils die Ableitung $D\varphi$ der Funktion $\varphi$ zu berechnen:
\begin{parts}
\item $f(u,v,w)=u^2+v^2+w^2;\qquad u(t)=e^t\cos t,~v(t)=e^t\sin t,~w(t)=e^t$,
\item $f(u,v)=\ln(u^2+v^2)$ f\"{u}r $(u,v)\neq (0,0)$; $u(x,y)=xy,~v(x,y)=\sqrt{x} / y$ f\"{u}r $x,y>0$,
\item $f(u,v,w)=uv+vw-uw;\qquad u(x,y)=x+y,v(x,y)=x+y^2,w(x,y)=x^2+y$.
\end{parts}
\end{Problem}
\begin{proof}
	\begin{parts}
	\item Sei $g:\R\to \R^3,t\to (u(t), v(t), w(t))^T$. Es gilt
\begin{align*}
	D(f\circ g)(t)=&Df(g(t))Dg(t)\\
	=&(2u, 2v, 2w)(u'(t), v'(t), w'(t))^T\\
	=&2uu'(t)+2vv'(t)+2ww'(t)\\
	=&2(e^t\cos t)(e^t\cos t-e^t\sin t)\\
	 &+2(e^t\sin t)(e^t\sin t+e^t\cos t)+2e^{2t}\\
	=&4e^{2t}.
\end{align*}
\item Sei $g:\R^2\to \R^2, (x,y)\to (u(x,y),v(x,y))^T$. Es gilt
	\begin{align*}
		Df=&\left( \frac{2u}{u^2+v^2}, \frac{2v}{u^2+v^2} \right)\\
		Dg=&\begin{pmatrix} y & x \\ \frac{1}{2\sqrt{x} y} & \sqrt{x}  \end{pmatrix} 
	\end{align*}
und daher
	\begin{align*}
		D(f\circ g)(x,y)=&Df(g(x,y))Dg(x,y)\\
		=&\left( \frac{2xy}{x^2y^2+\frac{x}{y^2}}, \frac{2\sqrt{x} /y}{x^2y^2+\frac{x}{y^2}} \right) \begin{pmatrix} y & x \\ \frac{1}{2\sqrt{x} y}& \sqrt{x}  \end{pmatrix} \\
		=&\left( \frac{1+2xy^4}{x+x^2y^4}, \frac{2y(1+xy^2)}{1+xy^4} \right) 
	\end{align*}
\item Sei $g:\R^2\to \R^3, (x,y)\to (u(x,y), v(x,y), w(x,y))^T$. Es gilt
	\begin{align*}
		Df(u,v,w)=&(v-w, u+w, v-u)\\
		Dg(x,y,z)=&\begin{pmatrix} 1 & 1\\ 1 & 2y\\2x & 1 \end{pmatrix} 
	\end{align*}
	Daraus folgt:
	\begin{align*}
		D(f\circ g)(x,y)=&Df(g(x,y))Dg(x,y)\\
		=&(x+y^2-x^2-y, x^2+x+2y, y^2-y)\\
		 &\cdot\begin{pmatrix} 1 & 1 \\ 1 & 2y \\ 2x & 1 \end{pmatrix}\\
		=&(y(1+y)+2x(1-y+y^2),\\
		 &x+2xy+x^2(2y-1)+2y(3y-1)).\qedhere
	\end{align*}
	\end{parts}
\end{proof}
\begin{Problem}
Beweisen Sie die Differenzierbarkeit der folgenden Funktionen und geben Sie die Ableitung an:	
\begin{parts}
\item $f(x)=x^TAx$ f\"{u}r $x\in \R^n$ und ein $A\in \R^{n\times n}$,
\item $f(X,Y)=XY$ f\"{u}r $(X,Y)\in \R^{m\times n}\times \R^{n\times k}$.
\end{parts}
\end{Problem}
\begin{proof}
	\begin{parts}
	\item Wir berechnen $f'(x_0)$. Es gilt
		\begin{align*}
			f(x_0+\delta x)=&(x_0+\delta x)^T A(x_0+\delta x)\\
			=&x_0Ax_0+(\delta x)^TAx_0+(x_0)^TA(\delta x)\\
			 &+(\delta x)^T A(\delta x)\\
			=&f(x_0)+((x_0)^TA^T\delta x)^T+(x_0)^TA(\delta x)\\
			 &+(\delta x)^T A(\delta x)\\
			=&f(x_0)+x_0^T A^T(\delta x)+(x_0)^TA(\delta x) & (x_0)^TA^T \delta x\in \R\\
			 &+(\delta x)^TA(\delta x)\\
			=&f(x_0)+x_0^T(A^T+A)(\delta x)+(\delta x)^T A(\delta x)
		\end{align*}
		Wir idenfizieren $Df(x_0)=x_0^T(A^T+A)$. Es bleibt zu zeigen, dass $(\delta x)^T A(\delta x)$ eigentlich die Restabbildung ist. Da
		\[
			\lim_{\|\delta x\| \to 0} \left|\frac{(\delta x)^T}{\|\delta x\|} A\delta x\right|\le\lim_{\|\delta x\| \to 0} \|A\| \|\delta x\|=0
		,\]
		gilt die Behauptung.
	\item Ähnlich berechnen wir $f'(X_0,Y_0)$. Es gilt
		\begin{align*}
			f(X_0+\delta X, Y_0+\delta Y)=&(X_0+\delta X)(Y_0+\delta Y)\\
			=&X_0Y_0+(\delta X)Y_0+X_0(\delta Y)+(\delta X)(\delta Y)
		\end{align*}
		Da $\frac{(\delta X)(\delta Y)}{|(\delta X, \delta Y)}\to 0$ wenn $|(\delta X, \delta Y)|\to 0$, ist die Ableitung
		\[
		J_f((\delta X, \delta Y))=(\delta X)Y_0+X_0(\delta Y)
		.\qedhere\] 
	\end{parts}
\end{proof}

\begin{Problem}
	Zeigen Sie, dass die Funktion $f(x,y)=xy$ f\"{u}r $(x,y)\in \R^2$ einen kritischen Punkt in $(x,y)=(0,0)$ besitzt, aber kein Extremum.
\end{Problem}
\begin{proof}
	\[
	f'(x,y)=(y,x)^T
\]
und $f'(0,0)=(0,0)$. Die Funktion besitzt in $(0,0)$ daher einen kritischen Punkt. Es ist aber kein Extremum. Es gilt $f(0,0)=0$. Es ist kein Maximum, weil auf der Gerade $x=y=t$ gilt $f(t,t)=t^2>0$ f\"{u}r $t\neq 0$, also in jede offene Menge bzw. offenem Kugel gibt es mindestens ein Punkt $(x,y)=(t,t)$, so dass $f(x,y)>0=f(0,0)$. Ähnlich gilt, auf der Gerade $(x,y)=(t,-t)$, $f(t,-t)=-t^2<0$, also $f(0,0)$ ist kein Minimum. Dann  besitzt $f$ kein Extremum in $(0,0)$.
\end{proof}
