\begin{Problem}
	Begründen Sie, warum die Determinante $\text{det}:\R^{n\times n}\to \R$ unendlich oft differenzierbar ist, und bestimmen Sie die Ableitung im Punkt $X\in \R^{n\times n}$. Welche besondere Form nimmt $(D\text{det})(\text{Id})$ an?  	
\end{Problem}
\begin{proof}
	Per Definition ist
	\[
		\text{det}(X)=\sum_{\sigma\in S_n} \text{sgn}(\sigma)X_{1\sigma(1)}X_{2\sigma(2)}\dots X_{n\sigma(n)}
	.\] 
	Da dies eine Summe von Produkte der Komponenten sind, ist $\text{det}$ unendlich oft differenzierbar.

	Es gilt $\text{det}':\R^{n\times n}\to\text{Hom}(\R^{n\times n},\R)$. Wir betrachten eine parameterabhängige Matrix $A(t)$. Es gilt
	\begin{align*}
	\dv{t}\text{det}A(t)=&\dv{t}\sum_{\sigma\in S_n}\text{sgn}(\sigma)A_{1\sigma(1)}(t)\dots A_{n\sigma(n)}(t)\\
		=&\sum_{\sigma\in S_n}\left[ (A_{1\sigma(1)}\dots A_{n\sigma(n)}) \sum_{j=1}^n\frac{A_{j\sigma(j)}'(t)}{A_{j\sigma(j)}} \right] \\
				=&\sum_{j=1}^n\sum_{\sigma\in S_n}\left[ (A_{1\sigma(1)}\dots A_{n\sigma(n)}) \frac{A_{j\sigma(j)}'(t)}{A_{j\sigma(j)}} \right] 
	\end{align*}
	F\"{u}r festes $j$ kann dieser Ausdruck so vorgestellt werden: Wir setzten in der $j$-te Zeile $A'$ statt $A$, also 
			\begin{center}
		\begin{tikzpicture}
			\matrix (M)[matrix of math nodes, row sep = 0.5cm, column sep = 0.5cm,left delimiter={(},right delimiter={)}]{
				A_{11}(t) & A_{12}(t) & \dots & A_{1n}(t)\\
				A_{21}(t) & A_{22}(t) & \dots & A_{2n}(t)\\
				\vdots & \vdots  & \ddots & \vdots\\
				A_{j1}'(t) & A_{j2}'(t) & \dots & A_{jn}'(t)\\
				\vdots & \vdots & \ddots & \vdots\\
				A_{n1}(t) & A_{n2}(t) & \dots & A_{nn}(t)\\
			};
			\draw[thick, blue] (-3.5,-0.5) -- (3.5,-0.5);
			\draw (3.6,-0.5) node[anchor=west,blue] {$j$-te Zeile};
		\end{tikzpicture}
	\end{center}
	und berechnen deren Determinante. Wir führen eine Laplaceentwicklung um die $j$-te Zeile durch und erhalten, mit $A^\#$ die adjugierte Matrix
	\begin{align*}
		\dv{t}\text{det} A(t)=&\sum_{j=1}^n\sum_{k=1}^n A_{jk}'(t)A^{\#}_{kj}\\
		=&\sum_{k=1}^n\sum_{j=1}^n A^{\#}_{kj}A_{jk}'(t)\\
		=&\sum_{k=1}^n (A^{\#} A')_{kk}\\
		=&\text{tr}(A^{\#}A')
	\end{align*}
	Also die Ableitung $\text{det}'$ im Punkt $A$ ist die Abbildung, die $B$ nach $\text{tr}(A^{\#}B)$ abbildet. 

	Im Punkt $I$: Es gilt $I^\#=I$, also $D\text{det}(I)=\text{tr}$.
\end{proof}
\begin{Problem}
	Ist $U\subset \R^n$ offen und $f:U\to \R$ zweimal differenzierbar, so kann die zweite Ableitung $D^2f$ in jedem Punkt $x\in U$ durch eine blineare Abbildung $\text{Hom}(\R^n,\R^n;\R)$ darstellen. F\"{u}r eine gegebene Basis auf dem $\R^n$-wir wählen die kanonische Basis hier - lassen sich bilineare Abbildungen durch eine Matrix $A\in \R^{n\times n}$ darstellen, die sog. Hesse-Matrix $Hf$ mit
	\[
		Hf(x,y)=\left[ \pdv[2]{f}{x_i}{x_j}(x) \right]_{i,j=1,\dots, n}=\begin{pmatrix} \pdv[2]{f}{x_1}{x_1} & \dots & \pdv[2]{f}{x_1}{x_n} \\ \vdots & \ddots & \vdots \\ \pdv[2]{f}{x_1}{x_n} & \dots & \pdv[2]{f}{x_n}{x_n}\end{pmatrix} 
	.\] 
Hier wurde schon ausgenutzt, dass die partiellen Ableitungen nach dem Satz von Schwarz vertauschen, die Hesse-Matrix ist also symmetrisch.

Es sei nun $f:\R^n\to \R$ mit
\[
	f(x)=\frac{1}{2}x^T Ax=\frac{1}{2}\sum_{i,j=1}^n a_{ij}x_i x_j
\]
f\"{u}r eine Matrix $A\in \R^{n\times n}$.
\begin{parts}
	\item Zeigen Sie, dass $f$ in $(0,0)$ ein Minimum, Maximum oder Sattelpunkt genau dann besitzt, wenn die Hesse-Matrix von $f$ (in (0,0)) positiv semi- negativ semi- bzw. indefinit ist. 
	\item Zeigen Sie durch ein Beispiel, dass für allgemeine Funktionen mit positiv (bzw. negativ) semidefiniter Hesse-Matrix im kritischen Punkt kein lokales Minimum (bzw. Maximum) vorliegen muss.  
\end{parts}
\end{Problem}
\begin{proof}
	\begin{parts}
	\item Offensichtlich ist $A$ die Hesse-Matrix und $f(0)=0$. Wir drücken $f(x)$ als $\frac{1}{2}\langle x, Ax\rangle$ aus.

	Dann per Definition: Falls $A$ positiv semidefinit ist, ist $f(x)=\frac{1}{2}\langle x, Ax\rangle\ge 0$ f\"{u}r alle $x$, also $f$ besitzt ein Minimum in $(0,0)$. Die Umkehrrichtung gilt auch per Definition. Sei $U$ eine Umgebung. F\"{u}r $x\in U$ ist $\langle x, Ax\rangle \ge 0$ per Definition eines Minimums. 

	Das gleiche Argument (per Definition) gilt f\"{u}r $f$ ein Maximum in $(0,0)$.
\item Die Funktion $f:\R\to\R$ mit $f(x)=x^3$ besitzt in $0$ Ableitung $0$. $0$ ist sowohl positiv als auch negativ definit. Die Funktion besitzt aber weder ein Maximum noch ein Minimum in $0$.
\end{parts}
\end{proof}
\begin{Problem}
	Es sei
	\[
		F:\R^2\to \R^2, \qquad F(x,y)=(x^2-y^2,2xy)^T
	.\] 
	\begin{parts}
	\item Berechnen Sie die Jacobi-Matrix von $F$.
	\item In welchem Punkt $p\in \R^2$ existiert die Inverse von $JF(p)$?
	\item Finden Sie eine lokale inverse Abbildung $F^{-1}$ von $F$ in einer Umgebung von $p=(1,0)=F(1,0)$ und berechnen Sie die Ableitung von $F^{-1}$ in $p$.
	\item Ist $F$ auf dem ganzen Gebiet $\{p\in \R^2|JF(p)\text{ invertierbar}\} $ global invertierbar?
	\end{parts}
\end{Problem}
\begin{proof}
	\begin{parts}
	\item
		\[
			JF=\begin{pmatrix} 2x & -2y \\ 2y & 2x \end{pmatrix} 
		.\] 
	\item $\text{det}(JF)=4x^2+4y^2$. Dies ist genau dann Null, wenn $x=y=0$. Sonst ist $JF$ invertierbar.
	\item Wir lösen das Gleichungssystem:
		\begin{align*}
			x^2-y^2=&\xi\\
			2xy=&\zeta\\
			y=&\frac{\zeta}{2x}\\
			x^2-\left( \frac{\zeta}{2x} \right)^2=&\xi\\
			x^4-\frac{\zeta^2}{4}=&\xi x^2\\
			x^4 - \xi x^2-\frac{\zeta^2}{4}=&0\\
			x^2=\frac{1}{2}\left[ \xi\pm \sqrt{\xi^2+\zeta^2}  \right] 
		\end{align*}
		Beim Punkt $(x,y)=(1,0)$ ist $(\xi,\zeta)=(1,0)$. In einer genug kleinen Umgebung nehmen wir daher die positive Nullstelle und danach positive Würzel.
		\[
			x=\left[ \frac{\xi+\sqrt{\xi^2+\zeta^2} }{2} \right]^{1 / 2}
		.\] 
		Daraus folgt:
		\[
			y=\frac{\zeta}{2}\left[ \frac{2}{\xi+\sqrt{\xi^2+\zeta^2} } \right]^{1 / 2}
		.\] 
		Dies ist die Umkehrabbildung.	
	\item Nein, weil sie nicht injektiv ist. Es gilt $F(1,1)=(0,2)$ und $F(-1,-1)=(0,2)$, aber $(1,1)\neq (-1,-1)$.\qedhere 
	\end{parts}
\end{proof}
\begin{Problem}
	Mithilfe des Satzes über implizite Funktionen beweisen wir die Glattheit der Inversen-Abbildung $\text{inf}:GL(n)\to GL(n)$. Gehen Sie wie folgt vor:
	\begin{parts}
	\item Begründen Sie, dass die Abbildung $A\cdot B\to AB$ auf $\R^{(n\times n)^2}$ unendlich oft differenzierbar ist.
	\item Nutzen Sie den Satz über implizite Funktionen, um $\text{inf}\in \mathcal{C}^\infty (GL(n),GL(n))$ zu beweisen.

	\emph{Hinweis: Betrachten Sie} $A\cdot B=\text{Id}$
	\end{parts}
\end{Problem}
\begin{proof}
	\begin{parts}
	\item Das Produkt $AB$ besitzt Komponente $(AB)_{ij}=\sum_{k=1}^n A_{ik}B_{kj}$. Dann sind alle Komponenten von $AB$ nach alle Komponenten von $A$ und $B$ unendlich stetig differenzierbar, da die lineare Kombinationen von Produkten von $AB$ sind.
	\item Wir betrachten $g:\R^{(n\times n)^2}\to \R^{n\times n}$ mit $g(A,B)\to AB-I$. Falls die Ableitung $(D_2g)(A,B)$ invertierbar ist, ist die Abbildung $\text{inv}:A\mapsto B$ glatt. 

		Da $I$ konstant ist, leiten wir also das Produkt $AB$ nach $B$ ab. Dies ist aber bekanntermaßen glatt, also nach dem Satz von inverse Funktionen ist $\text{inv}$ glatt.\qedhere
	\end{parts}
\end{proof}
