\documentclass[prb,12pt]{revtex4-2}

%font
\usepackage[T1]{fontenc}
\usepackage[osf,sc]{mathpazo}
%actual preamble
\usepackage{amsmath, amssymb,physics,amsfonts,amsthm}
\usepackage{enumitem}
\usepackage{cancel}
\usepackage{booktabs}
\usepackage{tikz}
\usepackage{hyperref}
\usepackage{enumitem}
\usepackage{transparent}
\usepackage{float}
\usepackage{multirow}
\newtheorem{Theorem}{Theorem}
\newtheorem{Proposition}[Theorem]{Proposition}
\newtheorem{Lemma}[Theorem]{Lemma}
\newtheorem{Corollary}[Theorem]{Corollary}
\newtheorem{Example}[Theorem]{Example}
\newtheorem{Remark}[Theorem]{Remark}
\theoremstyle{definition}
\newtheorem{Problem}{Problem}
\theoremstyle{definition}
\newtheorem{Definition}[Theorem]{Definition}
\newenvironment{parts}{\begin{enumerate}[label=(\alph*)]}{\end{enumerate}}
%tikz
\usetikzlibrary{patterns}
% definitions of number sets
\newcommand{\N}{\mathbb{N}}
\newcommand{\R}{\mathbb{R}}
\newcommand{\Z}{\mathbb{Z}}
\newcommand{\Q}{\mathbb{Q}}
\newcommand{\C}{\mathbb{C}}
\begin{document}
	\title{Analysis 2 Hausaufgabenblatt Nr. 2}
	\author{Jun Wei Tan}
	\email{jun-wei.tan@stud-mail.uni-wuerzburg.de}
	\affiliation{Julius-Maximilians-Universit\"{a}t W\"{u}rzburg}
	\author{Lukas Then}
	\affiliation{Julius-Maximilians-Universit\"{a}t W\"{u}rzburg}
	\date{\today}
	\maketitle
\begin{Problem}
	Es seien $f, g : D \to \R$  $n$-mal differenzierbare Funktionen für $n \in \N\backslash 0$ und $D \subset  \mathbb{K}$ offen. Zeigen Sie, dass $f \cdot g$ ebenfalls $n$-mal differenzierbar ist und weiterhin
	\[
		(f\cdot g)^{(n)}(x)=\sum_{k=0}^{n} f^{(k)}(x)g^{(n-k)}(x)\] 
		für jedes $x\in D$ gilt.
\end{Problem}
\begin{proof}
	Wir zeigen es per Induktion, für $n=1$ ist es das Produktregel. Nehme jetzt an, dass $f,g$ $(n+1)-$ mal differenzierbar Funktionen sind und 
	\[
		(f\cdot g)^{(n)}(x)=\sum_{k=0}^{n} f^{(k)}(x)g^{(n-k)}(x)\]
		gilt (weil alle $(n+1)$-mal differenzierbar Funktionen sind auch $n$-mal differenzierbar). Dann ist $(f\cdot g)^{(n)}(x)$ differenzierbar, weil die rechte Seite eine Linearkombination von Produkte aus (zumindest) einmal differenzierbar Funktionen. Es gilt auch,
		\begin{align*}
			(f\cdot g)^{(n)}(x)=&\dv{x}\sum_{k=0}^{n}\binom{n}{k} f^{(k)}(x)g^{(n-k)}(x)\\
			=& \sum_{k=0}^{n} \binom{n}{k}\left( f^{(k+1)}(x)g^{(n-k)}(x)+f^{(k)}(x)g^{(n-k+1)}(x) \right) & n=1\text{ Fall} \\
			=& \sum_{k=0}^{n} \binom{n}{k}f^{(k+1)}(x)g^{(n-k)}(x)+\sum_{k=0}^{n} f^{(k)}(x)g^{(n-k+1)}(x)\\
			=&\sum_{k=1}^{n+1} \binom{n}{k-1}f^{(k)}(x)g^{(n-k+1)}(x)+\sum_{k=0}^{n}\binom{n}{k} f^{(k)}(x)g^{(n-k+1)}(x)\\
			=&\sum_{k=0}^{n+1} \binom{n}{k}f^{(k)}g^{(n-k)}(x)
		\end{align*}
\end{proof}
\begin{Problem}
	\begin{enumerate}[label=\roman*)]
		\item Betrachten Sie die Funktionenfolge $f_n:\R \to \R$ definiert durch
\[
f_n(x)=\frac{1}{n}\sqrt{n^2x^2+1} 
.\] 
Beweisen Sie, dass $(f_n),n\in \N$ gegen eine zu bestimmende Grenzfunktion $f : \R \to \R$  gleichmäßig konvergiert, diese jedoch nicht differenzierbar auf $\R$  ist. Warum ist das kein Widerspruch zu Proposition 5.5.2?  
\item Untersuchen Sie
	\[
	f(x)=\sum_{n=1}^{\infty} \frac{\cos(nx)}{n^3}, x\in \R
	.\] 
	auf Differenzierbarkeit.
	\end{enumerate}
\end{Problem}
\begin{proof}
	\begin{enumerate}[label=\roman*)]
		\item \[
		f_n(x)=\sqrt{x^2+\frac{1}{n^2}} 
		.\] 
		Es ist klar, dass $f_n(x)$ konvergiert gegen $\sqrt{x^2} =|x|$. Sei dann $r(x)=\sqrt{x^2+\frac{1}{n^2}} -|x|$. F\"{u}r $x>0$ gilt
		\begin{align*}
			r(x)=& \sqrt{x^2+\frac{1}{n^2}}-x\\
			r'(x)=& \frac{x}{\sqrt{x^2+\frac{1}{n^2}} }-1\le 0
		\end{align*}
		Deswegen ist $r(x)$ monoton fallend auf $(0,\infty)$. Ähnlich beweist man, dass $r(x)$ monoton wachsend auf $(-\infty, 0)$ ist. Deswegen ist $x=0$ ein globales Maximum, und $r(x)\le r(0)=\frac{1}{n}$. Daher konvergiert $(f_n)$ gleichmäßig.

		Man berechnet:
		\[
			f_n'(x)=\frac{x}{\sqrt{x^2+\frac{1}{n^2}} }
		.\] 
		Die Folge der Ableitungen konvergiert gegen $\frac{x}{\sqrt{x^2} }=\text{sgn}(x)$, falls $x\neq 0$, und $0$, falls $x=0$. Es konvergiert aber nicht lokal gleichmäßig in eine Umgebung $U$ auf $0$.

		Sei $1>\epsilon>0$ gegeben, und nehme an, dass existiere $N\in \N$, f\"{u}r die gilt, 
		\[
		|f_n(x)'-g(x)|\le \epsilon\qquad n>N,x\in U
		,\] 
		wobei
		\[
		f'(x)=\begin{cases}
			1 & x > 0\\
			-1 & x < 0\\
			0 & x = 0
		\end{cases}
		.\] 
Nehme eine solche Abbildung $f_n'(x)$. Weil $f'_n$ stetig ist, und $f_n'(0)=0$, gibt es eine Umgebung $0\in V$, in der gilt, dass $|f_n'(x)-f_n'(0)|=f_n'(x)\le 1-\epsilon, x\in V$. Sei dann  $0\neq x\in V$, und $|1-f_n'(x)|>\epsilon$. Deswegen ist es kein Widerspruch.

\item Es gilt $\left|\frac{\cos(nx)}{n^3}\right|\le \frac{1}{n^3}$. Daher konvergiert die Reihe gleichmäßig (Weierstraßsches Majorantenkriterium).

	Jetzt ist $\dv{x} \frac{\cos(nx)}{n^3}=-\frac{\cancel{n}\sin(nx)}{n^{\cancelto{2}{3}}}$. Weil $\left|\frac{\sin(nx)}{n^2}\right|\le \frac{1}{n^2}$, konvergiert $\sum_{n=1}^{\infty} \left| \frac{\sin(nx)}{n^2} \right|$ gleichmäßig. Deswegen ist $f$ differenzierbar, mit Ableitung
	\[
		f'(x)=\sum_{n=1}^{\infty} \left[ -\frac{\sin(nx)}{n^2} \right] 
	.\qedhere\] 
\end{enumerate}
\end{proof}
\begin{Problem}
	Zeigen Sie, dass die Funktion
	\[f:[-1,1]\to\R\qquad f(x)=\text{max}\left\{ x,0 \right\} \]
	gleichmäßig durch Polynome approximiert werden kann.
\end{Problem}
\begin{proof}
	Wir wissen schon, dass es $q_n(x)$ existiert, $q_n(x)$ Polynome, und $q_n(x)\to |x|$ gleichmäßig. Es gilt auch
	\[
	f(x)=\frac{|x|}{2}+\frac{x}{2}
	.\] 
	Daher konvergiert gleichmäßig
	\[
		\frac{q_n(x)}{2}+\frac{x}{2}\to f(x)
	.\qedhere\] 
\end{proof}
\begin{Problem}
\begin{enumerate}[label=\roman*)]
	\item  Es seien $f : (a, b) \to (c, d)$ und $g : (c, d) \to  R$  $n$-mal differenzierbare Funktionen mit $n \in \N_0$. Zeigen Sie, dass auch $g \circ f$ $n$-mal differenzierbar ist.  
	\item Zeigen Sie, dass $f : \R \to \R$ mit
		\[
		f(x)=\begin{cases}
			\exp\left( -\frac{1}{x^2} \right) & x\neq 0\\
			0 & x = 0 
		\end{cases}
		\]
		eine unendlich oft differenzierbare Funktion definiert ist. Bestimmen Sie zudem $f^{(n)}(0)$ für alle $n\in \N$
\end{enumerate}	
\end{Problem}
\begin{proof}
	\begin{enumerate}[label=\roman*)]
		\item \ldots
		\item 
			\begin{Lemma}
				\[
				\lim_{x \to \infty} \frac{(\ln x)^p}{x^k}=0, k > 0
				.\] 
			\end{Lemma}
			\begin{proof}
				Wir beweisen es per Induktion auf $p$. F\"{u}r $p=1$ verwenden wir den Satz von L'Hopital
				\[
					\lim_{x \to \infty} \frac{\ln x}{x^k}=\lim_{x \to \infty} \frac{1}{kx^{k-1}(k)}=\lim_{x \to \infty} \frac{1}{kx^k}=0
				.\qedhere\] 
				Jetzt nehme an, dass es f\"{u}r $p$ gilt. Wir zeigen, dass es f\"{u}r $p\to p+1$ auch gilt. 
				\[
					\lim_{x \to \infty} \frac{(\ln x)^{p+1}}{x^k}=\lim_{x \to \infty} \frac{(p+1)(\ln x)^p}{kx^{k-1}(x)}=\frac{p+1}{k}\lim_{\xi \to \infty} \frac{(\ln x)^p}{x^k}=0
				.\] 
			\end{proof}
			\begin{Lemma}
			\[
				\lim_{x \to \infty} x^p e^{-kx}=0, k>0
			.\] 	
			\end{Lemma}
			\begin{proof}
				Nimm $x=e^\xi$. Dann gilt
				\[
					\lim_{x \to \infty} x^{-k}(\ln x)^p=\lim_{x \to \infty} e^{-k\xi}\xi^p=0
				.\qedhere\] 
			\end{proof}
			Die Ableitungen $f^{(n)}(x), x\neq 0$ haben den Form $p_n\left( \frac{1}{x} \right) \exp\left( \frac{1}{x^2} \right) $, wobei $p_n(x)$ eine Polynome ist. 
			\begin{Proposition}
				$f^{(n)}(0)=0$
			\end{Proposition}
			\begin{proof}
				Wir beweisen es per Induktion. $f^{(0)}(0)=0$ per Definition.
				\begin{align*}
					f^{(n)}(0)=&\lim_{x \to 0} \frac{f^{(n-1)}(x)-f^{(n-1)}(0)}{x-0}\\
					=&\lim_{x \to 0}  \frac{1}{x}\left( f^{(n-1)}(x) \right)\\
					=&\lim_{x \to 0} \frac{1}{x}p_{n-1}\left(\frac{1}{x}\right)e^{-\frac{1}{x^2}}\\
					=&\lim_{x \to \infty} xp_n(x)e^{-x}\\
					=&0
				\end{align*}
				Deswegen ist $f$ überall (inkl. 0) differenzierbar, mit alle Ableitungen $f^{(n)}(0)=0$
			\end{proof}
	\end{enumerate}
\end{proof}
\begin{Problem}
	Es seien $K_1 , K_2 \subset \mathbb{K}$  nichtleere, kompakte Mengen und die Folgen stetiger Funktionen $f_n : K_1 \to K_2$ sowie $g_n : K_2 \to K$  seien gleichmäßig konvergent gegen $f : K_1 \to K_2$ bzw. $f : K_2 \to K$. Beweisen Sie, dass auch
	\[
	g_n\circ f_n \to g \circ f\]
	gleichmäßig auf $K_1$ gilt.
\end{Problem}

\begin{proof}
	Sei $\epsilon>0$ gegeben. Dann per Definition existiert $n_2\in\N$, sodass 
	\begin{equation}\label{eq:anal2u2eq1}
		|g_n(x)-g(x)|<\frac{\epsilon}{2}, x\in K_2, n\ge n_2 
	\end{equation}
	Weil $g$ stetig und auf eine kompakte Menge definiert ist, ist $g$ gleichmäßig stetig, und es existiert $\delta>0$, f\"{u}r die gilt
	\begin{equation}\label{eq:anal2u2eq2}
		|g(a)-g(b)|<\frac{\epsilon}{2}, \qquad |a-b|<\delta
	\end{equation}
	Es gibt auch $n_1\in\N$, $|f_n(x)-f(x)|<\delta, x\in K_1, n\ge n_1$. F\"{u}r $n>n_1$ gilt daher auch
	\begin{equation}\label{eq:anal2u2eq3}
		|g(f_n(x))-g(f(x))|<\frac{\epsilon}{2}, n>n_1, x\in K_1
	\end{equation}
	Sei $N=\max(n_1,n_2)$. F\"{u}r $n\ge N$ gilt Eq.~\eqref{eq:anal2u2eq1} und Eq.~\eqref{eq:anal2u2eq3} auch, weil $N\ge n_1$ und $N\ge n_2$. Dann f\"{u}r $n\ge N$ gilt.
	\begin{align*}
		|g(f(x))-g_n(f_n(x))|		=&|g(f(x))-g(f_n(x))+g(f_n(x))-g_n(f_n(x))|\\
		\le&\underbrace{|g(f(x))-g(f_n(x))|}_{< \epsilon / 2~\eqref{eq:anal2u2eq3}}+\underbrace{|g(f_n(x))-g_n(f_n(x))|}_{< \epsilon / 2~\eqref{eq:anal2u2eq1}}\\
		<\epsilon
	\end{align*}
	Also $g_n\circ f_n\to g\circ f$ gleichmäßig.
\end{proof}
	\end{document}
