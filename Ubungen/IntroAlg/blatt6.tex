\begin{Problem}
	Seien $G,H$ Gruppen und $\Phi:G\to H$ ein Homomorphismus.
	\begin{parts}
		\item Sei $g\in G$ ein Element mit Ordnung $n\in \N^*$. Zeigen Sie, dass $\text{ord}(\Phi(g))|n$ gilt.
		\item Sei $N\trianglelefteq G$. Ist dann stets auch $\Phi(N)\trianglelefteq H$?
	\end{parts}
\end{Problem}
\begin{proof}
	\begin{parts}
	\item Es gilt
		\[
		\Phi(g)^n=\Phi(g^n)=\Phi(1_G)=1_H
	,\]
	also $\text{ord}(\Phi(g))\le n$. Wir beweisen die Aussage per Widerspruch. Sei $\text{ord}(\Phi(g))=p\nmid n$. Wir schreiben
	\[
	n=qp+r,\qquad r<p
\]
(Division mit Rest). Es gilt dann
\begin{align*}
	\Phi(g)^n=&\Phi(g)^{qp+r}\\
	=&\Phi(g)^{qp}\Phi(g)^r\\
	=&\left( \Phi(g)^p \right)^q\Phi(g)^r\\
	=&1^q\Phi(g)^r\\
	=&\Phi(g)^r\\
	\neq& 1
\end{align*}
Im letzten Schritt haben wir benutzt, dass $r<p=\text{ord}(\Phi(g))$, also $\Phi(g)^r\neq 1$, sonst wäre $\text{ord}(\Phi(g))=r$.
\item Nein. Wir betrachten eine ``Einbettung" $\text{em}:S_n\to S_m$, wobei $m>n$. Sei $\sigma\in S_n$. Es gilt $\text{em}(\sigma)=\sigma'$ genau dann, wenn
	\[
	\sigma'(i)=\begin{cases}
		\sigma(i) & i\le n\\
		i & \text{sonst.}
	\end{cases}
	.\] 
	Es ist klar, dass $\Phi$ ein Homomorphismus ist. Sei $\sigma_1,\sigma_2\in S_n$, mit $\Phi(\sigma_1)=\sigma_1'$ und $\Phi(\sigma_2)=\sigma_2'$. Dann ist
	\[
	\sigma_1'\sigma_2'(i)=\sigma_1'(i)=i\qquad n < i \le m
,\]
also $\sigma_1'\sigma_2'|_{\left\{ 1,2,\dots, n \right\} }$ ein Element von $S_n$, dessen Bild $\sigma_1'\sigma_2'$ ist.

Wir wissen, dass $A_n\trianglelefteq S_n$. 

\textbf{Konkretes Gegenbeispiel:} Sei $n=3,m=5$. Wir betrachten $\text{em}:S_3\to S_4$, und $\Phi(A_3)$, wobei $A_3\trianglelefteq S_3$.

Dann ist $\Phi(A_3)$ kein Normalteiler von $S_4$. Unter Zweckentfremdung der Notation stellen wir $A_3$ und $\Phi(A_3)$ in Zykelnotation dar
\[
	A_3=\Phi(A_3)=\{(123), (132)\}
	.\]
Es gilt auch $(14)\in S_4$ und $(14)^{-1}=(14)$.

Es gilt aber
\[
	(14)(123)(14)=(14)(1234)=(234)\not\in  \Phi(A_3)
,\] 
also $\Phi(A_3)\not\trianglelefteq H$.\qedhere
	\end{parts}
\end{proof}
\begin{Problem}
	Seien $G$ eine Gruppe und $M:=\left\{ x^2|x\in G \right\} $.
	\begin{parts}
	\item Zeigen Sie, dass $N:=\left<M \right>$ ein Normailteiler von $G$ ist.
	\item Beweisen Sie, dass für jedes Element $gN$ der Faktorgruppe $G / N$ gilt: $\text{ord}(gN)\le 2$.
	\end{parts}
\end{Problem}
\begin{proof}
	\begin{parts}
	\item Sei $x,y\in G$ beliebig. Es gilt $x^2\in M$. Es ist $y^{-1}xy\in G$. Dann gilt
		\begin{align*}
			y^{-1}x^2y=&y^{-1}x xy\\
			=&y^{-1}xyy^{-1}xy\\
			=&(y^{-1}xy)^2\\
			\in& M,
		\end{align*}
		also $M$ ist ein Normalteiler von $G$.
	\item Wir betrachten das Quadrat eine Linksnebenklasse. Sei $a\in G$ beliebig, also $aN$ ist eine beliebige Linksnebenklasse.
		\[
		aN\cdot aN=(a^2)N
		.\] 
		Wir wissen aber, dass $a^2 \in N$ (per Definition). Weil $N$ ein Normalteiler ist, gilt dann $a^2N=N$. Es folgt, dass $\text{ord}(aN)\le 2$.\qedhere 
	\end{parts}
\end{proof}
\begin{Problem}
	Geben Sie ein Beispiel für eine Gruppe mit Untergruppen $A,B\le G$ an, so dass
	\[
		A\trianglelefteq B\qquad\text{und}\qquad B\trianglelefteq G
	\]
	gelten, nicht jedoch $A\trianglelefteq G$.

	{\footnotesize \textbf{Hinweis:} Dies zeigt, dass das Normalteiler-Sein im Allgemeinen nicht transitiv ist.}
\end{Problem}
\begin{proof}
	Wir betrachten $G=D_4$, $B=\left<r^2,s \right>$ und $A=\left<s \right>$. Wir zeigen die Eigenschaften
	\begin{enumerate}[label=(\roman*)]
		\item $B\trianglelefteq G$.

			Es gilt $|D_4|=8$ und $B=\left\{ 1,r^2,s,r^2s \right\} $, also $|B|=4,~[G:B]=2$. Dann ist $B$ stets ein Normalteiler (Bsp 2.35).

		\item $A\trianglelefteq B$.

			Wir betrachten $x^{-1}Ax$ f\"{u}r $x\in B$. F\"{u}r $x\in\left\{ 1,s \right\} $ ist es klar. F\"{u}r $x\in \left\{ r^2,r^2s \right\} $ muss wir direkt das Produkt mit $x^{-1}sx$ berechnen.

			Es gilt
			\[
				r^{-2}s r^{2}=s r^4=s
			,\]
			$(r^2s)^{-1}=s r^2$ (man kann das durch direktes Multiplikation verifizieren)
			\[
			s r^2 s s r^2=s r^4=s
			.\]
			Insgesamt ist $A\trianglelefteq B$.
		\item $A\not\trianglelefteq G$. 

			Es gilt $r^{-1} s r=r^{-2} s=r^2s\neq 0$. Wenn wir $D_4\subseteq S_{\C}$ betrachten, ist $r^2s(1_C)=-1$, wobei $1_C$ das $1$ in $\C$ ist (also nicht das neutrale Element in $\text{Sym}_{\C}$).
	\end{enumerate}
\end{proof}
\begin{Problem}
	Zeigen Sie, dass
	\[
	S_n=\left<(12),(123\dots n) \right>\] 
	f\"{u}r jedes $n\in \N^*$ gilt.
\end{Problem}
\begin{proof}
	Wir zeigen, dass Elemente mit bestimmte Formen Elemente von $\left<(12),(123\dots n) \right>$ sind. Im Beweis begründen wir alle Schritte mit ``Sonst wäre $\left<(12),(123\dots n) \right>$ keine Gruppe, weil es nicht abgeschlossen wäre".

	Außerdem bedeutet hier Addition immer die Addition in $\Z / n \Z$, aber durch 1 verschoben, also die mögliche Ergebnisse sind $1,2,\dots n$ statt $0,1,\dots, n-1$. Wir bezeichnen $(12)=s$ und $(123\dots n)=T$. 
	\begin{enumerate}[label=(\roman*)]
		\item Alle Transpositionen $(i(i+1))$. Die Proposition ist
			\[
				T^{x}sT^{n-x}=((x+1)(x+2))
			.\] 
			Es gilt
			\[
				T^{n-x}=\begin{pmatrix} 1 & 2 & \dots & n-1 & n \\ 1+(n-x) & 2+(n-x) & \dots & n-1+(n-x) & n+(n-x) \end{pmatrix} 
			.\]
			Dann ist
			\[
				sT^{n-x}=\begin{pmatrix} 1 & \dots & 1+x & 2 + x & \dots & n\\
				1+(n-x) & \dots & 2 & 1 & \dots & 2n-x\end{pmatrix} 
			,\] 
			also
			\begin{align*}
				T^xsT^{n-x}=& \begin{pmatrix} 1 & \dots & 1+x & 2+x & \dots & n \\ 1+(n-x)+x & \dots & 2+x & 1+x & \dots & 2n-x+x \end{pmatrix} \\
				=& \begin{pmatrix}  1 & \dots & 1+x & 2+x & \dots & n \\ 1 + n & \dots & 2 + x & 1 + x & \dots & 2n \end{pmatrix} \\
				=& \begin{pmatrix} 1 & \dots & 1+x & 2+x & \dots & n \\ 1 & \dots & 2+x & 1+x & \dots & n \end{pmatrix} 
			\end{align*}
		\item Alle Transpositionen $(i(i+k))$, f\"{u}r alle $i\in \left\{ 1,2,\dots, n \right\} $ und $k\in \left\{ 1,2,\dots,n-1 \right\} $.

			Wir beweisen es per Induktion über $k$. Wir wissen es schon f\"{u}r $k=1$. Jetzt nehmen wir an, dass alle Transpositionen $i(i+k')\in \left<(12)(123\dots n) \right>$ f\"{u}r alle $i\in \left\{ 1,2,\dots n \right\} $ und $k'\in \left\{ 1,2,\dots, k-1 \right\} $.

			Wir betrachten $(i(i+k))$ f\"{u}r $i$ beliebig. Ziel:
			\begin{equation}\label{eq:introalgblatt4-1}
				(i(i+k))=(i(i+k-1))((i+k-1)(i+k))(i(i+k-1))
			.\end{equation} 
			Wir betrachten die Wirkung auf $i$, $i+k-1$ und $i+k$. Es ist klar, dass keine andere Zahlen nicht davon bewegt werden. Es gilt
\begin{align*}
	&(i(i+k-1))((i+k-1)(i+k))(i(i+k-1))i\\
	&=(i(i+k-1))((i+k-1)(i+k))(i+k-1)\\
	&=(i(i+k-1))(i+k)\\
	&= i+k\\
	&(i(i+k-1))((i+k-1)(i+k))(i(i+k-1))(i+k)\\
	&=(i(i+k-1))((i+k-1)(i+k))(i+k)\\
	&=(i(i+k-1))(i+k-1)\\
	&=i\\
	&(i(i+k-1))((i+k-1)(i+k))(i(i+k-1))(i+k-1)\\
	&=(i(i+k-1))((i+k-1)(i+k))i\\
	&=(i(i+k-1))i\\
	&=i+k-1,
\end{align*}
also die Gleichheit in \eqref{eq:introalgblatt4-1} gilt.
\item Alle Elemente $\sigma \in S_n$.

	Wir schreiben ein beliebiges Element $\sigma\in S_n$ als Produkt von Transpositionen. Weil alle Transpositionen Elemente von $\left<(12)(123\dots n) \right>$ sind, müssen dann $\sigma\in \left<(12)(123\dots n) \right>$ auch.\qedhere
	\end{enumerate}
\end{proof}
