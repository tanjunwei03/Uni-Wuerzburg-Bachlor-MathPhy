\documentclass[prb,12pt]{revtex4-2}
%fonts
% Palatino for main text and math
\usepackage[osf,sc]{mathpazo}

% Helvetica for sans serif
% (scaled to match size of Palatino)
\usepackage[scaled=0.90]{helvet}

% Bera Mono for monospaced
% (scaled to match size of Palatino)
\usepackage[scaled=0.85]{beramono}
\usepackage{amsmath, amssymb,physics,amsfonts,amsthm}
\usepackage{enumitem}
\usepackage{cancel}
\usepackage[most]{tcolorbox}
\usepackage{booktabs}
\usepackage{tikz}
\usepackage{hyperref}
\usepackage{enumitem}
\usepackage{transparent}
\usepackage{float}
\usepackage{multirow}
\newtheorem{Theorem}{Theorem}
\newtheorem{Proposition}{Proposition}
\newtheorem{Lemma}[Theorem]{Lemma}
\newtheorem{Corollary}[Theorem]{Corollary}
\newtheorem{Example}[Theorem]{Example}
\newtheorem{Remark}[Theorem]{Remark}
\theoremstyle{definition}
\newtheorem{Problem}{Problem}
\theoremstyle{definition}
\newtheorem{Definition}[Theorem]{Definition}
\newenvironment{parts}{\begin{enumerate}[label=(\alph*)]}{\end{enumerate}}
%tikz
\usetikzlibrary{patterns}
\usepackage{tikz-cd}
% definitions of number sets
\newcommand{\N}{\mathbb{N}}
\newcommand{\R}{\mathbb{R}}
\newcommand{\Z}{\mathbb{Z}}
\newcommand{\Q}{\mathbb{Q}}
\newcommand{\C}{\mathbb{C}}
\begin{document}
	\title{Einf\"{u}rung in die Algebra Hausaufgaben Blatt Nr. 9}
	\author{Jun Wei Tan}
	\email{jun-wei.tan@stud-mail.uni-wuerzburg.de}
	\affiliation{Julius-Maximilians-Universit\"{a}t W\"{u}rzburg}
	\date{\today}
	\maketitle
	\section{Zahlentheorie}
	\begin{Theorem}
		(Division mit Rest)
	\[
	a=bq+r,~0\le r\le |b|-1
	.\] 
\end{Theorem}
\begin{Theorem}
	Sei $\Z^2\ni (a,b)\neq (0,0)$. Dann gibt es $s,t\in \Z$ mit
	\[
	ggT(a,b)=sa+tb
	.\] 
\end{Theorem}
\begin{Theorem}
	Sei $\Z^2\ni (a,b)\neq (0,0)$ und $d$ ein Teiler von $a$ und $b$. Es gilt
	\[
		d\cdot \text{ggT}\left( \frac{a}{b},\frac{b}{d} \right) =\text{ggT}(a,b)
	.\] 
\end{Theorem}
\begin{Theorem}
	Seien $a,b,c\in \Z$. Sind $a,b$ teilerfremd, gilt
	\begin{parts}
		\item $a|bc\implies a|c$ 
		\item $a|c$ und $b|c\implies ab|c$.
		\item $\text{ggT}(a,bc)=\text{ggT}(a,c)$
	\end{parts}
\end{Theorem}
\begin{Theorem}
	Sei $a,b\in \N^*$.
	\[
		ab=\text{ggT}(a,b)\cdot \text{kgV}(a,b)
	.\] 
\end{Theorem}
\begin{Theorem}
	 Sei $a\in \Z$, $p$ eine Primzahl, $a\nmid p$. Dann ist
	 \[
		 a^{p-1}-1\equiv 0\pmod{p}
	 .\] 
\end{Theorem}
	\section{Allgemein Gruppentheorie}
	\section{Gruppenhomomorphismen}
	\begin{Theorem}
		Sei $\phi:G\to H$ ein Homomorphismus. Es gilt $\text{ord}(\phi(g))|\text{ord}(g)$.
	\end{Theorem}
	\begin{Theorem}
		Sei $U\le G,N\trianglelefteq G$. 
		\[
		U / U\cap N \cong UN / N
		.\] 
	\end{Theorem}
	\begin{Theorem}
		Sei $K\trianglelefteq G, K\subseteq H\le G$. $H\trianglelefteq G$ genau dann, wenn $H / K \trianglelefteq G / K$. In diesem Fall ist
		\[
		\frac{G}{H}\cong \frac{G / K}{H / K}
		.\] 
	\end{Theorem}
	\section{Gruppenoperationen}
	\begin{Definition}
		Eine Operation ist ein Homomorphismus $G\to \text{Sym}(M)$.
	\end{Definition}
	\begin{Theorem}
		Die Länge der Bahn durch $m\in M$ ist $[G : G_m]$.
	\end{Theorem}
	\begin{Theorem}
		Sei $m,n\in M$ in der gleichen Bahn. Dann sind die Stabilisatoren konjugiert.
	\end{Theorem}
	\begin{Theorem}
		Sei $m_1,\dots, m_r$ Repräsentanten der Bahnen.
		\[
			|M|=\sum_{i=1}^r [G:G_{m_i}]
		.\] 
	\end{Theorem}
	\begin{Theorem}
		Sei $m_1,\dots, m_r$ Repräsentanten der Konjugationsklassen, die großer als $1$ sind. Es gilt
		\[
			|G|=|Z(G)|+\sum_{i=1}^r [G:G_{m_i}]
		.\] 
	\end{Theorem}
	\section{Abelsche Gruppen}
\begin{Theorem}
Sei n die größte Elementordnung in einer abelschen Gruppe G. Dann gilt $g^n = e$ für alle $g \in G$.	
\end{Theorem}
\begin{Theorem}
	$G$ ist genau dann abelsch, wenn die Zentrumsfaktorgruppe $G / Z(G)$ zyklisch ist.
\end{Theorem}
\begin{Theorem}
	Sei $p$ eine Primzahl. Alle Gruppen der Ordnung $p^2$ sind abelsch.
\end{Theorem}
	\section{Zyklische Gruppen}
	\begin{Theorem}
		Sei $a$ ein Erzeuger der zyklischen Gruppe $G$ mit $|G|=n$. $a^m$ ist genau dann Erzeuger, wenn $m$ und $n$ teilerfremd sind.
	\end{Theorem}
	\begin{Theorem}
		$G$ ist zyklisch genau dann, wenn $G$ zu jedem positiven Teiler $t$ von $|G|$ genau eine Untergruppe der Ordnung $t$ besitzt.
	\end{Theorem}
	\section{Symmetrische \& Alternierende Gruppen}
	\begin{Theorem}
		Sei $\sigma, \tau\in S_n$ disjunkt. Es gilt $\text{ord}(\sigma\tau)=\text{kgV}(\text{ord}(\sigma),\text{ord}(\tau))$
	\end{Theorem}
	\begin{Theorem}
		\[
		S_n=\langle (12),(123\dots n)\rangle
		.\] 
	\end{Theorem}
	\begin{Theorem}
		Sei $\phi=(a_1a_2\dots)(b_1b_2\dots)\dots\in S_n$ in Zykelnotation und $\psi\in S_n$. Es gilt
		\[
			\psi\phi\psi^{-1}=(\psi(a_1)\psi(a_2)\dots)(\psi(b_1)\psi(b_2)\dots)\dots
		.\] 
	\end{Theorem}
	\section{Einfache Gruppen}
	\section{Produktgruppen}
	\begin{Theorem}
		Sei  $A,B\le G$. $AB$ ist eine Gruppe genau dann, wenn $AB=BA$.

	\end{Theorem}
	\begin{Theorem}
		\[
		|AB|=\frac{|A| |B|}{|A\cap B|}
		.\] 
	\end{Theorem}
	\begin{Theorem}
		Internes direktes Produkt: $A,B\trianglelefteq G,~A\cap B=\{e\}\implies AB\cong A\times B $.
	\end{Theorem}
	\begin{Theorem}
		Internes semidirektes Produkt: $A\trianglelefteq G, B\le G,A\cap B=\{e\} \implies AB\cong A\rtimes B$
	\end{Theorem}
	\begin{Definition}
		$A\rtimes_{\varphi} B=(A\times B, \circ, (e,e))$, wobei $(u,v)\circ (\tilde{u},\tilde{v})=(u\varphi_{v}(\tilde{u}), v\tilde{v})$
	\end{Definition}
	\section{Beispielverzeichnis}
	\end{document}
