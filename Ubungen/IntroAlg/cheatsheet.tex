\documentclass[prb,12pt]{revtex4-2}
%fonts
% Palatino for main text and math
\usepackage[osf,sc]{mathpazo}

% Helvetica for sans serif
% (scaled to match size of Palatino)
\usepackage[scaled=0.90]{helvet}

% Bera Mono for monospaced
% (scaled to match size of Palatino)
\usepackage[scaled=0.85]{beramono}
\usepackage{amsmath, amssymb,physics,amsfonts,amsthm}
\usepackage{enumitem}
\usepackage{cancel}
\usepackage[most]{tcolorbox}
\usepackage{booktabs}
\usepackage{tikz}
\usepackage{hyperref}
\usepackage{enumitem}
\usepackage{transparent}
\usepackage{float}
\usepackage{multirow}
\newtheorem{Theorem}{Theorem}
\newtheorem{Proposition}{Proposition}
\newtheorem{Lemma}[Theorem]{Lemma}
\newtheorem{Corollary}[Theorem]{Corollary}
\newtheorem{Example}[Theorem]{Example}
\newtheorem{Remark}[Theorem]{Remark}
\theoremstyle{definition}
\newtheorem{Problem}{Problem}
\theoremstyle{definition}
\newtheorem{Definition}[Theorem]{Definition}
\newenvironment{parts}{\begin{enumerate}[label=(\alph*)]}{\end{enumerate}}
%tikz
\usetikzlibrary{patterns}
\usepackage{tikz-cd}
% definitions of number sets
\newcommand{\N}{\mathbb{N}}
\newcommand{\R}{\mathbb{R}}
\newcommand{\Z}{\mathbb{Z}}
\newcommand{\Q}{\mathbb{Q}}
\newcommand{\C}{\mathbb{C}}
\begin{document}
	\title{Einf\"{u}rung in die Algebra Hausaufgaben Blatt Nr. 9}
	\author{Jun Wei Tan}
	\email{jun-wei.tan@stud-mail.uni-wuerzburg.de}
	\affiliation{Julius-Maximilians-Universit\"{a}t W\"{u}rzburg}
	\date{\today}
	\maketitle
	\section{Direktes Produkt von Zyklische Gruppen}
	\begin{Remark}
		Sei $A,B\le G$. I.A. ist $AB$ keine Gruppe.
	\end{Remark}
	\begin{Theorem}
		Sei  $A,B\le G$. $AB$ ist eine Gruppe genau dann, wenn $AB=BA$.

		Erfüllt, wenn $A$ oder $B$ normal in $G$ sind.
	\end{Theorem}
	\begin{Theorem}
		\[
		|AB|=\frac{|A| |B|}{|A\cap B|}
		.\] 
	\end{Theorem}
	\begin{Theorem}
		Internes direktes Produkt: $A,B\trianglelefteq G,~A\cap B=\{e\}\implies AB\cong A\times B $.
	\end{Theorem}
	\begin{Theorem}
		Internes semidirektes Produkt: $A\trianglelefteq G, B\le G,A\cap B=\{e\} \implies AB\cong A\rtimes B$
	\end{Theorem}
	\end{document}
