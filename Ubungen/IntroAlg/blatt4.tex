\begin{Problem}
	Seien $n\in \N^*, T$ die Menge der positiven Teiler von $n$ und $G$ eine Gruppe der Ordnung $n$. F\"{u}r $t\in T$ definieren wir die Mengen
	\[
		M_t:=\left\{ g\in G|\text{ord}(g)=t \right\} \subseteq G
	.\] 
	\begin{parts}
		\item Zeigen Sie, dass jedes $g\in G$ in genau einer der Mengen $M_t$ mit $t\in T$ liegt.
		\item Sei nun zudem $G$ zyklisch. Zeigen Sie, dass dann $|M_t|=\varphi(t)$ f\"{u}r alle $t\in T$ gilt.
		\item Folgern Sie: F\"{u}r jedes $n\in \N^*$ gilt $n=\sum_{t|n, t>0}\varphi(t)$.
	\end{parts}
\end{Problem}
\begin{proof}
	\begin{parts}
	\item Sei $g\in G$ beliebig und $H=\left<g \right>$. $H$ ist eine Untergruppe von $G$. Es gilt auch, dass $|H|=\text{ord}(g)$. Wir wissen, dass $|H|$ teilt $|G|$. Daraus folgt, dass $\text{ord}(g)$ teilt $|G|$, und $g$ liegt in genau einer der Mengen $M_t$ mit $t\in T$.
	\item Weil $G$ zyklisch ist, gibt es f\"{u}r jede Teiler $t|n$ eine Untergruppe der Ordnung $t$. 

		Diese Untergruppe hat genau $\varphi(t)$ Erzeuger. F\"{u}r alle Erzeuger $p$gilt $\text{ord}(t)=|\left<t \right>|=t$. Es gilt daher, f\"{u}r jedes $t\in T$, $|M_t|\ge \varphi(t)$. 

		Außerdem ist die Untergruppe der Ordnung $t$ eindeutig. Deswegen ist genau $|M_t|=\varphi(t)$. Wir nehmen an, dass es $p\in G$ gibt, sodass $\text{ord}(p)=t$, also $|\left<p \right>|=t$. Weil die zyklische Untergruppe der Ordnung $t$ eindeutig ist, ist die erzeugte Gruppe genau die Gruppe, die wir vorher diskutiert haben, also $p$ ist einer der vorherigen $\varphi(t)$ Erzeuger.

		Daraus folgt, dass $|M_t|=\varphi(t)$ f\"{u}r alle $t\in T$.
	\item Es gibt f\"{u}r jedes $n$ eine zyklische Gruppe der Ordnung $n$, z.B. $\Z / n\Z$.

		Alle Element sind in einer der Mengen $M_t$. Die Zahl der Elemente ist einfach  $\sum_{t\in T} |M_t|=\varphi(t)=n$
	\end{parts}
\end{proof}
\begin{Problem}
	Zeigen Sie, dass für eine Gruppe $G$ der Ordnung $n \in \N^*$ äquivalent sind:
	\begin{parts}
	\item $G$ ist zyklisch.
	\item $G$ besitzt zu jedem positiven Teiler $t$ von $n$ genau eine Untergruppe der Ordnung $t$.
	\end{parts}
\end{Problem}
\begin{proof}
	(a)$\implies$(b) ist schon in der Vorlesung bewiesen. (Satz 2.18). Es bleibt (b)$\implies$(a) zu zeigen. Wir betrachten dann ein Element $a\in G$ und die erzeugte zyklische Gruppe $\left<a \right>$. Falls $\left<a \right> = G$ sind wir fertig. Also nehmen wir an, $\left<a \right>\neq G$. Sei $\text{ord}(a)=k$, und $\left<a \right> = k$. Diese Gruppe besitzt genau $\varphi(k)$ Erzeuger.
	
	Dann wählen wir ein anderes $b$ Element aus, das kein Erzeuger von $\left<a \right>$ ist. Wir betrachten $\left<b \right>$. Es muss gelten, dass $\text{ord}(b)=\left| \left<b \right> \right| \neq k$. Sonst wäre $\left<b \right> = \left<a \right>$, weil es nur \emph{eine} Untergruppe der Ordnung $k$ gibt.

	Ähnlich fahren wir fort. Wir wählen ein Element aus, das kein Erzeuger von der voherigen betrachten zyklischen Gruppen sind. Weil es f\"{u}r jeder Teiler von $n$ nur eine Untergruppe der Ordnung gibt, gilt $|M_t|=\varphi(t)$.

	Sei $T$ die Menge der positiven Teiler von $n$. Wir berechnen
	\[
	\sum_{n\neq t\in T}\varphi(t)<n
	,\] 
	also wir haben Elemente, f\"{u}r die gilt, deren Ordnung kein positiver Teiler von $n$, das weniger als $n$ ist. Weil die Ordnung ein Teiler von $n$ sein muss, muss die Ordnung $n$ sein, also solche Elemente sind Erzeuger der Gruppe $G$, und $G$ ist zyklisch.
\end{proof}
\begin{Problem}
	\begin{parts}
	\item  Bestimmen Sie die Ordnungen der Elemente für jede der Diedergruppen $D_n$ mit $n\ge 3$.
	\item  Zeigen Sie, dass Satz 2.23 für nicht-abelsche Gruppen im Allgemeinen falsch ist.
		\begin{tcolorbox}
			(Satz 2.23) Sei n die größte Elementordnung in einer abelschen Gruppe G. Dann gilt $g^n = e$ für alle $g \in G$.
		\end{tcolorbox}
	\end{parts}
\end{Problem}
\begin{proof}
	\begin{parts}
	\item Wir betracten zuerst Elemente mit dem Form $r^ks^0=r^k$. Wir haben per die letzte Übungsblatt
		\begin{tcolorbox}
			Zeigen Sie, dass für $k \in \N$ genau dann $r^k = e$ gilt, wenn $n|k$ ist.
		\end{tcolorbox}
		also wir brauchen die kleinste Zahl $\text{ord}(r^k)$, sodass $(r^k)^{\text{ord}(r^k)}=r^{k\text{ord}(r^k)}=r^{pn},p\in \N$. Per Definition des kleinsten gemeinsamen Vielfaches ist
		\[
			\text{ord}(r^k)=\frac{\text{kgV}(k, n)}{k}=\frac{n}{\text{ggT}(k,n)}
		.\] 
		Wir wissen auch, dass $s^2=e$, also $\text{ord}(s)=2$.
		
		Jetzt betrachten wir alle Elemente mit dem Form $r^ks$. Es gilt
		\[
			r^ks r^k s=r^kr^{-k}s s=ee=e
		,\]
		also alle Elemente mit dem Form $r^ks$ haben Ordnung $2$.
	\item Wir betrachten $D_3$. Die größte Elementordnung ist $3$, weil $\text{ggT}(3,2)=1$, und $\frac{3}{\text{ggT}(3,2)}=3$, also $r^2$ hat Ordnung $3$.

		Keine größere Elementordnung ist möglich, weil $\frac{n}{\text{ggT}(n,k)}\le n=3$, und $2 \le 3$. 

		Es gilt aber
		\[
		s^3=(s^2)s=es=s
		,\]
		also $s^3\neq e$, ein Widerspruch.\qedhere
	\end{parts}
\end{proof}
\begin{Problem}
	Sei $n\ge 3$. 
	\begin{parts}
	\item Zeigen Sie, dass die Menge $R:=\left\{ r^0, r^1, \dots, r^{n-1} \right\} $ ein Normalteiler von $D_n$ ist.
	\item Zeigen Sie, dass die Gruppe $\left<x \right>$ f\"{u}r kein $x\in D_n\backslash R$ ein Normalteiler von $D_n$ ist.
	\end{parts}
\end{Problem}
\begin{proof}
	\begin{parts}
	\item Wir betrachten $x^{-1}Rx$:
		\begin{enumerate}[label=(\roman*)]
			\item $x=r^p$:

				Es gilt $x^{-1}=r^{-p}$. und
				\[
					r^{-p}r^kr^p=r^k
				,\]
				f\"{u}r alle $k\in \Z$, also $r^{-p}Rr^p=R$, insbesondere $r^{-p}Rr^p\subseteq R$.
			\item $x=s$:

				Es gilt $s^{-1}=s$, und
				\[
					s r^k s= r^{-k} ss=r^{-k}\in S
				,\]
				(Die Behauptung, dass $r^{-k}\in S$, war im letzten Übungsblatt bewiesen).

			\item $x=r^ps$:

				Wir haben schon bewiesen, dass  $x^{-1}=r^ps$. Es gilt
				 \[
					 r^p s r^k r^p s=r^p s r^{k+p}s = r^p r^{-(k+p)}s s=r^{-k}\in R
				.\]
		\end{enumerate}
		Insgesamt gilt $x^{-1}Rx\subseteq R$ f\"{u}r alle $x\in D_n$, also $R$ ist ein Normalteiler.
	\item Noch einmal betrachten wir die unterschiedlichen Fälle
		\begin{enumerate}[label=(\roman*)]
			\item $x=s$, also $\left<x \right> = \left\{ e,s \right\} $:

				Es gilt
				\[
					r^{-1}s r=r^{-1}r^{-1}s=r^{-2}s\neq s
				.\] 
				(Es gilt $r^{-2}\neq r^0=e$ wenn $n\ge 3$).
			\item $x=r^ps$, also $\left<x \right> = \left\{ e, r^p s \right\} $.

				Es gilt
				\[
					r^{-k}r^p s r^k=r^{-k}r^pr^{-k}s=r^{p-2k}s
				,\] 
				was nur ein Element von $\left<x \right>$ ist, wenn $p-2k\equiv p\pmod{n}$. Sei zum Beispiel $k=1$. Weil $n\ge 3$, gilt die Gleichung nie.
				\begin{tcolorbox}
					\begin{Lemma}
						F\"{u}r $n\ge 3$ kann
						\[
							p-2\equiv p\pmod{n}
						\]
						nicht gelten, f\"{u}r alle $p\in \Z$
					\end{Lemma}
					\begin{proof}
						Die Gleichung $p-2\equiv p\pmod{n}$ genau dann, wenn es $k\in \Z$ existiert, sodass
						\[
						p-p-2=kn
						,\]
						aber das impliziert
\[
2=kn
,\] 
was unmöglich ist, weil $n\ge 3 > 2$.
					\end{proof}
				\end{tcolorbox}
				Also $r^{-1}\left<x \right>r\not\subseteq \left<x \right>$
		\end{enumerate}
		Die Gruppe $\left<x \right>$ ist dann keine Gruppe f\"{u}r $x\in D_n\backslash R$.\qedhere
	\end{parts}
\end{proof}
