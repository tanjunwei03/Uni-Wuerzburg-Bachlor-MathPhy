\begin{Problem}
	Seien $n\in \N^*, T$ die Menge der positiven Teiler von $n$ und $G$ eine Gruppe der Ordnung $n$. F\"{u}r $t\in T$ definieren wir die Mengen
	\[
		M_t:=\left\{ g\in G|\text{ord}(g)=t \right\} \subseteq G
	.\] 
	\begin{parts}
		\item Zeigen Sie, dass jedes $g\in G$ in genau einer der Mengen $M_t$ mit $t\in T$ liegt.
		\item Sei nun zudem $G$ zyklisch. Zeigen Sie, dass dann $|M_t|=\varphi(t)$ f\"{u}r alle $t\in T$ gilt.
		\item Folgern Sie: F\"{u}r jedes $n\in \N^*$ gilt $n=\sum_{t|n, t>0}\varphi(t)$.
	\end{parts}
\end{Problem}
\begin{proof}
	\begin{parts}
	\item Sei $g\in G$ beliebig und $H=\left<g \right>$. $H$ ist eine Untergruppe von $G$. Es gilt auch, dass $|H|=\text{ord}(g)$. Wir wissen, dass $|H|$ teilt $|G|$. Daraus folgt, dass $\text{ord}(g)$ teilt $|G|$, und $g$ liegt in genau einer der Mengen $M_t$ mit $t\in T$.
	\item h 
	\end{parts}
\end{proof}
\begin{Problem}
	Zeigen Sie, dass für eine Gruppe $G$ der Ordnung $n \in \N^*$ äquivalent sind:
	\begin{parts}
	\item $G$ ist zyklisch.
	\item $G$ besitzt zu jedem positiven Teiler $t$ von $n$ genau eine Untergruppe der Ordnung $t$.
	\end{parts}
\end{Problem}

\begin{Problem}
	\begin{parts}
	\item  Bestimmen Sie die Ordnungen der Elemente für jede der Diedergruppen $D_n$ mit $n\ge 3$.
	\item  Zeigen Sie, dass Satz 2.23 für nicht-abelsche Gruppen im Allgemeinen falsch ist.
		\begin{tcolorbox}
			(Satz 2.23) Sei n die größte Elementordnung in einer abelschen Gruppe G. Dann gilt $g^n = e$ für alle $g \in G$.
		\end{tcolorbox}
	\end{parts}
\end{Problem}

\begin{Problem}
	Sei $n\ge 3$. 
	\begin{parts}
	\item Zeigen Sie, dass die Menge $R:=\left\{ r^0, r^1, \dots, r^{n-1} \right\} $ ein Normalteiler von $D_n$ ist.
	\item Zeigen Sie, dass die Gruppe $\left<x \right>$ f\"{u}r kein $D_n\backslash R$ ein Normalteiler von $D_n$ ist.
	\end{parts}
\end{Problem}
