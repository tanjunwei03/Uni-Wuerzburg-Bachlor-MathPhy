\begin{Problem}
	Seien $p,q$ zwei (nicht notwendigerweise verschiedene) Primzahlen und $G$ eine Gruppe der Ordnung $pq$. Zeigen Sie, dass $G$ auflösbar ist.  
\end{Problem}
\begin{proof}
	Sei $p=q$. Dann ist $G$ eine Gruppe der Ordnung $p^2$. Wir wissen, dass solche Gruppen abelsch ist, also $G'=\{e\} $ und $G$ ist auflösbar.

	Sei jetzt $p\neq q$. Nach den Sylowsätze gibt es Untergruppen der Ordnung $p$, f\"{u}r deren Anzahl $n_p$ gilt:
	\begin{align*}
		n_p\equiv& 1\pmod{p}\\
		n_p|q
	\end{align*}
	Da $q$ eine Primzahl ist, muss $n_p=1$ gelten. Ähnlich gilt auch $n_q=1$. Sei $P$ die Untergruppe der Ordnung $p$. Als Gruppe einer Primzahlordnung ist $P$ zyklisch, insbesondere abelsch und daher auflösbar. $|G / P|=q$, also $|G / Q|$ ist zyklisch, abelsch und auflösbar. 

	Dann ist $G$ auflösbar.
\end{proof}
\begin{Problem}
	Zeigen Sie, dass jede Gruppe $G$ der Ordnung $12$ auflösbar ist.
\end{Problem}
\begin{proof}
	$12=3\times 2^2$, also es gibt nach den Sylowsätze Gruppen der Ordnung $4$ und $3$ von Anzahl $n_2$ bzw. $n_3$. Aus den vorherigen Übungsblätter wissen wir, dass $n_2= 1$ oder $n_3=1$ gilt. 
	\begin{enumerate}
		\item $n_2=1$. Sei $H$ die Untergruppe der Ordnung $4$. Als Gruppe der Ordnung $4=2^2$ ist $H$ abelsch und auflösbar. Weil $|G / H|=3$, ist $G / H$ zyklisch, abelsch und auflösbar. Da sowohl $H$ als auch $G / H$ auflösbar sind, und $H$ ein Normalteiler ist, ist $G$ auflösbar.
		\item $n_3=1$. Sei $H$ die Untergruppe der Ordnung $3$. Als Gruppe von Primordnung ist $H$ zyklisch, abelsch und auflösbar. Da $|G / H|=4=2^2$, ist $G / H$ abelsch und daher auflösbar. Da $H$ normal ist und sowohl $H$ als auch $G / H$ auflösbar sind, ist $G$ auflösbar.\qedhere 
	\end{enumerate}
\end{proof}
\begin{Problem}
	Sei $R$ ein Ring, und seien $a,b\in R$. Es gelte $ab=1$ und $ba\neq 1$. Ein Element $x\in R$ heißt \emph{nilpotent}, falls es ein $s\in \N$ mit $x^s=0$ gibt. Ein Element $x\in R$ heißt idempotent, falls $x^2=x$ gilt.
	\begin{parts}
		\item Zeigen Sie, dass das Element $1-ba$ idempotent ist.
		\item Zeigen Sie, dass das Element $b^n(1-ba)$ f\"{u}r alle $n\in \N^*$ nilpotent ist.
		\item Zeigen Sie, dass es unendlich viele nilpotente Elemente in $R$ gibt.
	\end{parts}
\end{Problem}
\begin{proof}
	\begin{parts}
	\item 
		\begin{align*}
			(1-ba)^2=&(1-ba)(1-ba)\\
			=&1-ba-ba+(-ba)(-ba) & \text{Distributivgesetz}\\
			=&1-2(ba) + baba &\text{Lemma 3.1}\\
			=&1-2(ba)+b(ab)a\\
			=&1-2ba+ba\\
			=&1-ba
		\end{align*}
	\end{parts}
\end{proof}
\begin{Problem}
\begin{parts}
\item Sei $R$ ein kommutativer Ring. Zeigen Sie die Äquivalenz der folgenden Aussagen:
	\begin{enumerate}[label=(\arabic*)]
		\item F\"{u}r alle $r,s\in R$ gilt $(r+s)^4=r^4+s^4$.
		\item In $R$ gilt $2=0$.
	\end{enumerate}
\item Geben Sie ein Beispiel für einen kommutativen Ring an, der den Bedingungen aus (a) genügt, aber kein Körper ist. 
\end{parts}
\end{Problem}
\begin{proof}
	\begin{parts}
	\item Es gilt
		\begin{align*}
			(r+s)^2=&(r+s)(r+s)\\
			=&r^2+s r+rs+s^2\\
			(r+s)^4=&(r+s)^2(r+s)^2\\
			=&(r^2+s r+rs+s^2)(r^2+s r+rs+s^2)\\
			=&r^4+r^2 s r+r^3 s + r^2s^2\\
			 &+s r^3+s r s r+s r r s+s r s^2\\
			 &+r s r^2+rss r+rs rs+ rs^3\\
			 &+s^2r^2+s^3 r+s^2 rs+s^4\\
			=&r^4 + r^3s+r^3s+r^2s^2\\
			 &+s r^3+ s^2r^2 + s^2 r^2 + s^3 r\\
			 &+r^3 s + r^2 s^2+ r^2 s^2 + r s^3\\
			 &+s^2r^2 + s^3 r + s^3 r + s^4 & \text{Kommutativgesetz}\\
			=&r^4+4r^3 s+6r^2 s^2+4r s^3+s^4
		\end{align*}
		Die Behauptung $(s+r)^4=r^4+s^4$ ist dann äquivalent zu
		\[
		4r^3s+6r^2s^2+4rs^3=0~\forall s,r\in R
		.\] 
		Die Rückrichtung ist jetzt klar: Falls $2=0$, ist
		\[
			2r^3s+6r^2s^2+4rs^2=\cancelto{0}{2}(2r^3s+3r^2s^2+2rs^3)=0
		.\] 
		Die andere Richtung: Wir nehmen an, dass
		\[
		2(2r^3s+3r^2s^2+2rs^3)=0~\forall r,s\in R
		.\] 
		Insbesondere betrachten wir $r=-1$ und $s=1$. Dann ist $r^3=1$ und $r^2=1$. Alle Potenzen von $s$ sind $1$. Es gilt
		\[
		2(2r^3 s+3r^2s^2+2 rs^3)=2(-1)=0
		.\] 
		Aber $-1\neq 0$, also $2=0$.
	\end{parts}
\end{proof}
