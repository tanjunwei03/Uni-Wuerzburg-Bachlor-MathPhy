\begin{Problem}
	\begin{parts}
		\item Eine Gruppe $G$ der Ordnung 21 operiere auf einer Menge $M$ mit 11 Elementen. Zeigen Sie, dass diese Operation eine Bahn der Länge 1 besitzt.

			{\footnotesize (Ist $\left\{ m \right\} \subseteq M$ eine solche einelementige Bahn, dann gilt $g.m=m$ f\"{u}r alle $g\in G$. Jedes $g\in G$ fixiert also $m$. Man nennt $m$ daher auch einen Fixpunkt der Operation.)} 
		\item Sei $G:=\text{GL}(2,\C)$ die Gruppe der invertierbaren komplexen $(2\times 2)$-Matrizen und $M$ die Menge aller komplexen $(2\times 2)$-Matrizen, die nur reelle Eigenwerte besitzen. Dann operiert $G$ per Konjukation auf $M$. (Dies brauchen Sie nicht zu zeigen.) Geben Sie ein Repräsentantensystem der Bahnen der Operation an)
		\end{parts}
\end{Problem}
\begin{proof}
	\begin{parts}
	\item Wir schreiben die Klassengleichung
		\[
			|M|=\sum_{i=1}^r [G:G_m]
		.\] 
		Jeder Term im Summe ist eine Teiler von $21$, also $1,3,7$ oder $21$. Die Operation besitzt eine Bahn der Länge 1 genau dann, wenn $1$ zumindest einmal vorkommt. Wir schreiben die möglichen Summen:
	\begin{align*}
		11=&1\times 11\\
		11=&3+1\times 8\\
		11=&3\times 2+1\times 5\\
		11=&3\times 3+1\times 2\\
		11=&7+1\times 4\\
		11=&7+3+1
	\end{align*} 
	Weil $1$ immer vorkommt, gibt es immer eine Bahn der Länge 1.
	\item Konjukation ist genau eine Ähnlichkeitstransformation. Wir brauchen: 
		\begin{tcolorbox}
			Zwei Matrizen sind ähnlich genau dann, wenn sie die gleichen Jordan-Normalform haben.
		\end{tcolorbox}
Daraus ergibt sich ein Repräsentantensystem der Bahnen:
		\[
			\left\{ \left.\begin{pmatrix} a & 0 \\ 0 & b \end{pmatrix} \right|a,b\in \R \right\} \cup \left\{ \left.\begin{pmatrix} a & 1 \\ 0 & a \end{pmatrix} \right|a\in \R \right\} 
		,\] 
		wobei die erste diagonalisierbare Matrizen darstellt und die zweite nicht diagonalisierbare Matrizen darstellt.\qedhere
	\end{parts}
\end{proof}

\begin{Problem}
Von der endlichen Gruppe $G$ sei bekannt, dass sie nicht abelsch ist und zu jedem positiven Teiler $t$ von $|G|$ mindestens eine Untergruppe der Ordnung $t$ besitzt. Zeigen Sie, dass $G$ nicht einfach ist. (Hinweis: Sei $p$ die kleinste Primzahl, die $|G|$ teilt, und $U$ eine Untergruppe von $G$ vom Index $p$. Lassen Sie $G$ auf den Nebenklassen von $U$ operieren und betrachten Sie den Kern des zugehörigen Homomorphismus.)	
\end{Problem}
\begin{proof}
	Wie im Hinweis: Sei $p$ die kleinste Primzahl, die $|G|$ teilt und $U$ eine Untergruppe von $G$ vom Index $p$. $G$ operiere auf den Linksnebenklassen von $U$. Die Operation ist ein Homomorphismus $G\to S_p$. Deren Kern $K$ ist eine Untergruppe von $U$, weil $xU\neq U$ f\"{u}r $x\not\in  U$. Das Ziel ist: Der Kern ist nicht trivial.

Die Faktorgruppe $G / K$ ist isomorph zu eine Untergruppe von $S_p$, also es teilt $|S_p|=p!$. Als Faktorgruppe teilt $|G / K|$ den Ordnung von $G$ auch. Weil $p\neq 1$, ist $U$ eine echte Untergruppe, also $|G / K| \neq 1$. Dann ist die einzige Zahl, die sowohl $p!$ als auch $|G|$ teilt, $p$. Daraus folgt: $|K|=|U|$. Weil $K\subseteq U$, ist $K=U$, also $U$ ist ein Normalteiler.
\end{proof}
\begin{Problem}
Benutzen Sie die Beweisidee aus Korollar 2.79, um folgende Aussage zu zeigen: Seien $p$ eine Primzahl, $n \in \N^*$, $G$ eine Gruppe der Ordnung $p^n$ und $\{e\} < N \trianglelefteq G$ ein nicht-trivialer Normalteiler von G. Dann gilt $|Z(G) \cap N| > 1$.	
\end{Problem}
\begin{proof}
	Als Normalteiler (insbesondere Untergruppe) teilt der Ordnung von $N$ den Ordnung von $G$, also $|N|=p^m,~0<m\le n$. $0$ ist ausgeschlossen, weil $N$ nicht trivial ist. Dann ist $p$ ein Teiler von $|N|$. Wir schreiben noch einmal die Klassengleichung:
	\[
	|G|=|Z(G)|+\sum_{i=1}^r [G:C_G(x_i)]
	.\] 
	Als Normalteiler ist $N$ per Definition eine Vereinigung von Konjugationsklassen, sonst wäre $N$ unter Konjugation nicht abgeschlossen. Eine solche Konjukationsklasse ist $\{e\} $, weil $N$ eine Untergruppe ist. Der Ordnung von Konjugationsklassen sind Teiler von $p^n$, also Potenzen von $p$. Dann ist der Ordnung von $N$ eine Summe
	\[
		|N|=\sum_{m=1}^k [G:G_m]=1+p^{m_1}+p^{m_2}+\dots+p^{m_k}
	.\] 
	Es ist nicht möglich, dass alle $m_1,\dots, m_k>0$ sind, weil dann $p$ ein Teiler von alle $p^{m_i}$ und $|N|$ wäre, jedoch kein Teiler von $1$ und daher kein Teiler von die rechte Seite. Dann muss es f\"{u}r eine $m_i$ gelten, dass $m_i=0$. Dann enthält $N$ Konjugationsklassen der größe $1$ bzw. Elememente im Zentrum, die nicht $e$ sind, also $N\cap Z(G)\neq \{e\} $.
\end{proof}
\begin{Problem}
	Die Gruppe $G$ operiere auf einer Menge $M$. Sei $\Phi:G\to \text{Sym}(M)$ der zugehörige Homomorphismus und $K$ sein Kern. Zeigen Sie, dass durch die Abbildung
	\[
G / K \times M \to M,\qquad gK.m:=g.m
\]
eine treue Operation von $G / K$ auf $M$ gegeben ist.
\end{Problem}

\begin{proof}
	Wir zeigen zuerst, dass es wohldefiniert ist. Sei $k_1,k_2\in K$ und $m\in M$. Es gilt
	\begin{align*}
		gk_1.m=&\Phi(gk_1)(m)\\
		=&\Phi(g)(\Phi(k_1)(m))\\
		=&\Phi(g)(e(m))\\
		=&\Phi(g)(m)\\
		=&g.m
	\end{align*}
	und ähnlich für $gk_2.m=g.m$. Sei jetzt $g_1,g_2\in G$, so dass $g_1K.m=g_2K.m$ f\"{u}r alle $m\in M$. Dann ist
	\[
		g_2^{-1}g_1.m=m
	\]
	f\"{u}r alle $m\in M$ oder $g_2^{-1}g_1\in K$. Daraus folgt: $g_1$ und $g_2$ liegen in der gleichen Nebenklasse.
\end{proof}
