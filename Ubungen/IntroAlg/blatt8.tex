\begin{Problem}
	\begin{parts}
		\item Eine Gruppe $G$ der Ordnung 21 operiere auf einer Menge $M$ mit 11 Elementen. Zeigen Sie, dass diese Operation eine Bahn der Länge 1 besitzt.

			{\footnotesize (Ist $\left\{ m \right\} \subseteq M$ eine solche einelementige Bahn, dann gilt $g.m=m$ f\"{u}r alle $g\in G$. Jedes $g\in G$ fixiert also $m$. Man nennt $m$ daher auch einen Fixpunkt der Operation.)} 
		\item Sei $G:=\text{GL}(2,\C)$ die Gruppe der invertierbaren komplexen $(2\times 2)$-Matrizen und $M$ die Menge aller komplexen $(2\times 2)$-Matrizen, die nur reelle Eigenwerte besitzen. Dann operiert $G$ per Konjukation auf $M$. (Dies brauchen Sie nicht zu zeigen.) Geben Sie ein Repräsentantensystem der Bahnen der Operation an)
		\end{parts}
\end{Problem}
\begin{proof}
	\begin{parts}
	\item Wir schreiben die Klassengleichung
		\[
			|M|=\sum_{i=1}^r [G:G_m]
		.\] 
		Jeder Term im Summe ist eine Teiler von $21$, also $1,3,7$ oder $21$. Die Operation besitzt eine Bahn der Länge 1 genau dann, wenn $1$ zumindest einmal vorkommt. Wir schreiben die mögliche Summen:
	\begin{align*}
		11=&1\times 11\\
		11=&3+1\times 8\\
		11=&3\times 2+1\times 5\\
		11=&3\times 3+1\times 2\\
		11=&7+1\times 4\\
		11=&7+3+1
	\end{align*} 
	Weil $1$ immer vorkommt, gibt es immer eine Bahn der Länge 1.
	\item Konjukation ist genau eine Ähnlichkeitstransformation. Trotz der Aufgabenstellung brauchen wir noch die Eigenschaften.
		\begin{tcolorbox}
			\begin{Lemma}
			Sind zwei Matrizen $A$ und $B$ ähnlich, dann haben sie dieselben Eigenwert.	
			\end{Lemma}
			\begin{proof}
			Sei $A=Q^{-1}BQ$. Sei außerdem $v$ ein Eigenvektor von $A$ mit Eigenwert $\lambda$. Es gilt $QA=BQ$ und
		\begin{align*}
			QAv=&Q\lambda v=\lambda(Qv)\\
			=&BQv=B(Qv)
		\end{align*}
		also $Qv$ ist ein Eigenvektor von $B$ mit Eigenwert $\lambda$. Wir können die Rollen von $A$ und $B$ vertauschen, um die andere Richtung zu zeigen. 	
			\end{proof}
			\begin{Remark}
				Die Umkehrrichtung gilt nicht immer. Es gilt wenn die Matrizen diagonalisierbar sind.
			\end{Remark}
		\end{tcolorbox}
		Es folgt sofort: Wenn zwei Matrizen in der gleichen Bahn liegen, haben die die gleichen Eigenwerte. Wenn die Matrizen nicht diagonaliserbarsind, schreiben wir die in Jordan-Normalform. Daraus ergibt sich ein Repräsentantensystem der Bahnen:
		\[
			\left\{ \begin{pmatrix} a & 0 \\ 0 & b \end{pmatrix} |a,b\in \R \right\} \cup \left\{ \begin{pmatrix} a & 1 \\ 0 & a \end{pmatrix} |a\in \R \right\} 
		.\] 
	\end{parts}
\end{proof}

\begin{Problem}
Von der endlichen Gruppe $G$ sei bekannt, dass sie nicht abelsch ist und zu jedem positiven Teiler $t$ von $|G|$ mindestens eine Untergruppe der Ordnung $t$ besitzt. Zeigen Sie, dass $G$ nicht einfach ist. (Hinweis: Sei $p$ die kleinste Primzahl, die $|G|$ teilt, und $U$ eine Untergruppe von $G$ vom Index $p$. Lassen Sie $G$ auf den Nebenklassen von $U$ operieren und betrachten Sie den Kern des zugehörigen Homomorphismus.)	
\end{Problem}

\begin{Problem}
Benutzen Sie die Beweisidee aus Korollar 2.79, um folgende Aussage zu zeigen: Seien $p$ eine Primzahl, $n \in \N^*$, $G$ eine Gruppe der Ordnung $p^n$ und $\{e\} < N \trianglelefteq G$ ein nicht-trivialer Normalteiler von G. Dann gilt $|Z(G) \cap N| > 1$.	
\end{Problem}

\begin{Problem}
	Die Gruppe $G$ operiere auf einer Menge $M$. Sei $\Phi:G\to \text{Sym}(M)$ der zugehörige Homomorphismus und $K$ sein Kern. Zeigen Sie, dass durch die Abbildung
	\[
G / K \times M \to M,\qquad gK.m:=g.m
\]
eine treue Operation von $G / K$ auf $M$ gegeben ist.
\end{Problem}
\begin{proof}
	Wir schreiben noch einmal die Klassengleichung:
	\[
		|G|=|Z(G)|+\sum_{i=1}^r [G:C_G(x_i)]
	.\] 
	Als Normalteiler ist $N$ eine Vereinigung von Konjugationsklassen. Wir nehmen die solchen Konjugationsklassen raus, und schreibe stattdessen
	\[
		|G|=|Z(G)|+[G:N]+\sum_{i=1}^{r'}[G:C_G(x_i)]
	.\] 
\end{proof}
\begin{proof}
	Wir zeigen zuerst, dass es wohldefiniert ist. Sei $k_1,k_2\in K$ und $m\in M$. Es gilt
	\begin{align*}
		gk_1.m=&\Phi(gk_1)(m)\\
		=&\Phi(g)(\Phi(k_1)(m))\\
		=&\Phi(g)(e(m))\\
		=&\Phi(g)(m)\\
		=&g.m
	\end{align*}
	und ähnlich für $gk_2.m=g.m$. Sei jetzt $g_1,g_2\in G$, so dass $g_1K.m=g_2K.m$ f\"{u}r alle $m\in M$. Dann ist
	\[
		g_2^{-1}g_1.m=m
	\]
	f\"{u}r alle $m\in M$ oder $g_2^{-1}g_1\in K$. Daraus folgt: $g_1$ und $g_2$ liegen in der gleichen Nebenklasse.
\end{proof}
