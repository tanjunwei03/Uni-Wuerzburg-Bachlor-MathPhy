\begin{Problem}\label{pr:introalgblatt7-1}
	Sei $G$ eine Gruppe.
	\begin{parts}
		\item Zeigen Sie, dass die Menge aller Automorphismen von $G$ zusammen mit der Verkettung von Funktionen als Verknüpfung eine Gruppe bildet. 
		
			Man nennt diese Gruppe die \emph{Automorphismengruppe} von $G$ und schreibt $\text{Aut}(G)$ f\"{u}r sie.
		\item Zeigen Sie, dass f\"{u}r jedes $g\in G$ durch
			\[
				k_g:G\to G\qquad x\to gxg^{-1}
			\]
			ein Automorphismus von $G$ gegeben ist.

Dies zeigt, dass die Konjugation mit einem beliebigen Gruppenelement $g$ einen Automorphismus von $G$ liefert. 
	\end{parts}
\end{Problem}
\begin{proof}
	\begin{parts}
	\item Wir beweisen die Eigenschaften
		\begin{enumerate}[label=(\roman*)]
			\item Neutrales Element:

				Sei $1:G\to G,~1(x)=x~\forall x\in G$. Es ist klar, dass $1$ bijektiv ist. Außerdem ist
				\[
				1(xy)=xy=1(x)1(y)
			,\]
			also $1$ ist ein Automorphismus. Außerdem gilt f\"{u}r alle $f\in \text{Aut}(G)$ :
			 \[
			f 1(x)=f(x)~\forall x\in G
		,\]
		also $1$ ist das neutrale Element.
	\item Existenz des Inverses: Sei $f\in\text{Aut}(G)$. Weil $f$ bijektiv ist, gibt es auch eine bijektive inverse Abbildung $f^{-1}$. Es bleibt zu zeigen, dass $f^{-1}$ eine Homomorphismus ist. Sei $x,y\in G$ beliebig. Weil $f$ bijektiv ist, gibt es Elemente $a,b\in G$, so dass $x=f(a)$ und $y=f(b)$ gilt. Per Definition eine inverse Abbildung ist $f^{-1}(x)=a,~f^{-1}(y)=b$.

		Es folgt:
		\begin{align*}
			f^{-1}(xy)&=f^{-1}(f(a)f(b))\\
				  &=f^{-1}(f(ab)) & f\text{ ist ein Homomorphismus}\\
				  &=ab=f^{-1}(x)f^{-1}(y)
		\end{align*}
		also $f^{-1}\in \text{Aut}(G)$.
	\item Assoziatiativität

		Folgt sofort aus der Assoziativität von Funktionverknüpfungen.
	\item Abgeschlossenheit

		Die Verkettung bijektive Abbildungen ist noch einmal bijektiv. Die Verkettung ist auch ein Homomorphismus (Definition 2.58), also $\text{Aut}(G)$ ist abgeschlossen.
		\end{enumerate}
	\item Noch einmal zeigen wir alle Eigenschaften. Sei $g\in G$ beliebig. Wir betrachten $k_g$.
		\begin{enumerate}[label=(\roman*)]
			\item Sie ist ein Homomorphismus.

				Sei $x,y\in G$. Es gilt $k_g(x)=gxg^{-1}$ und $k_g(y)=gyg^{-1}$. Daraus folgt:
				\[
					k_g(x)k_g(y)=gxg^{-1}gyg^{-1}=gxyg^{-1}=k_g(xy)
				.\] 
			\item Sie ist injektiv.

				Wir zeigen, dass $\text{Ker}(k_g)=\left\{ 1 \right\} $. Wir nehmen an, dass es $1\neq x\in G$ gibt, so dass $k_g(x)=1$. Dann ist
				\[
					gxg^{-1}=1\implies gx=g
				.\] 
				Aus der Kurzungsregel folgt $x=1$, ein Widerspruch.
			\item Sie ist surjektiv. 
				Sei $y\in G$ und $x=g^{-1}yg$. Dann gilt
				\[
					k_g(x)=gxg^{-1}=gg^{-1}ygg^{-1}=y
				,\]
				also sie ist surjektiv.
		\end{enumerate}
	\end{parts}
\end{proof}
\begin{Problem}
	Unter dem \emph{Zentrum} $Z(G)$ einer Gruppe $G$ versteht man die Menge aller Elemente von $G$, die mit allen Elementen von $G$ vertauschen, also die Menge $Z(G)=\left\{ g\in G|\text{es gilt }gxg^{-1}=x\text{ f\"{u}r alle }x\in G \right\} $. Wir definieren die Menge der \emph{inneren Automorphismen von} $G$ durch
\[
	\text{Inn}(G):=\left\{ k_g|g\in G \right\} \qquad\text{mit }k_g\text{ wie in \ref{pr:introalgblatt7-1}(b)}
.\] 
Zeigen Sie, dass $Z(G)\trianglelefteq G,\text{Inn}(G)\le \text{Aut}(G)$ und $G / Z(G)\cong \text{Inn}(G)$ gelten.
\end{Problem}
\begin{proof}
	Wir schreiben zuerst einen alternativen Definition:
	\[
		Z(G)=\left\{ g\in G|\text{ es gilt }gx=xg\text{ f\"{u}r alle }x\in G \right\} 
	.\] 
	Dann ist auch $1\in Z(G)$, weil $1x=x 1=x$ f\"{u}r alle $x\in G$ gilt.
	\begin{parts}
	\item $Z(G)\le G$.

		Sei $g,h\in Z(G)$. Dann gilt, f\"{u}r alle $x\in G$:
		\begin{align*}
			ghx(gh)^{-1}=&ghxh^{-1}g^{-1}\\
			=&gxg^{-1} & h\in Z(G)\\
			=&x & g\in Z(G),
		\end{align*}
		also $Z(G)$ ist abgeschlossen.

		Sei jetzt $g\in Z(G)$ mit Inverse $g^{-1}$ (momentan nicht angenommen, dass es in $Z(G)$ ist). Das Ziel ist:
		\[
			g^{-1}xg=x~\forall x\in G
		.\] 
		Weil $G$ eine Gruppe ist, können wir $x$ als $y^{-1}$ schreiben f\"{u}r ein $y\in G$. Es gilt
		\begin{align*}
			g^{-1}y^{-1}g=&(g^{-1}yg)^{-1}\\
			=&y^{-1} & g\in Z(G).
		\end{align*}
		Das heißt: $g^{-1}xg=x$ f\"{u}r alle $x\in G$, und alle Elemente in $Z(G)$ sind invertierbar.
	\item $Z(G)\trianglelefteq G$.

		Folgt fast sofort per Definition: Wir betrachten die Nebenklassen. Sei $x\in G$. Es gilt
		\begin{align*}
			xZ(G)=&\left\{ xg|g\in Z(G) \right\}\\
		=&\left\{ gx|g\in Z(G) \right\} \\
		=& Z(G)x
		\end{align*}
	\item $\text{Inn}(G)\le \text{Aut}(G)$

		Sei $f_1,f_2\in\text{Inn}(G)$, also es gibt $g_1,g_2\in Z(G)$, so dass
		\begin{align*}
			f_1(x)=&g_1xg_1^{-1}\\
			f_2(x)=&g_2xg^{-1}
		\end{align*}
	\end{parts}
	Dann ist $(f_1\circ f_2)(x)=g_1g_2xg_2^{-1}g_1^{-1}=g_1g_2x(g_1g_2)^{-1}$, also $f_1\circ f_2\in\text{Inn}(G)$.

	Die Abbildung $f:G\to G,~f\left( x \right) =x=1x 1^{-1}$ ist auch ein Element von $\text{Inn}(G)$. Es ist klar, dass es das neutrale Element ist. 

	Sei jetzt
	\begin{align*}
		f_1(x)=&g_1xg_1^{-1}\\
		f_1^{-1}(x)=g_1^{-1}xg
	\end{align*}
	$f_1^{-1}\in \text{Inn}(G)$, weil $g_1^{-1}\in G$ und $f_1=k_{g_1^{-1}}$. Es gilt f\"{u}r alle $x\in X$, dass
	\[
		(f_1\circ f_1^{-1})(x)=g_1g_1^{-1}xg_1g_1^{-1}=x
	\]
	Also $\text{Inn}(G)$ ist eine Gruppe.
\item $G / Z(G)\cong \text{Inn}(G)$. 

	Per Definition ist $\Lambda: g\to k_g$ eine surjektive Abbildung. Weiter ist $\Lambda$ ein Homomorphismus: Sei $g,h\in G$. Dann ist $\Lambda(gh)=k_{gh}$, wobei $k_{gh}(x)=(gh)x(gh)^{-1}$ f\"{u}r alle $x\in G$. Es gilt
	\begin{align*}
		k_{gh}(x)=&(gh)x(gh)^{-1}\\
		=&(gh)xh^{-1}g^{-1}\\
		=&g(hxh^{-1})g^{-1}\\
		=&(k_g\circ k_h)(x)
	\end{align*}
	Was ist der Kern des Homomorphismus? $k_g=1$ genau dann, wenn
	\[
		k_g(x)=gxg^{-1}=x~\forall x\in G
	.\] 
	Der Kern ist per Definition genau das Zentrum. Dann folgt $G / Z(G)\cong \text{Inn}(G)$ aus dem Homomorphiesatz.
	\begin{center}
\begin{tikzcd}[scale=2]
			G \arrow[r, "g\to k_g"] \arrow[d] & \text{Inn}(G)\\
			G / Z(G) \arrow[ur,"\cong"]
	\end{tikzcd}
	\end{center}
\end{proof}
\begin{Problem}
	\begin{parts}
	\item Nach Beispiel 2.71 operiert $S_n$ auf sich selbst per Konjugation. Beschreiben Sie die Bahnen dieser Operation.
	\item Wir nennen eine Transposition der $S_3$ \emph{schön}, wen Sie von der Form $(1x)$ mit $x\in \left\{ 2,3 \right\} $ ist. Sei das Neutrale von $S_3$ als Produkt schöner Transpositionen dargestellt. Kommen in diesem Produkt die einzelnen schönen Transpositionen dann stets in gerader Anzahl vor? 
	\end{parts}
\end{Problem}
\begin{proof}
	\begin{parts}
	\item Wir brauchen hier
		\begin{tcolorbox}[title=Satz 2.52]
Sei $\varphi = (a_1a_2\dots)(b_1b_2\dots)\dots\in S_n$ in Zykelnotation und $\psi \in S_n$. Dann
\[
	\psi\varphi\psi^{-1}=(\psi(a_1)\psi(a_2)\dots)(\psi(b_1)\psi(b_2)\dots)\dots
.\] 
		\end{tcolorbox}
		Per Definition sind die Bahnen definiert durch
		\[
			\left\{ \sigma'\sigma\sigma^{\prime-1}|\sigma'\in S_n \right\} 
		.\] 
		Zu jedem $\sigma\in S_n$ gehört eine (vielleicht nicht eindeutige) Bahn. Wir schreiben $\sigma$ als Produkt disjunkte Zykeln (im Sinne von Satz 2.41) 

		Dann zeigen wir zwei Richtungen
		\begin{enumerate}[label=(\roman*)]
			\item Wenn zwei Permutationen die gleichen Zykellänge haben, sind sie konjugiert (Satz 2.53)

				Dies zeigt, dass sie in der gleichen Bahn sind.
			\item Wenn zwei Permutationen bzw. Elemente in $S_n$ unterschiedliche Darstellungen als Produkt disjunkte Zykeln haben, wobei unterschiedliche heißt \emph{nicht} bis auf die Reihenfolge der Faktoren), sind die Permutation nicht konjugiert.

				Dies folgt aus die Eindeutigkeit der Darstellung von $\sigma$ als Produkt von disjunkte Zykeln. 
		\end{enumerate}
		Insgesamt gilt: Zwei Elemente liegen in der gleichen Bahn genau dann, wenn die Elemente die gleiche Darstellung als Produkt disjunkte Zykeln haben. Dann gibt es, f\"{u}r jede endliche monoton wachsende Folge
		\[
			(a_1,a_2,\dots, a_p),~a_i\in \N,~\sum_{i=1}^p a_i\le n,~a_1\le a_2\le\dots\le a_p
		\]
		eine Bahn.
	\item Nein. Es gilt $(12)(23)=(123)$, ein Zyklus der Länge 3. Also gilt $\left[ (12)(23) \right]^3=(123)^3=1$, aber die einzelnen Transpositionen kommen in ungerade Zahlen (3) vor.\qedhere
	\end{parts}
\end{proof}
\begin{Problem}
	Die Gruppe $G$ operiere auf der Menge $M$. Weiter sei $U$ eine Untergruppe von $G$, so dass die auf $U$ eingeschränkte Operation transitiv auf $M$ sei.

	Zeigen Sie, dass dann $G=U\cdot G_m$ f\"{u}r alle $m\in M$ gilt.
\end{Problem}
\begin{proof}
	Sei $m\in M$ fest, aber beliebig. Sei $U'\subseteq U$ eine minimale Teilmenge von $U$, so dass $\left\{ xm|x\in U' \right\} =M$ (möglich weil die Operation transitiv ist). Wir können eine solche Teilmenge konstruieren, indem wir das Ergebnis der Operation $xm$ betrachten und alle andere Elemente mit dem gleichen Ergebnis wegwerfen.  
	\begin{enumerate}[label=(\roman*)]
		\item Sei $a,b\in G_m,~a\neq b$ und $x\in U'$. Dann gilt $xa\neq xb$ (Kürzungsregel)
		\item Sei $x,y\in U',~x\neq y$ und $a,b\in G_m$. Dann gilt $xa\neq yb$, weil $xam=xm$ und $ybm=ym$, aber $xm\neq ym$ per Definition von $U'$ als minimale Teilmenge.
		\item Dann betrachten wir alle Elemente von der Form $xa,x\in U', a\in G_m$. Wir haben schon gezeigt, dass alle solche Elemente unterschiedlich sind. Wie viele Elemente gibt es? $|U'|=|M|$ per Konstruktion und $|G_m|=|G| / |M|$ (Satz 2.71). Daraus folgt, dass
			\[
			|U\cdot G_m|\ge |M|(|G| / |M|)=|G|
			,\]
			also $G\subseteq U\cdot G_M$. Weil wir offenbar per Definition keine Elemente außerhalb $G$ erreichen können, haben wir also $U\cdot G_m\subseteq G$ und daher Gleichheit.
	\end{enumerate}
\end{proof}
