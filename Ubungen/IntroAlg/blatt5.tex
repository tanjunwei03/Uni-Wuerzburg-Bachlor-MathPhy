\begin{Problem}
	\begin{parts}
		\item Begr\"{u}nden Sie, dass die Permutation $\sigma=\begin{pmatrix} 1 & 2 & 3 & 4 & 5 & 6 & 7 & 8 & 9 \\ 7 & 5 & 8 & 3 & 9 & 1 & 6 & 4 & 2 \end{pmatrix} \in S_9$ in der alternierenden Gruppe $A_9$ liegt.
		\item Finden Sie $i$ und $k$, so dass die Permutation $\begin{pmatrix}  1 & 2 & 3 & 4 & 5 & 6 & 7 & 8 & 9\\ 1 & 2 & 7 & 4 & i & 5 & 6 & k & 9 \end{pmatrix} \in S_9$ gerade ist.
		\end{parts}
\end{Problem}

\begin{proof}
	\begin{parts}
	\item Wir schreiben zuerst $\sigma$ als Zyklus
	\[
	\sigma=(176)(259)(384)
	.\] 
	Dann stellen wir die Zyklus als Produkte von Transpositionen dar, wie im Beweis von 2.44
	\[
	\sigma=(17)(76)(25)(59)(38)(84)
	.\]
	Es gibt $6$ Transpositionen, also $\sigma$ ist gerade, und $\sigma\in A_9$.
\item Weil jede Zahl nur einmal vorkommen darf, gibt es nur zwei M\"{o}glichkeiten
	\begin{align*}
	& i=3 & j=8,\\
	& i=8 & j =3.
	\end{align*} 
	Wir betrachten die zwei F\"{a}lle:
	\begin{enumerate}[label=(\roman*)]
		\item $i=3, j=8$ :

			Wir schreiben es als Zyklus, und dann von Transpositionen
			\[
				(3765)=(37)(76)(65)
			,\]
			also es ist gerade.
		\item $i=8,j=3$ wir machen ähnlich
			\[
				(37658)=(37)(76)(65)(58)
			,\] 
			also es ist in diesem Fall nicht gerade.\qedhere
	\end{enumerate}
\end{parts}
\end{proof}

\begin{Problem}
	Es sei $n\in \N^*$. Die Permutationen $\sigma, \tau\in S_n$ seien disjunkt.
	\begin{parts}
	\item Beweisen Sie Lemma 2.41: Es gilt $\sigma\tau=\tau\sigma$.
	\item Folgern Sie: Es ist $\text{ord}(\sigma\tau)=\text{kgV}(\text{ord}(\sigma),\text{ord}(\tau))$.
	\end{parts}
\end{Problem}
\begin{proof}
	\begin{parts}
	\item Kurze Erinnerung am Definition von disjunkter Permutationen:
		\begin{tcolorbox}
			\begin{Definition}\label{def:introalgblatt5-1}
			Zwei Permutationen $\sigma,\tau\in S_n$ heißen \emph{disjunkt}, falls gilt
			\begin{align*}
				\sigma(i)\neq i\implies& \tau(i)=i,\text{ und}\\
				\tau(i)\neq i\implies& \sigma(i)=i
			\end{align*}
		\end{Definition}
		\end{tcolorbox}
		Wir brauchen außerdem eine Ergebnis
		\begin{tcolorbox}
			\begin{Lemma}
				Sei $\sigma(i)=j\neq i$. Es gilt dann $\sigma(j)\neq j$.
			\end{Lemma}
			\begin{proof}
				Sonst wäre es ein Widerspruch zu die Definition, dass $S_n$ die Gruppe alle bijektive funktionen $\left\{ 1,\dots, n \right\} \to \left\{ 1, \dots, n \right\} $ ist. Die Permutation wäre dann nicht injektiv, weil $\sigma(i)=\sigma(j)$, aber per Annahme $i\neq j$ gilt.
			\end{proof}
			\begin{Corollary}\label{corollary:introalgblatt5-1}
			Sei $\sigma,\tau\in S_n$ disjunkter Permutation. Falls $\sigma(i)\neq i$ gilt $\tau\sigma\left( i \right) =\sigma(i)$.
			\end{Corollary}
			\begin{Remark}
				Alle Aussagen here gelten natürlich noch, wenn man die Rollen von $\sigma$ und $\tau$ vertauschen.
			\end{Remark}
		\end{tcolorbox}Die Ergebnis folgt jetzt fast sofort. Wir betrachten drei F\"{a}lle:
		\begin{enumerate}[label=(\roman*)]
			\item $\sigma(i)\neq i$, also $\tau(i)=i$.

				Es gilt dann
				\[
					\sigma\tau(i)\overset{\ref{def:introalgblatt5-1}}{=}\sigma(i)\overset{\ref{corollary:introalgblatt5-1}}{=}\tau\sigma(i)
				.\] 
			\item $\tau(i)\neq i$, also $\sigma(i)=i$.

				\[
					\tau\sigma(i)\overset{\ref{def:introalgblatt5-1}}{=}\tau(i)\overset{\ref{corollary:introalgblatt5-1}}{=}\sigma\tau(i)
				.\] 
			\item $\tau(i)=i$ und $\sigma(i)=i$.

				\[
				\tau\sigma(i)=i=\sigma\tau(i)
				.\] 
		\end{enumerate}
		Insgesamt gilt $\tau\sigma=\sigma\tau$.
	\end{parts}
\end{proof}
