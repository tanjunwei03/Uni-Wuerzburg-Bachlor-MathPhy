\begin{Problem}
	Sei $R:=K^{n\times n}$ der Ring der $n\times n$-Matrizen über einem Körper $K$. Zeigen Sie, dass $\{0\} $ und $R$ die einzigen Ideale von $R$ sind, vgl. Bemerkung 3.13.
\end{Problem}
\begin{proof}
	Sei $I\neq \{0\} $ ein Ideal von $R$. Wir nehmen an, dass $m\in I$ gibt mit $m$ invertierbar. Dann ist $m^{-1}\in R$ und $1 = m m^{-1}\in I$. Sei jetzt $m\in R$ beliebig. Da $1\in I$, ist $1m=m\in I$, also $I=R$.

	Deswegen nehmen wir an, dass $I$ nur nichtinvertierbare Elemente enthält.Sei $m\in I$. Aus dem Spektralsatz bzw. Jordannormalform haben wir eine Zerlegung $m=m_d+m_n$ mit $[m_d,m_n]=0$, $m_d$ diagonalisierbar und $m_n$ nilpotent. Da $m^n=m_d+m_n\in I$ f\"{u}r alle $n\in \N^*$, enthält $I$ ein diagonalisierbares Element. 

	Wir diagonalisieren das Element, also $I$ enthält ein diagonales Element, z.B. $\text{diag}(\lambda_1,\lambda_2,\dots, \lambda_n)\in I$. Wir wissen, dass Zeilenumformungen durch Matrixmulplikationen dargestellt werden kann. Durch diese Zeilenumformungen erhalten wir f\"{u}r jedes Diagonalelement zumindest ein Element, das keine Nullstelle in diesem Eintrag besitzt. 

	Wir additieren alle solche Elemente und erhalten daher eine diagonale Matrix, mit alle diagonale Elemente ungleich Null. Dies ist invertierbar, und die Behauptung folgt.
\end{proof}
\begin{Problem}
	Seien $R$ ein kommutativer Ring und $g\in R[x]$ ein normiertes Polynom vom Grad $n\in \N^*$. Sei $I:=gR\trianglelefteq R[X]$ das von $g$ erzeugte Ideal von $R[X]$. Zeigen Sie:
	\begin{parts}
		\item Jedes Element des Faktorrings $R[X] / I$ lässt sich in der Form
			\[
				r_0+r_1X+\dots, +r_{n-1}X^{n-1}+I\qquad\text{mit gewissen}~r_i\in R
			\]
			darstellen. Es gilt also $R[X] / I=\{r_0+r_1X+\dots+r_{n-1}X^{n-1}|r_0,\dots, r_{n-1}\in R\} $.
		\item Die Elemente $\sum_{i=0}^{n-1}r_i X^i +I$ aus der Menge in Teil (a) sind paarweise verschieden, d.h. f\"{u}r beliebige $r_0,\dots, r_{n-1},s_0,\dots, s_{n-1}$ gilt
			\[
				(r_0,\dots, r_{n-1})\neq (s_0,\dots, s_{n-1})\implies \sum_{i=0}^{n-1} r_i X^i+I\neq \sum_{i=0}^{n-1}s_i X^i+I
			.\] 
	\end{parts}
\end{Problem}
\begin{proof}
	\begin{parts}
	\item Per Definition können wir die Elemente als $p+I,p\in R[X]$. Es ist einfach zu zeigen, dass wir oBdA $\text{grad}(p)<n$. Wir betrachten $p+I$ mit $\text{grad }p\ge n$. Dann führen wir Division mit Rest durch und erhalten
		\[
			p=gq+r,~g,r\in R[x],\text{grad }r<n
		.\] 
		Es gilt deswegen
		\[
		p+I=(gq+r)+I
		.\] 
		Weil $gq\in I$ per Definition, ist $gq+I=I$ und
		\[
		p+I=r+I
	,\]
	also wir dürfen oBdA annehmen, dass $\text{grad }p<n$.
\item Sei die Polynome $r=(r_0,\dots, r_{n-1}$ und $s=(s_0,\dots, s_{n-1})$. Falls $r+I=s+I$, wäre $r-s\in I$. Da $\text{grad}r$ und $\text{grad}(s)<n$ sind, ist $\text{grad}(r-s)<n$. $r-s\in I$ liefert dann $r-s=gq$. Aber $g$ ist vom Grad $n$ und $gq$ ist daher vom Grad $\ge n$, ein Widerspruch.\qedhere
	\end{parts}
\end{proof}
\begin{Problem}
	Zur Abkürzung schreiben wir $\Z_n$ f\"{u}r den Ring $\Z / n\Z$. Zeigen Sie, dass die vier Ringe
	\[
		R_1:=\Z_4,~R_2:=\Z_2\times \Z_2,~R_3:=\Z_2[X] / X^2 \Z_2[X],~R_4:=\Z_2[X] / (X^2+X+1)\Z_2[X]
	\]
	paarweise nicht isomorph sind.
\end{Problem}
\begin{proof}
	Die Elemente sind
	\begin{align*}
		\Z_4:&\overline{0},\overline{1},\overline{2},\overline{3}\\
		\Z_2\times \Z_2:&(\overline{0},\overline{0}),(\overline{0},\overline{1})\\
				&(\overline{1},\overline{0}),(\overline{1},\overline{1})\\
		\Z_2[X] / X^2:&0+X^2,1+X^2\\
			      &X+X^2,X+1+X^2\\
		\Z_2[X] / (X^2+X+1):&0+(X^2+X+1),\\
				    &1+(X^2+X+1),\\
				    &X+(X^2+X+1),\\
				    &X+1+(X^2+X+1)
	\end{align*}
	Es gilt $\overline{1}+\overline{1}=\overline{2}\neq \overline{0}$ in $\Z_4$, aber das additive Neutrale hat immer die Ordnung $2$ in die letzte drei.
\end{proof}
\begin{Problem}
	Sei $R$ ein kommutativer Ring. Ein Element $a\in R$ heißt \emph{nilpotent}, falls $a^n=0$ f\"{u}r ein $n\ge 1$ gilt.
	\begin{parts}
	\item Begründen Sie, warum Null in einem Körper das einzige nilpotente Element ist.
	\item Zeigen Sie, dass das \emph{Nullradikal} $N:=\{a\in R|a\text{ ist nilpotent}\} $ ein Ideal ist.
	\item Zeigen Sie, dass das Nilradikal in jedem Primideal $P$ von $R$ enthalten ist.
	\end{parts}
\end{Problem}
\begin{proof}
	\begin{parts}
	\item Sei $K$ ein Körper mit $K\ni a\neq 0$. Da $K \backslash\{0\} $ multiplikativ abgeschlossen sein sollte, wäre $a^k\in K \backslash \{0\} $, also  $a^k\neq 0$ f\"{u}r alle $k\in \N^*$
	\item $N$ ist eine additive Untergruppe:
		\begin{enumerate}[label=(\roman*)]
			\item $0$ ist nilpotent (siehe oben)
			\item Sei $a$ nilpotent mit $a^n=0$. Es gilt dann $(-a)^n=(-1)^n a^n=(-1)^n 0=0$.
			\item Sei $a,b$ nilpotent mit $a^n=b^m=0$. Da $R$ kommutativ ist, dürfen wir den Binomialsatz verwenden. Wir betrachten
				\[
					(a+b)^{n+m}=\sum_{k=0}^n \binom{n}{k} a^k b^{n+m-k}
				.\]
				Falls $k\ge n$ wäre, gälte dann $a^k=0$ und $a^k b^{n+m-k}=0$. Falls $k\le n$ wäre, gälte $n+m-k\ge m$ und $b^{n+m-k}=0$, daher $a^k b^{n+m-k}=0$. Dann ist $(a+b)^{n+m}=0$, und $a+b$ ist nilpotent.
		\end{enumerate}
		Sei jetzt $a\in N,b\in R$ mit $a^n=0$. Es gilt $(ab)^n=a^nb^n=0b^n=0$.
	\item Sei $P$ ein Ideal von $R$ mit $N\not\subseteq P$. Wir zeigen, dass $P+N\supseteq P\cup N$ ein Ideal ist, also $P$ ist nicht maximal. 

		Die additive Eigenschaften folgen, da sowohl $P$ als auch $N$ Untergruppen sind. Da die additive Gruppe abelsch ist, ist $P+N=N+P$, also $P+N$ ist eine abelsche Untergruppe der additive Gruppe.

		Sei jetzt $p\in P, a\in N,r\in R$, also $p+a\in P+N$. Da sowohl $P$ als auch $N$ Ideale sind, ist
		\[
			r(p+a)=\underbrace{rp}_{\in P}+\underbrace{ra}_{\in N}\in P+N
		.\] 
		$P+N$ ist also ein Ideal. Da $N\not\subseteq P$, ist $P\subsetneq P+N$, also $P$ ist nicht maximal.\qedhere
	\end{parts}
\end{proof}
