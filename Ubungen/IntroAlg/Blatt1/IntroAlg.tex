\documentclass[prb,12pt]{revtex4-2}

\usepackage{amsmath, amssymb,physics,amsfonts,amsthm}
\usepackage{enumitem}
\usepackage{cancel}
\usepackage{booktabs}
\usepackage{tikz}
\usepackage{hyperref}
\usepackage{enumitem}
\usepackage{transparent}
\usepackage{float}
\usepackage{multirow}
\newtheorem{Theorem}{Theorem}
\newtheorem{Proposition}{Theorem}
\newtheorem{Lemma}[Theorem]{Lemma}
\newtheorem{Corollary}[Theorem]{Corollary}
\newtheorem{Example}[Theorem]{Example}
\newtheorem{Remark}[Theorem]{Remark}
\theoremstyle{definition}
\newtheorem{Problem}{Problem}
\theoremstyle{definition}
\newtheorem{Definition}[Theorem]{Definition}
\newenvironment{parts}{\begin{enumerate}[label=(\alph*)]}{\end{enumerate}}
%tikz
\usetikzlibrary{patterns}
% definitions of number sets
\newcommand{\N}{\mathbb{N}}
\newcommand{\R}{\mathbb{R}}
\newcommand{\Z}{\mathbb{Z}}
\newcommand{\Q}{\mathbb{Q}}
\newcommand{\C}{\mathbb{C}}
\begin{document}
\title{Einf\"{u}rung in die Algebra Hausaufgabenblatt Nr. 1}
	\author{Jun Wei Tan}
	\email{jun-wei.tan@stud-mail.uni-wuerzburg.de}
	\affiliation{Julius-Maximilians-Universit\"{a}t W\"{u}rzburg}
	\date{\today}
	\maketitle
\begin{Problem}
	Sei $G := 2\N^* := \{2n | n \in \N^* \}$ die Menge der positiven geraden Zahlen. Wir nennen $a \in G$ \emph{zerlegbar}, falls sich $a$ als Produkt zweier Elemente aus $G$ schreiben lässt. Ansonsten nennen wir $a$ unzerlegbar. Beispielsweise sind 4 zerlegbar und 6 unzerlegbar. Zeigen Sie:
	\begin{parts}
		\item $G$ ist multiplikativ abgeschlossen.
		\item Jedes $a\in G$ l\"{a}sst sich als Produkt unzerlegbarer Elemente aus $G$ schreiben.
		\item Selbst wenn man die Reihenfolge der Faktoren nicht berücksichtigt, so ist die Zerlegung nach (b) im Allgemeinen nicht eindeutig.
	\end{parts}
\end{Problem}

\begin{proof}
	\begin{parts}
	\item $2n\times 2n'=4nn'=2(nn')$
	\item Wir beweisen es per Induktion. Nehme an, dass jede Elemente  $2n, n < k$ entweder unzerlegbar ist, oder als Produkt unzerlegbare Elemente aus $G$ geschrieben werden kann. F\"{u}r $2(1)=2$ ist es klar - $2$ ist unzerlegbar.
		
		Sei $M_k\subseteq G =\left\{ m\in G| \exists n\in G, mn=2k\right\} $
		
		Entweder ist $M=\varnothing$, also $k$ ist unzerlegbar, oder es existiert $m,n\in G, mn=2k$. Weil $m$ und $n$ ein Produkt unzerlegbarer Elemente aus $G$ sind, ist $2k$ auch ein Produkt unzerlegbarer Elemente.
	\item Gegenbeispiel:
		\[
		G\ni 1020=30\times 34=102\times 10
		.\] 
	\end{parts}
\end{proof}
\begin{Problem}
	In dieser Aufgabe stellen wir den Euklidischen Algorithmus zur Berechung des größten gemeinsamen Teilers vor. Seien hierzu zwei natürliche Zahlen $a, b \in \N$ mit $b \neq 0$ vorgelegt. Wir setzen $r_0 := a, r1 := b$ und rekursiv für alle $i \in \N^*$ mit $r_i \neq 0$.
		\[r_{i+1}:=\text{ Rest von }r_{i-1}\text{ bei der Division durch }r_i\]
	\begin{enumerate}[label=(\alph*)]
	\item	Zeigen Sie, dass es ein $n\ge 2$ mit  $r_n=0$ gibt.
	\end{enumerate}	
Da die Rekursionsformel für $i = n$ nicht mehr anwendbar ist, bricht die Folge $(r_i)$ der Reste beim Index $n$ ab. Daher gibt es nur genau einen Index $n \ge 2$ mit $r_n = 0$. Beweisen Sie nun:
\begin{enumerate}[resume*]
\item F\"{u}r alle $i\in \left\{ 1,2,3,\dots,n \right\} $ gilt $ggT(a,b)=ggT(r_{i-1},r_i)$.
\item Es ist $ggT(a,b)=r_{n-1}$.
\item Berechnen Sie $ggT(210,45)$ mit Hilfe des Euklidschen Algorithmus.
\end{enumerate}
\end{Problem}

\begin{proof}
	\begin{parts}
	\item

		\[r_{i-1}=qr_{i}+r_{i+1}\qquad 0\le r_{i+1}<r_i\]
		per Definition. Weil $r_{i-1}<r_i$, ist die Folge monoton fallend. Da es endlich viele nat\"{u}rliche Zahlen $k<b$ gibt, muss $r_n=0$.
 

	\item Wir beweisen:
		\[
			ggT(r_{i-1}, r_i)=ggT(r_i,r_{i+1})
		.\]
		Die gewünschte Ergebnisse folgt daraus per Induktion.

		Es gilt $r_{i-1}-qr_i=r_{i+1}$. Dann folgt: $ggT(r_{i-1}, r_i)$ teilt $r_{i-1}$ und $r_i$ und daher auch $r_{i-1}-qr_i$. Deshalb ist $ggT(r_{i-1},r_i)$ auch einen Teiler von $r_{i+1}\implies ggT(r_{i-1},r_i)\le ggT(r_i, r_{i+1})$. 

		Weil $r_{i-1}=qr_i+r_{i+1}$, ist $ggT(r_i,r_{i+1})$ einen Teiler von $r_i$ und $r_{i+1}$ und daher auch von $qr_i+r_{i+1}$. Deshalb ist es auch einen Teiler von $r_{i-1}$, und $ggT(r_{i},r_{i+1})\le ggT(r_{i-1},r_i)$
	\item Es gilt
		\[
			r_{n-2}=qr_{n-1}+\cancel{r_{n}}
		,\] 
		also $r_{n-1}$ teilt $r_{n-2}$. Daraus folgt
		\[
			ggT(r_{n-1},r_{n-2})=r_{n-1}=ggT(a,b)
		.\] 
	\item 
		\begin{align*}
			210&=4 \times 45+30\\
			45&=1\times 30+15\\
			30&=2\times 15+0\\
			15&\\
			0&
		\end{align*}
		\[
		ggT(210,45)=15
		.\]

	\end{parts}
\end{proof}
\begin{Problem}
	(Bonus Problem) Wir wissen von dem Lemma von Bezout, dass f\"{u}r jeder $x,y,\in\N$ es $a,b\in \Z$ gibt, so dass
	\[
	ax+by=ggT(x,y)
	.\] 
	Zum Beispiel ist $-210+5\times 45=15$. Kann man von das Euklidische Algorithmus die Zahlen $a,b$ rechnen?
\end{Problem}
\begin{proof}
	Wir berechnen zuerst eine andere Beispiel
	\begin{align*}
		427=&1\times 264+163\\
		264=&1\times 163+101\\
		163=&1\times 101+62\\
		101=&1\times 62+39\\
		62=&1\times 39+23\\
		39=&1\times 23+16\\
		23=&1\times 16+7\\
		16=&2\times 7+2\\
		7=&3\times 2+1\\
		2=&1\times 2 +0
	\end{align*}
	Wir kehren zur\"{u}ck:
	\begin{align*}
		7-1=&3\times 2\\
		3\times 16=&6\times 7 +3\times 2\\
		=&6\times 7+(7 - 1)\\
		=&7\times 7-1\\
		6\times 16 =& 14\times 7-1\\
		6\times 16+1=& 14\times 7\\
		14\times 23 =& 14\times 16+14\times 7\\
		=& 14\times 16+(6\times 16+1)\\
		=& 20\times 16 +1
	\end{align*}
	In der letzte Gleichung bleibt $ggT(427,264)$ (1). Wir setzen immer wieder ein, bis zu wir eine Gleichung des Forms $427a+264b=1$ haben
\end{proof}
\begin{Problem}
	Sei $k\in\N$ gegeben. F\"{u}r welche Zahlen $\N\ni a,b<k$ braucht das Euklidische Algorithismus die meiste Schritte?
\end{Problem}
\begin{proof}
Wir möchten, dass die Folge $r_n\to 0$ nicht so schnell. 
	\begin{align*}
		13&=1\times 8+5\\
		8&=1\times 5+3\\
		5&=1\times 3 +2\\
		3&=1\times 2 +1\\
		2&=1\times 2+0\\
		1&\\
		0&
	\end{align*}
	ist die Fibonacci Folge.
\end{proof}
\begin{Problem}
	Seien $p$ und $q$ zwei ungerade und aufeinanderfolgende Primzahlen, so dass also zwischen $p$ und $q$ keine weiteren Primzahlen existieren. Zeigen Sie, dass $p + q$ ein Produkt von mindestens drei (nicht notwendig verschiedenen) Primzahlen ist.
\end{Problem}

\begin{proof}
	Sei obdA $p<q$. Weil $p$ und $q$ ungerade sind, ist $p+q$ gerade, also $p+q=2k, k\in \N$. Nehme an, dass $p+q$ ein Produkt von zwei Primzahlen ist, also $k\in \mathbb{P}$. Dann gilt
	\[
		p < k < q, \qquad k\in \mathbb{P}
	,\]
	ein Widerspruch. Deshalb ist $k\not\in \mathbb{P}$ und $k$ ist ein Produkt von mindestens zwei Primzahlen, also $p+q$ ist ein Produkt von mindestens drei Primzahlen. 
\end{proof}

\begin{Problem}
	Seien $n \in \N^*$ und $a \in \Z$. Zeigen Sie, dass es genau dann ein $x \in \Z$ mit $ax \equiv 1\pmod{n}$ gibt, wenn $ggT(a, n) = 1$ gilt.
\end{Problem}
\begin{proof}
	$ax\equiv 1\pmod{n}\iff ax-1=kn, k\in\Z$, also $ax-kn=1$.

	Weil $ggT(a,n)=1$, gibt es so zwei Zahlen $a, -k$, so dass $ax-kn=1$ (Lemma von Bezout)
\end{proof}
\end{document}
