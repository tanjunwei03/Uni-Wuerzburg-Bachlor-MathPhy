\begin{Problem}
	Es sei $X$ eine Zufallsvariable, die diskret uniform verteilt ist auf den ganzen Zahlen $\{-n, \ldots, n\}$ zwischen $-n$ und $n$, für ein $n \in \mathbb{N}$.
	
	\begin{parts}
		\item Zeigen Sie, dass $X \overset{d}{=} -X$, also dass $X$ und $-X$ dieselbe Verteilung haben.  
		Was ist $\mathbb{P}(X = -X)$?
		
		\item Bestimmen Sie den Erwartungswert $\mathbb{E}[X]$.
		
		\item Leiten Sie die Verteilung von $X^2$ her.
		
		\item Bestimmen Sie $\mathbb{E}[X^2]$.
	\end{parts}
\end{Problem}
\begin{proof}
	\begin{parts}
		\item \[\mathbb{P}(-X=a) = \underbrace{\mathbb{P}(X = -a)=\mathbb{P}(X=a)}_\text{uniform verteilt}\]
		Das Ereignis $\{X=-X\}$ ist auch das Ereignis $\{X=0\}$ mit Wahrscheinlichkeit $\frac 1{2n+1}$.
		\item Wir bezeichnen die Wahrscheinlichkeit $\mathbb{P}(X=a)=\frac{1}{2n+1}=p$ (unabhängig von $a$, da $X$ uniform verteilt ist).
		
		Der Erwartungswert ist
		\[\mathbb{E}(X)=\sum_{k=-n}^n pk = p(0)=0.\]
		\item Es gilt, f\"{u}r $0 \le a \le n$
		\[\mathbb{P}(X^2<a^2)=\mathbb{P}(|X| < a)=\frac{2\lfloor a \rfloor + 1}{2n+1}\]
		und damit ist
		\[\mathbb{P}(X^2 < a)=\frac{2\lfloor a^2\rfloor + 1}{2n+1}\]
		und die Verteilung von $X$
		\[\mathbb{P}(X^2 < a)=\begin{cases}
			0 & a < 0\\
			1 & a \ge n^2\\
			\frac{2\lfloor a^2\rfloor +1}{2n+1} & \text{sonst.}
		\end{cases}\]
		\item
		\begin{align*}
			\mathbb{E}[X^2]&=\sum_{k=1}^n \frac{2}{2n+1}k^2\\
			&=\frac 13 n(1+n).\qedhere
		\end{align*}
	\end{parts}
\end{proof}
\begin{Problem}
	Es sei $F^{-1} : (0, 1) \to \mathbb{R}$ eine Quantilsfunktion. Der Abstand $F^{-1}(3/4) - F^{-1}(1/4)$ heißt Interquartilsabstand der zugehörigen Verteilung.
	
	\begin{parts}
		\item Was ist der minimal mögliche Interquartilsabstand einer Verteilung? Geben Sie ein entsprechendes Beispiel an.
		\item Leiten Sie den Interquartilsabstand einer uniformen Verteilung $\mathcal{U}(a, b)$ auf $(a, b)$ her.
		\item Leiten Sie den Interquartilsabstand einer Exponentialverteilung $\text{Exp}(\lambda)$ mit Parameter $\lambda > 0$ her.
		\item Leiten Sie den Interquartilsabstand einer Normalverteilung $\mathcal{N}(\mu, \sigma^2)$ mit Parametern $\mu \in \mathbb{R}, \sigma^2 > 0$ her. Sie können benutzen, dass $\Phi(0{,}675) \approx 3/4$ ist für die Verteilungsfunktion $\Phi$ von $\mathcal{N}(0, 1)$.
		\item Welche Werte kann $\mathbb{P}(F^{-1}(1/4) \leq X \leq F^{-1}(3/4))$ minimal und maximal annehmen, falls $X$ mit Verteilungsfunktion $F$ verteilt ist? Geben Sie entsprechende Beispiele an.
		\item Bei welchen der drei Verteilungen aus (b), (c) und (d) liegt der Median in der Mitte des Intervalls $[F^{-1}(1/4), F^{-1}(3/4)]$?
	\end{parts}
	\end{Problem}
	\begin{proof}
		\begin{parts}
			\item $0$, sei $X$ eine konstante Zufallsvariabel $X=1$. Die Verteilungsfunktion ist
			\[F(x)= \begin{cases}
				0 & x < 1\\
				1 & \text{sonst.}
			\end{cases}\]
			Die Quantilsfunktion ist
			\[F^{-1}(x) = 1\]
			f\"{u}r $x\in (0,1)$ und damit $F^{-1}(3/4) - F^{-1}(1/4)=0$.
			\item Die Verteilungsfunktion ist
			\[F(x) = \frac{x-a}{b-a}\]
			f\"{u}r $x\in (a,b)$. Da die Verteilungsfunktion stetig ist, ist die Quantilsfunktion einfach deren Inverse:
			\[F^{-1}(x)= (b-a)x +a.\]
			\item Die Verteilungsfunktion ist
			\[F(x)=1-e^{-\lambda x}\]
			was wieder stetig und monoton steigend ist. Damit ist
			\[F^{-1}(x) =-\frac{1}{\lambda}\ln (1-x).\]
			\item Die Verteilungsfunktion $\Phi$ der Normalverteilung ist stetig. Damit ist $\Phi^{-1}(3/4)=0,675$. 
			
			Wir wissen auch, dass die Verteilung um $0$ symmetrisch ist: Falls
			\[\Phi(x) = \frac 12 + a,\]
			ist
			\[\Phi(-x) = \frac 12 - a.\]
			Damit ist
			\[\Phi(-0,695)=\frac 14,\]
			$\Phi^{-1}(1/4)=-0,695$ und
			\[F^{-1}\left(\frac 34\right)-F^{-1}\left(\frac 14\right)\]
			\item Die Wahrscheinlichkeit kann maximal 1 sein (siehe Beispiel aus (a)).
			
			Die Wahrscheinlichkeit kann minimal 1/2 sein. Irgendeine stetige Verteilungsfunktion passt, beispielsweise $\mathcal{U}((0,1))$.
			\item  (b) Ja, da $F^{-1}$ linear ist.
			
			(c) Nein, da $\ln \frac 34 + \ln \frac 14\neq \ln \frac 12$.
			
			(d) Ja, da $\Phi^{-1}\left(\frac 34\right)+\Phi^{-1}\left(\frac 14\right)=0=\Phi^{-1}\left(\frac 12\right)$.\qedhere
		\end{parts}
	\end{proof}