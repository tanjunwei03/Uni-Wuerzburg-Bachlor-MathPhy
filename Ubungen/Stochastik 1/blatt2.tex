\begin{Problem}
	Bei einer Sportveranstaltung wird ein Dopingtest durchgeführt. Falls eine Person gedopt ist, so fällt der Test zu 99\% auch positiv aus. Hat eine Person nicht gedopt, zeigt der Test trotzdem mit 5\% Wahrscheinlichkeit ein positives Ergebnis an. Aus Erfahrung sei bekannt, dass 20\% der Teilnehmenden gedopt sind.
	\begin{parts}
		\item Bestimmen Sie die Wahrscheinlichkeit dafur, dass eine Dopingprobe positiv ausf\"{a}llt.
		\item Bestimmen Sie die Wahrscheinlichkeit dafur, dass der Test negativ ausf\"{a}llt, obwohl die getestete Person gedopt ist.
		\item Berechnen Sie die Wahrscheinlichkeit dafur, dass eine Person gedopt ist, deren Test negativ ausgefallen ist.
	\end{parts}
\end{Problem}
\begin{proof}
	\begin{parts}
	\item Da eine Person entweder gedopt or nicht gedopt ist, können wir die Wahrscheinlichkeit zerlegen
		\begin{align*}
			P(\text{positiv})&=P(\text{positiv}|\text{gedopt})P(\text{gedopt}) + P(\text{positiv}|\text{nicht gedopt})P(\text{nicht gedopt})\\
					 &=0.99\cdot 0.2+0.05\cdot 0.8\\
					 &=0.238
		\end{align*}
	\item Die Wahrscheinlichkeit ist
		\[
			P(\text{negativ}|\text{gedopt})=1-P(\text{positiv}|\text{gedopt})=1-0.99=0.01
		.\] 
	\item Die Wahrscheinlichkeit ist
		\[
			P(\text{negativ}\cap \text{gedopt})=P(\text{negativ}|\text{gedopt})\cap P(\text{gedopt})=0.01\cdot 0.2=0.002
		.\qedhere\] 
	\end{parts}
\end{proof}
\begin{Problem}
	Betrachten Sie die Menge $\Omega = \{1, 2, 3\}$ und die Abbildung $p : \Omega \to [0, 1]$ mit $p(\omega) = \omega/6$, f\"{u}r $\omega\in\Omega$
	\begin{parts}
		\item Geben Sie vier verschiedene $\sigma$-Algebren auf $\Omega$ an.
		\item Geben Sie ein Beispiel an f\"{u}r $\sigma$-Algebren uber der Menge $\Omega$, so dass
		\[\mathcal{A}\sqcup \mathcal{B}:=\{A\cup B:A\in\mathcal{A}\text{ und }B\in\mathcal{B}\},\]
		\emph{keine} $\sigma$-Algebra ist.
		\item Zeigen Sie, dass ein Wahrscheinlichkeitsmaß $\mathbb{P}$ auf $(\Omega,\mathcal{P}(\Omega))$ existiert, so dass $\mathbb{P}(\{\omega\}) =p(\omega)=\omega/6$.
		\item Bestimmen Sie die $\sigma$-Algebra, welche von $\mathcal{E} = \{A\in \mathcal{P}(\Omega)|\mathbb{P}(A) = 1/2\}$ erzeugt wird
	\end{parts}
	\end{Problem}
\begin{proof}
	\begin{parts}
		\item \begin{gather*}
			\{\varnothing, \Omega\}\\
			\{\varnothing, \{1\}, \{2,3\}, \Omega\}\\
			\{\varnothing, \{1,2\}, \{3\},\Omega\}\\
			\mathcal{P}(\Omega)
		\end{gather*}
	\item Wir betrachten die zweite und dritte $\sigma$-Algebren:
	\begin{align*}
		\mathcal{A}&=\{\varnothing, \{1\}, \{2,3\}, \Omega\}\\
		\mathcal{B}&=\{\varnothing, \{1,2\}, \{3\},\Omega\}\\
		\mathcal{A}\sqcup \mathcal{B}&=\{\varnothing, \{1\}, \{2,3\}, \{1,2\}, \{3\}, \{1,3\}, \Omega \}
	\end{align*}
was keine $\sigma$-Algebra ist, da $\Omega \setminus \{1,3\}=\{2\}\not\in \mathcal{A}\sqcup \mathcal{B}$.
\item Wir können die Funktion durch deren Wirkung auf jeder Teilmenge definieren:
\begin{align*}
	\mathbb{P}(\varnothing)&=0\\
	\mathbb{P}(\{\omega\})&=\omega/6\\
	\mathbb{P}(\{1,2\})&=1/2\\
	\mathbb{P}(\{2,3\})&=5/6\\
	\mathbb{P}(\{1,3\})&=2/3\\
	\mathbb{P}(\Omega)&=1
\end{align*}
Durch Betrachtung alle Permutationen kann man zeigen, dass $\mathbb{P}$ $\sigma$-additiv ist. Daher ist $\mathbb{P}$ ein Wahrscheinlichkeitsmaß.
\item $\mathcal{E}=\{\{1,2\},\{3\}\}$. Daher ist die von $\mathcal{E}$ erzeugte $\sigma$-Algebra
\[\mathcal{A}_\sigma(\mathcal{E})=\{\varnothing, \{1,2\}, \{3\}, \Omega\}\]
Dies ist die kleinste $\sigma$-Algebra, die $\mathcal{E}$ enthält, da eine $\sigma$-Algebra die Nullmenge und die gesamte Menge enthalten muss. Man darf die anderen 2 Mengen auch nicht weglassen, da die Mengen aus $\mathcal{E}$ sind. Man verfiziere auch, dass Komplementen und Vereinigungen noch in $\mathcal{A}_\sigma(\mathcal{E})$ sind.\qedhere
	\end{parts}
\end{proof}
