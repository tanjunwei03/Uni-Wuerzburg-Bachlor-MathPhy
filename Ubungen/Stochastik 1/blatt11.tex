\begin{Problem}
	Betrachte den Wahrscheinlichkeitsraum \(( [0, 1], \mathcal{B}([0, 1]), \mathcal{U}([0, 1]))\), mit \(\Omega = [0, 1]\) und der uniformen Verteilung \(\mathcal{U}([0, 1])\). Mit den Teilmengen 
	\[
	E_1 = [0, 1/4] \cup [1/2, 3/4], \quad E_2 = [0, 1/3] \cup [2/3, 1] \quad \text{und} \quad E_3 = [0, 1/2],
	\]
	seien zwei Mengensysteme \(\mathcal{E}_1\) und \(\mathcal{E}_2\) gegeben durch 
	\[
	\mathcal{E}_1 = \{E_1, E_2\}, \quad \mathcal{E}_2 = \{E_3\}.
	\]
	
	\begin{enumerate}
		\item[(a)] Zeigen Sie, dass \(\mathcal{E}_1\) und \(\mathcal{E}_2\) unabhängig sind.
		\item[(b)] Zeigen Sie, dass \(\sigma(\mathcal{E}_1)\) und \(\mathcal{E}_2\) nicht unabhängig sind.
		\item[(c)] Folgern Sie, dass die von \(\mathcal{E}_1\) und \(\mathcal{E}_2\) erzeugten \(\sigma\)-Algebren nicht unabhängig sind. Wieso folgt aus (a) nicht die Unabhängigkeit der erzeugten \(\sigma\)-Algebren?
	\end{enumerate}
\end{Problem}
\begin{proof}
	\begin{parts}
		\item Wir betrachten
	\end{parts}
\end{proof}

\begin{Problem}
	Es sei \(X\) eine exponentialverteilte Zufallsvariable, \(X \sim \text{Exp}(\lambda)\), \(\lambda > 0\). Die Zufallsvariable \(Y\) sei unabhängig von \(X\) mit 
	\[
	\mathbb{P}(Y = 1) = \mathbb{P}(Y = -1) = \frac{1}{2}.
	\]
	Leiten Sie die Verteilungsfunktion der Zufallsvariablen \(Z = X \cdot Y\) her.
\end{Problem}