\begin{Problem}
	Seien $X_1$, $X_2$ und $X_3$ Bernoulli-verteilte Zufallsvariablen mit Parametern $p_1$, $p_2$ und $p_3$, $p_j \in [0, 1], 1 \le j \le 3$. Es gilt also $\mathbb{P}(X_j = 1) = 1 - \mathbb{P}(X_j = 0) = p_j , 1 \le j \le 3$. Ereignisse der Form $\{X_i = k\}, \{X_j = l\}, l, k \in \{0, 1\}$, seien für alle $i \neq j$ unabhängig.
	\begin{parts}
		\item Bestimmen Sie die Wahrscheinlichkeit des Ereignisses $\{X_1 + X_2 + X_3 = 1\}$.
		\item Leiten Sie die Verteilung der Zufallsvariable $S = X_1 +X_2 +X_3$ her. Geben Sie insbesondere an, um welche Art von Zufallsvariable es sich hier handelt.
		\item Bestimmen Sie die Wahrscheinlichkeit des Ereignisses $\{X_1 < X_3\}$.
		\item Wir definieren die Zufallsvariable $Y = \mathbbm{1}\{X_1 < X_2\}$. Wie ist $Y$ verteilt?
	\end{parts}
\end{Problem}
\begin{proof}
	\begin{parts}
		\item Wir berechnen die Wahrscheinlichkeit des Ereignisses $\{X_1=1,X_2=0,X_3=0\}$. Die Wahrscheinlichkeit ist
		\[\mathbb{P}(X_1=1,X_2=0,X_3=0)=p_1(1-p_2)(1-p_3).\]
		Analog finden wir
		\[\mathbb{P}(X_1=0,X_2=1,X_3=0)=(1-p_1)p_2(1-p_3)\]
		und
		\[\mathbb{P}(X_1=0,X_2=0,X_3=1)=(1-p_1)(1-p_2)p_3.\]
		Da diese Ereignisse disjunkt sind, und deren Vereinigung das Ereignis $\{X_1+X_2+X_3=1\}$ ist, ist die Wahrscheinlichkeit einfach
		\[\mathbb{P}(X_1+X_2+X_3=1)=p_1(1-p_2)(1-p_3)+(1-p_1)p_2(1-p_3)+(1-p_1)(1-p_2)p_3\]
		\item Es gilt $X_1+X_2+X_3\in \{0,1,2,3\}$. Also wir suchen die Zähldichte
		\[\mathbb{P}(X_1+X_2+X_3=2).\]
		Ähnlich betrachten wir die drei Ereignisse
		\begin{align*}
			\mathbb{P}(X_1=1,X_2=1,X_3=0)&=p_1p_2(1-p_3)\\
			\mathbb{P}(X_1=0,X_2=1,X_3=1)&=(1-p_1)p_2p_3\\
			\mathbb{P}(X_1=1,X_2=0,X_3=1)&=p_1(1-p_2)p_3
		\end{align*}
		und
		\[\mathbb{P}(X_1+X_2+X_3=2)=p_1p_2(1-p_3)+(1-p_1)p_2p_3+p_1(1-p_2)p_3\]
		Letztlich ist
		\[\mathbb{P}(X_1+X_2+X_3=3)=\mathbb{P}(X_1=1,X_2=1,X_3=1)=p_1p_2p_3\]
		und
		\[\mathbb{P}(X_1+X_2+X_3=0)=\mathbb{P}(X_1=0,X_2=0,X_3=0)=(1-p_1)(1-p_2)(1-p_3).\]
		Das Wahrscheinlichkeitsmaß ist definiert durch die Summe
		\[\mathbb{P}(A)=\sum_{x\in A}\mathbb{P}(\{x\}).\]
		\item Es gilt
		\[\{X_1<X_3\}=\{X_1=0,X_3=1\}\]
		und damit
		\[\mathbb{P}(X_1<X_3)=(1-p_1)p_3.\]
		\item $Y$ is Bernoulli-verteilt mit Parameter $(1-p_1)p_3$.\qedhere
	\end{parts}
\end{proof}
\begin{Problem}
	\begin{parts}
	\item Zeigen Sie, dass es sich bei
	\[F(x)=\begin{cases}
		0 & x < 0\\
		\frac{7}{20}+\frac{1}{5}x+\frac{1}{20}x^2 & 0 \le x < 1\\
		1 & x \ge 1
	\end{cases}\]
um eine Verteilungsfunktion handelt.
\item Es handelt sich hier um eine Mischung aus diskreter und stetiger Verteilung, also ist $F$ von der Form
\[F(x)=a F^d(x)+b F^s(x)\]
mit positiven Zahlen $a$ und $b$, $a+b=1$, und $F^d$ Verteilungsfunktion einer diskreten und $F^s$ einer stetigen Verteilung.

Bestimmen Sie $a,b$, sowie $F^d$ und $F^s$. Skizzieren Sie $F^d, F^s$ und $F$. Geben Sie die Wahrscheinlichkeitsfunktion der diskreten Verteilung an.
\item Ein Median $m$ einer Verteilung wird häufig eingeführt als eine Zahl, für welche
\[\mathbb{P}(X\le m)\ge 1/2,\text{ und }\mathbb{P}(X\ge m)\ge 1/2,\]
gilt. Bestimmen Sie einen so definierten Median zu der durch $F$ charakterisierten Verteilung.
\end{parts}
\end{Problem}
\begin{proof}
	\begin{parts}
		\item \begin{enumerate}[label=(\arabic*)]
			\item Monoton wachsend: Wir betrachten die Ableitung
			\[F'(x)=\begin{cases}
				0 & x < 0\\
				\frac 15 + \frac{1}{10} x & 0 < x < 1\\
				0 & x > 1.
			\end{cases}\]
			Da die Funktion fast überall stetig differenzierbar ist mit positiv semidefiniter Ableitung, ist sie monoton wachsend.
			\item Die Funktion ist per Definition rechtseitig stetig.
			\item $F(x)$ geht gegen 1 bzw. Null when $x$ geht gegen $+\infty$ bzw. $-\infty$.
		\end{enumerate}
		\item Wir sehen, dass es Atome bei $x=0$ und $x=1$ gibt:
		\begin{align*}
			F(0+)-F(0-)&=\frac{7}{20}\\
			F(1+)-F(1-)&=1-\frac{3}{5}=\frac{2}{5}.
		\end{align*}
		Diese Atome stellen die diskrete Verteilung dar. Wir müssen die Verteilung normieren
		\[\frac{7}{20}+\frac{2}{5}=\frac{3}{4}\]
		und
		\[F^d(x)=\begin{cases}
			0 & x < 0\\
			\frac{7}{15} & 0 \le x < 1\\
			1 & x \ge 1
		\end{cases}.\]
		Die stetige Verteilung ergibt sich durch den streng wachsenden Term zwischen $0$ und $1$. Die Normierung ist
		\[\left[\frac{7}{20}+\frac{1}{5}x+\frac{1}{20}x^2\right]_{0}^1=\frac 14\]
		und damit
		\[F^s(x)=\frac{4}{5}x+\frac{1}{5}x^2.\]
		Die Gewichtung ist $a=\frac 34$, $b=\frac 14$.
		\item Wir bestimmen $m$ durch die Gleichung $F(x)=\frac 12$. Die L\"{o}sung ist $m=-2+\sqrt{7}$.\qedhere
	\end{parts}
\end{proof}