\begin{Problem}
Was ist beim 7-fachen Wurf eines fairen Wurfels die Wahrscheinlichkeit daf ¨ ur, dass	
\begin{parts}
\item mindestens zwei Sechsen geworfen werden?
\item jede Augenzahl 1, . . . , 6 unter den Wurfergebnissen erscheint?
\item mindestens eines der beiden obigen Ereignisse eintritt?
Geben Sie zu den L¨osungen auch einen passenden Wahrscheinlichkeitsraum an, in welchem Sie die Ereignisse modellieren.
\end{parts}
\end{Problem}
\begin{proof}
	Der Wahrscheinlichkeitsraum ist $\{0,\cdots, 6\}^7$ mit dem diskreten Maß.
	\begin{parts}
	\item Wir betrachten die Möglichkeit, dass weniger als 2 Sechsen geworfen werden. Daraus ergeben sich 2 Fälle: Genau ein $6$ wurde geworfen und kein $6$ wurden geworfen. 

		Wenn kein $6$ geworfen werden soll, gibt es für jedes Wurf 5 Möglichkeiten, also die Wahrscheinlichkeit dieses Fälles ist $(5 / 6)^7$.

		Wenn genau ein $6$ geworfen werden soll, ist ein Wurf festgelegt. Es gibt $5^6$ Möglichkeiten für die andere Würfe. Aber mit Berücksichtigung der Reihenfolge müssen wir dieses Ergebnis mit $7$ multiplizieren, da das $6$ bei jedem Wurf erscheinen kann. Damit ist die gesamte Wahrscheinlichkeit $7\cdot (5 / 6)^6 \cdot (1 / 6)$.

		Maßtheoretisch haben wir dieses Ereignis in $7$ Ereignisse geteilt, also dass das $6$ in dem $n$-ten Wurf erscheint. Da diese Ereignisse disjunkt sind, können wir die Summe der Wahrscheinlichkeiten als die gesamte Wahrscheinlichkeit betrachten. Aus der Symmetrie sind deren Wahrscheinlichkeiten gleich, also wir multiplizieren einfach mit $7$. 
	
Damit ist die Wahrscheinlichkeit, dass wir weniger als 2 $6$ bekommen
\[\frac{15625}{23328}\]
und die Wahrscheinlichkeit, dass wir mindestens 2 $6$ bekommen ist
\[
\frac{7703}{23328}
.\] 
	\end{parts}
\end{proof}
