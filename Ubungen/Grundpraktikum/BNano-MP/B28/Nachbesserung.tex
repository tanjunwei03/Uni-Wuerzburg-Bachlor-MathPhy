\documentclass[prb,12pt]{revtex4-2}

\usepackage{amsmath, amssymb,physics,amsfonts,amsthm}
\usepackage[most]{tcolorbox}
\usepackage{enumitem}
\usepackage{cancel}
\usepackage{booktabs}
\usepackage{polynom}
\usepackage{tabularx}
\usepackage{tikz}
\usepackage{hyperref}
\usepackage{enumitem}
\usepackage[normalem]{ulem}
\usepackage{transparent}
\usepackage{caption}
\usepackage{float}
\usepackage{multirow}
\newtheorem{Theorem}{Theorem}
\newtheorem{Proposition}{Theorem}
\newtheorem{Lemma}[Theorem]{Lemma}
\newtheorem{Corollary}[Theorem]{Corollary}
\newtheorem{Example}[Theorem]{Example}
\newtheorem{Remark}[Theorem]{Remark}
\theoremstyle{definition}
\newtheorem{Problem}{Problem}
\theoremstyle{definition}
\newtheorem{Definition}[Theorem]{Definition}
\newenvironment{parts}{\begin{enumerate}[label=(\alph*)]}{\end{enumerate}}
%tikz
\usetikzlibrary{patterns}
\usetikzlibrary{matrix}
%tcolorbox
\tcbset{breakable=true,toprule at break = 0mm,bottomrule at break = 0mm}
% definitions of number sets
\newcommand{\N}{\mathbb{N}}
\newcommand{\R}{\mathbb{R}}
\newcommand{\Z}{\mathbb{Z}}
\newcommand{\Q}{\mathbb{Q}}
\newcommand{\C}{\mathbb{C}}
\allowdisplaybreaks
\setlength{\parindent}{0cm}
\captionsetup[table]{name=Tabelle}

\begin{document}
\title{Why am I doing this}
	\author{Jun Wei Tan}
	\email{jun-wei.tan@stud-mail.uni-wuerzburg.de}
	\affiliation{Julius-Maximilians-Universit\"{a}t W\"{u}rzburg}
	\date{\today}
	\maketitle

\section{Spannungsempfindlichkeit}
\subsection{Gleichspannung}
\begin{enumerate}
	\item Multimeter besser geeignet.
\end{enumerate}
\subsection{Wechselspannung}
\begin{enumerate}
	\item Im Vergleich zu einem Multimeter mit einem größeren Messbereich ist das Oszilloskop besser geeignet.
\end{enumerate}
\section{AC/DC-Kopplung}
\begin{enumerate}
	\item Geeignet, wenn es ein Superposition von einer kleinen Schwingung und einer konstanten Spannung (Gleichspannung) gibt.\cite{Keysight}
	\item z.B. Messung der Rippelstrom aus einem DC-Netzgerät wegen Gleichrichtung einer netzfrequenten Wechselspannung~	\cite{zhu1996switched,scoles1980handbook}
\end{enumerate}
\bibliographystyle{ieeetr}

\bibliography{ref}	
\end{document}
